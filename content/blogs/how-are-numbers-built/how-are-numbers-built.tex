% Options for packages loaded elsewhere
\PassOptionsToPackage{unicode,linktoc=all}{hyperref}
\PassOptionsToPackage{hyphens}{url}
\PassOptionsToPackage{dvipsnames,svgnames,x11names}{xcolor}
%
\documentclass[
  a4paper,
]{article}
\usepackage{amsmath,amssymb}
\usepackage{iftex}
\ifPDFTeX
  \usepackage[T1]{fontenc}
  \usepackage[utf8]{inputenc}
  \usepackage{textcomp} % provide euro and other symbols
\else % if luatex or xetex
  \usepackage{unicode-math} % this also loads fontspec
  \defaultfontfeatures{Scale=MatchLowercase}
  \defaultfontfeatures[\rmfamily]{Ligatures=TeX,Scale=1}
\fi
\usepackage{lmodern}
\ifPDFTeX\else
  % xetex/luatex font selection
\fi
% Use upquote if available, for straight quotes in verbatim environments
\IfFileExists{upquote.sty}{\usepackage{upquote}}{}
\IfFileExists{microtype.sty}{% use microtype if available
  \usepackage[]{microtype}
  \UseMicrotypeSet[protrusion]{basicmath} % disable protrusion for tt fonts
}{}
\makeatletter
\@ifundefined{KOMAClassName}{% if non-KOMA class
  \IfFileExists{parskip.sty}{%
    \usepackage{parskip}
  }{% else
    \setlength{\parindent}{0pt}
    \setlength{\parskip}{6pt plus 2pt minus 1pt}}
}{% if KOMA class
  \KOMAoptions{parskip=half}}
\makeatother
\usepackage{xcolor}
\usepackage[margin=25mm]{geometry}
\usepackage{color}
\usepackage{fancyvrb}
\newcommand{\VerbBar}{|}
\newcommand{\VERB}{\Verb[commandchars=\\\{\}]}
\DefineVerbatimEnvironment{Highlighting}{Verbatim}{commandchars=\\\{\}}
% Add ',fontsize=\small' for more characters per line
\usepackage{framed}
\definecolor{shadecolor}{RGB}{48,48,48}
\newenvironment{Shaded}{\begin{snugshade}}{\end{snugshade}}
\newcommand{\AlertTok}[1]{\textcolor[rgb]{1.00,0.81,0.69}{#1}}
\newcommand{\AnnotationTok}[1]{\textcolor[rgb]{0.50,0.62,0.50}{\textbf{#1}}}
\newcommand{\AttributeTok}[1]{\textcolor[rgb]{0.80,0.80,0.80}{#1}}
\newcommand{\BaseNTok}[1]{\textcolor[rgb]{0.86,0.64,0.64}{#1}}
\newcommand{\BuiltInTok}[1]{\textcolor[rgb]{0.80,0.80,0.80}{#1}}
\newcommand{\CharTok}[1]{\textcolor[rgb]{0.86,0.64,0.64}{#1}}
\newcommand{\CommentTok}[1]{\textcolor[rgb]{0.50,0.62,0.50}{#1}}
\newcommand{\CommentVarTok}[1]{\textcolor[rgb]{0.50,0.62,0.50}{\textbf{#1}}}
\newcommand{\ConstantTok}[1]{\textcolor[rgb]{0.86,0.64,0.64}{\textbf{#1}}}
\newcommand{\ControlFlowTok}[1]{\textcolor[rgb]{0.94,0.87,0.69}{#1}}
\newcommand{\DataTypeTok}[1]{\textcolor[rgb]{0.87,0.87,0.75}{#1}}
\newcommand{\DecValTok}[1]{\textcolor[rgb]{0.86,0.86,0.80}{#1}}
\newcommand{\DocumentationTok}[1]{\textcolor[rgb]{0.50,0.62,0.50}{#1}}
\newcommand{\ErrorTok}[1]{\textcolor[rgb]{0.76,0.75,0.62}{#1}}
\newcommand{\ExtensionTok}[1]{\textcolor[rgb]{0.80,0.80,0.80}{#1}}
\newcommand{\FloatTok}[1]{\textcolor[rgb]{0.75,0.75,0.82}{#1}}
\newcommand{\FunctionTok}[1]{\textcolor[rgb]{0.94,0.94,0.56}{#1}}
\newcommand{\ImportTok}[1]{\textcolor[rgb]{0.80,0.80,0.80}{#1}}
\newcommand{\InformationTok}[1]{\textcolor[rgb]{0.50,0.62,0.50}{\textbf{#1}}}
\newcommand{\KeywordTok}[1]{\textcolor[rgb]{0.94,0.87,0.69}{#1}}
\newcommand{\NormalTok}[1]{\textcolor[rgb]{0.80,0.80,0.80}{#1}}
\newcommand{\OperatorTok}[1]{\textcolor[rgb]{0.94,0.94,0.82}{#1}}
\newcommand{\OtherTok}[1]{\textcolor[rgb]{0.94,0.94,0.56}{#1}}
\newcommand{\PreprocessorTok}[1]{\textcolor[rgb]{1.00,0.81,0.69}{\textbf{#1}}}
\newcommand{\RegionMarkerTok}[1]{\textcolor[rgb]{0.80,0.80,0.80}{#1}}
\newcommand{\SpecialCharTok}[1]{\textcolor[rgb]{0.86,0.64,0.64}{#1}}
\newcommand{\SpecialStringTok}[1]{\textcolor[rgb]{0.80,0.58,0.58}{#1}}
\newcommand{\StringTok}[1]{\textcolor[rgb]{0.80,0.58,0.58}{#1}}
\newcommand{\VariableTok}[1]{\textcolor[rgb]{0.80,0.80,0.80}{#1}}
\newcommand{\VerbatimStringTok}[1]{\textcolor[rgb]{0.80,0.58,0.58}{#1}}
\newcommand{\WarningTok}[1]{\textcolor[rgb]{0.50,0.62,0.50}{\textbf{#1}}}
\usepackage{longtable,booktabs,array}
\usepackage{calc} % for calculating minipage widths
% Correct order of tables after \paragraph or \subparagraph
\usepackage{etoolbox}
\makeatletter
\patchcmd\longtable{\par}{\if@noskipsec\mbox{}\fi\par}{}{}
\makeatother
% Allow footnotes in longtable head/foot
\IfFileExists{footnotehyper.sty}{\usepackage{footnotehyper}}{\usepackage{footnote}}
\makesavenoteenv{longtable}
\setlength{\emergencystretch}{3em} % prevent overfull lines
\providecommand{\tightlist}{%
  \setlength{\itemsep}{0pt}\setlength{\parskip}{0pt}}
\setcounter{secnumdepth}{-\maxdimen} % remove section numbering
% definitions for citeproc citations
\NewDocumentCommand\citeproctext{}{}
\NewDocumentCommand\citeproc{mm}{%
  \begingroup\def\citeproctext{#2}\cite{#1}\endgroup}
\makeatletter
 % allow citations to break across lines
 \let\@cite@ofmt\@firstofone
 % avoid brackets around text for \cite:
 \def\@biblabel#1{}
 \def\@cite#1#2{{#1\if@tempswa , #2\fi}}
\makeatother
\newlength{\cslhangindent}
\setlength{\cslhangindent}{1.5em}
\newlength{\csllabelwidth}
\setlength{\csllabelwidth}{3em}
\newenvironment{CSLReferences}[2] % #1 hanging-indent, #2 entry-spacing
 {\begin{list}{}{%
  \setlength{\itemindent}{0pt}
  \setlength{\leftmargin}{0pt}
  \setlength{\parsep}{0pt}
  % turn on hanging indent if param 1 is 1
  \ifodd #1
   \setlength{\leftmargin}{\cslhangindent}
   \setlength{\itemindent}{-1\cslhangindent}
  \fi
  % set entry spacing
  \setlength{\itemsep}{#2\baselineskip}}}
 {\end{list}}
\usepackage{calc}
\newcommand{\CSLBlock}[1]{\hfill\break#1\hfill\break}
\newcommand{\CSLLeftMargin}[1]{\parbox[t]{\csllabelwidth}{\strut#1\strut}}
\newcommand{\CSLRightInline}[1]{\parbox[t]{\linewidth - \csllabelwidth}{\strut#1\strut}}
\newcommand{\CSLIndent}[1]{\hspace{\cslhangindent}#1}
\ifLuaTeX
\usepackage[bidi=basic]{babel}
\else
\usepackage[bidi=default]{babel}
\fi
\babelprovide[main,import]{british}
% get rid of language-specific shorthands (see #6817):
\let\LanguageShortHands\languageshorthands
\def\languageshorthands#1{}
% $HOME/.pandoc/defaults/latex-header-includes.tex
% Common header includes for both lualatex and xelatex engines.
%
% Preliminaries
%
% \PassOptionsToPackage{rgb,dvipsnames,svgnames}{xcolor}
% \PassOptionsToPackage{main=british}{babel}
\PassOptionsToPackage{english}{selnolig}
\AtBeginEnvironment{quote}{\small}
\AtBeginEnvironment{quotation}{\small}
\AtBeginEnvironment{longtable}{\centering}
%
% Packages that are useful to include
%
\usepackage{graphicx}
\usepackage{subcaption}
\usepackage[inkscapeversion=1]{svg}
\usepackage[defaultlines=4,all]{nowidow}
\usepackage{etoolbox}
\usepackage{fontsize}
\usepackage{newunicodechar}
\usepackage{pdflscape}
\usepackage{fnpct}
\usepackage{parskip}
  \setlength{\parindent}{0pt}
\usepackage[style=american]{csquotes}
% \usepackage{setspace} Use the <fontname-plus.tex> files for setspace
%
\usepackage{hyperref} % cleveref must come AFTER hyperref
\usepackage[capitalize,noabbrev]{cleveref} % Must come after hyperref
\let\longdivision\relax
\usepackage{longdivision}
% noto-plus.tex
% Font-setting header file for use with Pandoc Markdown
% to generate PDF via LuaLaTeX.
% The main font is Noto Serif.
% Other main fonts are also available in appropriately named file.
\usepackage{fontspec}
\usepackage{setspace}
\setstretch{1.3}
%
\defaultfontfeatures{Ligatures=TeX,Scale=MatchLowercase,Renderer=Node} % at the start always
%
% For English
% See also https://tex.stackexchange.com/questions/574047/lualatex-amsthm-polyglossia-charissil-error
% We use Node as Renderer for the Latin Font and Greek Font and HarfBuzz as renderer ofr Indic fonts.
%
\babelfont{rm}[Script=Latin,Scale=1]{NotoSerif}% Config is at $HOME/texmf/tex/latex/NotoSerif.fontspec
\babelfont{sf}[Script=Latin]{SourceSansPro}% Config is at $HOME/texmf/tex/latex/SourceSansPro.fontspec
\babelfont{tt}[Script=Latin]{FiraMono}% Config is at $HOME/texmf/tex/latex/FiraMono.fontspec
%
% Sanskrit, Tamil, and Greek fonts
%
\babelprovide[import, onchar=ids fonts]{sanskrit}
\babelprovide[import, onchar=ids fonts]{tamil}
\babelprovide[import, onchar=ids fonts]{greek}
%
\babelfont[sanskrit]{rm}[Scale=1.1,Renderer=HarfBuzz,Script=Devanagari]{NotoSerifDevanagari}
\babelfont[sanskrit]{sf}[Scale=1.1,Renderer=HarfBuzz,Script=Devanagari]{NotoSansDevanagari}
\babelfont[tamil]{rm}[Renderer=HarfBuzz,Script=Tamil]{NotoSerifTamil}
\babelfont[tamil]{sf}[Renderer=HarfBuzz,Script=Tamil]{NotoSansTamil}
\babelfont[greek]{rm}[Script=Greek]{GentiumBookPlus}
%
% Math font
%
\usepackage{unicode-math} % seems not to hurt % fallabck
\setmathfont[bold-style=TeX]{STIX Two Math}
\usepackage{amsmath}
\usepackage{esdiff} % for derivative symbols
% \renewcommand{\mathbf}{\symbf}
%
%
% Other fonts
%
\newfontfamily{\emojifont}{Symbola}
%

\usepackage{titling}
\usepackage{fancyhdr}
    \pagestyle{fancy}
    \fancyhead{}
    \fancyfoot{}
    \renewcommand{\headrulewidth}{0.2pt}
    \renewcommand{\footrulewidth}{0.2pt}
    \fancyhead[LO,RE]{\scshape\thetitle}
    \fancyfoot[CO,CE]{\footnotesize Copyright © 2006\textendash\the\year, R (Chandra) Chandrasekhar}
    \fancyfoot[RE,RO]{\thepage}
%
\usepackage{newunicodechar}
\newunicodechar{√}{\textsf{√}}
\usepackage {caption}
    \captionsetup{font={sf,stretch=1.4}}
\ifLuaTeX
  \usepackage{selnolig}  % disable illegal ligatures
\fi
\IfFileExists{bookmark.sty}{\usepackage{bookmark}}{\usepackage{hyperref}}
\IfFileExists{xurl.sty}{\usepackage{xurl}}{} % add URL line breaks if available
\urlstyle{sf}
\hypersetup{
  pdftitle={How Are Numbers Built?},
  pdfauthor={R (Chandra) Chandrasekhar},
  pdflang={en-GB},
  colorlinks=true,
  linkcolor={DarkOliveGreen},
  filecolor={Purple},
  citecolor={DarkKhaki},
  urlcolor={Maroon},
  pdfcreator={LaTeX via pandoc}}

\title{How Are Numbers Built?}
\author{R (Chandra) Chandrasekhar}
\date{2024-03-21 | 2024-04-21}

\begin{document}
\maketitle

\thispagestyle{empty}


``How are numbers built?'' This was the simple but profound question
that my \href{https://www.vocabulary.com/dictionary/polymath}{polymath}
friend Solus ``Sol'' Simkin asked me when we met unexpectedly at an
evening function.

``Sol, have a
\href{https://www.collinsdictionary.com/dictionary/english/sense-of-occasion}{sense
of occasion}, of time and place, please! This is a social event at a
Music School, not the
\href{https://en.wikipedia.org/wiki/Ancient_Agora_of_Athens}{Agora of
Athens}! Your question is too deep to be discussed here and now. We are
planning to go on a tour of Santorini in a month's time. Let our
thoughts mingle with those of the ancient, philosophical Greeks. Until
then, I will take a
\href{https://dictionary.cambridge.org/dictionary/english/raincheck}{raincheck}
on your question,'' I remonstrated.

And so it was that Sol next resumed this conversational thread while we
gazed upon the azure sea, from under the shade of an
\href{http://wedding.beleon.com/media/15172-beleontoursgreeceweddingsantoriniolivegrove.jpg}{olive
grove}, atop a hillock on
\href{https://en.wikipedia.org/wiki/Santorini}{Santorini}.

\subsection{How would you build a
world?}\label{how-would-you-build-a-world}

``If you were given the power to build a world, how would you do it?''
Sol asked me without forewarning.

``Why so
\href{https://dictionary.cambridge.org/dictionary/english/outlandish}{outlandish}
a question? Enjoy the sun and the breeze, and the bleats of the sheep,''
I replied.

``Have you heard of the
\href{https://worldbuilding.stackexchange.com/}{Worldbuilding
Stackexchange}? It `is a question and answer site for writers/artists
using science, geography and culture to construct imaginary worlds and
settings'.''

``No,'' I said.

``It is a serious site on the Web where bizarre worlds, with negative
gravity and entropy, may be conceived and discussed, before being
constructed and populated. My question is not a flippant one.''

``I stand educated. But what has mathematics to do with those flights of
fancy?'' I queried.

Sol said, ``Everything''. ``One cannot construct a world without the
laws of physics, or the laws of mind. Or the laws of cause and effect.
As long as structure, consistency, repeatability, and durability are
desired, one cannot do without numbers. More than light or atoms,
numbers are the building blocks of the world.''

We had launched at last into the discussion proper. And what a majestic
premise: that the world is built with numbers, before it could be built
with light or atoms. I asked Sol to let his canons of unassailable
argument boom, while I waited passively to be informed and entertained.

\subsection{Lessons from observing
life}\label{lessons-from-observing-life}

``You must have heard of my paternal cousin, once removed, Hieronymus
Septimus Simkin, whom I affectionately call Seven. He it was who opened
my eyes first to the unguarded secrets staring at us from Nature. He
introduced me to books like D'Arcy Wentworth Thompson's classic \emph{On
Growth and Form} {[}\citeproc{ref-thompson-1992}{1}{]} and the
interestingly titled \emph{The Parsimonious Universe}
{[}\citeproc{ref-parsimonious-1996}{2}{]}. These books postulate, with
incontrovertible evidence, that the
\href{https://en.wikipedia.org/wiki/Book_of_Nature}{Book of Nature}
derives its intelligence from adaptation, powered by mathematics.

``If Nature is constructed from---or using---mathematics, how are
numbers constructed? Are numbers themselves the very first creation of a
colossal intelligence? Numbers. Before light, before atoms, before cause
and effect?''

But, Sol did not stop there.

\subsection{``God made the integers''}\label{god-made-the-integers}

``Nature is varied and variegated in a way that defies monotony. There
is pattern but also variation.
\href{https://www.treehugger.com/amazing-fractals-found-in-nature-4868776}{Fractals}
typify what I am trying to convey. Perhaps, you will remember that
\href{https://en.wikipedia.org/wiki/Leopold_Kronecker}{Leopold
Kronecker} was reputed to have said \emph{`Die ganzen Zahlen hat der
liebe Gott gemacht, alles andere ist Menschenwerk'}, meaning `(The Dear)
God made the integers, all else is the work of man'
{[}\citeproc{ref-kronecker}{3}{]},'' Sol thundered on.

He waited for his exposition to sink in. Given our idyllic surroundings,
it was hard not to day dream and slip silently into slumber. He ordered
two glasses of
\href{https://en.wikipedia.org/wiki/Frapp\%C3\%A9_coffee}{Frappé} to
keep me from descending into somnolence.

\subsection{The integers have their place, but
\ldots{}}\label{the-integers-have-their-place-but}

``The integers are fundamental because all mathematics begins with
counting. The quantitative fields are all founded on the natural numbers
we count with. And
\href{https://swanlotus.netlify.app/blogs/the-two-most-important-numbers-zero-and-one}{zero
and one are the two most important integers}---that I grant you. But can
we stop with the integers, and exclude everything else?'' asked Sol.

``Are you trying to play Devil's advocate, Sol?'' I asked somewhat
confused by the change in tenor of his argument.

``Aha! So, you are still awake enough to follow what I say,'' he
laughed. ``Yes, that was a deliberate rhetorical question, and a segue
to my next observation.''

\subsection{The square and the circle}\label{the-square-and-the-circle}

``The square is \emph{the} four-sided regular polygon,'' Sol observed.
``If we consider a square with a side length equal to one unit, by the
theorem of Pythagoras, we know that its diagonal has a length equal to
\(\sqrt{1^2 + 1^2} = \sqrt{2}\) units. And there are proofs aplenty on
the Web that this number is in no way an integer. Indeed, it is not even
the ratio of two integers. How could something as basic as the diagonal
of a square cause the first chink in Kronecker's armour?

``Moving from the finite to the infinite, the circle may be viewed as
the limiting case of a regular polygon of \(n\) sides as
\(n \to \infty\). And if we tried to find out how many diameters would
fit into the circumference of a circle, we do not get an integer, or
even an exact fraction, but rather a number that sits between \(3\) and
\(4\), having decimal places without end, namely,
\(\pi \approx 3.141592654\dots\). And that number is not an integer by a
country mile.

``The
\href{https://swanlotus.netlify.app/blogs/the-two-most-important-numbers-zero-and-one}{natural
numbers, the integers, and the rationals}---all of these come under
Kronecker's integers, but where do we stash \(\sqrt{2}\) and \(\pi\)
amongst them?''

Sol's earnest question was met by bemused silence from me.

\subsection{\texorpdfstring{How about the number
\(e\)?}{How about the number e?}}\label{how-about-the-number-e}

Never one to leave a thread of thought half-fleshed out, Sol mounted his
next hobby horse, and expounded on
\href{https://en.wikipedia.org/wiki/E_(mathematical_constant)}{\(e\)},
the mystical number, sometimes called
\href{https://en.wikipedia.org/wiki/E_(mathematical_constant)}{Euler's
number}.

``The number \(e\) is probably \emph{the} most important number after
\(0\) and \(1\). And do you know what it is? It is both
\href{https://mathworld.wolfram.com/IrrationalNumber.html}{irrational}
and
\href{https://en.wikipedia.org/wiki/Transcendental_number}{transcendental}.
If you differentiate or integrate, you will find that the exponential
function \(\exp(x) = e^x\) is an
\href{https://swanlotus.netlify.app/blogs/eigenvalues-and-eigenvectors-why-are-they-important}{eigenfunction}
of each operation. If you look into Nature, \(e\) holds the pride of
place in the
\href{https://www.khanacademy.org/math/statistics-probability/modeling-distributions-of-data/normal-distributions-library/a/normal-distributions-review}{normal
distribution}. If you are into
\href{https://www.cns.nyu.edu/~david/handouts/linear-systems/linear-systems.html}{linear
system theory} you cannot escape \(e\).

``But what exactly is the value of \(e\)? It cannot be confined like an
integer: \(e \approx 2.718281828\dots\), again in a never ending decimal
sequence. This number pervades all of Nature and yet it cannot be
bottled into a finite number of digits! Were the legions of integers to
duel with this puny expeditionary force of three numbers, \(\sqrt{2}\),
\(\pi\), and \(e\), which group would you expect to win?

``It appears that Nature has inserted into the foundations of Creation,
non-integers like \(\sqrt{2}\), \(\pi\), and \(e\). \emph{But how are
these numbers built?} If you had to create a universe, what method would
you use to \emph{exactly construct} these three convoluted numbers at
the bedrock of Creation?''

``Very penetrating,'' I nodded in appreciation.

``Let me digress a little,'' Sol continued.

\subsection{Open secrets}\label{open-secrets}

``\href{https://en.wikipedia.org/wiki/Helen_Keller}{Helen Keller} is
reputed to have exclaimed, when she felt the warm glow of a wood-fire,
that it was the release of sunbeams that had been trapped long ago in
the wood. Her statement is remarkably perceptive, poetic, and precise,''
Sol continued.

``Unlike ancient sunlight trapped in wood, \(\sqrt{2}\), \(\pi\), and
\(e\), cannot be caged in a finite box. These three numbers---that
pervade Nature---have decimal forms that clearly announce that they are
\emph{not} integers. Their value defies finite expression; only with
symbols may we do them justice.

``Do you know why they are open secrets? They are public, staring at us
from every square, circle, and electrical signal, and yet, their full
form is never revealed. They cannot be contained except in infinity. To
know the next decimal place of \(\sqrt{2}\), or \(\pi\), or \(e\), one
needs to \emph{compute it} using some formula. Or one could look up a
table. But there is no \emph{knowing} that sought after next decimal
place, as we know \(\frac{1}{2} = 0.5\), with as many zeros stacked at
the end as we wish. That sort of closed form is not baked into nature.
She prefers the indescribable exactitude of numbers without end, like
\(e\).'' Sol \href{https://www.thefreedictionary.com/set+forth}{set
forth}.

The rest of Sol's dialogue was intricately mathematical. I have recorded
it here, substantially as a logical exposition---complete with
references---for the benefit of the casual reader, with bits of direct
speech thrown in.

\subsection{The square root of two}\label{the-square-root-of-two}

Of the triad---\(\sqrt{2}\), \(\pi\), and \(e\)---we first consider
\(\sqrt{2}\). It is the most within our everyday grasp. It evokes
geometry rather than number for its precise expression. It is the
diagonal of a unit square. And we know that its square root must lie
between \(1\) and \(1.5\), as the latter squared is \(2.25\). It may be
evaluated painstakingly using algorithms from the
age-before-calculators. So, let us look at one of those first.

\subsubsection{Manual extraction of root
two}\label{manual-extraction-of-root-two}

The manual extraction of square roots is analogous to long division. The
process is both tedious and error-prone. The algorithm uses the fact
that the factor \(2\) figures in any square, witness:
\((x + a) = x^2 + 2ax +a^2\). So, this particular method makes use of
this fact at each step in the ``long division'' that is done. To see the
end result and the working, please see
\href{https://www.cuemath.com/algebra/square-root-of-2/}{this}
{[}\citeproc{ref-cuemathsqrt}{4}{]}. For a deeper explanation,
\href{https://www.cantorsparadise.com/the-square-root-algorithm-f97ab5c29d6d}{read
this blog} {[}\citeproc{ref-ujjwalsingh2021}{5}{]}. ``I consider this
form of working, with pencil and paper, a sophisticated form of torture.
\href{https://en.wikipedia.org/wiki/Leonhard_Euler}{Euler} or
\href{https://en.wikipedia.org/wiki/Carl_Friedrich_Gauss}{Gauss} might
have revelled in such pursuits, but count me out!'' Sol added as a snide
aside.

\subsection{Different ways of expressing a
number}\label{different-ways-of-expressing-a-number}

The decimal representation of a number is not the only way to express
it. For example, the integer \(5\) may be expressed as:
\(5 = \frac{5}{1} = 5.0 = 101_2\)\footnote{Binary for 5.}, but being
prime, it cannot be decomposed into factors. And even with decimals, we
may rightfully claim \(5.0 = 4.9999\dots\),
{[}\citeproc{ref-courant-robbins-1996}{6}{]}, which only muddies the
waters a little more. So, how would the Creator have defined our triad
of numbers in the most succinct way?

The decimal representation comes from expressing a number as the sum of
fractions whose denominators are powers of ten. And if the decimal is
never ending, the process of division and summation does not terminate.
Recall that the \hyperref[manual-extraction-of-root-two]{Manual
extraction of root two} also relied on division of sorts. \emph{So, does
division hold the key to how numbers are built?}

Sol then confessed, ``I had forgotten that the decimal system is not the
only way to represent irrationals and transcendentals in never-ending
glory. And I don't mean a change of base. Can you guess what I had
forgotten?''

``Nothing from me to egg you on,'' I said in a sleepy tone. The time,
place, and weather had lulled me into a restful somnolence that was
ill-suited to mathematical head-scratching, even with the Frappé.

``It is something that we learn at high school, more as a curiosity than
as useful mathematics,'' Sol continued by way of enticing me with a
clue. ``Can you guess what it is?''

When I shook my head with a dazed stare, Sol said, ``Come on. One last
clue. It has to do with division and fractions.''

When I refused to be drawn into guessing what it was, Sol exclaimed,
``\href{https://en.wikipedia.org/wiki/Continued_fraction}{Continued
Fractions}!''
{[}\citeproc{ref-olds1963}{7}--\citeproc{ref-loya2017}{11}{]} rousing me
into full wakefulness with his thunderous voice.

``Apart from a change of base, there are basically \emph{two} ways I
know of representing real numbers: decimals, and continued fractions.
Patterns not discernible in the decimal representation suddenly pop out
with pellucid clarity when the same number is expressed as a continued
fraction. The advent of computers and 64-bit computation has diverted
our attention away from experiencing the periodic beauty of a
\href{https://en.wikipedia.org/wiki/Quadratic_irrational_number}{quadratic
irrational}, expressed as a continued fraction,'' Sol went on,
lyrically.

``Practically, every irrational, when pressed to computational use, is
really a rational approximation to the irrational, to an accuracy that
serves the purpose. In that sense, Kronecker was not far from the truth.
But the full glory of \(\sqrt{2}\), or \(\pi\), or \(e\) can only be
encapsulated by the symbols we use for them. Every other, rational
expression is but a costumed appearance, not the true persona.'' Sol was
in his element as he expounded.

\subsection{The charm of continued
fractions}\label{the-charm-of-continued-fractions}

Sol then went on to demonstrate his preferred method of evaluating
\(\sqrt{2}\), using continued fractions. The method seemed like sleight
of hand, but it is well-founded, and is also an example of how integers
are used to tame the irrationals.

Continued fractions are curious mathematical entities that have
surprising properties. They are an alternative rational number
representation of real numbers. No finite continued fraction can equate
to an irrational number. But a never-ending continued fraction can
indeed represent an irrational number. ``This is why I say that the
rationals and the irrationals meet at infinity,'' Sol said with panache.

\subsubsection{Continued fraction expansion of a rational
number}\label{continued-fraction-expansion-of-a-rational-number}

``Let us start modestly and try to expand a \emph{rational} number using
continued fractions,'' said Sol. ``Give me a scary or hairy rational
number, preferably larger than one,'' he said.

``What about \(\frac{3257}{106}\)?'' I answered, choosing the two
numbers that randomly came to mind.

``Taken,'' replied Sol. We start off by doing plain long division to
get: \begin{equation}\phantomsection\label{eq:part}{
\begin{aligned}
\tfrac{3257}{106} &= 30 + \tfrac{77}{106}\\
&= 30 + \frac{1}{\frac{106}{77}}\\
\end{aligned}
}\end{equation} Why do we write it like this? We want to get whole
number quotients and whole number remainders. The trick is to always
divide the larger number by the smaller, by \emph{inverting} the
remainder fraction. If you keep in mind that our goal always is an
improper fraction, you are good to go.

``Because \(\frac{3257}{106}\) is a rational number, the continued
fraction terminates. The full expansion is shown in \cref{eq:full}
below: \begin{equation}\phantomsection\label{eq:full}{
\begin{aligned}
\tfrac{3257}{106} &= 30 + \tfrac{77}{106}\\
&= 30 + \frac{1}{\frac{106}{77}}\\
&= 30 + \cfrac{1}{1 + \tfrac{77}{29}}\\
&= 30 + \cfrac{1}{1 + \cfrac{1}{2 + \cfrac{1}{1 + \cfrac{1}{1 + \cfrac{1}{1 + \tfrac{1}{9}}}}}}
\end{aligned}
}\end{equation}

You will agree that this form---more easily written by hand than
typed---is a little cumbersome. So, the convention for writing a
continued fraction is to enclose the quotients and remainders in square
brackets and express it as \([30; 1, 2, 1, 1, 1, 9]\), with a semi-colon
after the integer part, and commas separating the other digits. Note
that we have exact equality: \[
\tfrac{3257}{106} = [30; 1, 2, 1, 1, 1, 9].
\] We conclude that---in addition to a decimal representation---a number
may be expressed as a continued fraction. We assert the equality below:
\[
\tfrac{3257}{106} = 30 \tfrac{77}{106} = 30.\overline{72641509433962} = [30; 1, 2, 1, 1, 1, 9].
\] Note that the decimal expansion is recurring with a period
{[}\citeproc{ref-period}{12}{]} of \(13\) digits, whereas, the continued
fraction expansion terminates.

\subsubsection{Continued fraction expansion of
√2}\label{continued-fraction-expansion-of-2}

The irrational number \(\sqrt{2}\) is amenable to a simply derived
continued fraction expansion. Consider:
\begin{equation}\phantomsection\label{eq:sqrt2}{
\begin{aligned}
\sqrt{2} &= \sqrt{2} &\text{ (add and subtract $1$ on the RHS)}\\
&= 1 + \sqrt{2} - 1 &\text{ (multiply second term on RHS by $\frac{\sqrt{2}+1}{\sqrt{2}+1} = 1$)}\\
&= 1 + \frac{(\sqrt{2} - 1)(\sqrt{2} + 1)}{\sqrt{2} + 1} &\text{ (difference of two squares)}\\
&= 1 + \frac{1}{1 + \textcolor{Maroon}{\sqrt{2}}}.
\end{aligned}
}\end{equation} This is a
\href{https://en.wikipedia.org/wiki/Recursion_(computer_science)}{recursion}
embodying \(\sqrt{2}\). Since the LHS\footnote{RHS and LHS stand for
  Right Hand Side and Left Hand Side resectively.} in \cref{eq:sqrt2} is
\(\sqrt{2}\), we may substitute the entire RHS in place of the term
\(\textcolor{Maroon}{\sqrt{2}}\) on the RHS. So doing, we get the
following infinite descending staircase of continued fractions:
\begin{equation}\phantomsection\label{eq:sqrt2cfinfty}{
\begin{aligned}
\sqrt{2} &= 1 + \frac{1}{1 + \textcolor{Maroon}{\sqrt{2}}}\\
&= 1 + \cfrac{1}{1 + \textcolor{Maroon}{1 + \cfrac{1}{1+\sqrt{2}}}}\\
&= 1 + \cfrac{1}{2 + \cfrac{1}{1 + \sqrt{2}}} &\text{ (and recursively substituting for $\sqrt{2}$ again)}\\
&= 1 + \cfrac{1}{2 + \cfrac{1}{1 + 1 + \cfrac{1}{1 + \sqrt{2}}}}\\
&= 1 + \cfrac{1}{2 + \cfrac{1}{2 + \cfrac{1}{1 + \sqrt{2}}}}\\
&= 1 + \cfrac{1}{2 + \cfrac{1}{2 + \cfrac{1}{2 + \cfrac{1}{1+\sqrt{2}}}}}\\
&\hskip 100pt\ddots\\ % Care!
\end{aligned}
}\end{equation} Because the continued fraction repeats itself, we may
write: \[
\sqrt{2} = [1; 2, 2, 2, 2\dots] = [1; \overline{2}].
\] This is an \emph{exact, succinct, and elegant} representation.

\subsection{Congruents}\label{congruents}

The \emph{congruents} or \emph{approximants} from a continued fraction
are partial sums that we may accumulate as \emph{successive rational
approximations} to the irrational number---\(\sqrt{2}\) in our
case---that we seek to represent. Unfurling the continued fractions into
partial sums is a tricky exercise. There are also recurrence relations
for them {[}\citeproc{ref-olds1963}{7}--\citeproc{ref-loya2017}{11}{]}.
In our particular case, we ignore the terminal
\(\frac{1}{1 + \sqrt{2}}\) terms that occur in the \emph{denominator} of
\cref{eq:sqrt2cfinfty} but count the numerator terms to get a sequence
of fractions.

In this way, we start off with \(1\), followed by
\(1 + \frac{1}{2} = \frac{3}{2}\). Working our way down, we encounter
\(1 + \frac{1}{2 + \frac{1}{2}} = 1+\frac{1}{\frac{5}{2}} = 1 + \frac{2}{5} = \frac{7}{5}\).
The next convergent after this, when simplified, is
\(1 + \frac{1}{2+\frac{2}{5}} = 1 + \frac{5}{12} = \frac{17}{12}\).

The first fifteen convergents are tabulated in
\cref{tbl:sqrt2convergents}. Note that these values oscillate about the
true value as consecutive congruents successively overestimate and
underestimate the irrational number. Some of the congruents have large
numerators and denominators. In many cases, the decimal representations
have recurring decimals that could have \emph{very} long periods, as
indicated in the third column of the table.

\begin{longtable}[]{@{}clr@{}}
\caption{\label{tbl:sqrt2convergents}The first fifteen convergents for
\(\sqrt{2}\). The periods of the repeating portions of the decimals were
obtained from the \href{https://www.wolframalpha.com}{Wolfram Alpha}
website.}\tabularnewline
\toprule\noalign{}
Convergent & Decimal Value & Period \\
\midrule\noalign{}
\endfirsthead
\toprule\noalign{}
Convergent & Decimal Value & Period \\
\midrule\noalign{}
\endhead
\bottomrule\noalign{}
\endlastfoot
\(\frac{1}{1}\) & \(1.0\) & \(0\) \\
\(\frac{3}{2}\) & \(1.5\) & \(0\) \\
\(\frac{7}{5}\) & \(1.4\) & \(0\) \\
\(\frac{17}{12}\) & \(1.41\overline{6}\) & \(1\) \\
\(\frac{41}{29}\) & \(1.\overline{4137931034\dots}\) & \(28\) \\
\(\frac{99}{70}\) & \(1.4\overline{142857}\) & \(6\) \\
\(\frac{239}{169}\) & \(1.\overline{4142011834\dots}\) & \(78\) \\
\(\frac{577}{408}\) & \(1.414\overline{2156862745098039}\) & \(16\) \\
\(\frac{1393}{985}\) & \(1.41421319796954314\dots\) & \(98\) \\
\(\frac{3363}{2378}\) & \(1.4142136248\) & \(140\) \\
\(\frac{8119}{5741}\) & \(1.4142135516\dots\) & \(5740\) \\
\(\frac{19601}{13860}\) & \(1.41\overline{421356}\) & \(6\) \\
\(\frac{47321}{33461}\) & \(1.4142135620\dots\) & \(4780\) \\
\(\frac{114243}{80782}\) & \(1.4142135624\dots\) & \(546\) \\
\(\frac{275807}{195025}\) & \(1.4142135623\dots\) & \(1876\) \\
\end{longtable}

Sol said that working out the fractions in \cref{tbl:sqrt2convergents}
could be a form of torture, unless you are particularly fond of, or
adept at computing them by hand. He himself did not relish such hand
computations, but preferred to program to get a solution. The link to a
program is given toward the end of this blog.

The rational fractions above are tabulated with their decimal versions
to provide an idea of how the convergents do indeed converge to the
``benchmark'' decimal value of \(\sqrt{2}\) as available on a
\texttt{Julia}
\href{https://en.wikipedia.org/wiki/Read\%E2\%80\%93eval\%E2\%80\%93print_loop}{REPL},
which is shown below. There is agreement at best to about ten decimal
places.

\begin{Shaded}
\begin{Highlighting}[]
\FunctionTok{sqrt}\NormalTok{(}\FunctionTok{big}\NormalTok{(}\FloatTok{2}\NormalTok{))}
\FloatTok{1.414213562373095048801688724209698078569671875376948073176679737990732478462102}
\end{Highlighting}
\end{Shaded}

\subsection{Elegant and inelegant
representations}\label{elegant-and-inelegant-representations}

Sol said, ``It is clear that \(\tfrac{1}{3} = 0.\overline{3}\) is an
elegant representation for the rational number \(\tfrac{1}{3}\). The
recurring decimals are not an issue; it is whether the digits may be
\emph{predicted} beforehand.''

``Likewise, \(\sqrt{2} = [1, \overline{2}]\) is an elegant
representation for the irrational number \(\sqrt{2}\).

``Two different approaches have led to two different representations of
two different numbers---one rational and the other irrational---that are
\emph{both elegant}. I consider that a marvel.

``This leads me to think that there might be other ways in which the
important numbers in Nature may be expressed using only integers. We
know only of decimals and continued fractions,'' Sol mused. ``But there
must be other \emph{identities} as yet undiscovered.''

``What about infinite series and such for \(\pi\) and \(e\)?'' I
ventured.

``Spot on,'' said Sol. ``It is my belief that the Creator built each
number that plays a major role in Nature using some elegant and succinct
representation. If the act of Creation were not efficient or
parsimonious, I do not think we will have the diversity we experience
today. Let us talk about \(\pi\) and \(e\) and \(\phi\) some other
time.''

``My only quibble is with \emph{prime numbers}. They cannot be built
from anything except by adding \(1\) to their predecessors. The day we
solve the mystery of how the primes are built, we would have understood
a major mystery of Creation, and learned to think like the Creator!''

And on that final note, Sol and I wrapped up our discussion on how
numbers are built, while enjoying the idyllic environment of Santorini.

\subsection{Program link}\label{program-link}

A simple \texttt{Julia} program,
\href{auxiliary/ContFrac.jl}{ContFrac.jl}, is available. It provides
functions related to continued fractions, but no claims are made as to
its absolute correctness. \emojifont {😉} \normalfont Take a look if you
wish.

\subsection{Acknowledgements}\label{acknowledgements}

The free \href{https://www.wolframalpha.com}{Wolfram Alpha} website is a
valuable resource. I am unsure, though if---with the march of time---its
multitudinous functions will be gradually furled up behind a paywall.

\subsection{Feedback}\label{feedback}

Please \href{mailto:feedback.swanlotus@gmail.com}{email me} your
comments and corrections.

\noindent A PDF version of this article is
\href{./how-are-numbers-built.pdf}{available for download here}:

\begin{small}

\begin{sffamily}

\url{https://swanlotus.netlify.app/blogs/how-are-numbers-built.pdf}

\end{sffamily}

\end{small}

\section*{References}\label{bibliography}
\addcontentsline{toc}{section}{References}

\phantomsection\label{refs}
\begin{CSLReferences}{0}{0}
\bibitem[\citeproctext]{ref-thompson-1992}
\CSLLeftMargin{{[}1{]} }%
\CSLRightInline{D'Arcy Wentworth Thompson. 1992. \emph{On growth and
form} (The Complete Revised Edition ed.). Dover Publications.}

\bibitem[\citeproctext]{ref-parsimonious-1996}
\CSLLeftMargin{{[}2{]} }%
\CSLRightInline{Stefan Hildebrandt and Anthony Tromba. 1996. \emph{{The
Parsimonious Universe: Shape and Form in the Natural World}} (1st ed.).
Copernicus Books.}

\bibitem[\citeproctext]{ref-kronecker}
\CSLLeftMargin{{[}3{]} }%
\CSLRightInline{Wikipedia contributors. 2023. {Leopold
Kronecker---Wikipedia, The Free Encyclopedia}. Retrieved 6 December 2023
from \url{https://en.wikipedia.org/wiki/Leopold_Kronecker}}

\bibitem[\citeproctext]{ref-cuemathsqrt}
\CSLLeftMargin{{[}4{]} }%
\CSLRightInline{---. 2023. {Square Root of 2}. Retrieved 9 December 2023
from \url{https://www.cuemath.com/algebra/square-root-of-2/}}

\bibitem[\citeproctext]{ref-ujjwalsingh2021}
\CSLLeftMargin{{[}5{]} }%
\CSLRightInline{Ujjwal Singh. 2021. {The Square Root Algorithm}.
Retrieved 8 December 2023 from
\url{https://www.cantorsparadise.com/the-square-root-algorithm-f97ab5c29d6d}}

\bibitem[\citeproctext]{ref-courant-robbins-1996}
\CSLLeftMargin{{[}6{]} }%
\CSLRightInline{Richard Courant and Herbert Robbins. 1996. \emph{{What
is Mathematics?}: {An Elementary Approach to Ideas and Methods}} (2nd
ed.). Oxford University Press.}

\bibitem[\citeproctext]{ref-olds1963}
\CSLLeftMargin{{[}7{]} }%
\CSLRightInline{C D Olds. 1963. \emph{{Continued Fractions}}. Random
House.}

\bibitem[\citeproctext]{ref-niven1991}
\CSLLeftMargin{{[}8{]} }%
\CSLRightInline{Ian Niven and Herbert S Zuckerman and Hugh L Montgomery.
1991. \emph{{An Introduction to the Theory of Numbers}} (5th ed.). John
Wiley \& Sons.}

\bibitem[\citeproctext]{ref-davenport2008}
\CSLLeftMargin{{[}9{]} }%
\CSLRightInline{Harold Davenport and James H Davenport. 2008. \emph{{The
Higher Arithmetic}: {An Introduction to the Theory of Numbers}} (8th
ed.). Cambridge University Press.}

\bibitem[\citeproctext]{ref-simoson2019}
\CSLLeftMargin{{[}10{]} }%
\CSLRightInline{Andrew J Simoson. 2019. \emph{{Exploring Continued
Fractions}: {From the Integers to Solar Eclipses}}. MAA Press/American
Mathematical Society.}

\bibitem[\citeproctext]{ref-loya2017}
\CSLLeftMargin{{[}11{]} }%
\CSLRightInline{Paul Loya. 2017. \emph{{Amazing and Aesthetic Aspects of
Analysis}}. Springer.}

\bibitem[\citeproctext]{ref-period}
\CSLLeftMargin{{[}12{]} }%
\CSLRightInline{Aslan986, Ross Millikan, and Bill Dubuque. 2012.
{Compute the period of a decimal number a priori}. Retrieved 13 December
2023
from\href{\%0A\%20\%20\%20\%20\%20\%20\%20\%20\%20https://math.stackexchange.com/questions/140583/compute-the-period-of-a-decimal-number-a-priori\%0A\%20\%20\%20\%20\%20\%20\%20\%20\%20}{
https://math.stackexchange.com/questions/140583/compute-the-period-of-a-decimal-number-a-priori
}}

\end{CSLReferences}



\end{document}
