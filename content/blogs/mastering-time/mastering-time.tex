% Options for packages loaded elsewhere
\PassOptionsToPackage{unicode,linktoc=all}{hyperref}
\PassOptionsToPackage{hyphens}{url}
\PassOptionsToPackage{dvipsnames,svgnames,x11names}{xcolor}
%
\documentclass[
  a4paper,
]{article}
\usepackage{amsmath,amssymb}
\usepackage{lmodern}
\usepackage{iftex}
\ifPDFTeX
  \usepackage[T1]{fontenc}
  \usepackage[utf8]{inputenc}
  \usepackage{textcomp} % provide euro and other symbols
\else % if luatex or xetex
  \usepackage{unicode-math}
  \defaultfontfeatures{Scale=MatchLowercase}
  \defaultfontfeatures[\rmfamily]{Ligatures=TeX,Scale=1}
\fi
% Use upquote if available, for straight quotes in verbatim environments
\IfFileExists{upquote.sty}{\usepackage{upquote}}{}
\IfFileExists{microtype.sty}{% use microtype if available
  \usepackage[]{microtype}
  \UseMicrotypeSet[protrusion]{basicmath} % disable protrusion for tt fonts
}{}
\makeatletter
\@ifundefined{KOMAClassName}{% if non-KOMA class
  \IfFileExists{parskip.sty}{%
    \usepackage{parskip}
  }{% else
    \setlength{\parindent}{0pt}
    \setlength{\parskip}{6pt plus 2pt minus 1pt}}
}{% if KOMA class
  \KOMAoptions{parskip=half}}
\makeatother
\usepackage{xcolor}
\IfFileExists{xurl.sty}{\usepackage{xurl}}{} % add URL line breaks if available
\IfFileExists{bookmark.sty}{\usepackage{bookmark}}{\usepackage{hyperref}}
\hypersetup{
  pdftitle={Mastering Time},
  pdfauthor={R (Chandra) Chandrasekhar},
  pdflang={en-GB},
  colorlinks=true,
  linkcolor={DarkOliveGreen},
  filecolor={Purple},
  citecolor={DarkKhaki},
  urlcolor={Maroon},
  pdfcreator={LaTeX via pandoc}}
\urlstyle{same} % disable monospaced font for URLs
\usepackage[margin=25mm]{geometry}
\usepackage{longtable,booktabs,array}
\usepackage{calc} % for calculating minipage widths
% Correct order of tables after \paragraph or \subparagraph
\usepackage{etoolbox}
\makeatletter
\patchcmd\longtable{\par}{\if@noskipsec\mbox{}\fi\par}{}{}
\makeatother
% Allow footnotes in longtable head/foot
\IfFileExists{footnotehyper.sty}{\usepackage{footnotehyper}}{\usepackage{footnote}}
\makesavenoteenv{longtable}
\setlength{\emergencystretch}{3em} % prevent overfull lines
\providecommand{\tightlist}{%
  \setlength{\itemsep}{0pt}\setlength{\parskip}{0pt}}
\setcounter{secnumdepth}{-\maxdimen} % remove section numbering
\ifLuaTeX
\usepackage[bidi=basic]{babel}
\else
\usepackage[bidi=default]{babel}
\fi
\babelprovide[main,import]{british}
% get rid of language-specific shorthands (see #6817):
\let\LanguageShortHands\languageshorthands
\def\languageshorthands#1{}
% $HOME/.pandoc/defaults/latex-header-includes.tex
% Common header includes for both lualatex and xelatex engines.
%
% Preliminaries
%
% \PassOptionsToPackage{rgb,dvipsnames,svgnames}{xcolor}
% \PassOptionsToPackage{main=british}{babel}
\AtBeginEnvironment{quote}{\small}
\AtBeginEnvironment{quotation}{\small}
\AtBeginEnvironment{longtable}{\centering}
%
% Packages that are useful to include
%
\usepackage{graphicx}
\usepackage{subcaption}
\usepackage[inkscapeversion=1]{svg}
\usepackage[defaultlines=4,all]{nowidow}
\usepackage[capitalize,noabbrev]{cleveref}
\usepackage{etoolbox}
\usepackage{fontsize}
\usepackage{newunicodechar}
\usepackage{pdflscape}
\usepackage{fnpct}
\usepackage{parskip}
  \setlength{\parindent}{0pt}
\usepackage[style=american]{csquotes}
% \usepackage{setspace} Use the <fontname-plus.tex> files for setspace
%
\usepackage{esdiff} % for derivative symbols
\usepackage{amsmath}
% noto-plus.tex
% Font-setting header file for use with Pandoc Markdown
% to generate PDF via LuaLaTeX.
% The main font is Noto Serif.
% Other main fonts are also available in appropriately named file.
\usepackage{fontspec}
\usepackage{setspace}
\setstretch{1.3}
%
\defaultfontfeatures{Ligatures=TeX,Scale=MatchLowercase,Renderer=Node} % at the start always
%
% For English
% See also https://tex.stackexchange.com/questions/574047/lualatex-amsthm-polyglossia-charissil-error
% We use Node as Renderer for the Latin Font and Greek Font and HarfBuzz as renderer ofr Indic fonts.
%
\babelfont{rm}[Script=Latin,Scale=1]{NotoSerif}% Config is at $HOME/texmf/tex/latex/NotoSerif.fontspec
%
\babelfont{sf}[Script=Latin]{SourceSansPro}% Config is at $HOME/texmf/tex/latex/SourceSansPro.fontspec
%
\babelfont{tt}[Script=Latin]{FiraMono}% Config is at $HOME/texmf/tex/latex/FiraMono.fontspec
%
% Sanskrit, Tamil, and Greek fonts
%
\babelprovide[import, onchar=ids fonts]{sanskrit}
\babelprovide[import, onchar=ids fonts]{tamil}
\babelprovide[import, onchar=ids fonts]{greek}
%
\babelfont[sanskrit]{rm}[Scale=1.1,Renderer=HarfBuzz,Script=Devanagari]{NotoSerifDevanagari}
\babelfont[sanskrit]{sf}[Scale=1.1,Renderer=HarfBuzz,Script=Devanagari]{NotoSansDevanagari}
\babelfont[tamil]{rm}[Renderer=HarfBuzz,Script=Tamil]{NotoSerifTamil}
\babelfont[tamil]{sf}[Renderer=HarfBuzz,Script=Tamil]{NotoSansTamil}
\babelfont[greek]{rm}[Script=Greek]{GentiumBookPlus}
%
% Math font
%
\usepackage{unicode-math} % seems not to hurt % fallabck
\setmathfont[bold-style=TeX]{STIX Two Math}
%
%
% Other fonts
%
\newfontfamily{\emojifont}{Symbola}
%

\usepackage{titling}
\usepackage{fancyhdr}
    \pagestyle{fancy}
    \fancyhead{}
    \fancyfoot{}
    \renewcommand{\headrulewidth}{0.2pt}
    \renewcommand{\footrulewidth}{0.2pt}
    \fancyhead[LO,RE]{\scshape\thetitle}
    \fancyfoot[CO,CE]{\footnotesize Copyright © 2006\textendash\the\year, R (Chandra) Chandrasekhar}
    \fancyfoot[RE,RO]{\thepage}
\newfontfamily{\regulariconfont}{Font Awesome 6 Free Regular}[Color=Grey]
\newfontfamily{\solidiconfont}{Font Awesome 6 Free Solid}[Color=Grey]
\newfontfamily{\brandsiconfont}{Font Awesome 6 Brands}[Color=Grey]
%
% Direct input of Unicode code points
%
\newcommand{\faEnvelope}{\regulariconfont\ ^^^^f0e0\normalfont}
\newcommand{\faMobile}{\solidiconfont\ ^^^^f3cd\normalfont}
\newcommand{\faLinkedin}{\brandsiconfont\ ^^^^f0e1\normalfont}
\newcommand{\faGithub}{\brandsiconfont\ ^^^^f09b\normalfont}
\newcommand{\faAtom}{\solidiconfont\ ^^^^f5d2\normalfont}
\newcommand{\faPaperPlaneRegular}{\regulariconfont\ ^^^^f1d8\normalfont}
\newcommand{\faPaperPlaneSolid}{\solidiconfont\ ^^^^f1d8\normalfont}

%
% The block below is commented out because of Tofu glyphs in HTML
%
% \newcommand{\faEnvelope}{\regulariconfont\ \normalfont}
% \newcommand{\faMobile}{\solidiconfont\ \normalfont}
% \newcommand{\faLinkedin}{\brandsiconfont\ \normalfont}
% \newcommand{\faGithub}{\brandsiconfont\ \normalfont}
\ifLuaTeX
  \usepackage{selnolig}  % disable illegal ligatures
\fi

\title{Mastering Time}
\author{R (Chandra) Chandrasekhar}
\date{2023-04-15 | 2023-04-16}

\begin{document}
\maketitle

\thispagestyle{empty}


\hypertarget{the-burden-of-heat-and-time}{%
\subsection{The burden of heat and
time}\label{the-burden-of-heat-and-time}}

It was a
\href{https://dictionary.cambridge.org/dictionary/english/sweltering}{sweltering}\footnote{Coloured
  links take you to the meanings of a number of unfamiliar words and
  phrases. This way, not only can you conquer procrastination, but you
  can also expand your vocabulary, as you read this blog. I hope that,
  \href{https://dictionary.cambridge.org/dictionary/english/by-and-by}{by
  and by}, you will seek out the meanings of unexplained words and
  phrases as well.} summer afternoon, when, heavy with perspiration, and
the burden of an unfinished task, I was about to nod off to an
\href{https://www.thefreedictionary.com/alluring}{alluring} but
unscheduled
\href{https://www.collinsdictionary.com/dictionary/english/siesta}{siesta},
when who should
\href{https://www.thefreedictionary.com/saunter}{saunter} over, but my
\href{https://www.vocabulary.com/dictionary/redoubtable}{redoubtable}
friend, the one and only Solus ``Sol'' Simkin.

Finding a
\href{https://www.macmillandictionary.com/dictionary/british/a-sympathetic-ear}{sympathetic
ear} in Sol, I unburdened myself. There was this report, due tomorrow,
and the weather today was humid and
\href{https://www.macmillandictionary.com/dictionary/british/stultifying}{stultifying}.
With the
\href{https://www.merriam-webster.com/dictionary/lethargy}{lethargy} of
the \href{https://www.thefreedictionary.com/listless}{listless}, and the
resistance of the unwilling, I felt no desire to
\href{https://idioms.thefreedictionary.com/I+wouldn\%27t+touch+with+a+barge+pole}{touch
the report with a barge pole}. I pleaded to being a
\href{https://www.dictionary.com/browse/hapless}{hapless} victim of my
circumstances.

Sol shot me a quick question. ``How long ago were you told to prepare
the report?''

``O! that was three months ago,'' I said, feeling a little sheepish.

``And in the ninety days that you had to prepare it, you felt that you
should tackle the report only on the eighty-ninth day?'' Sol quizzed me
like a lawyer for the prosecution

I threw up my hands in despair and said that each time I \emph{thought}
about the report, I felt a distinct sadness, akin to grief itself. A
wave of melancholia washed over me every time I contemplated the report,
so much so that I avoided it like the plague.

Sol chimed in, ``Then, I need to instruct you about the foolproof
technique of Professor Stavros Karavitis. He had a wonderful way to
knock out the blues from any task, and what is more, he always had time
for me, even if his schedule was full. Do you want to hear about his
well-known technique for achieving mastery over time?''

``Why not?'' I responded. Anything to draw me away from report writing,
and distract me from the sweaty humidity of this afternoon, was most
welcome. Thus began my rehabilitation from the ``Land of the
Procrastinators'', into the fair realm of ``Punctuality''.

\hypertarget{develop-kinship-with-your-tasks}{%
\subsection{Develop kinship with your
tasks}\label{develop-kinship-with-your-tasks}}

``I worked closely with Professor Karavitis for six years, first as a
doctoral candidate, and later as a researcher,'' continued Sol. ``The
first thing he made me understand was that \emph{I could not succeed at
what I did not enjoy}. I had to develop a kinship with my tasks, rather
than view them as punishments sent my way by a vengeful Fate.''

The good professor told me that I had two options when faced with a
task. One was to enjoy performing it. The other was to walk away, and go
and do something else that better captured my interest and enthusiasm.

If I did not like doing doctoral research---with all its
vicissitudes---I would fare much better by abandoning my PhD degree, and
shepherding sheep, if that was what gave me the greatest joy and
exhilaration. It is a cliched phrase, but \emph{`Do what you love, and
love what you do'}, works like magic.

\hypertarget{shed-the-need-for-approval}{%
\subsubsection{Shed the need for
approval}\label{shed-the-need-for-approval}}

Professor Kravitis added poignantly, ``Others do not care who you are,
what you are, what you do, or how you fare. You wrongly think---like the
\href{https://www.princeton.edu/~hos/mike/texts/ptolemy/ptolemy.html}{Ptolemaic
astronomers}---that you are the center of everyone's universe. Nothing
could be further from the truth.

Each person is deeply involved in thinking about his or her \emph{own}
life. The approval or disapproval of others is as short-lived as their
memories. Do not live your life to satisfy others. Rather, strive to
live in harmony with your innermost self. That is the sure road to
contentment and peace with yourself.

``The moment you stop thinking `What will others think of me', you have
opened for yourself the doors to true freedom.'' Enjoy that freedom and
do what you cherish. There is no other way to success, even as
\href{https://idioms.thefreedictionary.com/royal+road+to}{there is no
royal road to Geometry}. \emph{Shed the need for approval}.''

\hypertarget{prepare-well-in-advance}{%
\subsection{Prepare well in advance}\label{prepare-well-in-advance}}

Having metaphorically yeasted the dough of my mind, Professor
Kravitis---he actually insisted that I should call him Stavros---said
the next step was to start early, and start well.

``Never leave anything to the last minute,'' he said. ``No one likes a
half-baked cake. So, bake your cake well. Start early, start well, and
start secure with your own inner, sublime reassurance that you are going
to do a fine job. And hang on to that redeeming thought until you
finish.''

\hypertarget{do-it-once-and-do-it-well}{%
\subsection{Do it once and do it well}\label{do-it-once-and-do-it-well}}

Stavros said that anything worth doing should be done well.
Re-engineering a poor design was for feeble minds. The masterminds
always imagined clearly, and unhurriedly, until they were convinced that
their designs were sound. When the result took shape, form and function
worked seamlessly to marry strength with beauty. \emph{``Do it once and
do it well,''} was his mantra\footnote{This means doing it so well that
  you do not have to ``repair'' your work after it is completed. All
  good work must be checked or revised before being concluded. The
  ``once'' here includes this checking or revision as well. Quality has
  to be assured. \emojifont {😉} \normalfont}.

\hypertarget{done-and-dusted-out-of-mind}{%
\subsection{Done and dusted: out of
mind}\label{done-and-dusted-out-of-mind}}

Professor Kravitis was unsentimental about his own accomplishments,
which were staggering, given that when he became a full professor, he
was half the average age of his peers.

Sol said, ``One day, I picked up enough courage to ask him why he did
not celebrate his considerable academic accomplishments. He seemed
almost diffident about his own capabilities''

The answer that Professor Kravitis gave me was supremely insightful. He
said, ``Once a task is done and dusted, forget about it. Even if what
you did is worth a Nobel Prize, the fame that attends upon you lasts a
paltry 15 minutes, if even that.

``Sports stars, film actors, politicians, criminals, influencers, and
such like are the modern glitterati, who bask in the glare of public
attention and thrive on adulation.

``Why seek fame from your work? Your own quiet, inner satisfaction that
you have done your very best is the greatest reward you can ever earn.
\emph{Remember that you are competing only with yourself.} The world
does not care, nor should you. \emph{Done and dusted; out of mind}. That
is the way to work and live.''

\hypertarget{sols-personal-experience}{%
\subsection{Sol's personal experience}\label{sols-personal-experience}}

Sol continued, ``Do not for a moment think that this concentrated wisdom
was all delivered to me in a single enlightening dose. No chance of
that. Stavros was unassuming and genial. He treated me more like a
friend than as a student. I do not know why, but I suspect that he saw
me imbibing and putting into practice the sage truths that he imparted.
I believe that I absorbed the quiet wisdom that flowed from him by
osmosis rather than from overtly didactic efforts.''

``And what was most impressive was that he practised what he preached.
One Friday afternoon, in the early days of my research, I diffidently
approached Ms Juanita Peres, the secretary of Professor Kravitis, to ask
if he could spare me a half hour, as I had been stuck with some
experiments, and needed to bounce ideas off him, to better set my
bearings for future work.'' I was totally engrossed by Sol's tale by
now, and my lethargy had been replaced by keen interest.

``Of course, you can see Stavros now,'' Juanita said with confidence.
``He does not have any scheduled meeting this afternoon. So, you may go
right in.''

Sol said, ``I entered the professor's room and had a good forty minutes
of academic to-and-fro that opened windows in my mind. I had a clear
picture of what I should be doing, why, and how, as my work approached
the''real research'' part of my PhD programme''.

And again, Professor Kravitis made it seem that I was the one who came
up with all those brilliant insights. Paraphrasing Socrates, and without
any pretentiousness, he simply said, ``Those ideas are yours. I was
simply the midwife who helped birth them.''

``And I was flattered by the total attention he gave me during our
meeting,'' continued Sol.

After that inspiring session, I asked Juanita a very pertinent question.
I knew that Professor Kravitis was the Opening Keynote Speaker at a
conference that was taking place the following Monday at the university.
And it was a prestigious professional conference.

I myself could not have thought about anything else that Friday
afternoon, had I been tasked to deliver the opening keynote address for
the conference the following Monday morning. So, I asked Juanita how
Stavros managed to give me his whole time and attention for those forty
minutes, when in sixty hours or so, he had to deliver an important
keynote address. What Juanita told me next both shocked and educated me
in one go.

She said, ``I have known Professor Kravitis for more than twenty years.
In all that time, I have never seen him allow one event to interfere
with another. \emph{He has compartmentalized his mind}.

``So, he is well aware that he has this profound keynote address on
Monday morning. But he has no anxiety about it. He set aside ample time,
weeks ago, to research, prepare, and rehearse delivering the paper. Once
that was done, he has forgotten about it. It does not evoke any worry in
him because he has bequeathed adequate time and preparation to it. The
result is his best effort. And that leaves him free to attend to
everything else that intervenes between then and Monday next.''

``\,`Done and dusted; out of mind,' echoed in my mind again. Professor
Kravitis was indeed one who practised what he preached. A mentor for the
ages, whom we can all beneficially emulate,'' concluded Sol.

\hypertarget{closing-thoughts}{%
\subsection{Closing thoughts}\label{closing-thoughts}}

As Sol wound up his enriching tale, I resolved not to fall into the trap
of procrastination ever again.

\begin{enumerate}
\tightlist
\item
  First, I should prime my mind to enjoy the task.
\item
  Then, I should shed the need for approval from others.
\item
  Next, I would allocate adequate time to prepare for it, well ahead of
  the deadline.
\item
  Then I would do it once, and do it well.
\item
  And finally when it was done and dusted, I would dismiss it from my
  mind.
\end{enumerate}

``Whew! What a teaching!'' I said solemnly to Sol. He smiled the beaming
smile of the
\href{https://www.dictionary.com/browse/cognoscenti}{cognoscenti}.

\hypertarget{feedback}{%
\subsection{Feedback}\label{feedback}}

Please \href{mailto:feedback.swanlotus@gmail.com}{email me} your
comments and corrections.

\noindent A PDF version of this article is
\href{./mastering-time.pdf}{available for download here}:

\begin{small}

\begin{sffamily}

\url{https://swanlotus.netlify.app/blogs/mastering-time.pdf}

\end{sffamily}

\end{small}



\end{document}
