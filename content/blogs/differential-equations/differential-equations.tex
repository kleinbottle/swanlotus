% Options for packages loaded elsewhere
\PassOptionsToPackage{unicode,linktoc=all}{hyperref}
\PassOptionsToPackage{hyphens}{url}
\PassOptionsToPackage{dvipsnames,svgnames,x11names}{xcolor}
%
\documentclass[
  a4paper,
]{article}
\usepackage{amsmath,amssymb}
\usepackage{iftex}
\ifPDFTeX
  \usepackage[T1]{fontenc}
  \usepackage[utf8]{inputenc}
  \usepackage{textcomp} % provide euro and other symbols
\else % if luatex or xetex
  \usepackage{unicode-math} % this also loads fontspec
  \defaultfontfeatures{Scale=MatchLowercase}
  \defaultfontfeatures[\rmfamily]{Ligatures=TeX,Scale=1}
\fi
\usepackage{lmodern}
\ifPDFTeX\else
  % xetex/luatex font selection
\fi
% Use upquote if available, for straight quotes in verbatim environments
\IfFileExists{upquote.sty}{\usepackage{upquote}}{}
\IfFileExists{microtype.sty}{% use microtype if available
  \usepackage[]{microtype}
  \UseMicrotypeSet[protrusion]{basicmath} % disable protrusion for tt fonts
}{}
\makeatletter
\@ifundefined{KOMAClassName}{% if non-KOMA class
  \IfFileExists{parskip.sty}{%
    \usepackage{parskip}
  }{% else
    \setlength{\parindent}{0pt}
    \setlength{\parskip}{6pt plus 2pt minus 1pt}}
}{% if KOMA class
  \KOMAoptions{parskip=half}}
\makeatother
\usepackage{xcolor}
\usepackage[margin=25mm]{geometry}
\usepackage{longtable,booktabs,array}
\usepackage{calc} % for calculating minipage widths
% Correct order of tables after \paragraph or \subparagraph
\usepackage{etoolbox}
\makeatletter
\patchcmd\longtable{\par}{\if@noskipsec\mbox{}\fi\par}{}{}
\makeatother
% Allow footnotes in longtable head/foot
\IfFileExists{footnotehyper.sty}{\usepackage{footnotehyper}}{\usepackage{footnote}}
\makesavenoteenv{longtable}
\usepackage{graphicx}
\makeatletter
\def\maxwidth{\ifdim\Gin@nat@width>\linewidth\linewidth\else\Gin@nat@width\fi}
\def\maxheight{\ifdim\Gin@nat@height>\textheight\textheight\else\Gin@nat@height\fi}
\makeatother
% Scale images if necessary, so that they will not overflow the page
% margins by default, and it is still possible to overwrite the defaults
% using explicit options in \includegraphics[width, height, ...]{}
\setkeys{Gin}{width=\maxwidth,height=\maxheight,keepaspectratio}
% Set default figure placement to htbp
\makeatletter
\def\fps@figure{htbp}
\makeatother
\usepackage{svg}
\setlength{\emergencystretch}{3em} % prevent overfull lines
\providecommand{\tightlist}{%
  \setlength{\itemsep}{0pt}\setlength{\parskip}{0pt}}
\setcounter{secnumdepth}{-\maxdimen} % remove section numbering
% definitions for citeproc citations
\NewDocumentCommand\citeproctext{}{}
\NewDocumentCommand\citeproc{mm}{%
  \begingroup\def\citeproctext{#2}\cite{#1}\endgroup}
\makeatletter
 % allow citations to break across lines
 \let\@cite@ofmt\@firstofone
 % avoid brackets around text for \cite:
 \def\@biblabel#1{}
 \def\@cite#1#2{{#1\if@tempswa , #2\fi}}
\makeatother
\newlength{\cslhangindent}
\setlength{\cslhangindent}{1.5em}
\newlength{\csllabelwidth}
\setlength{\csllabelwidth}{3em}
\newenvironment{CSLReferences}[2] % #1 hanging-indent, #2 entry-spacing
 {\begin{list}{}{%
  \setlength{\itemindent}{0pt}
  \setlength{\leftmargin}{0pt}
  \setlength{\parsep}{0pt}
  % turn on hanging indent if param 1 is 1
  \ifodd #1
   \setlength{\leftmargin}{\cslhangindent}
   \setlength{\itemindent}{-1\cslhangindent}
  \fi
  % set entry spacing
  \setlength{\itemsep}{#2\baselineskip}}}
 {\end{list}}
\usepackage{calc}
\newcommand{\CSLBlock}[1]{\hfill\break#1\hfill\break}
\newcommand{\CSLLeftMargin}[1]{\parbox[t]{\csllabelwidth}{\strut#1\strut}}
\newcommand{\CSLRightInline}[1]{\parbox[t]{\linewidth - \csllabelwidth}{\strut#1\strut}}
\newcommand{\CSLIndent}[1]{\hspace{\cslhangindent}#1}
\ifLuaTeX
\usepackage[bidi=basic]{babel}
\else
\usepackage[bidi=default]{babel}
\fi
\babelprovide[main,import]{british}
% get rid of language-specific shorthands (see #6817):
\let\LanguageShortHands\languageshorthands
\def\languageshorthands#1{}
% $HOME/.pandoc/defaults/latex-header-includes.tex
% Common header includes for both lualatex and xelatex engines.
%
% Preliminaries
%
% \PassOptionsToPackage{rgb,dvipsnames,svgnames}{xcolor}
% \PassOptionsToPackage{main=british}{babel}
\PassOptionsToPackage{english}{selnolig}
\AtBeginEnvironment{quote}{\small}
\AtBeginEnvironment{quotation}{\small}
\AtBeginEnvironment{longtable}{\centering}
%
% Packages that are useful to include
%
\usepackage{graphicx}
\usepackage{subcaption}
\usepackage[inkscapeversion=auto]{svg}
\usepackage{nowidow}
\usepackage{etoolbox}
\usepackage{fontsize}
\usepackage{newunicodechar}
\usepackage{pdflscape}
\usepackage{fnpct}
\usepackage{parskip}
  \setlength{\parindent}{0pt}
\usepackage[style=american]{csquotes}
% \usepackage{setspace} Use the <fontname-plus.tex> files for setspace
%
\usepackage{hyperref} % cleveref must come AFTER hyperref
\usepackage[capitalize,noabbrev]{cleveref} % Must come after hyperref
\let\longdivision\relax
\usepackage{longdivision}
\newcommand{\dd}{\ensuremath{mathrm d}}
%
% Assume that amsmath is already loaded via \usepackage{amsmath}
% in the standard LaTeX template foe Pandoc
% 
\DeclareMathOperator{\sech}{sech}
\DeclareMathOperator{\csch}{csch}
\DeclareMathOperator{\arcsec}{arcsec}
\DeclareMathOperator{\arccot}{arccot}
\DeclareMathOperator{\arccsc}{arccsc}
\DeclareMathOperator{\arccosh}{arccosh}
\DeclareMathOperator{\arcsinh}{arcsinh}
\DeclareMathOperator{\arctanh}{arctanh}
\DeclareMathOperator{\arcsech}{arcsech}
\DeclareMathOperator{\arccsch}{arccsch}
\DeclareMathOperator{\arccoth}{arccoth} 
% noto-plus.tex
% Font-setting header file for use with Pandoc Markdown
% to generate PDF via LuaLaTeX.
% The main font is Noto Serif.
% Other main fonts are also available in appropriately named file.
\usepackage{fontspec}
\usepackage{setspace}
\setstretch{1.3}
%
\defaultfontfeatures{Ligatures=TeX,Scale=MatchLowercase,Renderer=Node} % at the start always
%
% For English
% See also https://tex.stackexchange.com/questions/574047/lualatex-amsthm-polyglossia-charissil-error
% We use Node as Renderer for the Latin Font and Greek Font and HarfBuzz as renderer ofr Indic fonts.
%
\babelfont{rm}[Script=Latin,Scale=1]{NotoSerif}% Config is at $HOME/texmf/tex/latex/NotoSerif.fontspec
\babelfont{sf}[Script=Latin]{SourceSansPro}% Config is at $HOME/texmf/tex/latex/SourceSansPro.fontspec
\babelfont{tt}[Script=Latin]{FiraMono}% Config is at $HOME/texmf/tex/latex/FiraMono.fontspec
%
% Sanskrit, Tamil, and Greek fonts
%
\babelprovide[import, onchar=ids fonts]{sanskrit}
\babelprovide[import, onchar=ids fonts]{tamil}
\babelprovide[import, onchar=ids fonts]{greek}
%
\babelfont[sanskrit]{rm}[Scale=1.1,Renderer=HarfBuzz,Script=Devanagari]{NotoSerifDevanagari}
\babelfont[sanskrit]{sf}[Scale=1.1,Renderer=HarfBuzz,Script=Devanagari]{NotoSansDevanagari}
\babelfont[tamil]{rm}[Renderer=HarfBuzz,Script=Tamil]{NotoSerifTamil}
\babelfont[tamil]{sf}[Renderer=HarfBuzz,Script=Tamil]{NotoSansTamil}
\babelfont[greek]{rm}[Script=Greek]{GentiumBookPlus}
%
% Math font
%
\usepackage{unicode-math} % seems not to hurt % fallabck
\setmathfont[bold-style=TeX]{STIX Two Math}
\usepackage{amsmath}
\usepackage{esdiff} % for derivative symbols
% \renewcommand{\mathbf}{\symbf}
%
%
% Other fonts
%
\newfontfamily{\emojifont}{Symbola}
%

\usepackage{titling}
\usepackage{fancyhdr}
    \pagestyle{fancy}
    \fancyhead{}
    \fancyfoot{}
    \renewcommand{\headrulewidth}{0.2pt}
    \renewcommand{\footrulewidth}{0.2pt}
    \fancyhead[LO,RE]{\scshape\thetitle}
    \fancyfoot[CO,CE]{\footnotesize Copyright © 2006\textendash\the\year, R (Chandra) Chandrasekhar}
    \fancyfoot[RE,RO]{\thepage}
%
\usepackage{newunicodechar}
\newunicodechar{√}{\textsf{√}}
\usepackage {caption}
    \captionsetup{font={sf,stretch=1.4}}
\ifLuaTeX
  \usepackage{selnolig}  % disable illegal ligatures
\fi
\IfFileExists{bookmark.sty}{\usepackage{bookmark}}{\usepackage{hyperref}}
\IfFileExists{xurl.sty}{\usepackage{xurl}}{} % add URL line breaks if available
\urlstyle{sf}
\hypersetup{
  pdftitle={Differential Equations},
  pdfauthor={R (Chandra) Chandrasekhar},
  pdflang={en-GB},
  colorlinks=true,
  linkcolor={DarkGreen},
  filecolor={Purple},
  citecolor={Teal},
  urlcolor={Maroon},
  pdfcreator={LaTeX via pandoc}}

\title{Differential Equations}
\author{R (Chandra) Chandrasekhar}
\date{2025-02-17 | 2025-02-17}

\begin{document}
\maketitle

\thispagestyle{empty}


\begin{small}

\begin{quote}
Isaac Newton was the first to glimpse this secret of the universe. He
found that the orbits of the planets, the rhythm of the tides, and the
trajectories of cannonballs could all be described, explained, and
predicted by a small set of differential equations. Today we call them
Newton's laws of motion and gravity. Ever since Newton, we have found
that the same pattern holds whenever we uncover a new part of the
universe. From the old elements of earth, air, fire, and water to the
latest in electrons, quarks, black holes, and superstrings, every
inanimate thing in the universe bends to the rule of differential
equations. I bet this is what Feynman meant when he said that calculus
is the language God talks. If anything deserves to be called the secret
of the universe, calculus is it.
\end{quote}

\end{small}

\begin{flushright}

\begin{footnotesize}

\textsc{Steven Strogatz}\\
\emph{Infinite Powers}, {[}\citeproc{ref-strogatz-2019}{1}{]}

\end{footnotesize}

\end{flushright}

\subsection{A Promise kept}\label{a-promise-kept}

You might recall from my blog
\href{https://swanlotus.netlify.app/blogs/expressions-equations-and-formulae}{Expressions,
Equations, and Formulae} that my friend Solus ``Sol'' Simkin had asked
me to write on differential equations as well, when he first requested a
blog. This blog is my promise kept to him.

\subsection{A Gentle Entry}\label{a-gentle-entry}

When you make a cup of tea, you must allow it to infuse. Likewise, with
a cup of coffee, you need to let it percolate. With new knowledge, it is
no different. You need time to allow the new knowledge to steep into
your mind and find its place alongside old knowledge. It is only then
that you become comfortable with what is new. Be gentle on yourself if
things do not pop out of the page at once. You are making friends with
mathematical ideas that took centuries to evolve and mature.

To many students, mathematics is daunting, calculus even more so, and
differential equations are downright terrifying. Nevertheless, there are
authors who tame the subject rather than taunt the student. They
approach the subject leisurely, allowing the reader to enjoy the view,
and explaining interesting sights along the way. There are others who
use pictures in preference to words. I have chosen two books---one from
each category---as books to read, along with this blog, or even before
it. Here are my two selections.

\subsubsection{Steven Strogatz}\label{steven-strogatz}

The applied mathematician
\href{https://en.wikipedia.org/wiki/Steven_Strogatz}{Steven Strogatz} is
one of my favourite authors active in the popular mathematics genre
today. He knows how to interest, educate, and sometimes, even entertain.
Learning becomes easy and enjoyable as a consequence. He has written a
masterful popularization of calculus, its history, techniques,
significance, and applications, entitled
\href{https://www.stevenstrogatz.com/books/infinite-powers}{\emph{Infinite
Powers: How calculus reveals the secrets of the universe}}
{[}\citeproc{ref-strogatz-2019}{1}{]} . Make the effort to read this
book---even if it is a little at a time---and complete it. You will
become a better scholar as a result, and calculus will become your
friend.

\subsubsection{Blanchard, Devaney, and
Hall}\label{blanchard-devaney-and-hall}

The second recommendation---a textbook by three authors, henceforth
called BDH---is the result of a radical departure in the teaching of
differential equations---from being a hodgepodge of mundane and special
techniques, memorized and regurgitated, to a fascinating picture of how
nature unfolds in time. The pedagogy of the book leverages the ubiquity
of computing technology to enable a syncretic view and more profound
appreciation of differential equations as the time-evolution of natural
systems. Their book is called, unsurprisingly,
\href{http://math.bu.edu/odes/4ed-TOC.html}{\emph{Differential
Equations}} and is in its fourth edition
{[}\citeproc{ref-blanchard-devaney-hall-2012}{2}{]}. It is not a book
that you need to work through, but one you should consult for its
approach and ideas. The earnest student will also doubtless, tackle a
few problems! \emojifont {😉} \normalfont.

\subsection{Newton's Second Law}\label{newtons-second-law}

Perhaps the very first encounter with differential equations occurs for
most people with the statement of
\href{https://www.britannica.com/science/Newtons-laws-of-motion/Newtons-second-law-F-ma}{Newton's
Second Law of Motion}. It is usually stated as \emph{the net force on a
body is proportional to the rate of change of the body's momentum}.

Wait a minute! Is this not physics? Yes, it is. But is it also
mathematics? Yes indeed. Let us not quibble over how we divide knowledge
into compartments. Let us instead consider knowledge as a porthole with
which to view Nature. Nature behaves as Nature does. But humankind has
imposed disciplinary barriers into the framework of human knowledge. Let
us not be deterred by those barriers.

There are many terms here that must be \emph{named}, i.e., defined and
explained. Let us go through them in sequence, using SI units, and
mostly avoiding implications arising from the Theory of Relativity:

\begin{enumerate}
\item
  \textbf{Space}: This is the three dimensional space in which we live
  and through which matter moves. The length between two points in space
  is their \emph{distance}. The set of points in a one-dimensional line
  is usually denoted by \(\mathbb{R}\). For two-dimensions, the set is
  the \href{https://en.wikipedia.org/wiki/Cartesian_product}{Cartesian
  product} \(\mathbb{R} \times \mathbb{R} = \mathbb{R}^2\), and the
  points are denoted \((x, y)\) where each \(x \in \mathbb{R}\) and each
  \(y \in \mathbb{R}\). Likewise for three dimensional space where
  \((x, y, z) \in \mathbb{R}^3\).\footnote{If these symbols and notation
    are new to you please read my blog
    \href{https://swanlotus.netlify.app/blogs/the-two-most-important-numbers-zero-and-one}{The
    Two Most Important Numbers: Zero and One}.}
\item
  \textbf{Time}: This is the time we experience---24 hours or 86,400
  seconds in a day---usually denoted by the letter \(t\).
\item
  \textbf{Scalar}: A quantity that only has magnitude. Examples include
  time, temperature, distance, mass, energy, etc.
\item
  \textbf{Vector}: A quantity that has both magnitude and direction.
  While \emph{distance} is a scalar that defines the length between two
  points, \emph{displacement} is a vector and encodes both the length
  between two points \emph{and} their direction. Vectors are usually
  denoted in boldface, like \(\mathbf{a}\), or when written, with a
  squiggle atop or underneath the symbol.
\item
  \textbf{Distance}: The difference in length between two points in
  space, without concern for their orientation.
\item
  \textbf{Displacement}: Displacement is the vector analogue of
  distance, and is often denoted by \(\mathbf{s}\). The displacement
  from \(A\) to \(B\) is the negative of the displacement from \(B\) to
  \(A\).
\item
  \textbf{Speed}: When a body moves a distance \(d\) metres in time
  \(t\) seconds, its average speed is \(\frac{d}{t}\) metres per second.
\item
  \textbf{Velocity}: When a body moves, \emph{velocity} captures both
  its speed and the direction in which the change occurs. Velocity is
  the vector analogue to speed, and is usually denoted by
  \(\mathbf{v}\). A body whose speed is constant but whose direction is
  changing constantly, like a ball revolving on a string, experiences a
  \emph{changing} velocity, and is hence undergoing \emph{acceleration}.
\item
  \textbf{Mass}: Mass represents the resistance matter presents to
  motion. Roughly, on Earth, it is proportional to weight, which is a
  vector. Mass is generally denoted by \(m\). So, the heavier an object,
  the greater its
  mass.\footnote{The \emph{inertial mass}, \(m_i\), is an object's
    resistance to acceleration when a force is applied on it, as in
    \(F = m_{i}a\). The \emph{gravitational mass}, \(m_g\), is the
    strength of an object's interaction with a gravitational field, as
    in \(G\frac{m_g}{M}{R^2}\). Both these entities are the same, i.e.,
    \(m_{i} = m_{g}\).}(https://en.wikipedia.org/wiki/Equivalence\_principle.){]}
  Mass multiplied by velocity equals momentum.
\item
  \textbf{Momentum}: Momentum, often denoted by \(\mathbf{p}\), is a
  vector and is the product of mass and velocity,
  \(\mathbf{p} = m\mathbf{v}\).
\item
  \textbf{Acceleration}: Acceleration, \(\mathbf{a}\) is defined as the
  time rate of change of the velocity vector. Using the symbology of
  calculus, \[
  \mathbf{a} \triangleq \frac{\mathrm{d}\mathbf{v}}{\mathrm{d}t}.
  \] The symbol \(\triangleq\) is often used to denote a definition.
\item
  \textbf{Force}: Newton's Second Law states that force is proportional
  to the rate of change of momentum it induces. Stated mathematically,
  bearing in mind that force is a vector, \[
  \mathbf{F} \propto \frac{\mathrm{d}\mathbf{p}}{\mathrm{d}t} \triangleq k\frac{\mathrm{d}\mathbf{p}}{\mathrm{d}t}.
  \] The symbol \(\propto\) stands for \emph{is proportional to}. To
  convert it to an equality, we must introduce a constant \(k\). By
  judiciously choosing units, it is possible to make the value of \(k\)
  equal to one. We then have the famous vector equation
  \begin{equation}\phantomsection\label{eq:second-law}{
  \begin{aligned}
  \mathbf{F} &= \frac{\mathrm{d}\mathbf{p}}{\mathrm{d}t}\\
  &= \frac{\mathrm{d}\left(m\mathbf{v}\right)}{\mathrm{d}t}; \text{ extracting the $m$ out as a constant}\\
  &= m\frac{\mathrm{d}\mathbf{v}}{\mathrm{d}t}; \text{ acceleration is the derivative of velocity}\\
  &= m\mathbf{a}.
  \end{aligned}
  }\end{equation} And there you have it: Newton's second law stated
  mathematically in vector form as a differential equation in the second
  last line of \cref{eq:second-law}. Force is measured in newtons, mass
  in kilograms, and acceleration in metres per second per second,
  written \emph{m s\textsuperscript{-2}} or
  \emph{m/s\textsuperscript{2}}.
\end{enumerate}

\subsection{The Skydiver problem}\label{the-skydiver-problem}

The BDH book in its fourth edition extensively treats what I call the
\emph{skydiver problem}. I have borrowed freely from the book both in
terms of problem statement and approaches to its solution.

Differential equations arise naturally whenever mathematics is
\emph{applied} to problems that arise in nature. Therefore, one needs to
become aware of the technical nomenclature of fields like physics,
chemistry, biology, etc., and their units where applicable.

\subsubsection{Prelude}\label{prelude}

The classical example of how objects fall from a height to the earth is
the (apocryphal?) experiment of
\href{https://en.wikipedia.org/wiki/Galileo's_Leaning_Tower_of_Pisa_experiment}{Galileo
throwing rocks from atop the leaning tower of Pisa}. If the sizes of the
two rocks are similar, and air resistance is negligible, they will
strike the ground together.

In \href{https://en.wikipedia.org/wiki/Mechanics}{Mechanics} we
generally abstract out the object of interest, and show all forces
acting on it, via a
\href{https://www.physicsclassroom.com/class/newtlaws/Lesson-2/Drawing-Free-Body-Diagrams}{free-body
diagram}, to the exclusion of other objects.

\subsubsection{Gravity and the acceleration it
induces}\label{gravity-and-the-acceleration-it-induces}

We already know that the earth exerts a gravitational force on all
earthbound objects. Newton proposed and concluded that, for an object on
the surface of the Earth, the attraction of the earth on a body of mass
\(m\) is \begin{equation}\phantomsection\label{eq:universal-gravity}{
F = G\frac{mM}{R^2}
}\end{equation} where \(G\) is the universal gravitational constant,
\(m\) is the mass of the object of interest, \(M\) is the mass of the
earth, and \(R\) is the radius of the earth, assuming a perfect sphere.
The acceleration, \(g\), experienced by the mass \(m\) in free-fall is,
from \cref{eq:second-law,eq:universal-gravity}, the force of
gravitational attraction divided by the mass of the object of interest.
We may now write
\begin{equation}\phantomsection\label{eq:gravity-acceleration}{
\begin{aligned}
\frac{F}{m} &= G\frac{M}{R^2}; \text{ noting that the LHS is $g$}\\
g &= G\frac{M}{R^2}.
\end{aligned}
}\end{equation} Let us substitute the values og \(G\), \(M\), and \(R\)
in \href{https://en.wikipedia.org/wiki/International_System_of_Units}{SI
units} to get a value for \(g\) the acceleration due to gravity felt by
all objects on earth. \begin{equation}\phantomsection\label{eq:g-value}{
\begin{aligned}
G &= 6.6743 × 10^{-11}\; \mathrm{m^{3} kg^{−1} s^{−2}}\\
M &= 5.9722 × 10^{24}\; \mathrm{kg}\\
R &= 6.3781 × 10^{6}\; \mathrm{m} \text{  giving}\\
g &= G\frac{M}{R^2} = 9.7985 \approx 9.8\; \mathrm{m\,s^{-2}}.
\end{aligned}
}\end{equation} The value of \(g\) we have calculated is in
\href{https://dictionary.cambridge.org/dictionary/english/be-in-the-right-ballpark}{the
right ball park}.

The force exerted by gravity on an object of mass \(m\) is its \(W\)
\href{https://www.thefreedictionary.com/weight}{weight} which is \(mg\),
i.e., \[
W = mg\; \mathrm{N}.
\]

\subsubsection{Air resistance}\label{air-resistance}

Air resistance is usually ignored in elementary problems, but in the
case of the parachuting skydiver, it is air resistance and the drag it
produces that protects the skydiver from a serious accident. The problem
asks us to assume that the drag, or retarding force, from air resistance
is proportional to the square of the velocity, and may be set to
\(kv^2\), where \(k\) is some constant.

Note that force is a vector, as is velocity, and we need to use vector
equations. To keep things simple, we will instead keep careful tabs on
the \emph{signs} of the various quantities. The weight is directed
earthwards. The drag is directed away from the earth. If we assume that
earthwards is positive, we have \(v\) positive, \(W = mg\) positive, and
\(R = kv^2\) negative.

https://en.wikipedia.org/wiki/Drag\_equation
https://math.libretexts.org/Bookshelves/Scientific\_Computing\_Simulations\_and\_Modeling/Physical\_Modeling\_in\_MATLAB\_(Downey)/11:\_Second-order\_systems/11.04:\_Air\_Resistance

\begin{figure}
\centering
\includesvg[width=0.8\linewidth,height=\textheight,keepaspectratio]{images/skydiver-elephant.svg}
\caption[Almost free-body diagram for a skydiving elephant a la
\href{https://en.wikipedia.org/wiki/Dumbo}{Dumbo}.]{Almost free-body
diagram\footnotemark{} for a skydiving elephant a la
\href{https://en.wikipedia.org/wiki/Dumbo}{Dumbo}.}\label{fig:skydiver}
\end{figure}
\footnotetext{To retain reader interest, I have included some extraneous
  details to enliven the picture!}

Parachuting and skydiving exist as high-risk, but allowed sports, only
because the air resistance on the parachute canopy prevents the skydiver
from plummeting to the Earth like a rock, with fatal or near fatal
consequences.

The

The reason skydiving exists as an exhilarating sport is because when the
parachute opens, it presents a large surface area to the air and engages
it. The result is air-resistance or drag that is an \emph{upward} force
that counteracts the \emph{downward} pull of gravity.

A simple diagram showing the forces on a skydiver is shown in XXXX

Benmiloud Mohammed

What makes skydiving a challenging and exhilarating sport is the
possibility of falling through air and landing without

The velocity v of a freefalling skydiver is well modeled by the
differential equation dv m = mg − kv 2 , dt where m is the mass of the
skydiver, g is the gravitational constant, and k is the drag coefficient
determined by the position of the diver during the dive. (Note that the
constants m, g, and k are positive.) (a) Perform a qualitative analysis
of this model. (b) Calculate the terminal velocity of the skydiver.
Express your answer in terms of m, g, and k.

In Exercise 12 of Section 1.1, we saw that the velocity v of a
freefalling skydiver is well modeled by the differential equation dv m =
mg − kv2 , dt where m is the mass of the skydiver, g is the
gravitational constant, and k is the drag coefficient determined by the
position of the driver during the dive. (a) Find the general solution of
this differential equation. (b) Confirm your answer to Exercise 12 of
Section 1.1 by calculating the limit of v(t) as t → ∞.

As we saw in Exercise 12 of Section 1.1, the velocity v of a freefalling
skydiver is well modeled by the differential equation dv m = mg − kv2 ,
dt where m is the mass of the skydiver, g = 9.8 m/s2 is the
gravitational constant, and k is the drag coefficient determined by the
position of the diver during the dive. Consider a diver of mass m = 54
kg (120 lb) with a drag coefficient of 0.18 kg/m. Use Euler's method to
determine how long it will take the diver to reach 95\% of her terminal
velocity after she jumps from the plane. {[}Hint: Use the formula for
terminal velocity that was derived in Exercise 12 of Section 1.1.{]}

\subsection{Acknowledgements}\label{acknowledgements}

I have freely adapted the figure
\href{https://latexdraw.com/tikz-free-body-diagram-skydiver-with-parachute/}{TikZ
Free Body Diagram Skydiver with Parachute} by Mohammed Benmiloud from
his website to depict \cref{fig:skydiver}. I express my gratitude for
his work and recommend his site for those who wish to learn more about
\href{https://en.wikipedia.org/wiki/PGF/TikZ}{TikZ}.

I also express my thanks to all the creators of
\href{https://github.com/samcarter/tikzlings}{TikZlings} packages for
their delightful depictions of animals and beings.

\subsection{Feedback}\label{feedback}

Please \href{mailto:feedback.swanlotus@gmail.com}{email me} your
comments and corrections.

\noindent A PDF version of this article is
\href{./differetial-equations.pdf}{available for download here}:

\begin{sffamily}

\url{https://swanlotus.netlify.app/blogs/differential-equations.pdf}

\end{sffamily}

\section*{References}\label{bibliography}
\addcontentsline{toc}{section}{References}

\phantomsection\label{refs}
\begin{CSLReferences}{0}{0}
\bibitem[\citeproctext]{ref-strogatz-2019}
\CSLLeftMargin{{[}1{]} }%
\CSLRightInline{Steven H Strogatz. 2019. \emph{{Infinite Powers}: {How
calculus reveals the secrets of the universe}}. Houghton Mifflin
Harcourt.}

\bibitem[\citeproctext]{ref-blanchard-devaney-hall-2012}
\CSLLeftMargin{{[}2{]} }%
\CSLRightInline{Paul Blanchard and Robert L Devaney and Glen R Hall.
2012. \emph{{Differential Equations}} (4th ed.). Brooks Cole/Cengage
Learning.}

\end{CSLReferences}



\end{document}
