% Options for packages loaded elsewhere
\PassOptionsToPackage{unicode,linktoc=all}{hyperref}
\PassOptionsToPackage{hyphens}{url}
\PassOptionsToPackage{dvipsnames,svgnames,x11names}{xcolor}
%
\documentclass[
  a4paper,
]{article}
\usepackage{amsmath,amssymb}
\usepackage{iftex}
\ifPDFTeX
  \usepackage[T1]{fontenc}
  \usepackage[utf8]{inputenc}
  \usepackage{textcomp} % provide euro and other symbols
\else % if luatex or xetex
  \usepackage{unicode-math} % this also loads fontspec
  \defaultfontfeatures{Scale=MatchLowercase}
  \defaultfontfeatures[\rmfamily]{Ligatures=TeX,Scale=1}
\fi
\usepackage{lmodern}
\ifPDFTeX\else
  % xetex/luatex font selection
\fi
% Use upquote if available, for straight quotes in verbatim environments
\IfFileExists{upquote.sty}{\usepackage{upquote}}{}
\IfFileExists{microtype.sty}{% use microtype if available
  \usepackage[]{microtype}
  \UseMicrotypeSet[protrusion]{basicmath} % disable protrusion for tt fonts
}{}
\makeatletter
\@ifundefined{KOMAClassName}{% if non-KOMA class
  \IfFileExists{parskip.sty}{%
    \usepackage{parskip}
  }{% else
    \setlength{\parindent}{0pt}
    \setlength{\parskip}{6pt plus 2pt minus 1pt}}
}{% if KOMA class
  \KOMAoptions{parskip=half}}
\makeatother
\usepackage{xcolor}
\usepackage[margin=25mm]{geometry}
\usepackage{longtable,booktabs,array}
\usepackage{calc} % for calculating minipage widths
% Correct order of tables after \paragraph or \subparagraph
\usepackage{etoolbox}
\makeatletter
\patchcmd\longtable{\par}{\if@noskipsec\mbox{}\fi\par}{}{}
\makeatother
% Allow footnotes in longtable head/foot
\IfFileExists{footnotehyper.sty}{\usepackage{footnotehyper}}{\usepackage{footnote}}
\makesavenoteenv{longtable}
\setlength{\emergencystretch}{3em} % prevent overfull lines
\providecommand{\tightlist}{%
  \setlength{\itemsep}{0pt}\setlength{\parskip}{0pt}}
\setcounter{secnumdepth}{-\maxdimen} % remove section numbering
% definitions for citeproc citations
\NewDocumentCommand\citeproctext{}{}
\NewDocumentCommand\citeproc{mm}{%
  \begingroup\def\citeproctext{#2}\cite{#1}\endgroup}
\makeatletter
 % allow citations to break across lines
 \let\@cite@ofmt\@firstofone
 % avoid brackets around text for \cite:
 \def\@biblabel#1{}
 \def\@cite#1#2{{#1\if@tempswa , #2\fi}}
\makeatother
\newlength{\cslhangindent}
\setlength{\cslhangindent}{1.5em}
\newlength{\csllabelwidth}
\setlength{\csllabelwidth}{3em}
\newenvironment{CSLReferences}[2] % #1 hanging-indent, #2 entry-spacing
 {\begin{list}{}{%
  \setlength{\itemindent}{0pt}
  \setlength{\leftmargin}{0pt}
  \setlength{\parsep}{0pt}
  % turn on hanging indent if param 1 is 1
  \ifodd #1
   \setlength{\leftmargin}{\cslhangindent}
   \setlength{\itemindent}{-1\cslhangindent}
  \fi
  % set entry spacing
  \setlength{\itemsep}{#2\baselineskip}}}
 {\end{list}}
\usepackage{calc}
\newcommand{\CSLBlock}[1]{\hfill\break#1\hfill\break}
\newcommand{\CSLLeftMargin}[1]{\parbox[t]{\csllabelwidth}{\strut#1\strut}}
\newcommand{\CSLRightInline}[1]{\parbox[t]{\linewidth - \csllabelwidth}{\strut#1\strut}}
\newcommand{\CSLIndent}[1]{\hspace{\cslhangindent}#1}
\ifLuaTeX
\usepackage[bidi=basic]{babel}
\else
\usepackage[bidi=default]{babel}
\fi
\babelprovide[main,import]{british}
% get rid of language-specific shorthands (see #6817):
\let\LanguageShortHands\languageshorthands
\def\languageshorthands#1{}
% $HOME/.pandoc/defaults/latex-header-includes.tex
% Common header includes for both lualatex and xelatex engines.
%
% Preliminaries
%
% \PassOptionsToPackage{rgb,dvipsnames,svgnames}{xcolor}
% \PassOptionsToPackage{main=british}{babel}
\PassOptionsToPackage{english}{selnolig}
\AtBeginEnvironment{quote}{\small}
\AtBeginEnvironment{quotation}{\small}
\AtBeginEnvironment{longtable}{\centering}
%
% Packages that are useful to include
%
\usepackage{graphicx}
\usepackage{subcaption}
\usepackage[inkscapeversion=auto]{svg}
\usepackage{nowidow}
\usepackage{etoolbox}
\usepackage{fontsize}
\usepackage{newunicodechar}
\usepackage{pdflscape}
\usepackage{fnpct}
\usepackage{parskip}
  \setlength{\parindent}{0pt}
\usepackage[style=american]{csquotes}
% \usepackage{setspace} Use the <fontname-plus.tex> files for setspace
%
\usepackage{hyperref} % cleveref must come AFTER hyperref
\usepackage[capitalize,noabbrev]{cleveref} % Must come after hyperref
\let\longdivision\relax
\usepackage{longdivision}
\newcommand{\dd}{\ensuremath{mathrm d}}
%
% Assume that amsmath is already loaded via \usepackage{amsmath}
% in the standard LaTeX template foe Pandoc
% 
\DeclareMathOperator{\sech}{sech}
\DeclareMathOperator{\csch}{csch}
\DeclareMathOperator{\arcsec}{arcsec}
\DeclareMathOperator{\arccot}{arccot}
\DeclareMathOperator{\arccsc}{arccsc}
\DeclareMathOperator{\arccosh}{arccosh}
\DeclareMathOperator{\arcsinh}{arcsinh}
\DeclareMathOperator{\arctanh}{arctanh}
\DeclareMathOperator{\arcsech}{arcsech}
\DeclareMathOperator{\arccsch}{arccsch}
\DeclareMathOperator{\arccoth}{arccoth} 
% noto-plus.tex
% Font-setting header file for use with Pandoc Markdown
% to generate PDF via LuaLaTeX.
% The main font is Noto Serif.
% Other main fonts are also available in appropriately named file.
\usepackage{fontspec}
\usepackage{setspace}
\setstretch{1.3}
%
\defaultfontfeatures{Ligatures=TeX,Scale=MatchLowercase,Renderer=Node} % at the start always
%
% For English
% See also https://tex.stackexchange.com/questions/574047/lualatex-amsthm-polyglossia-charissil-error
% We use Node as Renderer for the Latin Font and Greek Font and HarfBuzz as renderer ofr Indic fonts.
%
\babelfont{rm}[Script=Latin,Scale=1]{NotoSerif}% Config is at $HOME/texmf/tex/latex/NotoSerif.fontspec
\babelfont{sf}[Script=Latin]{SourceSansPro}% Config is at $HOME/texmf/tex/latex/SourceSansPro.fontspec
\babelfont{tt}[Script=Latin]{FiraMono}% Config is at $HOME/texmf/tex/latex/FiraMono.fontspec
%
% Sanskrit, Tamil, and Greek fonts
%
\babelprovide[import, onchar=ids fonts]{sanskrit}
\babelprovide[import, onchar=ids fonts]{tamil}
\babelprovide[import, onchar=ids fonts]{greek}
%
\babelfont[sanskrit]{rm}[Scale=1.1,Renderer=HarfBuzz,Script=Devanagari]{NotoSerifDevanagari}
\babelfont[sanskrit]{sf}[Scale=1.1,Renderer=HarfBuzz,Script=Devanagari]{NotoSansDevanagari}
\babelfont[tamil]{rm}[Renderer=HarfBuzz,Script=Tamil]{NotoSerifTamil}
\babelfont[tamil]{sf}[Renderer=HarfBuzz,Script=Tamil]{NotoSansTamil}
\babelfont[greek]{rm}[Script=Greek]{GentiumBookPlus}
%
% Math font
%
\usepackage{unicode-math} % seems not to hurt % fallabck
\setmathfont[bold-style=TeX]{STIX Two Math}
\usepackage{amsmath}
\usepackage{esdiff} % for derivative symbols
% \renewcommand{\mathbf}{\symbf}
%
%
% Other fonts
%
\newfontfamily{\emojifont}{Symbola}
%

\usepackage{titling}
\usepackage{fancyhdr}
    \pagestyle{fancy}
    \fancyhead{}
    \fancyfoot{}
    \renewcommand{\headrulewidth}{0.2pt}
    \renewcommand{\footrulewidth}{0.2pt}
    \fancyhead[LO,RE]{\scshape\thetitle}
    \fancyfoot[CO,CE]{\footnotesize Copyright © 2006\textendash\the\year, R (Chandra) Chandrasekhar}
    \fancyfoot[RE,RO]{\thepage}
%
\usepackage{newunicodechar}
\newunicodechar{√}{\textsf{√}}
\usepackage {caption}
    \captionsetup{font={sf,stretch=1.4}}
\ifLuaTeX
  \usepackage{selnolig}  % disable illegal ligatures
\fi
\IfFileExists{bookmark.sty}{\usepackage{bookmark}}{\usepackage{hyperref}}
\IfFileExists{xurl.sty}{\usepackage{xurl}}{} % add URL line breaks if available
\urlstyle{sf}
\hypersetup{
  pdftitle={Differential Equations},
  pdfauthor={R (Chandra) Chandrasekhar},
  pdflang={en-GB},
  colorlinks=true,
  linkcolor={DarkGreen},
  filecolor={Purple},
  citecolor={Teal},
  urlcolor={Maroon},
  pdfcreator={LaTeX via pandoc}}

\title{Differential Equations}
\author{R (Chandra) Chandrasekhar}
\date{2025-02-17 | 2025-02-17}

\begin{document}
\maketitle

\thispagestyle{empty}


\begin{small}

\begin{quote}
Isaac Newton was the first to glimpse this secret of the universe. He
found that the orbits of the planets, the rhythm of the tides, and the
trajectories of cannonballs could all be described, explained, and
predicted by a small set of differential equations. Today we call them
Newton's laws of motion and gravity. Ever since Newton, we have found
that the same pattern holds whenever we uncover a new part of the
universe. From the old elements of earth, air, fire, and water to the
latest in electrons, quarks, black holes, and superstrings, every
inanimate thing in the universe bends to the rule of differential
equations. I bet this is what Feynman meant when he said that calculus
is the language God talks. If anything deserves to be called the secret
of the universe, calculus is it.
\end{quote}

\end{small}

\begin{flushright}

\begin{footnotesize}

\textsc{Steven Strogatz}\\
\emph{Infinite Powers}, {[}\citeproc{ref-strogatz-2019}{1}{]}

\end{footnotesize}

\end{flushright}

\subsection{A Promise kept}\label{a-promise-kept}

You might recall from my blog
\href{https://swanlotus.netlify.app/blogs/expressions-equations-and-formulae}{Expressions,
Equations, and Formulae} that my friend Solus ``Sol'' Simkin had asked
me to write on differential equations as well, when he first requested a
blog. This blog is my promise kept to him.

\subsection{A Gentle Entry}\label{a-gentle-entry}

When you make a cup of tea, you must allow time for the tea to infuse.
Likewise, with a cup of coffee, you need percolation time. With new
knowledge, it is no different. You need time to allow the new knowledge
to steep into your mind and find its place alongside old knowledge. It
is only then that you become comfortable with what is new. Be gentle on
yourself if things do not pop out of the page at once. You are making
friends with mathematical ideas that took centuries to evolve and
mature.

To many students, mathematics is daunting, calculus even more so, and
differential equations are downright terrifying. Nevertheless, there are
authors who tame the subject rather than taunt the student. They
approach the subject leisurely, allowing the reader to enjoy the view,
and explaining interesting sights along the way. There are others who
use pictures in preference to words. I have chosen two books---one from
each category---as recommended books to read, along with this blog, or
even before it. Here are my two selections.

\subsubsection{Steven Strogatz}\label{steven-strogatz}

The applied mathematician
\href{https://en.wikipedia.org/wiki/Steven_Strogatz}{Steven Strogatz} is
one of my favourite authors active in the popular mathematics genre
today. He kows how to interest, educate, and sometimes, even entertain.
Learning becomes easy and enjoyable as a consequence. He has written a
masterful popularization of calculus, its history, techniques,
significance, and applications, entitled
\href{https://www.stevenstrogatz.com/books/infinite-powers}{\emph{Infinite
Powers: How calculus reveals the secrets of the universe}}
{[}\citeproc{ref-strogatz-2019}{1}{]} . Make the effort to read this
book---even if it is a little at a time---and complete it. You will
become a better scholar as a result, and calculus will become your
friend.

\subsubsection{Blanchard, Devaney, and
Hall}\label{blanchard-devaney-and-hall}

The second recommendation, a textbook by three authors, is the result of
a radical departure in the teaching of differential equations---from
being a hodgepodge of mundane and special techniques, memorized and
regurgitated, to a fascinating picture of how nature unfolds in time.
The pedagogy of the book leverages the ubiquity of computing technology
to enable a syncretic view and more profound appreciation of
differential equations as the time-evolution of natural systems. Their
book is called, unsurprisingly,
\href{http://math.bu.edu/odes/4ed-TOC.html}{\emph{Differential
Equations}} and is in its fourth edition
{[}\citeproc{ref-blanchard-devaney-hall-2012}{2}{]}. It is not a book
that you need to work through, but one you should consult for its
approach and ideas. The earnest student will also doubtless, tackle a
few problems! \emojifont {😉} \normalfont.

\subsection{Newton's Second Law}\label{newtons-second-law}

Perhaps the very first encounter with differential equations occurs, for
most people, with the statement of
\href{https://www.britannica.com/science/Newtons-laws-of-motion/Newtons-second-law-F-ma}{Newton's
Second Law of Motion}. It is usually stated as \emph{the net force on a
body is proportional to the rate of change of the body's momentum}.

Wait a minute! Is this not physics? Yes, it is. But is it also
mathematics? Yes indeed. Let us not quibble over how we divide knowledge
into compartments. Let us instead consider knowledge as a porthole with
which to view Nature. Nature behaves as Nature does. But humankind has
imposed disciplinary barriers into the framework of human knowledge. Let
us not respect those barriers or be deterred by them.

\section{Differential equations}\label{differential-equations}

Classification

ODE/homogeneous and links to polynomials
https://math.stackexchange.com/questions/3614864/why-are-most-natural-phenomena-described-using-differential-equations

https://math.stackexchange.com/questions/3782499/is-there-a-reason-it-is-so-rare-we-can-solve-differential-equations

Idea of a function must be clarified.

Where time and or space vary

DEs model physical phenomena where time and space vary along with
possibly other variables.

Linear differential equations with constant coefficients

\subsubsection{Examples of LDEs in real
life}\label{examples-of-ldes-in-real-life}

Spring under displacement

Wave motion

Simple pendulum

Electric circuits

\subsubsection{Physics}\label{physics}

https://www.quora.com/What-are-some-real-world-phenomena-that-can-be-described-by-using-differential-equations

https://www.physicsforums.com/insights/differential-equation-systems-and-nature/

https://math.stackexchange.com/questions/3614864/why-are-most-natural-phenomena-described-using-differential-equations

Schrödinger equation etc.

\subsubsection{Nonlinear DEs}\label{nonlinear-des}

Weather forecasting Chaos etc.

Population growth and Logistic equation

Predator-prey models

Catenary

\subsection{Acknowledgements}\label{acknowledgements}

\subsection{Feedback}\label{feedback}

Please \href{mailto:feedback.swanlotus@gmail.com}{email me} your
comments and corrections.

\noindent A PDF version of this article is
\href{./differetial-equations.pdf}{available for download here}:

\begin{sffamily}

\url{https://swanlotus.netlify.app/blogs/differential-equations.pdf}

\end{sffamily}

\section*{References}\label{bibliography}
\addcontentsline{toc}{section}{References}

\phantomsection\label{refs}
\begin{CSLReferences}{0}{0}
\bibitem[\citeproctext]{ref-strogatz-2019}
\CSLLeftMargin{{[}1{]} }%
\CSLRightInline{Steven H Strogatz. 2019. \emph{{Infinite Powers}: {How
calculus reveals the secrets of the universe}}. Houghton Mifflin
Harcourt.}

\bibitem[\citeproctext]{ref-blanchard-devaney-hall-2012}
\CSLLeftMargin{{[}2{]} }%
\CSLRightInline{Paul Blanchard and Robert L Devaney and Glen R Hall.
2012. \emph{{Differential Equations}} (4th ed.). Brooks Cole/Cengage
Learning.}

\end{CSLReferences}



\end{document}
