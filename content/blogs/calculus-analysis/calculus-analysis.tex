% Options for packages loaded elsewhere
\PassOptionsToPackage{unicode,linktoc=all}{hyperref}
\PassOptionsToPackage{hyphens}{url}
\PassOptionsToPackage{dvipsnames,svgnames,x11names}{xcolor}
%
\documentclass[
  a4paper,
]{article}
\usepackage{amsmath,amssymb}
\usepackage{iftex}
\ifPDFTeX
  \usepackage[T1]{fontenc}
  \usepackage[utf8]{inputenc}
  \usepackage{textcomp} % provide euro and other symbols
\else % if luatex or xetex
  \usepackage{unicode-math} % this also loads fontspec
  \defaultfontfeatures{Scale=MatchLowercase}
  \defaultfontfeatures[\rmfamily]{Ligatures=TeX,Scale=1}
\fi
\usepackage{lmodern}
\ifPDFTeX\else
  % xetex/luatex font selection
\fi
% Use upquote if available, for straight quotes in verbatim environments
\IfFileExists{upquote.sty}{\usepackage{upquote}}{}
\IfFileExists{microtype.sty}{% use microtype if available
  \usepackage[]{microtype}
  \UseMicrotypeSet[protrusion]{basicmath} % disable protrusion for tt fonts
}{}
\makeatletter
\@ifundefined{KOMAClassName}{% if non-KOMA class
  \IfFileExists{parskip.sty}{%
    \usepackage{parskip}
  }{% else
    \setlength{\parindent}{0pt}
    \setlength{\parskip}{6pt plus 2pt minus 1pt}}
}{% if KOMA class
  \KOMAoptions{parskip=half}}
\makeatother
\usepackage{xcolor}
\usepackage[margin=25mm]{geometry}
\usepackage{longtable,booktabs,array}
\usepackage{calc} % for calculating minipage widths
% Correct order of tables after \paragraph or \subparagraph
\usepackage{etoolbox}
\makeatletter
\patchcmd\longtable{\par}{\if@noskipsec\mbox{}\fi\par}{}{}
\makeatother
% Allow footnotes in longtable head/foot
\IfFileExists{footnotehyper.sty}{\usepackage{footnotehyper}}{\usepackage{footnote}}
\makesavenoteenv{longtable}
\setlength{\emergencystretch}{3em} % prevent overfull lines
\providecommand{\tightlist}{%
  \setlength{\itemsep}{0pt}\setlength{\parskip}{0pt}}
\setcounter{secnumdepth}{-\maxdimen} % remove section numbering
% definitions for citeproc citations
\NewDocumentCommand\citeproctext{}{}
\NewDocumentCommand\citeproc{mm}{%
  \begingroup\def\citeproctext{#2}\cite{#1}\endgroup}
\makeatletter
 % allow citations to break across lines
 \let\@cite@ofmt\@firstofone
 % avoid brackets around text for \cite:
 \def\@biblabel#1{}
 \def\@cite#1#2{{#1\if@tempswa , #2\fi}}
\makeatother
\newlength{\cslhangindent}
\setlength{\cslhangindent}{1.5em}
\newlength{\csllabelwidth}
\setlength{\csllabelwidth}{3em}
\newenvironment{CSLReferences}[2] % #1 hanging-indent, #2 entry-spacing
 {\begin{list}{}{%
  \setlength{\itemindent}{0pt}
  \setlength{\leftmargin}{0pt}
  \setlength{\parsep}{0pt}
  % turn on hanging indent if param 1 is 1
  \ifodd #1
   \setlength{\leftmargin}{\cslhangindent}
   \setlength{\itemindent}{-1\cslhangindent}
  \fi
  % set entry spacing
  \setlength{\itemsep}{#2\baselineskip}}}
 {\end{list}}
\usepackage{calc}
\newcommand{\CSLBlock}[1]{\hfill\break\parbox[t]{\linewidth}{\strut\ignorespaces#1\strut}}
\newcommand{\CSLLeftMargin}[1]{\parbox[t]{\csllabelwidth}{\strut#1\strut}}
\newcommand{\CSLRightInline}[1]{\parbox[t]{\linewidth - \csllabelwidth}{\strut#1\strut}}
\newcommand{\CSLIndent}[1]{\hspace{\cslhangindent}#1}
\ifLuaTeX
\usepackage[bidi=basic]{babel}
\else
\usepackage[bidi=default]{babel}
\fi
\babelprovide[main,import]{british}
% get rid of language-specific shorthands (see #6817):
\let\LanguageShortHands\languageshorthands
\def\languageshorthands#1{}
% $HOME/.pandoc/defaults/latex-header-includes.tex
% Common header includes for both lualatex and xelatex engines.
%
% Preliminaries
%
% \PassOptionsToPackage{rgb,dvipsnames,svgnames}{xcolor}
% \PassOptionsToPackage{main=british}{babel}
\PassOptionsToPackage{english}{selnolig}
\AtBeginEnvironment{quote}{\small}
\AtBeginEnvironment{quotation}{\small}
\AtBeginEnvironment{longtable}{\centering}
%
% Packages that are useful to include
%
\usepackage{graphicx}
\usepackage{subcaption}
\usepackage[inkscapeversion=1]{svg}
\usepackage[defaultlines=4,all]{nowidow}
\usepackage{etoolbox}
\usepackage{fontsize}
\usepackage{newunicodechar}
\usepackage{pdflscape}
\usepackage{fnpct}
\usepackage{parskip}
  \setlength{\parindent}{0pt}
\usepackage[style=american]{csquotes}
% \usepackage{setspace} Use the <fontname-plus.tex> files for setspace
%
\usepackage{hyperref} % cleveref must come AFTER hyperref
\usepackage[capitalize,noabbrev]{cleveref} % Must come after hyperref
\let\longdivision\relax
\usepackage{longdivision}
% noto-plus.tex
% Font-setting header file for use with Pandoc Markdown
% to generate PDF via LuaLaTeX.
% The main font is Noto Serif.
% Other main fonts are also available in appropriately named file.
\usepackage{fontspec}
\usepackage{setspace}
\setstretch{1.3}
%
\defaultfontfeatures{Ligatures=TeX,Scale=MatchLowercase,Renderer=Node} % at the start always
%
% For English
% See also https://tex.stackexchange.com/questions/574047/lualatex-amsthm-polyglossia-charissil-error
% We use Node as Renderer for the Latin Font and Greek Font and HarfBuzz as renderer ofr Indic fonts.
%
\babelfont{rm}[Script=Latin,Scale=1]{NotoSerif}% Config is at $HOME/texmf/tex/latex/NotoSerif.fontspec
\babelfont{sf}[Script=Latin]{SourceSansPro}% Config is at $HOME/texmf/tex/latex/SourceSansPro.fontspec
\babelfont{tt}[Script=Latin]{FiraMono}% Config is at $HOME/texmf/tex/latex/FiraMono.fontspec
%
% Sanskrit, Tamil, and Greek fonts
%
\babelprovide[import, onchar=ids fonts]{sanskrit}
\babelprovide[import, onchar=ids fonts]{tamil}
\babelprovide[import, onchar=ids fonts]{greek}
%
\babelfont[sanskrit]{rm}[Scale=1.1,Renderer=HarfBuzz,Script=Devanagari]{NotoSerifDevanagari}
\babelfont[sanskrit]{sf}[Scale=1.1,Renderer=HarfBuzz,Script=Devanagari]{NotoSansDevanagari}
\babelfont[tamil]{rm}[Renderer=HarfBuzz,Script=Tamil]{NotoSerifTamil}
\babelfont[tamil]{sf}[Renderer=HarfBuzz,Script=Tamil]{NotoSansTamil}
\babelfont[greek]{rm}[Script=Greek]{GentiumBookPlus}
%
% Math font
%
\usepackage{unicode-math} % seems not to hurt % fallabck
\setmathfont[bold-style=TeX]{STIX Two Math}
\usepackage{amsmath}
\usepackage{esdiff} % for derivative symbols
% \renewcommand{\mathbf}{\symbf}
%
%
% Other fonts
%
\newfontfamily{\emojifont}{Symbola}
%

\usepackage{titling}
\usepackage{fancyhdr}
    \pagestyle{fancy}
    \fancyhead{}
    \fancyfoot{}
    \renewcommand{\headrulewidth}{0.2pt}
    \renewcommand{\footrulewidth}{0.2pt}
    \fancyhead[LO,RE]{\scshape\thetitle}
    \fancyfoot[CO,CE]{\footnotesize Copyright © 2006\textendash\the\year, R (Chandra) Chandrasekhar}
    \fancyfoot[RE,RO]{\thepage}
%
\usepackage{newunicodechar}
\newunicodechar{√}{\textsf{√}}
\ifLuaTeX
  \usepackage{selnolig}  % disable illegal ligatures
\fi
\usepackage{bookmark}
\IfFileExists{xurl.sty}{\usepackage{xurl}}{} % add URL line breaks if available
\urlstyle{sf}
\hypersetup{
  pdftitle={From Calculus to Analysis},
  pdfauthor={R (Chandra) Chandrasekhar},
  pdflang={en-GB},
  colorlinks=true,
  linkcolor={DarkOliveGreen},
  filecolor={Purple},
  citecolor={DarkKhaki},
  urlcolor={Maroon},
  pdfcreator={LaTeX via pandoc}}

\title{From Calculus to Analysis}
\author{R (Chandra) Chandrasekhar}
\date{2024-03-24 | 2024-03-24}

\begin{document}
\maketitle

\thispagestyle{empty}


\subsection{A Baffling Transition}\label{a-baffling-transition}

The transition from calculus in high school to analysis at university is
often baffling to the student. The comfort of simple---if
ill-understood---rules that magically give the correct answer to a
calculus problem, is replaced with some symbolic mumbo jumbo involving
\emph{inequalities} and Greek letters like epsilon and delta, which
serve to obfuscate rather than clarify. Proofs supplant computation,
upsetting many students. A good number of them lose marks, and often
also hope, and veer away from mathematics because the change from
calculus to analysis seems uncomfortable, unfriendly, and even uncalled
for.\footnote{Why fix it if it ain't broke?}

The calculus of Newton and Leibnitz that is taught in schools today is
about three and a half centuries old. But the attempt to change the
innocent calculus that \href{}{just worked} into the sophisticated,
convoluted beast called analysis was itself two hundred years in the
making.

It was a painful and painstaking transformation whose history is replete
with ideas that were challenged, thrown out or recast, refined, tested,
validated, and finally accepted. So, if you are shellshocked by the
transition from calculus to analysis, you are in good company. Even
today, it is claimed that many who use the logic and tools of analysis
routinely may be \emph{familiar} with them even if they lack a deep,
foundational understanding of them. {[}citation{]}

\subsection{My own trek through
Numberland}\label{my-own-trek-through-numberland}

I lay no claim to expertise in analysis. As an engineer, I was able to
live my professional life without deep searching for the foundations of
mathematical truth. But as I started writing about numbers, I was
troubled by the succession of types of numbers that were successively
added.

If we start out with the natural numbers and progress to include zero
and the negatives numbers, and then onto the rational numbers or
fraction, to irrationals, to complex numbers, I had doubts about the
numbers being really done and dusted.

The number line with integers alone had huge gaps in it; let us call
them \emph{holes}. Those holes could be filled somewhat with the
fractions or rational numbers. But the number line still \emph{leaked},
i.e., it had holes. Then, the irrationals filled some more holes.
\emph{Was the number line completely filled, or does it still leak?}
This was the doubt that assailed me.

It was then that I became aware of how the numbers we use everyday are
in fact all rational numbers of mixed fractions. The irrational numbers
were really a class of numbers that filled out the number menagerie
philosophically and mathematically, but were not pressed into use in
everyday life. If you look at the nigh sky, you see stars: these are the
rationals. But you do not see the black holes: these are the
irrationals. But both are needed for a complete astronomical picture.
So, too, are the irrationals needed for a complete number line.

But what if there was a \emph{third} type of number as yet undiscovered?
Or, heaven forbid, and fourth and further types of number? My escapade
through the mathematics of the late 1800s led me to the ides of a
\href{https://en.wikipedia.org/wiki/Dedekind_cut}{Dedekind cut}
{[}\citeproc{ref-dedekind}{1}{]} and thanks to his insight, we do not
have further types of numbers on the real number line.

\subsection{Intuition, geometry, paradoxes, and
imprecision}\label{intuition-geometry-paradoxes-and-imprecision}

Gradient Geometry Arithmetize it as both geometry and algebra seemed to
grow varieties. Area of a rectangle with zero with is zero. What is
close to but not zero? Approaches means what?

\subsection{Limits and such}\label{limits-and-such}

The idea of \emph{limits} are not fully appreciated until the real
number line is comprehended, at least somewhat. And then, we can begin
understanding and resolving paradoxes, such as
\href{https://en.wikipedia.org/wiki/Zeno's_paradoxes}{Zeno's paradox of
Achilles and the Tortoise}.

Let us agree to cast aside for another day a deeper examination of the
real number system, start with sequences, series, and functions, and how
their encounters with vanishingly small quantities, the
\emph{infinitesimals}, and of the unbounded \emph{infinite} quantities
require taming by the idea of limits so that they may be coralled into
the \emph{paddock of precision} in the real number system.

\subsection{The Differential Calculus}\label{the-differential-calculus}

The \href{}{differential calculus} taught in high school allows us to
compute rates of change, like finding the instantaneous speed of a
moving object, using a sleight of hand called \emph{taking limits} in
which an \emph{average} speed over a finite time is allowed to
\emph{approach} the \emph{instantaneous} speed by reducing the time
interval so that it becomes \emph{inifinitesimal}, i.e., a value that is
zero to all intents and purposes. But, \emph{we cannot divide by zero
when dealing with real numbers}.

\subsection{The Integral Calculus}\label{the-integral-calculus}

Likewise, the \href{}{integral calculus} allows us to compute
arbitrarily shaped areas under curves by using the formula for the area
of a rectangle and allowing the with of the rectangle to slowly reduce
until the number of rectangles becomes \emph{infinite}, in which case we
get the area under the curve.

\subsection{Are epsilon and delta really
necessary?}\label{are-epsilon-and-delta-really-necessary}

But these are sleights of hand that upend the apple cart of logic. The
area of a \emph{vanishingly thin} rectangle must perforce equal zero and
the addition of arrays of such rectangles cannot give us a finite
non-zero area. And yet the magic of calculus allows us to compute finite
non-zero areas. And there is no logical road to these magical results.

This is the principal motivation for the progression from calculus to
analysis and those pesky \(\epsilon\) and \(\delta\) definitions that
feel strange, and sound like mumbo jumbo. Analysis sounds so arcane that
it seems that calculus is common sense and that analysis is the real
magic. No wonder mathematics at high school judiciously avoids analysis
whose innards seem like black hole when compared with the sunny meadows
of calculus.

\textless!--What is a limit?

Why is it important?

How does one find a limit?

Is a limit unique?

How does intuition fail with limits

What happens with six/x at x = 0?

What happens at point-discontinuities?

If a function is re-defined as sinx/x = 5 at x = 0, what happens to the
limit? To continuity?

Why do we start with epsilon rather than delta when defining a limit
rigorously?

What is a limit in Calculus?

What is a limit in Analysis?

Why did it take 200 years from the time of Leibnitz to the time of
Weierstrass to formalize a limit rigorously, i.e., Calculus to Analysis?

What are some examples of pathological functions and their limits?

Critique of texts on limits: no pictures; Moarsh is an exception

``From Calculus to Analysis'' books seldom live up to the promise of
their titles

Start with sequences so that why epsilon before delta becomes clear:
Hight's book.

``When the successively attributed values of the same variable
indefinitely approach a fixed value, so that finally they differ from it
by as little as desired, the last is called the limit of all the
others.''9 Cauchy in Grabiner.

``For those ultimate ratios with which quantities vanish are not truly
the ratios of ultimate quantities, but limits toward which the ratios of
quantities decreasing without limit do always converge.'' Newrton in
Grabiner. ``approach nearer than by any given difference, but never go
beyond, nor in effect attain to, till the quantities are diminished in
infinitum.''16

\section*{References}\label{bibliography}
\addcontentsline{toc}{section}{References}

\phantomsection\label{refs}
\begin{CSLReferences}{0}{0}
\bibitem[\citeproctext]{ref-dedekind}
\CSLLeftMargin{{[}1{]} }%
\CSLRightInline{Richard Dedekind. 1963. \emph{{Essays on the Theory of
Numbers}}. Dover Publications.}

\end{CSLReferences}



\end{document}
