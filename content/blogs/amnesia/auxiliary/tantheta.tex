% tan-theta.tex
% Ilustration for blog "Doron's Mathematical Amnesia"
%
% R (Chandra) Chandrasekhar
% lnesqrt1cos0@gmail.com | chyavana@gmail.com
% 
% First written: 2024-04-09
% Last revised : 2024-04-09
%
\PassOptionsToPackage{rgb,dvipsnames,svgnames,dvipsnames,hyperref}{xcolor}
\documentclass[border={0mm},tikz]{standalone}
% TikZ-Preamble.tex
% Drag-and-drop files with common settings for the MM boiok.
%
% Uncomment the line below for the MM book.
%

%\usepackage[mdbch,ttscaled=true]{mathdesign}% See https://tex.stackexchange.com/a/621285/1636; MMBook
%\usepackage{stix2}

% See
% https://tex.stackexchange.com/questions/701568/why-is-the-stix2-bold-math-font-not-being-used-here
%
\usepackage[bold-style=TeX]{unicode-math}% Loads fontspec automatically
	\defaultfontfeatures{Ligatures=TeX,Scale=MatchLowercase}
	\setmainfont{NotoSerif}% \setmainfont{CharisSIL} for book; NotoSerif for blog
	\setsansfont{SourceSansPro}
	\setmathfont{STIX Two Math}
\usepackage{amsmath}
%
% DO NOT PUT A \tikzset command here becauae it messes up everything
%
\usepackage{tikz}
\usetikzlibrary{angles, quotes, arrows.meta, backgrounds, bending, shapes.misc, matrix, positioning, calc, intersections, decorations.text, decorations.pathreplacing,decorations.markings, datavisualization.formats.functions, patterns.meta}
%
% See
%
% https://github.com/wunderlich/edu/issues/56
%
% https://tex.stackexchange.com/questions/123760/draw-crosses-in-tikz
%
\tikzset{cross/.style={cross out, draw=black, fill=none, minimum size=2*(#1-\pgflinewidth), inner sep=0pt, outer sep=0pt}, cross/.default={2pt}}
%
\usepackage{relsize}
\usepackage{varwidth}
\usepackage{array}
\usepackage{realscripts}
%

%
\begin{document}
%
\tikzset{every picture/.style = {line width = 1.5pt, line cap = round, line join = round}}
%
\begin{tikzpicture}[scale=4]
%
% Set up unobtrusive grid and label axes
%
\draw[line width = 0.5pt, step=1 cm, color=Silver] (-1.4, -1.2) grid (1.4, 1.4);
\draw[color=Silver, -{latex}](-1.4, 0) -- (1.4, 0);
\draw[color=Silver, -{latex}](0, -1.2) -- (0, 1.4);
\node [color=Silver] at (1.3, 0) [below]{$x$};
\node [color=Silver] at (0, 1.3) [right]{$y$};
%
% Draw unit circle at origin
% 
\coordinate (O) at (0, 0);
\node at (O) [below] {$O$};
\draw (O) circle [radius = 1];
%
% Identify and label the cardinal coordinates
% 
\coordinate (Q) at (-1, 0);
\node at (Q) [above, xshift = -0.5cm]{$(-1, 0)$};
\node at (Q) [below, xshift = -0.15cm]{$Q$};
\node at (1, 0) [above, xshift = 0.5cm] {$(1, 0)$};
\node at (0, 1) [above] {$(0, 1)$};
\node at (0, -1) [below] {$(0, -1)$};
% 
% Identify R, draw OR, and label it
% 
\coordinate (R) at (60:1);
\draw [color=SteelBlue] (O) -- (R);
\node at (R) [yshift = 0.25cm, xshift = 0.25cm]{$R(x, y)$};
\draw [color = SteelBlue] (O) -- (Q);
% 
% Identify P as perpendicular from R to QR
% Draw PR and label it
% 
\coordinate (P) at ($(Q)!(R)!(O)$);
\draw [color=Sienna] (P) -- (R);
\node at (P) [below] {$P$};
% 
% Draw OP
% 
\draw [color=DarkOliveGreen] (O) -- (P);
%
% Mark and label the lengths x and y
% 
\draw [line width = 1pt, color=DarkOliveGreen, |-|, yshift = -0.12cm] ([yshift = -0.12cm]O) -- node [midway, below]{$x$} ([yshift = -0.12cm] P);
%
\draw [line width = 1pt, color=Sienna, |-|] ([xshift = 0.1cm] R) -- node [midway, right,]{$y$} ([xshift = 0.1cm] P);
%
% Draw the angle POR with an arrow
% No spaces between letters in angle = P--O--R
% 
\pic (theta) [draw, angle radius = 0.8cm, line width = 1pt, BrickRed, -{latex}, "$\theta$"] {angle = P--O--R};
%
% Draw a righ angle symbol for angle RPO
% 
\pic [draw, angle radius = 0.25cm, line width = 1pt] {right angle = R--P--O};
% 
% Get S as the intersection of QR and the y-axis
% 
\draw [color = Maroon, name path = qr] (Q) -- (R);
\path [name path = yaxis] (O) -- (0, 1);
\path [name intersections = {of = qr and yaxis, by = S}];
\node at (S) [left] {$S (0, t)$};
% 
% Name the base angles of the isosceles triangle as \alpha
% and show that it is \theta/2
% 
\pic (alpha1) [draw, angle radius = 1.2cm, line width = 1pt, DarkOrchid, "$\alpha$"] {angle = O--Q--R};
%
\pic (alpha2) [draw, angle radius = 1.2cm, line width = 1pt, DarkOrchid, "$\alpha$"] {angle = Q--R--O};
% 
% Relate \alpha and \theta and t
% 
\node [text=DarkSlateGray] at (-0.5, -0.2) {$\theta = 2\alpha$};
\node [text=DarkSlateGray] at (-0.5, -0.4) {$\tan\alpha = \dfrac{t}{1} =t$};
% 
\end{tikzpicture}
\end{document}
