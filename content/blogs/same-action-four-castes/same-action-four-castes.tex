% Options for packages loaded elsewhere
\PassOptionsToPackage{unicode,linktoc=all}{hyperref}
\PassOptionsToPackage{hyphens}{url}
\PassOptionsToPackage{dvipsnames,svgnames,x11names}{xcolor}
%
\documentclass[
  a4paper,
]{article}
\usepackage{amsmath,amssymb}
\usepackage{iftex}
\ifPDFTeX
  \usepackage[T1]{fontenc}
  \usepackage[utf8]{inputenc}
  \usepackage{textcomp} % provide euro and other symbols
\else % if luatex or xetex
  \usepackage{unicode-math} % this also loads fontspec
  \defaultfontfeatures{Scale=MatchLowercase}
  \defaultfontfeatures[\rmfamily]{Ligatures=TeX,Scale=1}
\fi
\usepackage{lmodern}
\ifPDFTeX\else
  % xetex/luatex font selection
\fi
% Use upquote if available, for straight quotes in verbatim environments
\IfFileExists{upquote.sty}{\usepackage{upquote}}{}
\IfFileExists{microtype.sty}{% use microtype if available
  \usepackage[]{microtype}
  \UseMicrotypeSet[protrusion]{basicmath} % disable protrusion for tt fonts
}{}
\makeatletter
\@ifundefined{KOMAClassName}{% if non-KOMA class
  \IfFileExists{parskip.sty}{%
    \usepackage{parskip}
  }{% else
    \setlength{\parindent}{0pt}
    \setlength{\parskip}{6pt plus 2pt minus 1pt}}
}{% if KOMA class
  \KOMAoptions{parskip=half}}
\makeatother
\usepackage{xcolor}
\usepackage[margin=25mm]{geometry}
\usepackage{longtable,booktabs,array}
\usepackage{calc} % for calculating minipage widths
% Correct order of tables after \paragraph or \subparagraph
\usepackage{etoolbox}
\makeatletter
\patchcmd\longtable{\par}{\if@noskipsec\mbox{}\fi\par}{}{}
\makeatother
% Allow footnotes in longtable head/foot
\IfFileExists{footnotehyper.sty}{\usepackage{footnotehyper}}{\usepackage{footnote}}
\makesavenoteenv{longtable}
\setlength{\emergencystretch}{3em} % prevent overfull lines
\providecommand{\tightlist}{%
  \setlength{\itemsep}{0pt}\setlength{\parskip}{0pt}}
\setcounter{secnumdepth}{-\maxdimen} % remove section numbering
\ifLuaTeX
\usepackage[bidi=basic]{babel}
\else
\usepackage[bidi=default]{babel}
\fi
\babelprovide[main,import]{british}
% get rid of language-specific shorthands (see #6817):
\let\LanguageShortHands\languageshorthands
\def\languageshorthands#1{}
% $HOME/.pandoc/defaults/latex-header-includes.tex
% Common header includes for both lualatex and xelatex engines.
%
% Preliminaries
%
% \PassOptionsToPackage{rgb,dvipsnames,svgnames}{xcolor}
% \PassOptionsToPackage{main=british}{babel}
\PassOptionsToPackage{english}{selnolig}
\AtBeginEnvironment{quote}{\small}
\AtBeginEnvironment{quotation}{\small}
\AtBeginEnvironment{longtable}{\centering}
%
% Packages that are useful to include
%
\usepackage{graphicx}
\usepackage{subcaption}
\usepackage[inkscapeversion=auto]{svg}
\usepackage[defaultlines=4,all]{nowidow}
\usepackage{etoolbox}
\usepackage{fontsize}
\usepackage{newunicodechar}
\usepackage{pdflscape}
\usepackage{fnpct}
\usepackage{parskip}
  \setlength{\parindent}{0pt}
\usepackage[style=american]{csquotes}
% \usepackage{setspace} Use the <fontname-plus.tex> files for setspace
%
\usepackage{hyperref} % cleveref must come AFTER hyperref
\usepackage[capitalize,noabbrev]{cleveref} % Must come after hyperref
\let\longdivision\relax
\usepackage{longdivision}
\newcommand{\dd}{\ensuremath{mathrm d}}
% noto-plus.tex
% Font-setting header file for use with Pandoc Markdown
% to generate PDF via LuaLaTeX.
% The main font is Noto Serif.
% Other main fonts are also available in appropriately named file.
\usepackage{fontspec}
\usepackage{setspace}
\setstretch{1.3}
%
\defaultfontfeatures{Ligatures=TeX,Scale=MatchLowercase,Renderer=Node} % at the start always
%
% For English
% See also https://tex.stackexchange.com/questions/574047/lualatex-amsthm-polyglossia-charissil-error
% We use Node as Renderer for the Latin Font and Greek Font and HarfBuzz as renderer ofr Indic fonts.
%
\babelfont{rm}[Script=Latin,Scale=1]{NotoSerif}% Config is at $HOME/texmf/tex/latex/NotoSerif.fontspec
\babelfont{sf}[Script=Latin]{SourceSansPro}% Config is at $HOME/texmf/tex/latex/SourceSansPro.fontspec
\babelfont{tt}[Script=Latin]{FiraMono}% Config is at $HOME/texmf/tex/latex/FiraMono.fontspec
%
% Sanskrit, Tamil, and Greek fonts
%
\babelprovide[import, onchar=ids fonts]{sanskrit}
\babelprovide[import, onchar=ids fonts]{tamil}
\babelprovide[import, onchar=ids fonts]{greek}
%
\babelfont[sanskrit]{rm}[Scale=1.1,Renderer=HarfBuzz,Script=Devanagari]{NotoSerifDevanagari}
\babelfont[sanskrit]{sf}[Scale=1.1,Renderer=HarfBuzz,Script=Devanagari]{NotoSansDevanagari}
\babelfont[tamil]{rm}[Renderer=HarfBuzz,Script=Tamil]{NotoSerifTamil}
\babelfont[tamil]{sf}[Renderer=HarfBuzz,Script=Tamil]{NotoSansTamil}
\babelfont[greek]{rm}[Script=Greek]{GentiumBookPlus}
%
% Math font
%
\usepackage{unicode-math} % seems not to hurt % fallabck
\setmathfont[bold-style=TeX]{STIX Two Math}
\usepackage{amsmath}
\usepackage{esdiff} % for derivative symbols
% \renewcommand{\mathbf}{\symbf}
%
%
% Other fonts
%
\newfontfamily{\emojifont}{Symbola}
%

\usepackage{titling}
\usepackage{fancyhdr}
    \pagestyle{fancy}
    \fancyhead{}
    \fancyfoot{}
    \renewcommand{\headrulewidth}{0.2pt}
    \renewcommand{\footrulewidth}{0.2pt}
    \fancyhead[LO,RE]{\scshape\thetitle}
    \fancyfoot[CO,CE]{\footnotesize Copyright © 2006\textendash\the\year, R (Chandra) Chandrasekhar}
    \fancyfoot[RE,RO]{\thepage}
%
\usepackage{newunicodechar}
\newunicodechar{√}{\textsf{√}}
\usepackage {caption}
    \captionsetup{font={sf,stretch=1.4}}
\ifLuaTeX
  \usepackage{selnolig}  % disable illegal ligatures
\fi
\IfFileExists{bookmark.sty}{\usepackage{bookmark}}{\usepackage{hyperref}}
\IfFileExists{xurl.sty}{\usepackage{xurl}}{} % add URL line breaks if available
\urlstyle{sf}
\hypersetup{
  pdftitle={Same Action: Four Castes},
  pdfauthor={R (Chandra) Chandrasekhar},
  pdflang={en-GB},
  colorlinks=true,
  linkcolor={DarkGreen},
  filecolor={Purple},
  citecolor={Teal},
  urlcolor={Maroon},
  pdfcreator={LaTeX via pandoc}}

\title{Same Action: Four Castes}
\author{R (Chandra) Chandrasekhar}
\date{2008-10-24 | 2024-08-02}

\begin{document}
\maketitle

\thispagestyle{empty}


This is Deepavali week. We are also having dinner guests tonight. So, I
spent much of the day sprucing up the house from front porch to back
toilet while my wife looked after the more essential culinary side. As I
was doing the cleaning chores, I asked myself why I was doing them. As
different answers flashed in my mind, I realized that they were all
applicable, but each flavour of answer held within itself a secret: it
determined the caste of the action. Since we are talking about caste a
la Hinduism, I will first explain caste before telling you my tale of
today.

The ancient wise people of India discerned that there were four great
goals of life. They called these purushaarthas. These four great goals
are:

\begin{enumerate}
\def\labelenumi{\arabic{enumi}.}
\item
  fulfilment of desire or kama;
\item
  acquisition of wealth or artha;
\item
  establishment of righteousness or dharma; and
\item
  quest of spiritual liberation or moksha.
\end{enumerate}

Humankind was also partitioned or classified according to which of these
four major goals predominated in their individual lives. The linkage was
so:

\begin{enumerate}
\def\labelenumi{\arabic{enumi}.}
\item
  kama \textless=\textgreater{} shudra
\item
  artha \textless=\textgreater{} vaishya
\item
  dharma \textless=\textgreater{} kshatriya
\item
  moksha \textless=\textgreater{} brahmana
\end{enumerate}

This is the origin of the four castes. It was the mental attitude rather
than the physical body that determined caste. It was not the
birth-pedigree but the operating system of the mind that determined
caste. Moreover, because the mind constantly fluctuates with time, and
because human aspiration and motivation oscillate with time and deed,
the caste of a person changes accordingly. It is not fixed for all time
but varies with mood, moment, and movement.

Now, back to my cleaning chores. I asked the question, ``Why am I
cleaning house?'' My mind reflected back many answers from which I have
distilled these four archetypes:

\begin{enumerate}
\def\labelenumi{\arabic{enumi}.}
\item
  Because I want to impress the guests that we keep a very clean house.
\item
  A clean house and a clean bathroom will encourage guests to leave
  everything as clean as they saw it: so the house will remain clean
  after the event.
\item
  Cleanliness promotes health and well-being. My guests should not
  suffer allergic reactions from dust and dirt.
\item
  My guests are representatives of the Almighty. Therefore I must show
  my devotion and reverence to the Almighty by cleaning the house.
\end{enumerate}

These four attitudes of mind correspond respectively to the shudra,
vaishya, kshatriya, and brahmana states respectively. The action has not
changed; it remains the same. But the mental attitude or underlying
motivation is different in each case. It starts out by being wholly
self-centred and changes shade by shade into an action that is totally
selfless.

The interesting thing is that caste is not dependent on action but on
the attitude to action. To ligate caste to activity and make each trade
or profession into a caste is to grossly distort caste and its purpose
and meaning. There is no leather-working caste and no medical-doctor
caste any more than there is a thinking caste or a breathing caste.
There is only attitude to action and that attitude determines the
``instantaneous'' caste of the person performing that action.

It is in an attempt to make brahmanas of us all always, who will find
ultimate freedom, that Bhagavan Krishna has graciously said in the
Bhagavad Gita {[}9:27{]}:

yat karoshi yadashnaasi yaj juhoshi dadaasi yat yat tapasyasi kaunteya
tat kurushva mad-arpanam

Whatever you do, whatever you eat, whatever you offer in sacrifice,
whatever you donate Whatever austerities you practise, O son of Kunti,
do that as an offering unto Me.

Galileo's Thermometer: same device different temperatures mean different
bulbs at the top. The top bulb at any moment determines caste at any
moment. https://en.wikipedia.org/wiki/Galileo\_thermometer

\subsection{Acknowledgements}\label{acknowledgements}

\subsection{Feedback}\label{feedback}

Please \href{mailto:feedback.swanlotus@gmail.com}{email me} your
comments and corrections.

\noindent A PDF version of this article is
\href{./same-action-four-castes.pdf}{available for download here}:

\begin{small}

\begin{sffamily}

\url{https://swanlotus.netlify.app/blogs/same-action-four-castes.pdf}

\end{sffamily}

\end{small}



\end{document}
