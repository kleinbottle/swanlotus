% Options for packages loaded elsewhere
\PassOptionsToPackage{unicode,linktoc=all}{hyperref}
\PassOptionsToPackage{hyphens}{url}
\PassOptionsToPackage{dvipsnames,svgnames*,x11names*}{xcolor}
%
\documentclass[
  11pt,
  british,
  a4paper,
]{article}
\usepackage{lmodern}
\usepackage{amsmath}
\usepackage{ifxetex,ifluatex}
\ifnum 0\ifxetex 1\fi\ifluatex 1\fi=0 % if pdftex
  \usepackage[T1]{fontenc}
  \usepackage[utf8]{inputenc}
  \usepackage{textcomp} % provide euro and other symbols
  \usepackage{amssymb}
\else % if luatex or xetex
  \usepackage{unicode-math}
  \defaultfontfeatures{Scale=MatchLowercase}
  \defaultfontfeatures[\rmfamily]{Ligatures=TeX,Scale=1}
\fi
% Use upquote if available, for straight quotes in verbatim environments
\IfFileExists{upquote.sty}{\usepackage{upquote}}{}
\IfFileExists{microtype.sty}{% use microtype if available
  \usepackage[]{microtype}
  \UseMicrotypeSet[protrusion]{basicmath} % disable protrusion for tt fonts
}{}
\makeatletter
\@ifundefined{KOMAClassName}{% if non-KOMA class
  \IfFileExists{parskip.sty}{%
    \usepackage{parskip}
  }{% else
    \setlength{\parindent}{0pt}
    \setlength{\parskip}{6pt plus 2pt minus 1pt}}
}{% if KOMA class
  \KOMAoptions{parskip=half}}
\makeatother
\usepackage{xcolor}
\IfFileExists{xurl.sty}{\usepackage{xurl}}{} % add URL line breaks if available
\IfFileExists{bookmark.sty}{\usepackage{bookmark}}{\usepackage{hyperref}}
\hypersetup{
  pdftitle={A Peacock in Mid Flight},
  pdfauthor={R (Chandra) Chandrasekhar},
  pdflang={en-GB},
  colorlinks=true,
  linkcolor=TealBlue,
  filecolor=Purple,
  citecolor=DarkKhaki,
  urlcolor=Maroon,
  pdfcreator={LaTeX via pandoc}}
\urlstyle{same} % disable monospaced font for URLs
\usepackage[margin=25mm]{geometry}
\setlength{\emergencystretch}{3em} % prevent overfull lines
\providecommand{\tightlist}{%
  \setlength{\itemsep}{0pt}\setlength{\parskip}{0pt}}
\setcounter{secnumdepth}{-\maxdimen} % remove section numbering
\AtBeginEnvironment{quote}{\small}
\usepackage{etoolbox}
\usepackage{graphicx}
\usepackage{subcaption}
\usepackage{svg}
\usepackage[Latin,Tamil,Devanagari]{ucharclasses}
  \setmainfont[SmallCapsFont={Charis SIL Small Caps}]{Charis SIL}
  \setsansfont[Numbers=OldStyle,BoldFont={* Semibold}]{Source Sans Pro}
  \setmonofont[Scale=0.90]{Fira Mono}
  \defaultfontfeatures{Ligatures=TeX,Scale=MatchLowercase}
  \setmathfont[bold-style=ISO]{STIX Two Math}
  \newfontfamily\tamilfont[Script=Tamil]{Noto Sans Tamil}
  \setTransitionsFor{Tamil}{\tamilfont}{\normalfont}
  \setTransitionTo{Tamil}{\tamilfont}{}
  \setTransitionFrom{Tamil}{\normalfont}
  \newfontfamily\devfont[Script=Devanagari]{Noto Sans Devanagari}
  \setTransitionsFor{Devanagari}{\devfont}{\normalfont}
  \setTransitionTo{Devanagari}{\devfont}{}
  \setTransitionFrom{Devanagari}{\normalfont}
  \newfontfamily{\emojifont}{Symbola}
\usepackage[all]{nowidow}
\usepackage[margins=raggedright]{floatrow}
%
% Tables without rules
%
\let\addlinespace\relax
\let\toprule\relax
\let\bottomrule\relax
\let\midrule\relax
\let\addlinespace\relax
%
% Adjust punctuation for footnotes
%
\usepackage{fnpct} % footnote _before_ punctuation reversed and adjusted.
  \setfnpct{after-dot-space=-0.15em,after-comma-space=-0.150em,add-punct-marks=;[0.06em]![0.06em]?[0.06em]:[0.06em]}
%
% Flexible fontsizes
%
\usepackage{fontsize}
%
%% pandoc-eqnos
\usepackage[capitalise]{cleveref}
  \crefname{equation}{Equation}{Equations}
  \crefname{figure}{Figure}{Figures}
%% pandoc-fignos
\usepackage{caption}
%% pandoc-fignos: environment to disable figure caption prefixes
    \makeatletter
    \newcounter{figno}
    \newenvironment{fignos:no-prefix-figure-caption}{
      \caption@ifcompatibility{}{
        \let\oldthefigure\thefigure
        \let\oldtheHfigure\theHfigure
        \renewcommand{\thefigure}{figno:\thefigno}
        \renewcommand{\theHfigure}{figno:\thefigno}
        \stepcounter{figno}
        \captionsetup{labelformat=empty}
      }
    }{
      \caption@ifcompatibility{}{
        \captionsetup{labelformat=default}
        \let\thefigure\oldthefigure
        \let\theHfigure\oldtheHfigure
        \addtocounter{figure}{-1}
      }
    }
    \makeatother
%%
\usepackage{fancyhdr}
    \pagestyle{fancy}
    \fancyhead{}
    \fancyfoot{}
    \renewcommand{\headrulewidth}{0pt}
    \renewcommand{\footrulewidth}{0pt}
    %\fancyhead[CE,CO]{\title}
    \fancyfoot[CO,CE]{\small Copyright © 2006--2021, R (Chandra) Chandrasekhar}
    \fancyfoot[RE,RO]{\thepage}
\ifxetex
  % Load polyglossia as late as possible: uses bidi with RTL langages (e.g. Hebrew, Arabic)
  \usepackage{polyglossia}
  \setmainlanguage[variant=british]{english}
\else
  \usepackage[shorthands=off,main=british]{babel}
\fi
\ifluatex
  \usepackage{selnolig}  % disable illegal ligatures
\fi

\title{A Peacock in Mid Flight}
\author{R (Chandra) Chandrasekhar}
\date{2019-11-17}

\begin{document}
\maketitle

\thispagestyle{empty}


This morning, I saw a peacock in mid-flight. It was a wondrous and
unforgettable sight, the more so because unexpected. I was walking as
usual in the neighbourhood, which is known to be home to a large number
of peacocks. As I turned to my right, I saw what I thought was an
unusually long bird in flight. Only as it approached me did I realize
that it was a peacock in mid-flight.

Its passage through the air was fluid and effortless: a picture of grace
and elegance. It was like a silent airborne missile cruising at low
altitude. It flapped its wings barely once or twice when it banked to
turn right. And then it landed with aplomb. That was when I saw how long
its tail really was. It appeared longer than the body and should
certainly have contributed to the smoothest of landings of an airborne
bird that I have ever seen.

Whether the tail contributes more to lift or drag, or whether it is
simply a gorgeous rudder I cannot tell. But I marvelled at how Nature
has patiently crafted the peacock---with its beautiful opalescent cyan
neck and uniquely patterned feathers---over evolutionary time to
complete the perfect marriage of form and function, of beauty and
utility, of grace and majesty. Here was a bird that that flew like
poetry in motion.

We human beings, when confronted by the magnificent feathers of a
peacock, have have unfairly imposed our misshapen human values on them,
and coined expressions like ``proud as a peacock'' and ``strutting about
like a peacock'' even though we have no access to a peacock's feelings.

From a biological point of view, colourful plumage in birds is designed
to attract mates. Indeed, there are species who perform complicated
courting rituals by using their feathers to best advantage. In the case
of the peacock, it is the male that is spectacular; the female is more
modestly beplumed.

I have always \emph{heard} the neighbourhood peacocks more than
\emph{seen} them. And the call of the peacock does not not quite do
justice to its looks. There is an old description of female pulchritude
as embodying ``the voice of a nightingale and the plumage of a
peacock''. After hearing the peacocks, I have often chuckled to myself
that if the two birds had been interchanged, the result would certainly
be less than attractive.

If you had thought the peacock to be flightless, you would be excused.
It is a rather large bird, like the swan. Seeing the peacock in flight,
reminded me of the famed
\href{http://www.boeing.com/commercial/787family/background.html}{Boeing
787 Dreamliner} aircraft now making its début worldwide, and being
hailed for its efficiency and comfort. The peacock is proof that
engineering and art can and do meet with superlative results.

The swan and the peacock are both large birds. Each has a peculiar
beauty and grace. The swan serenely floating on water, and the peacock
standing with its tail spread out, are epitomes of natural beauty. And
when they fly, they both do so effortlessly and gracefully.

Interestingly, both birds feature in the mythology of Hinduism.
Subrahmaṇya or Kārtikeya, the Divine General, rides a peacock. And
Sarasvatī, the Goddess of Wisdom, is surrounded by both a peacock and a
swan. The peacock's call is supposed to act as a tuning fork to help her
tune her stringed instrument, the
\href{http://www.thefreedictionary.com/vina}{vīṇā,} while the white swan
represents divine discrimination.

I sorely wanted to capture the peacock in mid-flight, but it was too
swift and sudden for me to photograph it. ``Perhaps some other time,'' I
consoled myself. I then realized the wonders of the Web and decided that
there might sites with photographs of peacocks in flight.
\href{http://bennie-thesmiths.blogspot.in/2012/05/peacock-in-flight.html}{The
Smith's Bennie and Patsy blog entitled \emph{Peacock In Flight}} has
some magnificent shots. And there is another lovely image at
\href{http://www.trekearth.com/gallery/Asia/India/West/Rajasthan/Sujangarh/photo772964.htm}{Peacock
in Flight by Annu.}

For now, word pictures from me of the peacock in mid-flight must
suffice. The peacock I saw was a dream gliding through air. It was
effortless, efficient, smooth, graceful, unfluttered, unflustered,
powerful, silent, and exquisitely matched to the element. It was a
superb blend of engineering and art, of power and poise, honed to
perfection by millennia of evolution, a dream of heavenly beauty on
earth.



\end{document}
