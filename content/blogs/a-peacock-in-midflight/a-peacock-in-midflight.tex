% Options for packages loaded elsewhere
\PassOptionsToPackage{unicode,linktoc=all}{hyperref}
\PassOptionsToPackage{hyphens}{url}
\PassOptionsToPackage{dvipsnames,svgnames,x11names}{xcolor}
%
\documentclass[
  a4paper,
]{article}
\usepackage{amsmath,amssymb}
\usepackage{lmodern}
\usepackage{iftex}
\ifPDFTeX
  \usepackage[T1]{fontenc}
  \usepackage[utf8]{inputenc}
  \usepackage{textcomp} % provide euro and other symbols
\else % if luatex or xetex
  \usepackage{unicode-math}
  \defaultfontfeatures{Scale=MatchLowercase}
  \defaultfontfeatures[\rmfamily]{Ligatures=TeX,Scale=1}
\fi
% Use upquote if available, for straight quotes in verbatim environments
\IfFileExists{upquote.sty}{\usepackage{upquote}}{}
\IfFileExists{microtype.sty}{% use microtype if available
  \usepackage[]{microtype}
  \UseMicrotypeSet[protrusion]{basicmath} % disable protrusion for tt fonts
}{}
\makeatletter
\@ifundefined{KOMAClassName}{% if non-KOMA class
  \IfFileExists{parskip.sty}{%
    \usepackage{parskip}
  }{% else
    \setlength{\parindent}{0pt}
    \setlength{\parskip}{6pt plus 2pt minus 1pt}}
}{% if KOMA class
  \KOMAoptions{parskip=half}}
\makeatother
\usepackage{xcolor}
\IfFileExists{xurl.sty}{\usepackage{xurl}}{} % add URL line breaks if available
\IfFileExists{bookmark.sty}{\usepackage{bookmark}}{\usepackage{hyperref}}
\hypersetup{
  pdftitle={A Peacock in Mid Flight},
  pdfauthor={R (Chandra) Chandrasekhar},
  pdflang={en-GB},
  colorlinks=true,
  linkcolor={DarkOliveGreen},
  filecolor={Purple},
  citecolor={DarkKhaki},
  urlcolor={Maroon},
  pdfcreator={LaTeX via pandoc}}
\urlstyle{same} % disable monospaced font for URLs
\usepackage[margin=25mm]{geometry}
\usepackage{longtable,booktabs,array}
\usepackage{calc} % for calculating minipage widths
% Correct order of tables after \paragraph or \subparagraph
\usepackage{etoolbox}
\makeatletter
\patchcmd\longtable{\par}{\if@noskipsec\mbox{}\fi\par}{}{}
\makeatother
% Allow footnotes in longtable head/foot
\IfFileExists{footnotehyper.sty}{\usepackage{footnotehyper}}{\usepackage{footnote}}
\makesavenoteenv{longtable}
\setlength{\emergencystretch}{3em} % prevent overfull lines
\providecommand{\tightlist}{%
  \setlength{\itemsep}{0pt}\setlength{\parskip}{0pt}}
\setcounter{secnumdepth}{-\maxdimen} % remove section numbering
\ifLuaTeX
\usepackage[bidi=basic]{babel}
\else
\usepackage[bidi=default]{babel}
\fi
\babelprovide[main,import]{british}
% get rid of language-specific shorthands (see #6817):
\let\LanguageShortHands\languageshorthands
\def\languageshorthands#1{}
% $HOME/.pandoc/defaults/latex-header-includes.tex
% Common header includes for both lualatex and xelatex engines.
%
% Preliminaries
%
% \PassOptionsToPackage{rgb,dvipsnames,svgnames}{xcolor}
% \PassOptionsToPackage{main=british}{babel}
\AtBeginEnvironment{quote}{\small}
\AtBeginEnvironment{quotation}{\small}
\AtBeginEnvironment{longtable}{\centering}
%
% Packages that are useful to include
%
\usepackage{graphicx}
\usepackage{subcaption}
\usepackage[inkscapeversion=1]{svg}
\usepackage[defaultlines=4,all]{nowidow}
\usepackage[capitalize,noabbrev]{cleveref}
\usepackage{etoolbox}
\usepackage{fontsize}
\usepackage{newunicodechar}
\usepackage{pdflscape}
\usepackage{fnpct}
\usepackage{parskip}
  \setlength{\parindent}{0pt}
\usepackage[style=american]{csquotes}
% \usepackage{setspace} Use the <fontname-plus.tex> files for setspace
%
% noto-plus.tex
% Font-setting header file for use with Pandoc Markdown
% to generate PDF via LuaLaTeX.
% The main font is Noto Serif.
% Other main fonts are also available in appropriately named file.
\usepackage{fontspec}
\usepackage{setspace}
\setstretch{1.3}
%
\defaultfontfeatures{Ligatures=TeX,Scale=MatchLowercase,Renderer=Node} % at the start always
%
% For English
% See also https://tex.stackexchange.com/questions/574047/lualatex-amsthm-polyglossia-charissil-error
% We use Node as Renderer for the Latin Font and Greek Font and HarfBuzz as renderer ofr Indic fonts.
%
\babelfont{rm}[Script=Latin,Scale=1]{NotoSerif}% Config is at $HOME/texmf/tex/latex/NotoSerif.fontspec
%
\babelfont{sf}[Script=Latin]{SourceSansPro}% Config is at $HOME/texmf/tex/latex/SourceSansPro.fontspec
%
\babelfont{tt}[Script=Latin]{FiraMono}% Config is at $HOME/texmf/tex/latex/FiraMono.fontspec
%
% Sanskrit, Tamil, and Greek fonts
%
\babelprovide[import, onchar=ids fonts]{sanskrit}
\babelprovide[import, onchar=ids fonts]{tamil}
\babelprovide[import, onchar=ids fonts]{greek}
%
\babelfont[sanskrit]{rm}[Scale=1.1,Renderer=HarfBuzz,Script=Devanagari]{NotoSerifDevanagari}
\babelfont[sanskrit]{sf}[Scale=1.1,Renderer=HarfBuzz,Script=Devanagari]{NotoSansDevanagari}
\babelfont[tamil]{rm}[Renderer=HarfBuzz,Script=Tamil]{NotoSerifTamil}
\babelfont[tamil]{sf}[Renderer=HarfBuzz,Script=Tamil]{NotoSansTamil}
\babelfont[greek]{rm}[Script=Greek]{GentiumBookPlus}
%
% Math font
%
\usepackage{unicode-math} % seems not to hurt % fallabck
\setmathfont[bold-style=TeX]{STIX Two Math}
%
%
% Other fonts
%
\newfontfamily{\emojifont}{Symbola}
%

\usepackage{titling}
\usepackage{fancyhdr}
    \pagestyle{fancy}
    \fancyhead{}
    \fancyfoot{}
    \renewcommand{\headrulewidth}{0.2pt}
    \renewcommand{\footrulewidth}{0.2pt}
    \fancyhead[LO,RE]{\scshape\thetitle}
    \fancyfoot[CO,CE]{\footnotesize Copyright © 2006\textendash\the\year, R (Chandra) Chandrasekhar}
    \fancyfoot[RE,RO]{\thepage}
\newfontfamily{\regulariconfont}{Font Awesome 6 Free Regular}[Color=Grey]
\newfontfamily{\solidiconfont}{Font Awesome 6 Free Solid}[Color=Grey]
\newfontfamily{\brandsiconfont}{Font Awesome 6 Brands}[Color=Grey]
%
% Direct input of Unicode code points
%
\newcommand{\faEnvelope}{\regulariconfont\ ^^^^f0e0\normalfont}
\newcommand{\faMobile}{\solidiconfont\ ^^^^f3cd\normalfont}
\newcommand{\faLinkedin}{\brandsiconfont\ ^^^^f0e1\normalfont}
\newcommand{\faGithub}{\brandsiconfont\ ^^^^f09b\normalfont}
\newcommand{\faAtom}{\solidiconfont\ ^^^^f5d2\normalfont}
\newcommand{\faPaperPlaneRegular}{\regulariconfont\ ^^^^f1d8\normalfont}
\newcommand{\faPaperPlaneSolid}{\solidiconfont\ ^^^^f1d8\normalfont}

%
% The block below is commented out because of Tofu glyphs in HTML
%
% \newcommand{\faEnvelope}{\regulariconfont\ \normalfont}
% \newcommand{\faMobile}{\solidiconfont\ \normalfont}
% \newcommand{\faLinkedin}{\brandsiconfont\ \normalfont}
% \newcommand{\faGithub}{\brandsiconfont\ \normalfont}
\ifLuaTeX
  \usepackage{selnolig}  % disable illegal ligatures
\fi

\title{A Peacock in Mid Flight}
\author{R (Chandra) Chandrasekhar}
\date{2019-11-17}

\begin{document}
\maketitle

\thispagestyle{empty}


This morning, I saw a peacock in mid-flight. It was a wondrous and
unforgettable sight, the more so because unexpected. I was walking as
usual in the neighbourhood, which is known to be home to a large number
of peacocks. As I turned to my right, I saw what I thought was an
unusually long bird in flight. Only as it approached me did I realize
that it was a peacock in mid-flight.

Its passage through the air was fluid and effortless: a picture of grace
and elegance. It was like a silent airborne missile cruising at low
altitude. It flapped its wings barely once or twice when it banked to
turn right. And then it landed with aplomb. That was when I saw how long
its tail really was. It appeared longer than the body and should
certainly have contributed to the smoothest of landings of an airborne
bird that I have ever seen.

Whether the tail contributes more to lift or drag, or whether it is
simply a gorgeous rudder I cannot tell. But I marvelled at how Nature
has patiently crafted the peacock---with its beautiful opalescent cyan
neck and uniquely patterned feathers---over evolutionary time to
complete the perfect marriage of form and function, of beauty and
utility, of grace and majesty. Here was a bird that flew like poetry in
motion.

We human beings, when confronted by the magnificent feathers of a
peacock, have unfairly imposed our misshapen human values on them, and
coined expressions like ``proud as a peacock'' and ``strutting about
like a peacock'' even though we have no access to a peacock's feelings.

From a biological point of view, colourful plumage in birds is designed
to attract mates. Indeed, there are species who perform complicated
courting rituals by using their feathers to best advantage. In the case
of the peacock, it is the male that is spectacular; the female is more
modestly beplumed.

I have always \emph{heard} the neighbourhood peacocks more than
\emph{seen} them. And the call of the peacock does not quite do justice
to its looks. There is an old description of female pulchritude as
embodying ``the voice of a nightingale and the plumage of a peacock''.
After hearing the peacocks, I have often chuckled to myself that if the
two birds had been interchanged, the result would certainly be less than
attractive.

If you had thought the peacock to be flightless, you would be excused.
It is a rather large bird, like the swan. Seeing the peacock in flight,
reminded me of the famed
\href{http://www.boeing.com/commercial/787family/background.html}{Boeing
787 Dreamliner} aircraft now making its début worldwide, and being
hailed for its efficiency and comfort\footnote{This was true at the time
  of writing, but issues of quality and cost have surfaced since, to mar
  the \emph{dream}.}. The peacock is proof that engineering and art can
and do meet with superlative results.

The swan and the peacock are both large birds. Each has a peculiar
beauty and grace. The swan serenely floating on water, and the peacock
standing with its tail spread out, are epitomes of natural beauty. And
when they fly, they both do so effortlessly and gracefully.

Interestingly, both birds feature in the mythology of \emph{Sanātana
Dharma}, more commonly known as Hinduism. Subrahmaṇya or Kārtikeya, the
Divine General, rides a peacock. And Sarasvatī, the Goddess of Wisdom,
is surrounded by both a peacock and a swan. The peacock's call is
supposed to act as a tuning fork to help her tune her stringed
instrument, the \href{http://www.thefreedictionary.com/vina}{vīṇā,}
while the white swan represents divine discrimination.

I sorely wanted to capture the peacock in mid-flight, but it was too
swift and sudden for me to photograph it. ``Perhaps some other time,'' I
consoled myself. I then realized the wonders of the Web and decided that
there might sites with photographs of peacocks in flight.
\href{http://bennie-thesmiths.blogspot.in/2012/05/peacock-in-flight.html}{The
Smith's Bennie and Patsy blog entitled \emph{Peacock In Flight}} has
some magnificent shots. And there is another lovely image at
\href{http://www.trekearth.com/gallery/Asia/India/West/Rajasthan/Sujangarh/photo772964.htm}{Peacock
in Flight by Annu.}

For now, word pictures from me of the peacock in mid-flight must
suffice. The peacock I saw was a dream gliding through air. It was
effortless, efficient, smooth, graceful, unfluttered, unflustered,
powerful, silent, and exquisitely matched to the element. It was a
superb blend of engineering and art, of power and poise, honed to
perfection by millennia of evolution, a dream of heavenly beauty on
earth.

\hypertarget{feedback}{%
\subsection{Feedback}\label{feedback}}

Please \href{mailto:feedback.swanlotus@gmail.com}{email me} your
comments and corrections.

\noindent A PDF version of this article is
\href{./a-peacock-in-midflight.pdf}{available for download here}:

\begin{small}

\begin{sffamily}

\url{https://swanlotus.netlify.app/blogs/a-peacock-in-mid-flight.pdf}

\end{sffamily}

\end{small}



\end{document}
