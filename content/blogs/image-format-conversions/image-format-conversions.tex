% Options for packages loaded elsewhere
\PassOptionsToPackage{unicode,linktoc=all}{hyperref}
\PassOptionsToPackage{hyphens}{url}
\PassOptionsToPackage{dvipsnames,svgnames*,x11names*}{xcolor}
%
\documentclass[
  11pt,
  british,
  a4paper,
]{article}
\usepackage{lmodern}
\usepackage{amsmath}
\usepackage{ifxetex,ifluatex}
\ifnum 0\ifxetex 1\fi\ifluatex 1\fi=0 % if pdftex
  \usepackage[T1]{fontenc}
  \usepackage[utf8]{inputenc}
  \usepackage{textcomp} % provide euro and other symbols
  \usepackage{amssymb}
\else % if luatex or xetex
  \usepackage{unicode-math}
  \defaultfontfeatures{Scale=MatchLowercase}
  \defaultfontfeatures[\rmfamily]{Ligatures=TeX,Scale=1}
\fi
% Use upquote if available, for straight quotes in verbatim environments
\IfFileExists{upquote.sty}{\usepackage{upquote}}{}
\IfFileExists{microtype.sty}{% use microtype if available
  \usepackage[]{microtype}
  \UseMicrotypeSet[protrusion]{basicmath} % disable protrusion for tt fonts
}{}
\makeatletter
\@ifundefined{KOMAClassName}{% if non-KOMA class
  \IfFileExists{parskip.sty}{%
    \usepackage{parskip}
  }{% else
    \setlength{\parindent}{0pt}
    \setlength{\parskip}{6pt plus 2pt minus 1pt}}
}{% if KOMA class
  \KOMAoptions{parskip=half}}
\makeatother
\usepackage{xcolor}
\IfFileExists{xurl.sty}{\usepackage{xurl}}{} % add URL line breaks if available
\IfFileExists{bookmark.sty}{\usepackage{bookmark}}{\usepackage{hyperref}}
\hypersetup{
  pdftitle={Image format conversions},
  pdfauthor={R (Chandra) Chandrasekhar},
  pdflang={en-GB},
  colorlinks=true,
  linkcolor=TealBlue,
  filecolor=Purple,
  citecolor=DarkKhaki,
  urlcolor=Maroon,
  pdfcreator={LaTeX via pandoc}}
\urlstyle{same} % disable monospaced font for URLs
\usepackage[margin=25mm]{geometry}
\usepackage{color}
\usepackage{fancyvrb}
\newcommand{\VerbBar}{|}
\newcommand{\VERB}{\Verb[commandchars=\\\{\}]}
\DefineVerbatimEnvironment{Highlighting}{Verbatim}{commandchars=\\\{\}}
% Add ',fontsize=\small' for more characters per line
\usepackage{framed}
\definecolor{shadecolor}{RGB}{48,48,48}
\newenvironment{Shaded}{\begin{snugshade}}{\end{snugshade}}
\newcommand{\AlertTok}[1]{\textcolor[rgb]{1.00,0.81,0.69}{#1}}
\newcommand{\AnnotationTok}[1]{\textcolor[rgb]{0.50,0.62,0.50}{\textbf{#1}}}
\newcommand{\AttributeTok}[1]{\textcolor[rgb]{0.80,0.80,0.80}{#1}}
\newcommand{\BaseNTok}[1]{\textcolor[rgb]{0.86,0.64,0.64}{#1}}
\newcommand{\BuiltInTok}[1]{\textcolor[rgb]{0.80,0.80,0.80}{#1}}
\newcommand{\CharTok}[1]{\textcolor[rgb]{0.86,0.64,0.64}{#1}}
\newcommand{\CommentTok}[1]{\textcolor[rgb]{0.50,0.62,0.50}{#1}}
\newcommand{\CommentVarTok}[1]{\textcolor[rgb]{0.50,0.62,0.50}{\textbf{#1}}}
\newcommand{\ConstantTok}[1]{\textcolor[rgb]{0.86,0.64,0.64}{\textbf{#1}}}
\newcommand{\ControlFlowTok}[1]{\textcolor[rgb]{0.94,0.87,0.69}{#1}}
\newcommand{\DataTypeTok}[1]{\textcolor[rgb]{0.87,0.87,0.75}{#1}}
\newcommand{\DecValTok}[1]{\textcolor[rgb]{0.86,0.86,0.80}{#1}}
\newcommand{\DocumentationTok}[1]{\textcolor[rgb]{0.50,0.62,0.50}{#1}}
\newcommand{\ErrorTok}[1]{\textcolor[rgb]{0.76,0.75,0.62}{#1}}
\newcommand{\ExtensionTok}[1]{\textcolor[rgb]{0.80,0.80,0.80}{#1}}
\newcommand{\FloatTok}[1]{\textcolor[rgb]{0.75,0.75,0.82}{#1}}
\newcommand{\FunctionTok}[1]{\textcolor[rgb]{0.94,0.94,0.56}{#1}}
\newcommand{\ImportTok}[1]{\textcolor[rgb]{0.80,0.80,0.80}{#1}}
\newcommand{\InformationTok}[1]{\textcolor[rgb]{0.50,0.62,0.50}{\textbf{#1}}}
\newcommand{\KeywordTok}[1]{\textcolor[rgb]{0.94,0.87,0.69}{#1}}
\newcommand{\NormalTok}[1]{\textcolor[rgb]{0.80,0.80,0.80}{#1}}
\newcommand{\OperatorTok}[1]{\textcolor[rgb]{0.94,0.94,0.82}{#1}}
\newcommand{\OtherTok}[1]{\textcolor[rgb]{0.94,0.94,0.56}{#1}}
\newcommand{\PreprocessorTok}[1]{\textcolor[rgb]{1.00,0.81,0.69}{\textbf{#1}}}
\newcommand{\RegionMarkerTok}[1]{\textcolor[rgb]{0.80,0.80,0.80}{#1}}
\newcommand{\SpecialCharTok}[1]{\textcolor[rgb]{0.86,0.64,0.64}{#1}}
\newcommand{\SpecialStringTok}[1]{\textcolor[rgb]{0.80,0.58,0.58}{#1}}
\newcommand{\StringTok}[1]{\textcolor[rgb]{0.80,0.58,0.58}{#1}}
\newcommand{\VariableTok}[1]{\textcolor[rgb]{0.80,0.80,0.80}{#1}}
\newcommand{\VerbatimStringTok}[1]{\textcolor[rgb]{0.80,0.58,0.58}{#1}}
\newcommand{\WarningTok}[1]{\textcolor[rgb]{0.50,0.62,0.50}{\textbf{#1}}}
\usepackage{longtable,booktabs}
\usepackage{calc} % for calculating minipage widths
% Correct order of tables after \paragraph or \subparagraph
\usepackage{etoolbox}
\makeatletter
\patchcmd\longtable{\par}{\if@noskipsec\mbox{}\fi\par}{}{}
\makeatother
% Allow footnotes in longtable head/foot
\IfFileExists{footnotehyper.sty}{\usepackage{footnotehyper}}{\usepackage{footnote}}
\makesavenoteenv{longtable}
\usepackage{graphicx}
\makeatletter
\def\maxwidth{\ifdim\Gin@nat@width>\linewidth\linewidth\else\Gin@nat@width\fi}
\def\maxheight{\ifdim\Gin@nat@height>\textheight\textheight\else\Gin@nat@height\fi}
\makeatother
% Scale images if necessary, so that they will not overflow the page
% margins by default, and it is still possible to overwrite the defaults
% using explicit options in \includegraphics[width, height, ...]{}
\setkeys{Gin}{width=\maxwidth,height=\maxheight,keepaspectratio}
% Set default figure placement to htbp
\makeatletter
\def\fps@figure{htbp}
\makeatother
\setlength{\emergencystretch}{3em} % prevent overfull lines
\providecommand{\tightlist}{%
  \setlength{\itemsep}{0pt}\setlength{\parskip}{0pt}}
\setcounter{secnumdepth}{-\maxdimen} % remove section numbering
\usepackage{titling}
\usepackage{fancyhdr}
    \pagestyle{fancy}
    \fancyhead{}
    \fancyfoot{}
    \renewcommand{\headrulewidth}{0.2pt}
    \renewcommand{\footrulewidth}{0.2pt}
    \fancyhead[LO,RE]{\scshape\thetitle}
    \fancyfoot[CO,CE]{\small Copyright © 2006--\the\year, R (Chandra) Chandrasekhar}
    \fancyfoot[RE,RO]{\thepage}
%% pandoc-eqnos
\usepackage[capitalise]{cleveref}
  \crefname{equation}{Equation}{Equations}
  \crefname{figure}{Figure}{Figures}
  \crefname{table}{Table}{Tables}
%% pandoc-fignos
\usepackage{caption}
%% pandoc-fignos: environment to disable figure caption prefixes
    \makeatletter
    \newcounter{figno}
    \newenvironment{fignos:no-prefix-figure-caption}{
      \caption@ifcompatibility{}{
        \let\oldthefigure\thefigure
        \let\oldtheHfigure\theHfigure
        \renewcommand{\thefigure}{figno:\thefigno}
        \renewcommand{\theHfigure}{figno:\thefigno}
        \stepcounter{figno}
        \captionsetup{labelformat=empty}
      }
    }{
      \caption@ifcompatibility{}{
        \captionsetup{labelformat=default}
        \let\thefigure\oldthefigure
        \let\theHfigure\oldtheHfigure
        \addtocounter{figure}{-1}
      }
    }
    \makeatother
%%

\AtBeginEnvironment{quote}{\small}
\setlength{\parindent}{0pt} % block paragraphs
\usepackage{etoolbox}
\usepackage{graphicx}
\usepackage{subcaption}
\usepackage{svg}
\usepackage[Latin,Tamil,Devanagari]{ucharclasses}
  \setmainfont[SmallCapsFont={Charis SIL Small Caps}]{Charis SIL}
  \setsansfont[Numbers=OldStyle,BoldFont={* Semibold}]{Source Sans Pro}
  \setmonofont[Scale=0.90]{Fira Mono}
  \defaultfontfeatures{Ligatures=TeX,Scale=MatchLowercase}
  \setmathfont[bold-style=ISO]{STIX Two Math}
  \newfontfamily\tamilfont[Script=Tamil]{Noto Sans Tamil}
  \setTransitionsFor{Tamil}{\tamilfont}{\normalfont}
  \setTransitionTo{Tamil}{\tamilfont}{}
  \setTransitionFrom{Tamil}{\normalfont}
  \newfontfamily\devfont[Script=Devanagari]{Noto Sans Devanagari}
  \setTransitionsFor{Devanagari}{\devfont}{\normalfont}
  \setTransitionTo{Devanagari}{\devfont}{}
  \setTransitionFrom{Devanagari}{\normalfont}
  \newfontfamily{\emojifont}{Symbola}
\usepackage[all]{nowidow}
\usepackage[margins=raggedright]{floatrow}
%
%
% Adjust punctuation for footnotes
%
\usepackage{fnpct} % footnote _before_ punctuation reversed and adjusted.
  \setfnpct{after-dot-space=-0.15em,after-comma-space=-0.150em,add-punct-marks=;[0.06em]![0.06em]?[0.06em]:[0.06em]}
%
% Flexible fontsizes
%
\usepackage{fontsize}
%
\ifxetex
  % Load polyglossia as late as possible: uses bidi with RTL langages (e.g. Hebrew, Arabic)
  \usepackage{polyglossia}
  \setmainlanguage[variant=british]{english}
\else
  \usepackage[shorthands=off,main=british]{babel}
\fi
\ifluatex
  \usepackage{selnolig}  % disable illegal ligatures
\fi
\newlength{\cslhangindent}
\setlength{\cslhangindent}{1.5em}
\newlength{\csllabelwidth}
\setlength{\csllabelwidth}{3em}
\newenvironment{CSLReferences}[2] % #1 hanging-ident, #2 entry spacing
 {% don't indent paragraphs
  \setlength{\parindent}{0pt}
  % turn on hanging indent if param 1 is 1
  \ifodd #1 \everypar{\setlength{\hangindent}{\cslhangindent}}\ignorespaces\fi
  % set entry spacing
  \ifnum #2 > 0
  \setlength{\parskip}{#2\baselineskip}
  \fi
 }%
 {}
\usepackage{calc}
\newcommand{\CSLBlock}[1]{#1\hfill\break}
\newcommand{\CSLLeftMargin}[1]{\parbox[t]{\csllabelwidth}{#1}}
\newcommand{\CSLRightInline}[1]{\parbox[t]{\linewidth - \csllabelwidth}{#1}\break}
\newcommand{\CSLIndent}[1]{\hspace{\cslhangindent}#1}

\title{Image format conversions}
\author{R (Chandra) Chandrasekhar}
\date{2021-03-07 | 2021-03-07}

\begin{document}
\maketitle

\thispagestyle{empty}


Converting between image formats is something we could be required to do
at short notice. This tutorial explores the different tools that are
currently available and identifies the most efficient for each task. The
\texttt{ImageMagick} suite, the \texttt{cairo} backend, the
\texttt{poppler} utilities, \texttt{rsvg-convert}, and \texttt{CairoSVG}
are identified for the specific strengths that make them the tools of
choice for different image conversion tasks.

\hypertarget{two-varieties-of-digital-images}{%
\subsection{Two varieties of digital
images}\label{two-varieties-of-digital-images}}

Digital images come in two broad flavours:

\begin{itemize}
\tightlist
\item
  \href{https://en.wikipedia.org/wiki/Raster_graphics}{raster} or
  \href{https://en.wikipedia.org/wiki/Bitmap}{bitmap} graphics, and
\item
  \href{https://en.wikipedia.org/wiki/Vector_graphics}{vector graphics}.
\end{itemize}

The former leads to image blockiness or
\href{https://en.wikipedia.org/wiki/Pixelation}{pixellation} and loss of
definition at high magnifications, as shown in \cref{fig:raster}, while
the latter scales without degradation when magnified, as illustrated in
\cref{fig:vector}.

\begin{figure}
\hypertarget{fig:raster}{%
\centering
\includegraphics[width=0.5\textwidth,height=\textheight]{images/letter-O-150-dpi.png}
\caption[Raster graphics image of the letter O (150 dpi PNG
format).]{Raster graphics image of the letter O (150 dpi\footnotemark{}
PNG format).}\label{fig:raster}
}
\end{figure}
\footnotetext{Dots per inch.}

\begin{figure}
\hypertarget{fig:vector}{%
\centering
\includegraphics[width=0.5\textwidth,height=\textheight]{images/letter-O.svg}
\caption{Vector graphics image of the letter O (SVG
format).}\label{fig:vector}
}
\end{figure}

\hypertarget{raster-graphics}{%
\subsubsection{Raster Graphics}\label{raster-graphics}}

There are dozens of image formats, including these three:

\begin{enumerate}
\tightlist
\item
  \href{https://en.wikipedia.org/wiki/TIFF}{Tag(ged) Image File Format
  (TIFF)}

  \begin{itemize}
  \tightlist
  \item
    lossless compression
  \item
    large file sizes
  \item
    used in printing and professional graphics
  \item
    preferred for archival of scanned photographs
  \end{itemize}
\item
  \href{https://jpeg.org/about.html}{Joint Photographic Experts Group
  (JPEG)} format

  \begin{itemize}
  \tightlist
  \item
    small file sizes
  \item
    lossy compression
  \item
    good quality with fast downloads
  \item
    supported by web browsers
  \item
    preferred for scenes and portraits
  \item
    no transparency
  \end{itemize}
\item
  \href{http://www.libpng.org/pub/png/}{Portable Network Graphics (PNG)}
  format

  \begin{itemize}
  \tightlist
  \item
    lossless compression
  \item
    preferred for text and high definition images
  \item
    supported by most web browsers
  \item
    transparency
  \end{itemize}
\end{enumerate}

All three formats employ raster graphics in which image elements are
represented as rectangular arrays of
\href{https://en.wikipedia.org/wiki/Pixels}{pixels}.

\hypertarget{vector-graphics}{%
\subsubsection{Vector Graphics}\label{vector-graphics}}

The two principal vector graphics formats are:

\begin{enumerate}
\tightlist
\item
  \href{https://en.wikipedia.org/wiki/PDF}{Portable Document Format
  (PDF)}

  \begin{itemize}
  \tightlist
  \item
    preferred for archival quality electronic and printed documents
  \item
    supported by browsers with integrated PDF readers
  \item
    file sizes comparable to raster images
  \item
    machine-readable files
  \end{itemize}
\item
  \href{https://en.wikipedia.org/wiki/Scalable_Vector_Graphics}{Scalable
  Vector Graphics (SVG)} format

  \begin{itemize}
  \tightlist
  \item
    preferred for scalable graphics on web browsers
  \item
    used in digital image animations and digital art
  \item
    small file sizes
  \item
    human- and machine-readable files
  \end{itemize}
\end{enumerate}

Both these formats yield images which consist of mathematically defined
points, lines, curves, and shapes.

\hypertarget{format-conversions}{%
\subsection{Format conversions}\label{format-conversions}}

For a variety of reasons, it is often necessary to convert from one
image format to another. There are four broad possibilities for this:

\begin{enumerate}
\def\labelenumi{\alph{enumi}.}
\tightlist
\item
  raster to raster;
\item
  raster to vector;
\item
  vector to raster; and
\item
  vector to vector
\end{enumerate}

We consider each of these in turn using
\href{https://itlaw.wikia.org/wiki/Platform_neutral}{platform-neutral}
\href{https://opensource.com/resources/what-open-source}{open source}
tools. Since I run
\href{https://en.wikipedia.org/wiki/GNU/Linux_naming_controversy}{GNU/Linux}
on my desktop, my examples will feature commands from that
setup.\footnote{There are many websites that promise conversion online,
  requiring you to upload the input file and download the output file.
  These \emph{might be} fraught with security risks. Use them with
  caution.}

\hypertarget{tools-for-image-format-conversion}{%
\subsection{Tools for image format
conversion}\label{tools-for-image-format-conversion}}

Among the very many tools available, we examine below four that support
image format conversion:

\begin{enumerate}
\tightlist
\item
  \href{https://imagemagick.org/index.php}{\texttt{ImageMagick}}

  \begin{itemize}
  \tightlist
  \item
    graphics library for image manipulation and display
  \item
    standalone utilities like \texttt{convert}, \texttt{display},
    \texttt{identify}, \texttt{mogrify}, etc.
  \item
    scripting language
  \item
    pixel-based
  \item
    raster to raster conversions
  \item
    raster to vector conversions
  \end{itemize}
\item
  \href{https://www.cairographics.org/}{cairo}

  \begin{itemize}
  \tightlist
  \item
    vector-based 2D drawing and rendering library
  \item
    multiple output devices/formats
  \item
    used by other programs rather than in standalone mode
  \end{itemize}
\item
  \href{https://poppler.freedesktop.org/}{poppler}

  \begin{itemize}
  \tightlist
  \item
    vector-based PDF rendering library
  \item
    used by several PDF viewers
  \item
    uses cairo as backend
  \item
    standalone utilities like \texttt{pdftotext}, \texttt{pdftocairo},
    and \texttt{pdftoppm}
  \end{itemize}
\item
  \href{https://inkscape.org/}{Inkscape}

  \begin{itemize}
  \tightlist
  \item
    GUI-based vector graphics editor
  \item
    suitable both for technical illustration and digital art
  \item
    uses SVG as the main format
  \item
    can export to a wide variety of output formats
  \item
    option to use cairo for PDF export
  \end{itemize}
\end{enumerate}

\hypertarget{imagemagick-the-swiss-army-knife}{%
\subsection{\texorpdfstring{\texttt{ImageMagick}: the Swiss Army
knife}{ImageMagick: the Swiss Army knife}}\label{imagemagick-the-swiss-army-knife}}

\texttt{ImageMagick} is the name given to a suite of image processing
tools originally created in 1987 by John Cristy, then working for
\href{https://www.dupont.com/}{Du Pont}. In 1990, it was freely released
by Du Pont, who transferred copyright to
\href{https://imagemagick.org/script/contact.php}{\texttt{ImageMagick}
Studio LLC} who now maintain the project. It is distributed under a
derived Apache 2.0
\href{https://imagemagick.org/script/license.php}{license}. The
\href{https://github.com/\%60ImageMagick\%60/\%60ImageMagick\%60}{authoritative
source code repository} shows active development even today, 34 years
after the suite was first released {[}1{]}.

\texttt{ImageMagick} is so versatile and useful that it may rightfully
be called the
\href{https://www.thefreedictionary.com/Swiss-army+knife}{Swiss Army
knife} of the image processing world. It comes with several command line
utilities, each replete with options. Among these are:

\begin{itemize}
\tightlist
\item
  \href{https://imagemagick.org/script/convert.php}{\texttt{convert}}
  which converts from one format to another;
\item
  \href{https://imagemagick.org/script/display.php}{\texttt{display}}
  which displays one or more images;
\item
  \href{https://imagemagick.org/script/identify.php}{\texttt{identify}}
  which identifies the type of image and displays its characteristics;
\item
  \href{https://imagemagick.org/script/mogrify.php}{\texttt{mogrify}}
  which transforms an image, modifying its appearance; and
\item
  \href{https://imagemagick.org/script/montage.php}{\texttt{montage}}
  which generates an image montage from several images.
\end{itemize}

The above list is far from exhaustive. The interested reader is referred
to the
\href{https://imagemagick.org/script/command-line-tools.php}{excellent
online documentation} for further details. The power of
\texttt{ImageMagick} is enhanced with the
\href{https://imagemagick.org/script/magick-script.php}{magick-script}
Image Scripting Language. The examples in this blog use the command line
versions of invoking \texttt{ImageMagick}. If they seem daunting,
\href{https://imagemagick.org/script/command-line-processing.php}{refer
to this explanation} {[}2{]}.

\hypertarget{test-images}{%
\subsection{Test images}\label{test-images}}

Two quite different images are used to illustrate the format conversions
we perform here. The two test images are:

\begin{enumerate}
\tightlist
\item
  a coloured, text-only test image called \texttt{text-only.png}; and
\item
  a non-text, coloured, graphically rich image called
  \texttt{animals.jpg}.
\end{enumerate}

We will refer to these two images as \texttt{text-only} and
\texttt{animals}, respectively hereafter.

Note that, as \protect\hyperlink{non-text-test-image}{explained below},
\texttt{animals.jpg} is a cropped version of the original image
\texttt{animals-original.jpg} downloaded from the Web.

\hypertarget{text-only-image}{%
\subsubsection{Text-only image}\label{text-only-image}}

The \texttt{text-only} image was first generated as a PDF file,
\texttt{text-only.pdf}, by compiling a
\href{https://www.latex-project.org/}{LaTeX} source file. That file was
then converted to various raster formats using the methods
\protect\hyperlink{vector-to-raster-1}{discussed later} to yield the
images \texttt{text-only-600-dpi.png} and
\texttt{text-only-600-dpi.jpg}.

\begin{figure}
\hypertarget{fig:text-only}{%
\centering
\includegraphics[width=0.8\textwidth,height=\textheight]{images/text-only-600-dpi-cairo.png}
\caption{Text-only image in 600 dpi PNG format.}\label{fig:text-only}
}
\end{figure}

\hypertarget{non-text-test-image}{%
\subsubsection{Non-text test image}\label{non-text-test-image}}

The non-text image, \texttt{animals-original.jpg}, is a colourful,
graphically rich image with much detail. It is from a hand-drawn
illustration of microscopic marine animals by the German naturalist
\href{https://en.wikipedia.org/wiki/Ernst_Haeckel}{Ernst Haeckel},
scanned as a JPEG, and made available in the public domain.

\begin{figure}
\hypertarget{fig:animals-original}{%
\centering
\includegraphics[width=0.8\textwidth,height=\textheight]{images/animals-original.jpg}
\caption[Non-text, graphically rich \texttt{animals-original.jpg} image
in JPEG format.]{Non-text, graphically rich
\texttt{animals-original.jpg} image in JPEG
format.\footnotemark{}}\label{fig:animals-original}
}
\end{figure}
\footnotetext{These images are in the public domain and covered by the
  \href{https://creativecommons.org/publicdomain/zero/1.0/}{CC0
  licence}. They are available for download
  \href{https://www.rawpixel.com/image/2266608/free-illustration-image-ernst-haeckel-vintage-animals}{here}.}

\hypertarget{pre-processing-cropping}{%
\subsection{Pre-processing: Cropping}\label{pre-processing-cropping}}

Cropping is strictly not image format conversion, but is often a
necessary pre-processing step in image manipulations. For example,
\cref{fig:animals-original} has a whitish, non-monochromatic border
around the block print, containing annotations. For our purposes, this
border is at best a distraction. It may be removed altogether by
\emph{cropping}, leaving us with only the illustration. We will refer to
the resulting cropped image, \texttt{animals.jpg}, as \texttt{animals},
and use it as the source image in our examples below.

Cropping is usually better done interactively using a
\href{https://en.wikipedia.org/wiki/Graphical_user_interface}{GUI
(Graphical User Interface)}, than on the command line. However, the
latter, even if a bit tedious, is precisely repeatable.

\texttt{ImageMagick}'s \texttt{display} utility pops up a GUI when the
mouse is over the image and the left mouse button is clicked. We can
then drag and fit a window to the \emph{region we wish to keep},
clicking the \texttt{Crop} function, and saving the cropped image. The
steps are these:

\begin{enumerate}
\def\labelenumi{\alph{enumi}.}
\tightlist
\item
  left mouse click on the image to reveal the GUI (see \cref{fig:gui});
\item
  \texttt{Transform\ -\textgreater{}\ Crop};
\item
  put the mouse over the top left corner and drag until the bottom right
  corner to enclose the region of interest;
\item
  Click again on \texttt{Crop}; and
\item
  \texttt{File\ -\textgreater{}\ Save} with a different name.
\end{enumerate}

\begin{figure}
\hypertarget{fig:gui}{%
\centering
\includegraphics[width=0.2\textwidth,height=\textheight]{images/ImageMagick-display-gui.png}
\caption{\texttt{ImageMagick} interactive GUI.}\label{fig:gui}
}
\end{figure}

Alternatively, we may just position the cursor on the top left and
bottom right corners of the region we wish to \emph{retain}, noting the
co-ordinates in each case. If these coordinates are \((x_t, y_t)\) and
\((x_b, y_b)\), respectively, we have \(w = x_b - x_t\) and
\(h = y_b -y_t\). We may then invoke the convert command with crop as
the option so:

\begin{Shaded}
\begin{Highlighting}[]
\ExtensionTok{convert} \AttributeTok{{-}crop} \StringTok{\textquotesingle{}wxh+x\_t+y\_t\textquotesingle{}}\NormalTok{ animals{-}original.jpg animals.jpg}
\end{Highlighting}
\end{Shaded}

In our case, \((x_t, y_t) = (60, 84)\) and \((x_b, y_b) = (795, 1119)\)
giving \(w = 735\) and \(h = 1035\), leading to

\begin{Shaded}
\begin{Highlighting}[]
\ExtensionTok{convert} \AttributeTok{{-}crop} \StringTok{\textquotesingle{}735x1035+60+84\textquotesingle{}}\NormalTok{ animals{-}original.jpg animals.jpg}
\end{Highlighting}
\end{Shaded}

The resulting cropped image is shown in \cref{fig:cropped} below.

\begin{figure}
\hypertarget{fig:cropped}{%
\centering
\includegraphics[width=0.5\textwidth,height=\textheight]{images/animals.jpg}
\caption{Cropped version of the image in \cref{fig:animals-original}.
This is the \texttt{animals} image.}\label{fig:cropped}
}
\end{figure}

\hypertarget{file-sizes}{%
\subsubsection{File sizes}\label{file-sizes}}

The sizes of the original and cropped files are shown below in human
friendly numbers:

\begin{Shaded}
\begin{Highlighting}[]
\FunctionTok{ls} \AttributeTok{{-}Xsh}\NormalTok{ animals}\PreprocessorTok{*}\NormalTok{.jpg }\KeywordTok{|} \FunctionTok{awk} \StringTok{\textquotesingle{}\{print $1 "\textbackslash{}t" $2\}\textquotesingle{}}
\ExtensionTok{{-}{-}{-}}
\ExtensionTok{200K}\NormalTok{    animals.jpg}
\ExtensionTok{312K}\NormalTok{    animals{-}original.jpg}
\end{Highlighting}
\end{Shaded}

As a convention hereafter, when there is a \texttt{-\/-\/-} separator
between a command and some results, the latter are the results displayed
on execution of the command.

As expected, the original file \texttt{animals-original.jpg} is larger
than the cropped full-size version, \texttt{animals.jpg}, and all is
well.

\hypertarget{raster-to-raster-conversion}{%
\subsection{Raster to raster
conversion}\label{raster-to-raster-conversion}}

We now perform a sequence of image manipulations, including raster to
raster format conversions.

\hypertarget{resizing-format-conversion-and-montaging}{%
\subsubsection{Resizing, format-conversion, and
montaging}\label{resizing-format-conversion-and-montaging}}

We may invoke \texttt{ImageMagick}'s \texttt{convert} function not only
to convert from one format to another but also to accomplish cropping
(as we have already seen), image-resizing, making the background
transparent, and
\href{https://www.thefreedictionary.com/montage}{montaging}, etc.

Suppose we want to reduce the dimensions of the cropped image to half
their original values, and display the full-size and half-size images
side by side, we could run the following command:

\begin{Shaded}
\begin{Highlighting}[]
\ExtensionTok{convert}\NormalTok{ animals.jpg }\AttributeTok{{-}resize}\NormalTok{ 50\% animals{-}halfsize.jpg}

\CommentTok{\# Composite the two images aligning their bottoms}
\ExtensionTok{convert}\NormalTok{ +append }\AttributeTok{{-}gravity}\NormalTok{ south }\DataTypeTok{\textbackslash{}}
\NormalTok{animals.jpg animals{-}halfsize.jpg animals{-}both.jpg}
\end{Highlighting}
\end{Shaded}

\begin{figure}
\hypertarget{fig:animals-both-jpg}{%
\centering
\includegraphics[width=0.8\textwidth,height=\textheight]{images/animals-both.jpg}
\caption{Full-size cropped image on the left and half-sized image on the
right.}\label{fig:animals-both-jpg}
}
\end{figure}

\hypertarget{background-transparency}{%
\subsubsection{Background transparency}\label{background-transparency}}

Notice that there is a coloured white rectangle atop the half-size image
on the right in \cref{fig:animals-both-jpg}. We could remove it by
rendering the background transparent. However, because JPEG does not
support transparency, through an
\href{https://www.techopedia.com/definition/1945/alpha-channel}{alpha
channel}, we have to convert the composite image to the PNG format,
which does. This is one, real-life circumstance necessitating raster to
raster format conversion.

\begin{Shaded}
\begin{Highlighting}[]
\CommentTok{\# Non{-}transparent composite in PNG}
\ExtensionTok{convert}\NormalTok{ +append }\AttributeTok{{-}gravity}\NormalTok{ south }\DataTypeTok{\textbackslash{}}
\NormalTok{animals.jpg animals{-}halfsize.jpg animals{-}both{-}non{-}transparent.png}

\CommentTok{\# Transparent composite in PNG}
\ExtensionTok{convert}\NormalTok{ +append }\AttributeTok{{-}gravity}\NormalTok{ south }\AttributeTok{{-}background}\NormalTok{ transparent }\DataTypeTok{\textbackslash{}}
\NormalTok{animals.jpg animals{-}halfsize.jpg animals{-}both{-}transparent.png}
\end{Highlighting}
\end{Shaded}

\begin{figure}
\hypertarget{fig:animals-both-png}{%
\centering
\includegraphics[width=0.8\textwidth,height=\textheight]{images/animals-both-transparent.png}
\caption{Composite image converted to PNG format with transparent
background.}\label{fig:animals-both-png}
}
\end{figure}

\hypertarget{file-sizes-again}{%
\paragraph{File sizes again}\label{file-sizes-again}}

How do the file sizes of the three composite images compare? How does
the non-transparent JPEG compare with the non-transparent PNG? Also, how
high a price have we paid for the transparent background?

\begin{Shaded}
\begin{Highlighting}[]
\FunctionTok{ls} \AttributeTok{{-}Xsh}\NormalTok{ animals{-}both}\PreprocessorTok{*}\KeywordTok{|} \FunctionTok{awk} \StringTok{\textquotesingle{}\{print $1 "\textbackslash{}t" $2\}\textquotesingle{}}
\ExtensionTok{{-}{-}{-}}
\ExtensionTok{264K}\NormalTok{    animals{-}both.jpg}
\ExtensionTok{2.0M}\NormalTok{    animals{-}both{-}non{-}transparent.png}
\ExtensionTok{2.2M}\NormalTok{    animals{-}both{-}transparent.png}
\end{Highlighting}
\end{Shaded}

The PNG composite image is \emph{more than seven times larger} than its
JPEG counterpart, \emph{even without transparency}. And transparency
makes the PNG file size ten percent larger.

\hypertarget{compression-levels-and-file-sizes}{%
\paragraph{Compression levels and file
sizes}\label{compression-levels-and-file-sizes}}

The \href{https://en.wikipedia.org/wiki/Image_compression}{image
compression level} used above is the default compression level in
\texttt{ImageMagick}. Getting the right combination of image format,
image dimensions, image compression, and image quality so that the image
loads fast and looks good is
\href{https://www.smashingmagazine.com/2015/06/efficient-image-resizing-with-imagemagick/}{quite
an art}. {[}3{]}

To get an idea of the range of file sizes involved, let us try
generating a composite image with extremes of the compression level,
which can range from 0 to 9.

\begin{Shaded}
\begin{Highlighting}[]
\CommentTok{\# Least compression}
\ExtensionTok{convert} \AttributeTok{{-}define}\NormalTok{ PNG:compression{-}level=0 +append }\AttributeTok{{-}gravity}\NormalTok{ south }\DataTypeTok{\textbackslash{}}
\NormalTok{{-}background transparent }\DataTypeTok{\textbackslash{}}
\NormalTok{animals.jpg animals{-}halfsize.jpg }\DataTypeTok{\textbackslash{}}
\NormalTok{animals{-}both{-}compressed{-}0.png}

\CommentTok{\# Most compression}
\ExtensionTok{convert} \AttributeTok{{-}define}\NormalTok{ PNG:compression{-}level=9 +append }\AttributeTok{{-}gravity}\NormalTok{ south }\DataTypeTok{\textbackslash{}}
\NormalTok{{-}background transparent }\DataTypeTok{\textbackslash{}}
\NormalTok{animals.jpg animals{-}halfsize.jpg }\DataTypeTok{\textbackslash{}}
\NormalTok{animals{-}both{-}compressed{-}9.png}
\end{Highlighting}
\end{Shaded}

The file sizes are:

\begin{Shaded}
\begin{Highlighting}[]
\FunctionTok{ls} \AttributeTok{{-}Xsh}\NormalTok{ animals{-}both}\PreprocessorTok{*} \KeywordTok{|} \FunctionTok{awk} \StringTok{\textquotesingle{}\{print $1 "\textbackslash{}t" $2\}\textquotesingle{}}
\ExtensionTok{{-}{-}{-}}
\ExtensionTok{264K}\NormalTok{    animals{-}both.jpg}
\ExtensionTok{4.4M}\NormalTok{    animals{-}both{-}compressed{-}0.png}
\ExtensionTok{2.2M}\NormalTok{    animals{-}both{-}compressed{-}9.png}
\ExtensionTok{2.0M}\NormalTok{    animals{-}both{-}non{-}transparent.png}
\ExtensionTok{2.2M}\NormalTok{    animals{-}both{-}transparent.png}
\end{Highlighting}
\end{Shaded}

It appears that the default compression used by \texttt{ImageMagick}
gives a file size that is the same as the highest compression level.
Indeed, the uncompressed version---with a compression level of
zero---gives a file \emph{twice} the size of the uncompressed version
and \emph{sixteen times} the size of the JPEG. And we have not even used
two other related attributes:
\href{https://stackoverflow.com/questions/27267073/imagemagick-lossless-max-compression-for-png}{filter
and strategy} {[}4{]}. Getting the best tradeoff of image format, image
size, file size, loading time, and image quality is still more of an art
to be mastered than an algorithm to be applied.

We may conclude from the above that non-textual, detail-rich images are
better stored and displayed as JPEGs than PNGs.

\hypertarget{results-with-the-text-only-image}{%
\subsubsection{Results with the text-only
image}\label{results-with-the-text-only-image}}

For completeness, let us do a simple \emph{no quality loss} conversion
from PNG to JPEG for the \texttt{text-only} test image, and compare
image appearances and file sizes.

\begin{Shaded}
\begin{Highlighting}[]
\CommentTok{\# Lossless JPEG with a \textquotesingle{}quality\textquotesingle{} of 100}
\ExtensionTok{convert} \AttributeTok{{-}quality}\NormalTok{ 100 text{-}only{-}600{-}dpi{-}cairo.png }\DataTypeTok{\textbackslash{}}
\NormalTok{text{-}only{-}600{-}dpi{-}cairo.jpg}

\CommentTok{\# Composite both images into one with a transparent divider}
\ExtensionTok{convert}\NormalTok{ text{-}only{-}600{-}dpi{-}cairo.png text{-}only{-}600{-}dpi{-}cairo.jpg }\DataTypeTok{\textbackslash{}}
\NormalTok{{-}background transparent }\AttributeTok{{-}splice}\NormalTok{ 20x0+0+0 +append }\AttributeTok{{-}chop}\NormalTok{ 20x0+0+0 }\DataTypeTok{\textbackslash{}}
\NormalTok{text{-}only{-}both{-}600{-}dpi{-}cairo.png}

\FunctionTok{ls} \AttributeTok{{-}Xsh}\NormalTok{ text}\PreprocessorTok{*}\NormalTok{cairo}\PreprocessorTok{*} \KeywordTok{|} \FunctionTok{awk} \StringTok{\textquotesingle{}\{print $1 "\textbackslash{}t" $2\}\textquotesingle{}}
\ExtensionTok{{-}{-}{-}}
\ExtensionTok{148K}\NormalTok{    text{-}only{-}600{-}dpi{-}cairo.jpg}
\ExtensionTok{40K}\NormalTok{     text{-}only{-}600{-}dpi{-}cairo.png}
\ExtensionTok{120K}\NormalTok{    text{-}only{-}both{-}600{-}dpi{-}cairo.png}
\end{Highlighting}
\end{Shaded}

\begin{figure}
\hypertarget{fig:text-only-both}{%
\centering
\includegraphics[width=0.8\textwidth,height=\textheight]{images/text-only-both-600-dpi-cairo.png}
\caption{Composite PNG image of the PNG on the left, and JPEG on the
right.}\label{fig:text-only-both}
}
\end{figure}

\Cref{fig:text-only-both} does not reveal any degradation in quality
after conversion from PNG to JPEG. Note also that the file size of the
\emph{composite} PNG image is smaller than the file size of the
\emph{single} JPEG image.

\hypertarget{png-versus-jpeg}{%
\subsubsection{PNG versus JPEG}\label{png-versus-jpeg}}

We conclude from the \texttt{text-only} images that PNG is better suited
for textual images and provides a smaller file size for the same
quality.

Conversely, we know from the \texttt{animals} images that JPEG is more
suited to non-textual detail and yields good quality images at far
smaller file sizes than PNG.

\hypertarget{can-cairo-and-poppler-do-all-this}{%
\subsubsection{\texorpdfstring{Can \texttt{cairo} and \texttt{poppler}
do all
this?}{Can cairo and poppler do all this?}}\label{can-cairo-and-poppler-do-all-this}}

Can such processing be done using \texttt{cairo} or \texttt{poppler}?
Not really, in the case of the \texttt{animals} image. The
\emph{starting point} or \emph{input format} for \texttt{cairo} and
\texttt{poppler} is the PDF format. Our \texttt{animals} image is
scanned from an illustration and is therefore a JPEG raster image.
\texttt{ImageMagick}'s forte is the display, manipulation, and
processing of raster images; \texttt{cairo} and \texttt{poppler} start
with PDFs and have other goals.

\hypertarget{raster-to-vector-conversions}{%
\subsection{Raster to vector
conversions}\label{raster-to-vector-conversions}}

Let us say that we have a logo, designed and available as a raster
image. To use it on the Web, we could, if necessary, reformat it as a
JPEG or PNG file. But as we zoom into the page, the raster images will
start becoming less sharp and more blocky as shown in \cref{fig:raster}.

However, if the graphic were in SVG format, supported by most web
browsers, the logo would scale without visual degradation as we zoom
into the page.

How do we convert a raster image to a vector format like PDF or SVG?

\hypertarget{raster-to-pdf-with-convert}{%
\subsubsection{\texorpdfstring{Raster to PDF with
\texttt{convert}}{Raster to PDF with convert}}\label{raster-to-pdf-with-convert}}

We could do this using \texttt{ImageMagick}'s \texttt{convert} utility
again. For example,

\begin{Shaded}
\begin{Highlighting}[]
\CommentTok{\# Convert JPEG image to PDF}
\ExtensionTok{convert}\NormalTok{ animals.jpg animals.pdf}

\FunctionTok{ls} \AttributeTok{{-}Xsh}\NormalTok{ animals.jpg animals.pdf }\KeywordTok{|} \FunctionTok{awk} \StringTok{\textquotesingle{}\{print $1 "\textbackslash{}t" $2\}\textquotesingle{}}
\ExtensionTok{{-}{-}{-}}
\ExtensionTok{200K}\NormalTok{    animals.jpg}
\ExtensionTok{204K}\NormalTok{    animals.pdf}
\end{Highlighting}
\end{Shaded}

The converted image, \href{images/animals.pdf}{animals.pdf}, may be
viewed from the given link. Web browsers, while they may feature PDF
viewers on separate tabs, are still unable to display PDFs as part of a
web page. If the converted PDF is magnified by zooming, it will be seen
to reveal remarkable detail. And the difference between the JPEG and PDF
file sizes is negligible.

What happens, though, if the half-sized image is used to generate the
PDF? It is smaller and accordingly embodies less information than the
original.

\begin{Shaded}
\begin{Highlighting}[]
\ExtensionTok{convert}\NormalTok{ animals{-}halfsize.jpg animals{-}halfsize.pdf}

\FunctionTok{ls} \AttributeTok{{-}Xsh}\NormalTok{ animals{-}halfsize.jpg animals{-}halfsize.pdf }\KeywordTok{|} \DataTypeTok{\textbackslash{}}
\FunctionTok{awk} \StringTok{\textquotesingle{}\{print $1 "\textbackslash{}t" $2\}\textquotesingle{}}
\ExtensionTok{{-}{-}{-}}
\ExtensionTok{64K}\NormalTok{     animals{-}halfsize.jpg}
\ExtensionTok{64K}\NormalTok{     animals{-}halfsize.pdf}
\end{Highlighting}
\end{Shaded}

Again, no surprises here. The half-sized PDF tracks the file size of the
half-sized JPEG.

\hypertarget{raster-to-svg-with-convert}{%
\subsubsection{\texorpdfstring{Raster to SVG with
\texttt{convert}}{Raster to SVG with convert}}\label{raster-to-svg-with-convert}}

Let us try to harness \texttt{ImageMagick}'s \texttt{convert} again,
this time to change from JPEG to SVG:

\begin{Shaded}
\begin{Highlighting}[]
\ExtensionTok{convert}\NormalTok{ animals.jpg animals.svg}

\FunctionTok{ls} \AttributeTok{{-}Xsh}\NormalTok{ animals.jpg animals.svg }\KeywordTok{|} \FunctionTok{awk} \StringTok{\textquotesingle{}\{print $1 "\textbackslash{}t" $2\}\textquotesingle{}}
\ExtensionTok{{-}{-}{-}}
\ExtensionTok{200K}\NormalTok{    animals.jpg}
\ExtensionTok{436K}\NormalTok{    animals.svg}
\end{Highlighting}
\end{Shaded}

The SVG file is more than \emph{twice} the size of the original JPEG.
The question arises whether there is an alternative route to the SVG
that could give us smaller file sizes but comparable fidelity. What if
we did not convert from raster to SVG but from raster to PDF and thence
to SVG?

Since PDF to SVG conversion is really part of vector to vector
conversion, we will revisit this question
\protect\hyperlink{pdf-to-svg}{later}.

We will not consider the \texttt{text-only} image here as that image
originated not as a raster but as a PDF in the first place.

\hypertarget{vector-to-raster}{%
\subsection{Vector to raster}\label{vector-to-raster}}

The \texttt{poppler} utilities, with the \texttt{cairo} backend are the
primary resource here. Because the \texttt{text-only} image has been
produced natively in PDF, we will use it as the the example for these
conversions.

\hypertarget{pdf-to-png-and-jpeg-poppler-and-cairo}{%
\subsubsection{\texorpdfstring{PDF to PNG and JPEG: \texttt{poppler} and
\texttt{cairo}}{PDF to PNG and JPEG: poppler and cairo}}\label{pdf-to-png-and-jpeg-poppler-and-cairo}}

It was \protect\hyperlink{text-only-image}{mentioned above} that
\texttt{text-only} was originally generated as a PDF, vector graphics
image and subsequently converted to the PNG and JPEG formats. We explain
how that was done and also why the \texttt{ImageMagick} suite is not
used for this purpose.

The \texttt{poppler} suite contains utilities to convert from PDF to
several raster formats. Two versatile utilities called
\texttt{pdftocairo} and \texttt{pdftoppm} are available for our purpose.
One may view their usage by typing the name of the utility prefixed by
\texttt{man} or suffixed by \texttt{-help}, although the former is more
exhaustive.

\begin{Shaded}
\begin{Highlighting}[]
\ExtensionTok{pdftocairo} \AttributeTok{{-}png} \AttributeTok{{-}r}\NormalTok{ 600 }\AttributeTok{{-}singlefile}\NormalTok{ text{-}only.pdf }\DataTypeTok{\textbackslash{}}
\NormalTok{text{-}only{-}600{-}dpi{-}cairo}

\ExtensionTok{pdftoppm} \AttributeTok{{-}png} \AttributeTok{{-}r}\NormalTok{ 600 }\AttributeTok{{-}singlefile}\NormalTok{ text{-}only.pdf }\DataTypeTok{\textbackslash{}}
\NormalTok{text{-}only{-}600{-}dpi{-}ppm}

\ExtensionTok{pdftocairo} \AttributeTok{{-}jpeg} \AttributeTok{{-}jpegopt} \StringTok{"quality=100"} \AttributeTok{{-}r}\NormalTok{ 600 }\DataTypeTok{\textbackslash{}}
\NormalTok{{-}singlefile text{-}only.pdf text{-}only{-}600{-}dpi{-}cairo}

\ExtensionTok{pdftoppm} \AttributeTok{{-}jpeg} \AttributeTok{{-}jpegopt} \StringTok{"quality=100"} \AttributeTok{{-}r}\NormalTok{ 600 }\DataTypeTok{\textbackslash{}}
\NormalTok{{-}singlefile text{-}only.pdf text{-}only{-}600{-}dpi{-}ppm}

\ExtensionTok{convert} \AttributeTok{{-}units}\NormalTok{ pixelsperinch }\AttributeTok{{-}density}\NormalTok{ 600 }\AttributeTok{{-}quality}\NormalTok{ 100 }\DataTypeTok{\textbackslash{}}
\NormalTok{text{-}only{-}600{-}dpi{-}cairo.png text{-}only{-}600{-}dpi{-}cairo{-}IM.jpg}

\ExtensionTok{convert} \AttributeTok{{-}units}\NormalTok{ pixelsperinch }\AttributeTok{{-}density}\NormalTok{ 600 }\AttributeTok{{-}quality}\NormalTok{ 100 }\DataTypeTok{\textbackslash{}}
\NormalTok{text{-}only{-}600{-}dpi{-}ppm.png text{-}only{-}600{-}dpi{-}ppm{-}IM.jpg}
\end{Highlighting}
\end{Shaded}

The value \texttt{-r\ 600} signifies a resolution of 600 pixels per inch
(PPI), or alternatively, dots per inch (dpi). The default value is 150
PPI. The value of 600 is suitable for printing on laser printers to give
output that will visually rival the original PDF in quality. Note that
while raster images have inherent resolutions, PDF images have none:
they scale without loss of quality.

The \texttt{-singlefile} option is used because we are simply converting
a single ``page'' of PDF rather than a numbered page sequence. In all
cases, the destination filename is the ``root'' of the converted file
sequence, which in this case is the filename without any extension.

In addition, the JPEG version may feature lossy compression where
quality is traded for file size. Since PNG is lossless, to compare the
two formats on an even keel, we specify that the \texttt{-quality} of
the JPEG should be the maximum of 100.

Both \texttt{pdftocairo} and \texttt{pdftoppm} are used in the
conversions above, with appropriately named filenames.

We could also use \texttt{ImageMagick}'s \texttt{convert} to convert
from PNG to JPEG, and this is done in the last two commands above. Note
that this is strictly not a vector to raster conversion but merely
raster to raster. See
\protect\hyperlink{why-is-imagemagick-disallowed-for-pdf-to-raster-1}{below}
for why we cannot convert from PDF to raster with \texttt{convert}.

The files sizes that result are shown below:

\begin{Shaded}
\begin{Highlighting}[]
\FunctionTok{ls} \AttributeTok{{-}Xsh}\NormalTok{ text{-}only}\PreprocessorTok{*}\KeywordTok{|} \FunctionTok{awk} \StringTok{\textquotesingle{}\{print $1 "\textbackslash{}t" $2\}\textquotesingle{}}
\ExtensionTok{{-}{-}{-}}
\ExtensionTok{148K}\NormalTok{    text{-}only{-}600{-}dpi{-}cairo{-}IM.jpg}
\ExtensionTok{120K}\NormalTok{    text{-}only{-}600{-}dpi{-}cairo.jpg}
\ExtensionTok{140K}\NormalTok{    text{-}only{-}600{-}dpi{-}ppm{-}IM.jpg}
\ExtensionTok{112K}\NormalTok{    text{-}only{-}600{-}dpi{-}ppm.jpg}
\ExtensionTok{16K}\NormalTok{     text{-}only.pdf}
\ExtensionTok{40K}\NormalTok{     text{-}only{-}600{-}dpi{-}cairo.png}
\ExtensionTok{40K}\NormalTok{     text{-}only{-}600{-}dpi{-}ppm.png}
\end{Highlighting}
\end{Shaded}

The numbers tell their own story. I would have expected the two sets of
raster images output by \texttt{pdftocairo} and \texttt{pdftoppm} to be
roughly equal in size, given their identical options during invocation.
Strangely, they are not, at least for the JPEGs. This could be either
because of different defaults, or different algorithms, or something
else: I simply do not know.

It appears that \texttt{pdftoppm} gives marginally smaller file sizes
for JPEG than \texttt{pdftocairo}. Moreover, when \texttt{pdftoppm} is
used to convert directly from PDF to JPEG, the file size is smaller than
when PNG is used as an intermediate file format and conversion to JPEG
is by \texttt{ImageMagick}'s \texttt{convert}.

One other takeaway is that text-rich images are better rendered in PNG
than JPEG. The PDF and PNG image file sizes are of the same order of
magnitude, whereas the JPEGS are an order of magnitude larger.

\hypertarget{why-is-imagemagick-disallowed-for-pdf-to-raster}{%
\subsubsection{\texorpdfstring{Why is \texttt{ImageMagick} disallowed
for PDF to
raster?}{Why is ImageMagick disallowed for PDF to raster?}}\label{why-is-imagemagick-disallowed-for-pdf-to-raster}}

If you try to convert a PDF to any raster image format, you will get an
error:

\begin{verbatim}
convert text-only.pdf text-only.png
---
convert: attempt to perform an operation not allowed by the security policy `gs' @ error/delegate.c/ExternalDelegateCommand/378.
convert: no images defined `text-only.png' @ error/convert.c/ConvertImageCommand/3304.
\end{verbatim}

The reason why this is now disallowed is {[}explained in the
appendix{]}{[}Appendix: \texttt{ImageMagick}'s security
vulnerabilities{]}.

\hypertarget{svg-to-png-and-jpeg}{%
\subsubsection{SVG to PNG and JPEG}\label{svg-to-png-and-jpeg}}

At present, most converters from SVG to other formats do not support
fonts as well as PDF does. Bear this deficit in mind as SVG to raster
converters are invoked. Both \texttt{ImageMagick} and \texttt{Inkscape}
are able to convert SVG to PNG. Command line invocations for each and
their resultng file sizes are detailed below. Note that the
\texttt{cairo} backend might be used in many of these utilities.

\%\%\% Clean up below \%\%\%

The \texttt{Inkscape} GUI-based vector graphics editor supports SVG as
its native format and allows export of the generated SVG graphics both
as PDF and as PNG. Let us use the \texttt{text-only} image as an example
and convert it from its original PDF format to an SVG using
\texttt{pdftocairo}:

\begin{Shaded}
\begin{Highlighting}[]
\ExtensionTok{pdftocairo} \AttributeTok{{-}svg}\NormalTok{ text{-}only.pdf text{-}only.svg}

\FunctionTok{ls} \AttributeTok{{-}Xsh}\NormalTok{ text{-}only.}\PreprocessorTok{*}\KeywordTok{|} \FunctionTok{awk} \StringTok{\textquotesingle{}\{print $1 "\textbackslash{}t" $2\}\textquotesingle{}}
\ExtensionTok{{-}{-}{-}}
\ExtensionTok{16K}\NormalTok{     text{-}only.pdf}
\ExtensionTok{12K}\NormalTok{     text{-}only.svg}
\end{Highlighting}
\end{Shaded}

We could open up \texttt{text-only.svg} in \texttt{Inkscape} and save it
as a PDF, \texttt{text-only-inkscape.pdf} completing the round trip. Or
we could use the command line so:

\begin{verbatim}
inkscape text-only.svg -o text-only-inkscape.pdf

ls -Xsh text-only*.{svg,pdf}| awk '{print $1 "\t" $2}'
---
8.0K    text-only-inkscape.pdf
16K     text-only.pdf
12K     text-only.svg
\end{verbatim}

What exactly has been gained or lost in this round trip is a little too
recondite to consider here, and covers issues such as PDF version, font
embedding or its absence, default borders, etc.

More to the point, though, is how do we convert from SVG to PNG on the
command line using inkscape?

What about SVG to PNG? CairoSVG Inkscape SVG to PNG?
\texttt{ImageMagick} SVG to PNG?

The cairosvg module offers 4 functions:

\begin{verbatim}
svg2pdf,
svg2png,
svg2ps, and
svg2svg.
\end{verbatim}

CairoSVG is designed to parse well-formed SVG files, and draw them on a
Cairo surface. Cairo is then able to export them to PDF, PS, PNG, and
even SVG files.

\hypertarget{vector-to-vector}{%
\subsection{Vector to vector}\label{vector-to-vector}}

There are principally two format conversions here:

\begin{itemize}
\tightlist
\item
  PDF to SVG; and
\item
  SVG to PDF.
\end{itemize}

We look at each in turn below.

\hypertarget{pdf-to-svg-pdftocairo-and-pdftoppm}{%
\subsubsection{\texorpdfstring{PDF to SVG: \texttt{pdftocairo} and
\texttt{pdftoppm}}{PDF to SVG: pdftocairo and pdftoppm}}\label{pdf-to-svg-pdftocairo-and-pdftoppm}}

The \texttt{poppler} standalone utilities \texttt{pdftocairo} and
\texttt{pdftoppm} offer a route from PDF to SVG. So, if we had the file
\texttt{animals.pdf} we could convert it to SVG:

\begin{Shaded}
\begin{Highlighting}[]
\CommentTok{\# PDF to SVG using pdftocairo}
\ExtensionTok{pdftocairo} \AttributeTok{{-}svg}\NormalTok{ animals.pdf animals{-}pdftocairo.svg}

\FunctionTok{ls} \AttributeTok{{-}Xsh}\NormalTok{ animals.pdf animals{-}pdftocairo.svg }\KeywordTok{|} \FunctionTok{awk} \StringTok{\textquotesingle{}\{print $1 "\textbackslash{}t" $2\}\textquotesingle{}}
\ExtensionTok{{-}{-}{-}}
\ExtensionTok{204K}\NormalTok{    animals.pdf}
\ExtensionTok{268K}\NormalTok{    animals{-}pdftocairo.svg}
\end{Highlighting}
\end{Shaded}

Two observations are pertinent:

\begin{enumerate}
\item
  It is preferable to convert from a raster image to PDF using
  \texttt{convert} and then convert that PDF to SVG using
  \texttt{pdftocairo} rather than convert directly from a raster to an
  SVG using \texttt{convert} in a single step. The file sizes that
  result are very much smaller with the two-stage conversion.
\item
  \texttt{pdftoppm} is not set up to convert \emph{to} SVG, and hence
  cannot be used.
\end{enumerate}

\hypertarget{pdf2svg}{%
\subsubsection{\texorpdfstring{\texttt{pdf2svg}}{pdf2svg}}\label{pdf2svg}}

There is a utility called
\href{https://github.com/dawbarton/pdf2svg}{\texttt{pdf2svg}} that has
been available for some time now. It may be used to accomplish the same
PDF to SVG conversion as \texttt{pdftocairo}:

\begin{Shaded}
\begin{Highlighting}[]
\CommentTok{\# PDF to SVG using pdf2svg}
\ExtensionTok{pdf2svg}\NormalTok{ animals.pdf animals{-}pdf2svg.svg}

\FunctionTok{ls} \AttributeTok{{-}Xsh}\NormalTok{ animals.pdf animals{-}pdf2svg.svg }\KeywordTok{|} \FunctionTok{awk} \StringTok{\textquotesingle{}\{print $1 "\textbackslash{}t" $2\}\textquotesingle{}}
\ExtensionTok{{-}{-}{-}}
\ExtensionTok{204K}\NormalTok{    animals.pdf}
\ExtensionTok{268K}\NormalTok{    animals{-}pdf2svg.svg}
\end{Highlighting}
\end{Shaded}

The file sizes are identical to those from the \texttt{pdftocairo}
conversion. Drew Barton, the author of \texttt{pdf2svg},
\href{https://cityinthesky.co.uk/opensource/pdf2svg/}{has written}:

\begin{quote}
Note: since this utility was written, the maintainers of Poppler have
written a utility that works on the same principle: \texttt{pdftocairo}.
I recommend that you use their utility since it is better maintained
than mine.
\end{quote}

So, it appears that \texttt{pdftocairo} is sufficient for converting
from PDF to SVG.

\texttt{rsvg-convert} \texttt{pdftocairo}

https://en.wikipedia.org/wiki/Cairo\_(graphics)

https://www.cairographics.org/

https://cgit.freedesktop.org/cairo

\hypertarget{inkscape}{%
\subsubsection{Inkscape}\label{inkscape}}

https://wiki.inkscape.org/wiki/index.php/Tools

\hypertarget{pdf2svg-1}{%
\subsection{pdf2svg}\label{pdf2svg-1}}

https://cityinthesky.co.uk/opensource/pdf2svg/

https://github.com/dawbarton/pdf2svg

https://inkscape.org/develop/about-svg/

\hypertarget{poppler}{%
\subsection{poppler}\label{poppler}}

\texttt{pdftoppm}

pdftocairo

\hypertarget{raster-to-raster}{%
\subsection{Raster to raster}\label{raster-to-raster}}

\hypertarget{image-to-pdf}{%
\subsection{Image to PDF}\label{image-to-pdf}}

Still works. No strictures. But the PDF can get grungy. Use a pyramid of
resolutions.

\hypertarget{avoiding-blurry-pdfs}{%
\subsubsection{Avoiding blurry PDFs}\label{avoiding-blurry-pdfs}}

-units pixelsperinch -density 1200 etc., in conversion

Useful when a hgh resolution image is available. In any case: PDF and
png/jpg sizes are similar.

\hypertarget{choosing-the-optimal-image-resolution-for-a-clear-pdf}{%
\subsection{Choosing the optimal image resolution for a clear
PDF}\label{choosing-the-optimal-image-resolution-for-a-clear-pdf}}

96dpi for screen 150 dpi default 300 dpi for print {[}give references{]}

\begin{Shaded}
\begin{Highlighting}[]
\ExtensionTok{convert} \AttributeTok{{-}units}\NormalTok{ pixelsperinch }\AttributeTok{{-}density}\NormalTok{ 300 file.png file.pdf}
\end{Highlighting}
\end{Shaded}

\hypertarget{pdf-to-image-not-supported}{%
\subsection{PDF to image not
supported}\label{pdf-to-image-not-supported}}

\begin{Shaded}
\begin{Highlighting}[]
\CommentTok{\#! /bin/magick}
\ExtensionTok{convert}\NormalTok{ file.pdf file.png}
\end{Highlighting}
\end{Shaded}

\begin{verbatim}
convert test.pdf test.png
---
convert: unable to open image 'test.pdf': No such file or directory @ error/blob.c/OpenBlob/3537.
convert: no images defined `test.png' @ error/convert.c/ConvertImageCommand/3304.
\end{verbatim}

\hypertarget{security-considerations}{%
\subsubsection{Security considerations}\label{security-considerations}}

\texttt{ImageMagick} is no more the famed Swiss army knife for
conversions from PDFs to images.

Give references to security concerns.

\hypertarget{enter-poppler}{%
\subsection{Enter poppler}\label{enter-poppler}}

\hypertarget{vector-to-raster-1}{%
\subsection{Vector to raster}\label{vector-to-raster-1}}

It was \protect\hyperlink{text-only-image}{mentioned above} that
\texttt{text-only} was originally generated as a PDF, vector graphics
image and subsequently converted to the PNG and JPEG formats. We explain
how that was done and also why the \texttt{ImageMagick} suite is not
used for this purpose.

\hypertarget{poppler-and-cairo}{%
\subsubsection{\texorpdfstring{\texttt{poppler} and
\texttt{cairo}}{poppler and cairo}}\label{poppler-and-cairo}}

The \texttt{poppler} suite contains utilities to convert from PDF to
several raster formats. Two versatile utilities called
\texttt{pdftocairo} and \texttt{pdftoppm} are available for our purpose.
One may view their usage by typing the name of the utility prefixed by
\texttt{man} or suffixed by \texttt{-help}, although the former is more
exhaustive.

\begin{Shaded}
\begin{Highlighting}[]
\ExtensionTok{pdftocairo} \AttributeTok{{-}png} \AttributeTok{{-}r}\NormalTok{ 600 }\AttributeTok{{-}singlefile}\NormalTok{ text{-}only.pdf }\DataTypeTok{\textbackslash{}}
\NormalTok{text{-}only{-}600{-}dpi{-}cairo}

\ExtensionTok{pdftoppm} \AttributeTok{{-}png} \AttributeTok{{-}r}\NormalTok{ 600 }\AttributeTok{{-}singlefile}\NormalTok{ text{-}only.pdf }\DataTypeTok{\textbackslash{}}
\NormalTok{text{-}only{-}600{-}dpi{-}ppm}

\ExtensionTok{pdftocairo} \AttributeTok{{-}jpeg} \AttributeTok{{-}jpegopt} \StringTok{"quality=100"} \AttributeTok{{-}r}\NormalTok{ 600 }\DataTypeTok{\textbackslash{}}
\NormalTok{{-}singlefile text{-}only.pdf text{-}only{-}600{-}dpi{-}cairo}

\ExtensionTok{pdftoppm} \AttributeTok{{-}jpeg} \AttributeTok{{-}jpegopt} \StringTok{"quality=100"} \AttributeTok{{-}r}\NormalTok{ 600 }\DataTypeTok{\textbackslash{}}
\NormalTok{{-}singlefile text{-}only.pdf text{-}only{-}600{-}dpi{-}ppm}

\ExtensionTok{convert} \AttributeTok{{-}units}\NormalTok{ pixelsperinch }\AttributeTok{{-}density}\NormalTok{ 600 }\AttributeTok{{-}quality}\NormalTok{ 100 }\DataTypeTok{\textbackslash{}}
\NormalTok{text{-}only{-}600{-}dpi{-}cairo.png text{-}only{-}600{-}dpi{-}cairo{-}IM.jpg}

\ExtensionTok{convert} \AttributeTok{{-}units}\NormalTok{ pixelsperinch }\AttributeTok{{-}density}\NormalTok{ 600 }\AttributeTok{{-}quality}\NormalTok{ 100 }\DataTypeTok{\textbackslash{}}
\NormalTok{text{-}only{-}600{-}dpi{-}ppm.png text{-}only{-}600{-}dpi{-}ppm{-}IM.jpg}
\end{Highlighting}
\end{Shaded}

The value \texttt{-r\ 600} signifies a resolution of 600 pixels per inch
(PPI). The default value is 150 PPI. The value of 600 is suitable for
printing on laser printers to give output that will visually rival the
original PDF in quality. Note that while raster images have inherent
resolutions, PDF images have none: they scale without loss of quality.

The \texttt{-singlefile} option is used because we are simply converting
a single ``page'' of PDF rather than a numbered page sequence. In all
cases, the destination filename is the ``root'' of the converted file
sequence, which in this case is the filename without any extension.

In addition, the JPEG version may feature lossy compression where
quality is traded for file size. Since PNG is lossless, to compare the
two formats on an even keel, we specify that the \texttt{-quality} of
the JPEG should be the maximum of 100.

Both \texttt{pdftocairo} and \texttt{pdftoppm} are used in the
conversions above, with appropriately named filenames. One could also
use \texttt{ImageMagick}'s \texttt{convert} to convert from PNG to JPEG,
and this is done in the last two commands.

The files sizes that result are shown below:

\begin{Shaded}
\begin{Highlighting}[]
\FunctionTok{ls} \AttributeTok{{-}Xsh}\NormalTok{ text{-}only}\PreprocessorTok{*}\KeywordTok{|} \FunctionTok{awk} \StringTok{\textquotesingle{}\{print $1 "\textbackslash{}t" $2\}\textquotesingle{}}
\ExtensionTok{{-}{-}{-}}
\ExtensionTok{148K}\NormalTok{    text{-}only{-}600{-}dpi{-}cairo{-}IM.jpg}
\ExtensionTok{120K}\NormalTok{    text{-}only{-}600{-}dpi{-}cairo.jpg}
\ExtensionTok{140K}\NormalTok{    text{-}only{-}600{-}dpi{-}ppm{-}IM.jpg}
\ExtensionTok{112K}\NormalTok{    text{-}only{-}600{-}dpi{-}ppm.jpg}
\ExtensionTok{16K}\NormalTok{     text{-}only.pdf}
\ExtensionTok{40K}\NormalTok{     text{-}only{-}600{-}dpi{-}cairo.png}
\ExtensionTok{40K}\NormalTok{     text{-}only{-}600{-}dpi{-}ppm.png}
\end{Highlighting}
\end{Shaded}

The numbers tell their own story. I would have expected the two sets of
raster images output by \texttt{pdftocairo} and \texttt{pdftoppm} to be
roughly equal in size, given their identical options during invocation.
Strangely, they are not, at least for the JPEGs. This could be either
because of different defaults, or different algorithms, or something
else: I simply do not know.

It appears that \texttt{pdftoppm} gives marginally smaller file sizes
for JPEG than \texttt{pdftocairo}. Moreover, when \texttt{pdftoppm} is
used to convert directly from PDF to JPEG, the file size is smaller than
when PNG is used as an intermediate file format and conversion to JPEG
is by \texttt{ImageMagick}'s \texttt{convert}.

One other takeaway is that text-rich images are better rendered in PNG
than JPEG. The PDF and PNG image file sizes are of the same order of
magnitude, whereas the JPEGS are an order of magnitude larger.

\hypertarget{why-is-imagemagick-disallowed-for-pdf-to-raster-1}{%
\subsubsection{\texorpdfstring{Why is \texttt{ImageMagick} disallowed
for PDF to
raster?}{Why is ImageMagick disallowed for PDF to raster?}}\label{why-is-imagemagick-disallowed-for-pdf-to-raster-1}}

If you try to convert a PDF to any raster image format, you will get an
error:

\begin{verbatim}
convert text-only.pdf text-only.png
---
convert: attempt to perform an operation not allowed by the security policy `gs' @ error/delegate.c/ExternalDelegateCommand/378.
convert: no images defined `text-only.png' @ error/convert.c/ConvertImageCommand/3304.
\end{verbatim}

The reason why this is now disallowed is {[}explained in the
appendix{]}{[}Appendix: \texttt{ImageMagick}'s security
vulnerabilities{]}.

\hypertarget{svg-to-raster}{%
\subsubsection{SVG to raster}\label{svg-to-raster}}

What about SVG to PNG? CairoSVG Inkscape SVG to PNG?
\texttt{ImageMagick} SVG to PNG?

The cairosvg module offers 4 functions:

\begin{verbatim}
svg2pdf,
svg2png,
svg2ps, and
svg2svg.
\end{verbatim}

CairoSVG is designed to parse well-formed SVG files, and draw them on a
Cairo surface. Cairo is then able to export them to PDF, PS, PNG, and
even SVG files.

\hypertarget{vector-to-vector-1}{%
\subsection{Vector to vector}\label{vector-to-vector-1}}

There are basically two possibilities:

\begin{enumerate}
\def\labelenumi{\alph{enumi}.}
\tightlist
\item
  PDF to SVG; and
\item
  SVG to PDF.
\end{enumerate}

Both are possible with the \texttt{cairo} library and \texttt{poppler}
suite as well as other libraries and utilities.

\hypertarget{pdf-to-svg}{%
\subsubsection{PDF to SVG}\label{pdf-to-svg}}

pdftocairo -svg pdftoppm -svg? pdf2svg Any others?

\hypertarget{svg-to-pdf}{%
\subsubsection{SVG to PDF}\label{svg-to-pdf}}

CairoSVG https://cairosvg.org/: single executabe cairosvg from
python-cairosvg: safety rsvg-convert from librsvg Inkscape
\texttt{ImageMagick} any others?

https://wiki.gnome.org/Projects/LibRsvg

\begin{Shaded}
\begin{Highlighting}[]
\ExtensionTok{pdftoppm} \AttributeTok{{-}png}\NormalTok{ ernst{-}heackel{-}medium.pdf ernst{-}heackel{-}medium.png}
\ExtensionTok{convert}\NormalTok{ ernst{-}heackel{-}medium.jpg ernst{-}heackel{-}medium{-}direct.png}
\ExtensionTok{convert}\NormalTok{ ernst{-}heackel{-}medium.jpg ernst{-}heackel{-}medium{-}direct.png}
\end{Highlighting}
\end{Shaded}

How to use resize etc.

\hypertarget{summary}{%
\subsection{Summary}\label{summary}}

\texttt{ImageMagick}'s \texttt{convert} is the tool of choice for
converting from any raster format to another raster format or to PDF or
SVG.

When we start out with PDF as the source image format, and the
destination format is either a raster format or SVG, the tool of choice
is \texttt{pdftocairo} from the \texttt{poppler} utilities.

When the source format is SVG and the destination format is either PDF
or a raster format, the tool of choice is \texttt{cairosvg}.

*\textbf{tbl:formats?} summarizes this information, which is current at
the time of writing, but could change as image utilities landscape
changes with time.

\begin{longtable}[]{@{}ll@{}}
\caption{Tools for image format conversions.
\{\#tbl:formats\}}\tabularnewline
\toprule
Conversion Type & Tool \\
\midrule
\endfirsthead
\toprule
Conversion Type & Tool \\
\midrule
\endhead
raster to raster & \texttt{convert} \\
raster to PDF & \texttt{convert} \\
raster to SVG & \texttt{convert} \\
PDF to raster & \texttt{pdftocairo} \\
SVG to raster & \texttt{cairosvg} \\
PDF to SVG & \texttt{pdftocairo} \\
SVG to PDF & \texttt{cairosvg} \\
\bottomrule
\end{longtable}

\hypertarget{appendix-security-vulnerabilities-in-imagemagick}{%
\subsubsection{\texorpdfstring{Appendix: Security vulnerabilities in
\texttt{ImageMagick}}{Appendix: Security vulnerabilities in ImageMagick}}\label{appendix-security-vulnerabilities-in-imagemagick}}

Great power exacts a commensurate price. \texttt{ImageMagick}'s great
power and ease of use does come at a great price: vulnerability to
exploits by malicious remote actors.

\texttt{ImageMagick} uses external libraries or \emph{backend tools}
which are called via \texttt{system()} commands in accordance with
\emph{delegated} command strings specified in a configuration file
called \texttt{policy.xml}.

In April 2016, it was reported that because of insufficient validation
of delegated command strings, it was possible for someone to execute
malicious code remotely, to the detriment of the unwitting user of
\texttt{ImageMagick}. This was revealed at a website, interestingly
named \href{https://imagetragick.com/}{ImageTragick} to attract
sufficient attention and remedial action to the discovered bug {[}5{]}.

In November 2020,
\href{https://portswigger.net/daily-swig/imagemagick-pdf-parsing-flaw-allowed-attacker-to-execute-shell-commands-via-maliciously-crafted-image}{another
security vulnerability was discovered} {[}6{]}. It was
\href{https://insert-script.blogspot.com/2020/11/imagemagick-shell-injection-via-pdf.html}{reported
and promptly patched} by the \texttt{ImageMagick} maintainers {[}7{]}.

Recent versions of the \texttt{ImageMagick} suite, bundled with major
distributions, should have correctly configured \texttt{policy.xml}
files that will block known exploits.
\href{https://www.techopedia.com/definition/25266/sandboxing}{Sandboxing}
is another technique to quarantine the system from possible
vulnerabilities. Above all, it is vital to keep system and application
software up to date to avail of evolutions in performance and security.

\hypertarget{feedback}{%
\paragraph{Feedback}\label{feedback}}

Please \href{mailto:feedback.swanlotus@gmail.com}{email me} your
comments and corrections.

\noindent A PDF version of this article is
\href{./image-format-conversions.pdf}{available for download here.}

\hypertarget{bibliography}{%
\section*{References}\label{bibliography}}
\addcontentsline{toc}{section}{References}

\hypertarget{refs}{}
\begin{CSLReferences}{0}{0}
\leavevmode\hypertarget{ref-imagemagicksource}{}%
\CSLLeftMargin{{[}1{]} }
\CSLRightInline{ImageMagick Studio LLC, {`ImageMagick 7.'} {[}Online{]}.
Available: \url{https://github.com/ImageMagick/ImageMagick}.
{[}Accessed: 08-Mar-2021{]}}

\leavevmode\hypertarget{ref-imcli}{}%
\CSLLeftMargin{{[}2{]} }
\CSLRightInline{{`{ImageMagick---Command-line Processing}. {Anatomy of
the Command-line}.'} {[}Online{]}. Available:
\url{https://imagemagick.org/script/command-line-processing.php}.
{[}Accessed: 12-Mar-2021{]}}

\leavevmode\hypertarget{ref-newton2015}{}%
\CSLLeftMargin{{[}3{]} }
\CSLRightInline{D. Newton, {`{Efficient Image Resizing With
ImageMagick---Smashing Magazine},'} 25-Jun-2015. {[}Online{]}.
Available:
\url{https://www.smashingmagazine.com/2015/06/efficient-image-resizing-with-imagemagick/}.
{[}Accessed: 11-Mar-2021{]}}

\leavevmode\hypertarget{ref-setchell2014}{}%
\CSLLeftMargin{{[}4{]} }
\CSLRightInline{M. Setchell, {`{ImageMagick: Lossless max compression
for PNG?---Stack Overflow},'} 03-Dec-2014. {[}Online{]}. Available:
\url{https://stackoverflow.com/questions/27267073/imagemagick-lossless-max-compression-for-png/27269260\#27269260}.
{[}Accessed: 12-Mar-2021{]}}

\leavevmode\hypertarget{ref-imagetragick2016}{}%
\CSLLeftMargin{{[}5{]} }
\CSLRightInline{---, {`ImageMagick is on fire --- CVE-2016--3714,'}
12-May-2016. {[}Online{]}. Available: \url{https://imagetragick.com/}.
{[}Accessed: 08-Mar-2021{]}}

\leavevmode\hypertarget{ref-leyden2020}{}%
\CSLLeftMargin{{[}6{]} }
\CSLRightInline{J. Leyden, {`ImageMagick PDF-parsing flaw allowed
attacker to execute shell commands via maliciously crafted image,'}
23-Nov-2020. {[}Online{]}. Available:
\url{https://portswigger.net/daily-swig/imagemagick-pdf-parsing-flaw-allowed-attacker-to-execute-shell-commands-via-maliciously-crafted-image}.
{[}Accessed: 08-Mar-2021{]}}

\leavevmode\hypertarget{ref-infuhr2020}{}%
\CSLLeftMargin{{[}7{]} }
\CSLRightInline{A. Inführ, {`{ImageMagick - Shell injection via PDF
password},'} 21-Nov-2020. {[}Online{]}. Available:
\url{https://insert-script.blogspot.com/2020/11/imagemagick-shell-injection-via-pdf.html}.
{[}Accessed: 08-Mar-2021{]}}

\end{CSLReferences}



\end{document}
