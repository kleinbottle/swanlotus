% Options for packages loaded elsewhere
\PassOptionsToPackage{unicode,linktoc=all}{hyperref}
\PassOptionsToPackage{hyphens}{url}
\PassOptionsToPackage{dvipsnames,svgnames,x11names}{xcolor}
%
\documentclass[
  a4paper,
]{article}
\usepackage{amsmath,amssymb}
\usepackage{iftex}
\ifPDFTeX
  \usepackage[T1]{fontenc}
  \usepackage[utf8]{inputenc}
  \usepackage{textcomp} % provide euro and other symbols
\else % if luatex or xetex
  \usepackage{unicode-math} % this also loads fontspec
  \defaultfontfeatures{Scale=MatchLowercase}
  \defaultfontfeatures[\rmfamily]{Ligatures=TeX,Scale=1}
\fi
\usepackage{lmodern}
\ifPDFTeX\else
  % xetex/luatex font selection
\fi
% Use upquote if available, for straight quotes in verbatim environments
\IfFileExists{upquote.sty}{\usepackage{upquote}}{}
\IfFileExists{microtype.sty}{% use microtype if available
  \usepackage[]{microtype}
  \UseMicrotypeSet[protrusion]{basicmath} % disable protrusion for tt fonts
}{}
\makeatletter
\@ifundefined{KOMAClassName}{% if non-KOMA class
  \IfFileExists{parskip.sty}{%
    \usepackage{parskip}
  }{% else
    \setlength{\parindent}{0pt}
    \setlength{\parskip}{6pt plus 2pt minus 1pt}}
}{% if KOMA class
  \KOMAoptions{parskip=half}}
\makeatother
\usepackage{xcolor}
\usepackage[margin=25mm]{geometry}
\usepackage{longtable,booktabs,array}
\usepackage{calc} % for calculating minipage widths
% Correct order of tables after \paragraph or \subparagraph
\usepackage{etoolbox}
\makeatletter
\patchcmd\longtable{\par}{\if@noskipsec\mbox{}\fi\par}{}{}
\makeatother
% Allow footnotes in longtable head/foot
\IfFileExists{footnotehyper.sty}{\usepackage{footnotehyper}}{\usepackage{footnote}}
\makesavenoteenv{longtable}
\usepackage{graphicx}
\makeatletter
\def\maxwidth{\ifdim\Gin@nat@width>\linewidth\linewidth\else\Gin@nat@width\fi}
\def\maxheight{\ifdim\Gin@nat@height>\textheight\textheight\else\Gin@nat@height\fi}
\makeatother
% Scale images if necessary, so that they will not overflow the page
% margins by default, and it is still possible to overwrite the defaults
% using explicit options in \includegraphics[width, height, ...]{}
\setkeys{Gin}{width=\maxwidth,height=\maxheight,keepaspectratio}
% Set default figure placement to htbp
\makeatletter
\def\fps@figure{htbp}
\makeatother
\usepackage{svg}
\setlength{\emergencystretch}{3em} % prevent overfull lines
\providecommand{\tightlist}{%
  \setlength{\itemsep}{0pt}\setlength{\parskip}{0pt}}
\setcounter{secnumdepth}{-\maxdimen} % remove section numbering
\ifLuaTeX
\usepackage[bidi=basic]{babel}
\else
\usepackage[bidi=default]{babel}
\fi
\babelprovide[main,import]{british}
% get rid of language-specific shorthands (see #6817):
\let\LanguageShortHands\languageshorthands
\def\languageshorthands#1{}
% $HOME/.pandoc/defaults/latex-header-includes.tex
% Common header includes for both lualatex and xelatex engines.
%
% Preliminaries
%
% \PassOptionsToPackage{rgb,dvipsnames,svgnames}{xcolor}
% \PassOptionsToPackage{main=british}{babel}
\PassOptionsToPackage{english}{selnolig}
\AtBeginEnvironment{quote}{\small}
\AtBeginEnvironment{quotation}{\small}
\AtBeginEnvironment{longtable}{\centering}
%
% Packages that are useful to include
%
\usepackage{graphicx}
\usepackage{subcaption}
\usepackage[inkscapeversion=1]{svg}
\usepackage[defaultlines=4,all]{nowidow}
\usepackage{etoolbox}
\usepackage{fontsize}
\usepackage{newunicodechar}
\usepackage{pdflscape}
\usepackage{fnpct}
\usepackage{parskip}
  \setlength{\parindent}{0pt}
\usepackage[style=american]{csquotes}
% \usepackage{setspace} Use the <fontname-plus.tex> files for setspace
%
\usepackage{hyperref} % cleveref must come AFTER hyperref
\usepackage[capitalize,noabbrev]{cleveref} % Must come after hyperref
\let\longdivision\relax
\usepackage{longdivision}
% noto-plus.tex
% Font-setting header file for use with Pandoc Markdown
% to generate PDF via LuaLaTeX.
% The main font is Noto Serif.
% Other main fonts are also available in appropriately named file.
\usepackage{fontspec}
\usepackage{setspace}
\setstretch{1.3}
%
\defaultfontfeatures{Ligatures=TeX,Scale=MatchLowercase,Renderer=Node} % at the start always
%
% For English
% See also https://tex.stackexchange.com/questions/574047/lualatex-amsthm-polyglossia-charissil-error
% We use Node as Renderer for the Latin Font and Greek Font and HarfBuzz as renderer ofr Indic fonts.
%
\babelfont{rm}[Script=Latin,Scale=1]{NotoSerif}% Config is at $HOME/texmf/tex/latex/NotoSerif.fontspec
\babelfont{sf}[Script=Latin]{SourceSansPro}% Config is at $HOME/texmf/tex/latex/SourceSansPro.fontspec
\babelfont{tt}[Script=Latin]{FiraMono}% Config is at $HOME/texmf/tex/latex/FiraMono.fontspec
%
% Sanskrit, Tamil, and Greek fonts
%
\babelprovide[import, onchar=ids fonts]{sanskrit}
\babelprovide[import, onchar=ids fonts]{tamil}
\babelprovide[import, onchar=ids fonts]{greek}
%
\babelfont[sanskrit]{rm}[Scale=1.1,Renderer=HarfBuzz,Script=Devanagari]{NotoSerifDevanagari}
\babelfont[sanskrit]{sf}[Scale=1.1,Renderer=HarfBuzz,Script=Devanagari]{NotoSansDevanagari}
\babelfont[tamil]{rm}[Renderer=HarfBuzz,Script=Tamil]{NotoSerifTamil}
\babelfont[tamil]{sf}[Renderer=HarfBuzz,Script=Tamil]{NotoSansTamil}
\babelfont[greek]{rm}[Script=Greek]{GentiumBookPlus}
%
% Math font
%
\usepackage{unicode-math} % seems not to hurt % fallabck
\setmathfont[bold-style=TeX]{STIX Two Math}
\usepackage{amsmath}
\usepackage{esdiff} % for derivative symbols
% \renewcommand{\mathbf}{\symbf}
%
%
% Other fonts
%
\newfontfamily{\emojifont}{Symbola}
%

\usepackage{titling}
\usepackage{fancyhdr}
    \pagestyle{fancy}
    \fancyhead{}
    \fancyfoot{}
    \renewcommand{\headrulewidth}{0.2pt}
    \renewcommand{\footrulewidth}{0.2pt}
    \fancyhead[LO,RE]{\scshape\thetitle}
    \fancyfoot[CO,CE]{\footnotesize Copyright © 2006\textendash\the\year, R (Chandra) Chandrasekhar}
    \fancyfoot[RE,RO]{\thepage}
%
\usepackage{newunicodechar}
\newunicodechar{√}{\textsf{√}}
\ifLuaTeX
  \usepackage{selnolig}  % disable illegal ligatures
\fi
\usepackage{bookmark}
\IfFileExists{xurl.sty}{\usepackage{xurl}}{} % add URL line breaks if available
\urlstyle{sf}
\hypersetup{
  pdftitle={The Wonder That Is Pi},
  pdfauthor={R (Chandra) Chandrasekhar},
  pdflang={en-GB},
  colorlinks=true,
  linkcolor={DarkOliveGreen},
  filecolor={Purple},
  citecolor={DarkKhaki},
  urlcolor={Maroon},
  pdfcreator={LaTeX via pandoc}}

\title{The Wonder That Is Pi}
\author{R (Chandra) Chandrasekhar}
\date{2004-01-14 | 2024-06-29}

\begin{document}
\maketitle

\thispagestyle{empty}


\begin{quote}
This blog began life more than two decades ago, as part of a series of
lectures I delivered to very bright first-year engineering students at
an Australian university.

The number \(\pi\) (pronounced ``pie'') has been recognized from time
immemorial because its physical significance can be grasped easily: it
is the ratio of the circumference of a circle to its diameter. But who
would have thought that such an innocent ratio would exercise such
endless fascination because of the complexities enfolded into it?

Not surprisingly, some students I met recently wanted to know more about
\(\pi\). Accordingly, I have refreshed and revised my original
presentation to better accord with the form and substance of a blog. The
online references have also been updated to keep up with a rapidly
changing Web.

If there are any errors or omissions, please
\href{mailto:feedback.swanlotus@gmail.com}{email} me your feedback.
\end{quote}

\subsection{Circumference, diameter and
π}\label{circumference-diameter-and-ux3c0}

The straight line or
\href{https://mathworld.wolfram.com/Geodesic.html}{geodesic} is the
shortest distance between any two points on a plane, sphere, or other
space. The circle is the
\href{https://en.wikipedia.org/wiki/Locus_(mathematics)}{locus}
traversed by a moving point that is
\href{https://en.wikipedia.org/wiki/Equidistant}{equidistant} from
another fixed point on a two-dimensional plane. It is the most
\href{https://mathworld.wolfram.com/Symmetry.html}{symmetrical} figure
on the plane. The
\href{https://en.wikipedia.org/wiki/Diameter}{diameter} is the name
given both to any straight line passing through the centre of the
circle---intersecting it at two points---as well as to its length. When
we divide the \href{https://en.wikipedia.org/wiki/Perimeter}{perimeter}
of circle, more properly called its
\href{https://en.wikipedia.org/wiki/Circumference}{circumference},
\(C\), by its diameter, \(d\), we get the enigmatic constant \(\pi\),
which has a value between \(3.141\) and \(3.142\):
\begin{equation}\phantomsection\label{eq:pi-Cd}{
\frac{C}{d} = \pi.
}\end{equation} The diameter \(d\) is twice the radius \(r\), and
substituting for \(d\) into \cref{eq:pi-Cd}, we get the well-known
school formula: \begin{equation}\phantomsection\label{eq:two-pi-r}{
C = \pi d = 2\pi r \approx 2\left[\frac{22}{7}\right]r \approx 6.28r.
}\end{equation} Note, however, that \(\pi\) is \emph{not exactly equal}
to \(\frac{22}{7}\). This value is a convenient \emph{rational fraction
approximation} for \(\pi\) that serves well in elementary
contexts.\footnote{See
  \href{https://swanlotus.netlify.app/blogs/a-tale-of-two-measures-degrees-and-radians}{``A
  tale of two measures: degrees and radians''}.}

You might reasonably wonder whether the ratio of the circumference to
the diameter of \emph{any} circle is \emph{always} \(\pi\). The answer
is ``Yes'', because \emph{all circles are similar}. The ratios of
corresponding lengths of similar figures are equal. This idea is also
covered in my blog
\href{https://swanlotus.netlify.app/blogs/a-tale-of-two-measures-degrees-and-radians}{``A
tale of two measures: degrees and radians''}.

\begin{figure}
\centering
\includesvg[width=0.7\textwidth,height=\textheight]{images/C-over-d.svg}
\caption{The ratio of the circumference to the diameter of \emph{any}
circle is \(\pi\).}\label{fig:pi-circle}
\end{figure}

\cref{fig:pi-circle} shows the relationships in \cref{eq:pi-Cd} and
\cref{eq:two-pi-r} pictorially. The circumference of a circle is about
6.28 times its radius. Why this should be so is a mystery of Nature.

\subsection{Acknowledgements}\label{acknowledgements}

\subsection{Feedback}\label{feedback}

Please \href{mailto:feedback.swanlotus@gmail.com}{email me} your
comments and corrections.

\noindent A PDF version of this article is
\href{./the-wonder-that-is-pi.pdf}{available for download here}:

\begin{small}

\begin{sffamily}

\url{https://swanlotus.netlify.app/blogs/the-wonder-that-is-pi.pdf}

\end{sffamily}

\end{small}



\end{document}
