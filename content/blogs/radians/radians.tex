% Options for packages loaded elsewhere
\PassOptionsToPackage{unicode,linktoc=all}{hyperref}
\PassOptionsToPackage{hyphens}{url}
\PassOptionsToPackage{dvipsnames,svgnames,x11names}{xcolor}
%
\documentclass[
  a4paper,
]{article}
\usepackage{amsmath,amssymb}
\usepackage{iftex}
\ifPDFTeX
  \usepackage[T1]{fontenc}
  \usepackage[utf8]{inputenc}
  \usepackage{textcomp} % provide euro and other symbols
\else % if luatex or xetex
  \usepackage{unicode-math} % this also loads fontspec
  \defaultfontfeatures{Scale=MatchLowercase}
  \defaultfontfeatures[\rmfamily]{Ligatures=TeX,Scale=1}
\fi
\usepackage{lmodern}
\ifPDFTeX\else
  % xetex/luatex font selection
\fi
% Use upquote if available, for straight quotes in verbatim environments
\IfFileExists{upquote.sty}{\usepackage{upquote}}{}
\IfFileExists{microtype.sty}{% use microtype if available
  \usepackage[]{microtype}
  \UseMicrotypeSet[protrusion]{basicmath} % disable protrusion for tt fonts
}{}
\makeatletter
\@ifundefined{KOMAClassName}{% if non-KOMA class
  \IfFileExists{parskip.sty}{%
    \usepackage{parskip}
  }{% else
    \setlength{\parindent}{0pt}
    \setlength{\parskip}{6pt plus 2pt minus 1pt}}
}{% if KOMA class
  \KOMAoptions{parskip=half}}
\makeatother
\usepackage{xcolor}
\usepackage[margin=25mm]{geometry}
\usepackage{longtable,booktabs,array}
\usepackage{calc} % for calculating minipage widths
% Correct order of tables after \paragraph or \subparagraph
\usepackage{etoolbox}
\makeatletter
\patchcmd\longtable{\par}{\if@noskipsec\mbox{}\fi\par}{}{}
\makeatother
% Allow footnotes in longtable head/foot
\IfFileExists{footnotehyper.sty}{\usepackage{footnotehyper}}{\usepackage{footnote}}
\makesavenoteenv{longtable}
\setlength{\emergencystretch}{3em} % prevent overfull lines
\providecommand{\tightlist}{%
  \setlength{\itemsep}{0pt}\setlength{\parskip}{0pt}}
\setcounter{secnumdepth}{-\maxdimen} % remove section numbering
\ifLuaTeX
\usepackage[bidi=basic]{babel}
\else
\usepackage[bidi=default]{babel}
\fi
\babelprovide[main,import]{british}
% get rid of language-specific shorthands (see #6817):
\let\LanguageShortHands\languageshorthands
\def\languageshorthands#1{}
% $HOME/.pandoc/defaults/latex-header-includes.tex
% Common header includes for both lualatex and xelatex engines.
%
% Preliminaries
%
% \PassOptionsToPackage{rgb,dvipsnames,svgnames}{xcolor}
% \PassOptionsToPackage{main=british}{babel}
\PassOptionsToPackage{english}{selnolig}
\AtBeginEnvironment{quote}{\small}
\AtBeginEnvironment{quotation}{\small}
\AtBeginEnvironment{longtable}{\centering}
%
% Packages that are useful to include
%
\usepackage{graphicx}
\usepackage{subcaption}
\usepackage[inkscapeversion=1]{svg}
\usepackage[defaultlines=4,all]{nowidow}
\usepackage{etoolbox}
\usepackage{fontsize}
\usepackage{newunicodechar}
\usepackage{pdflscape}
\usepackage{fnpct}
\usepackage{parskip}
  \setlength{\parindent}{0pt}
\usepackage[style=american]{csquotes}
% \usepackage{setspace} Use the <fontname-plus.tex> files for setspace
%
\usepackage{esdiff} % for derivative symbols
\usepackage{amsmath}
\usepackage{hyperref} % cleveref must come AFTER hyperref
\usepackage[capitalize,noabbrev]{cleveref} % Must come after hyperref
% noto-plus.tex
% Font-setting header file for use with Pandoc Markdown
% to generate PDF via LuaLaTeX.
% The main font is Noto Serif.
% Other main fonts are also available in appropriately named file.
\usepackage{fontspec}
\usepackage{setspace}
\setstretch{1.3}
%
\defaultfontfeatures{Ligatures=TeX,Scale=MatchLowercase,Renderer=Node} % at the start always
%
% For English
% See also https://tex.stackexchange.com/questions/574047/lualatex-amsthm-polyglossia-charissil-error
% We use Node as Renderer for the Latin Font and Greek Font and HarfBuzz as renderer ofr Indic fonts.
%
\babelfont{rm}[Script=Latin,Scale=1]{NotoSerif}% Config is at $HOME/texmf/tex/latex/NotoSerif.fontspec
%
\babelfont{sf}[Script=Latin]{SourceSansPro}% Config is at $HOME/texmf/tex/latex/SourceSansPro.fontspec
%
\babelfont{tt}[Script=Latin]{FiraMono}% Config is at $HOME/texmf/tex/latex/FiraMono.fontspec
%
% Sanskrit, Tamil, and Greek fonts
%
\babelprovide[import, onchar=ids fonts]{sanskrit}
\babelprovide[import, onchar=ids fonts]{tamil}
\babelprovide[import, onchar=ids fonts]{greek}
%
\babelfont[sanskrit]{rm}[Scale=1.1,Renderer=HarfBuzz,Script=Devanagari]{NotoSerifDevanagari}
\babelfont[sanskrit]{sf}[Scale=1.1,Renderer=HarfBuzz,Script=Devanagari]{NotoSansDevanagari}
\babelfont[tamil]{rm}[Renderer=HarfBuzz,Script=Tamil]{NotoSerifTamil}
\babelfont[tamil]{sf}[Renderer=HarfBuzz,Script=Tamil]{NotoSansTamil}
\babelfont[greek]{rm}[Script=Greek]{GentiumBookPlus}
%
% Math font
%
\usepackage{unicode-math} % seems not to hurt % fallabck
\setmathfont[bold-style=TeX]{STIX Two Math}
%
%
% Other fonts
%
\newfontfamily{\emojifont}{Symbola}
%

\usepackage{titling}
\usepackage{fancyhdr}
    \pagestyle{fancy}
    \fancyhead{}
    \fancyfoot{}
    \renewcommand{\headrulewidth}{0.2pt}
    \renewcommand{\footrulewidth}{0.2pt}
    \fancyhead[LO,RE]{\scshape\thetitle}
    \fancyfoot[CO,CE]{\footnotesize Copyright © 2006\textendash\the\year, R (Chandra) Chandrasekhar}
    \fancyfoot[RE,RO]{\thepage}
\newfontfamily{\regulariconfont}{Font Awesome 6 Free Regular}[Color=Grey]
\newfontfamily{\solidiconfont}{Font Awesome 6 Free Solid}[Color=Grey]
\newfontfamily{\brandsiconfont}{Font Awesome 6 Brands}[Color=Grey]
%
% Direct input of Unicode code points
%
\newcommand{\faEnvelope}{\regulariconfont\ ^^^^f0e0\normalfont}
\newcommand{\faMobile}{\solidiconfont\ ^^^^f3cd\normalfont}
\newcommand{\faLinkedin}{\brandsiconfont\ ^^^^f0e1\normalfont}
\newcommand{\faGithub}{\brandsiconfont\ ^^^^f09b\normalfont}
\newcommand{\faAtom}{\solidiconfont\ ^^^^f5d2\normalfont}
\newcommand{\faPaperPlaneRegular}{\regulariconfont\ ^^^^f1d8\normalfont}
\newcommand{\faPaperPlaneSolid}{\solidiconfont\ ^^^^f1d8\normalfont}

%
% The block below is commented out because of Tofu glyphs in HTML
%
% \newcommand{\faEnvelope}{\regulariconfont\ \normalfont}
% \newcommand{\faMobile}{\solidiconfont\ \normalfont}
% \newcommand{\faLinkedin}{\brandsiconfont\ \normalfont}
% \newcommand{\faGithub}{\brandsiconfont\ \normalfont}
\ifLuaTeX
  \usepackage{selnolig}  % disable illegal ligatures
\fi
\IfFileExists{bookmark.sty}{\usepackage{bookmark}}{\usepackage{hyperref}}
\IfFileExists{xurl.sty}{\usepackage{xurl}}{} % add URL line breaks if available
\urlstyle{sf}
\hypersetup{
  pdftitle={Why radians when measuring angles?},
  pdfauthor={R (Chandra) Chandrasekhar},
  pdflang={en-GB},
  colorlinks=true,
  linkcolor={DarkOliveGreen},
  filecolor={Purple},
  citecolor={DarkKhaki},
  urlcolor={Maroon},
  pdfcreator={LaTeX via pandoc}}

\title{Why radians when measuring angles?}
\author{R (Chandra) Chandrasekhar}
\date{2023-10-17 | 2023-10-17}

\begin{document}
\maketitle

\thispagestyle{empty}


\hypertarget{acknowledgements}{%
\subsection{Acknowledgements}\label{acknowledgements}}

\hypertarget{feedback}{%
\subsection{Feedback}\label{feedback}}

Please \href{mailto:feedback.swanlotus@gmail.com}{email me} your
comments and corrections.

\noindent A PDF version of this article is
\href{./radians.pdf}{available for download here}:

\begin{small}

\begin{sffamily}

\url{https://swanlotus.netlify.app/blogs/radians.pdf}

\end{sffamily}

\end{small}

How do I measure an angle?

Most of us first hear of an angle when we encounter triangles. When two
lines that are not parallel meet, they enclose an angle. So, is an angle
a measure of lack of alignment? If a string is fixed to two points, and
you put your finger somewhere in between, the string will make an angle.
Can lines be used to measure angles?

If you chased angles by chasing lines, it is likely that your pursuit of
a robust definition of angle would hit some roadblocks. But f you
introduced the idea of \emph{rotation} on the plane, then you would
discover that two lines that are aligned, i.e., that are parallel, do
not make any angle, each with the other.

But the moment they move out of alignment or out of being parallel, you
would notice that one line \emph{turned} with respect to the other.
Turning is rotation. And angles measure this rotation. How do you rotate
your body? By moving in a circle, turning east, north, west, and south,
for example. Rotation implies movement in a circle.

So, we measure angles using the circle as the basis. And the largest
angle we can make is one full circle. After that, any further rotation
gives us more of the same. So, angles may be measured from zero to one
full circle, before everything starts repeating itself.

Why are radians used?

Radians are an angular measure in which the size of the angle is
expressed as a ratio of two lengths. These ratios are familiar to us
from trigonometry where the sin, cos, and tan functions are expressed as
the ratios of lengths in a right-angled triangle.

But if you think carefully, an angle involves \emph{rotation} and the
simplest and most symmetrical figure in two-dimensions involving a
rotation is a circle. So, angles are best measured using a circle as the
basis for denoting angular size rather than from a triangle, even if
triangles are the first time when you encountered angles.

All circles are similar. As are all squares, all equilateral triangles,
all regular n-gons. What exactly does similarity mean? It means that the
shape remains the same.Imagine a square or a circle. If you zoomed the
figure to enlarge or diminish it without distortion, and you could not
see a change pf shape, that figure exhibits similarity to every other
figure in that class.

When we attempt to define a standardized angular measure we require the
following:

\begin{enumerate}
\tightlist
\item
  the measure should be based on a circle or its arc
\item
  the measure should be independent of the radius of the circle
\end{enumerate}



\end{document}
