% Options for packages loaded elsewhere
\PassOptionsToPackage{unicode,linktoc=all}{hyperref}
\PassOptionsToPackage{hyphens}{url}
\PassOptionsToPackage{dvipsnames,svgnames,x11names}{xcolor}
%
\documentclass[
  a4paper,
]{article}
\usepackage{amsmath,amssymb}
\usepackage{iftex}
\ifPDFTeX
  \usepackage[T1]{fontenc}
  \usepackage[utf8]{inputenc}
  \usepackage{textcomp} % provide euro and other symbols
\else % if luatex or xetex
  \usepackage{unicode-math} % this also loads fontspec
  \defaultfontfeatures{Scale=MatchLowercase}
  \defaultfontfeatures[\rmfamily]{Ligatures=TeX,Scale=1}
\fi
\usepackage{lmodern}
\ifPDFTeX\else
  % xetex/luatex font selection
\fi
% Use upquote if available, for straight quotes in verbatim environments
\IfFileExists{upquote.sty}{\usepackage{upquote}}{}
\IfFileExists{microtype.sty}{% use microtype if available
  \usepackage[]{microtype}
  \UseMicrotypeSet[protrusion]{basicmath} % disable protrusion for tt fonts
}{}
\makeatletter
\@ifundefined{KOMAClassName}{% if non-KOMA class
  \IfFileExists{parskip.sty}{%
    \usepackage{parskip}
  }{% else
    \setlength{\parindent}{0pt}
    \setlength{\parskip}{6pt plus 2pt minus 1pt}}
}{% if KOMA class
  \KOMAoptions{parskip=half}}
\makeatother
\usepackage{xcolor}
\usepackage[margin=25mm]{geometry}
\usepackage{longtable,booktabs,array}
\usepackage{calc} % for calculating minipage widths
% Correct order of tables after \paragraph or \subparagraph
\usepackage{etoolbox}
\makeatletter
\patchcmd\longtable{\par}{\if@noskipsec\mbox{}\fi\par}{}{}
\makeatother
% Allow footnotes in longtable head/foot
\IfFileExists{footnotehyper.sty}{\usepackage{footnotehyper}}{\usepackage{footnote}}
\makesavenoteenv{longtable}
\setlength{\emergencystretch}{3em} % prevent overfull lines
\providecommand{\tightlist}{%
  \setlength{\itemsep}{0pt}\setlength{\parskip}{0pt}}
\setcounter{secnumdepth}{-\maxdimen} % remove section numbering
\ifLuaTeX
\usepackage[bidi=basic]{babel}
\else
\usepackage[bidi=default]{babel}
\fi
\babelprovide[main,import]{british}
% get rid of language-specific shorthands (see #6817):
\let\LanguageShortHands\languageshorthands
\def\languageshorthands#1{}
% $HOME/.pandoc/defaults/latex-header-includes.tex
% Common header includes for both lualatex and xelatex engines.
%
% Preliminaries
%
% \PassOptionsToPackage{rgb,dvipsnames,svgnames}{xcolor}
% \PassOptionsToPackage{main=british}{babel}
\PassOptionsToPackage{english}{selnolig}
\AtBeginEnvironment{quote}{\small}
\AtBeginEnvironment{quotation}{\small}
\AtBeginEnvironment{longtable}{\centering}
%
% Packages that are useful to include
%
\usepackage{graphicx}
\usepackage{subcaption}
\usepackage[inkscapeversion=1]{svg}
\usepackage[defaultlines=4,all]{nowidow}
\usepackage{etoolbox}
\usepackage{fontsize}
\usepackage{newunicodechar}
\usepackage{pdflscape}
\usepackage{fnpct}
\usepackage{parskip}
  \setlength{\parindent}{0pt}
\usepackage[style=american]{csquotes}
% \usepackage{setspace} Use the <fontname-plus.tex> files for setspace
%
\usepackage{esdiff} % for derivative symbols
\usepackage{amsmath}
\usepackage{hyperref} % cleveref must come AFTER hyperref
\usepackage[capitalize,noabbrev]{cleveref} % Must come after hyperref
% noto-plus.tex
% Font-setting header file for use with Pandoc Markdown
% to generate PDF via LuaLaTeX.
% The main font is Noto Serif.
% Other main fonts are also available in appropriately named file.
\usepackage{fontspec}
\usepackage{setspace}
\setstretch{1.3}
%
\defaultfontfeatures{Ligatures=TeX,Scale=MatchLowercase,Renderer=Node} % at the start always
%
% For English
% See also https://tex.stackexchange.com/questions/574047/lualatex-amsthm-polyglossia-charissil-error
% We use Node as Renderer for the Latin Font and Greek Font and HarfBuzz as renderer ofr Indic fonts.
%
\babelfont{rm}[Script=Latin,Scale=1]{NotoSerif}% Config is at $HOME/texmf/tex/latex/NotoSerif.fontspec
%
\babelfont{sf}[Script=Latin]{SourceSansPro}% Config is at $HOME/texmf/tex/latex/SourceSansPro.fontspec
%
\babelfont{tt}[Script=Latin]{FiraMono}% Config is at $HOME/texmf/tex/latex/FiraMono.fontspec
%
% Sanskrit, Tamil, and Greek fonts
%
\babelprovide[import, onchar=ids fonts]{sanskrit}
\babelprovide[import, onchar=ids fonts]{tamil}
\babelprovide[import, onchar=ids fonts]{greek}
%
\babelfont[sanskrit]{rm}[Scale=1.1,Renderer=HarfBuzz,Script=Devanagari]{NotoSerifDevanagari}
\babelfont[sanskrit]{sf}[Scale=1.1,Renderer=HarfBuzz,Script=Devanagari]{NotoSansDevanagari}
\babelfont[tamil]{rm}[Renderer=HarfBuzz,Script=Tamil]{NotoSerifTamil}
\babelfont[tamil]{sf}[Renderer=HarfBuzz,Script=Tamil]{NotoSansTamil}
\babelfont[greek]{rm}[Script=Greek]{GentiumBookPlus}
%
% Math font
%
\usepackage{unicode-math} % seems not to hurt % fallabck
\setmathfont[bold-style=TeX]{STIX Two Math}
%
%
% Other fonts
%
\newfontfamily{\emojifont}{Symbola}
%

\usepackage{titling}
\usepackage{fancyhdr}
    \pagestyle{fancy}
    \fancyhead{}
    \fancyfoot{}
    \renewcommand{\headrulewidth}{0.2pt}
    \renewcommand{\footrulewidth}{0.2pt}
    \fancyhead[LO,RE]{\scshape\thetitle}
    \fancyfoot[CO,CE]{\footnotesize Copyright © 2006\textendash\the\year, R (Chandra) Chandrasekhar}
    \fancyfoot[RE,RO]{\thepage}
\newfontfamily{\regulariconfont}{Font Awesome 6 Free Regular}[Color=Grey]
\newfontfamily{\solidiconfont}{Font Awesome 6 Free Solid}[Color=Grey]
\newfontfamily{\brandsiconfont}{Font Awesome 6 Brands}[Color=Grey]
%
% Direct input of Unicode code points
%
\newcommand{\faEnvelope}{\regulariconfont\ ^^^^f0e0\normalfont}
\newcommand{\faMobile}{\solidiconfont\ ^^^^f3cd\normalfont}
\newcommand{\faLinkedin}{\brandsiconfont\ ^^^^f0e1\normalfont}
\newcommand{\faGithub}{\brandsiconfont\ ^^^^f09b\normalfont}
\newcommand{\faAtom}{\solidiconfont\ ^^^^f5d2\normalfont}
\newcommand{\faPaperPlaneRegular}{\regulariconfont\ ^^^^f1d8\normalfont}
\newcommand{\faPaperPlaneSolid}{\solidiconfont\ ^^^^f1d8\normalfont}

%
% The block below is commented out because of Tofu glyphs in HTML
%
% \newcommand{\faEnvelope}{\regulariconfont\ \normalfont}
% \newcommand{\faMobile}{\solidiconfont\ \normalfont}
% \newcommand{\faLinkedin}{\brandsiconfont\ \normalfont}
% \newcommand{\faGithub}{\brandsiconfont\ \normalfont}
\ifLuaTeX
  \usepackage{selnolig}  % disable illegal ligatures
\fi
\IfFileExists{bookmark.sty}{\usepackage{bookmark}}{\usepackage{hyperref}}
\IfFileExists{xurl.sty}{\usepackage{xurl}}{} % add URL line breaks if available
\urlstyle{sf}
\hypersetup{
  pdftitle={The Two Most Important Numbers: Zero and One},
  pdfauthor={R (Chandra) Chandrasekhar},
  pdflang={en-GB},
  colorlinks=true,
  linkcolor={DarkOliveGreen},
  filecolor={Purple},
  citecolor={DarkKhaki},
  urlcolor={Maroon},
  pdfcreator={LaTeX via pandoc}}

\title{The Two Most Important Numbers: Zero and One}
\author{R (Chandra) Chandrasekhar}
\date{2023-10-31 | 2023-10-31}

\begin{document}
\maketitle

\thispagestyle{empty}


The unique properties of the numbers zero and one make them
mathematically indispensable. In this slow-paced stroll though the ideas
streaming out of these two numbers we uncover well-known as well as
relatively obscure facts about them. It is hoped that in the process we
are enabled to discover how they cement disparate areas of Mathematics.

\hypertarget{the-shy-one}{%
\subsection{The shy one}\label{the-shy-one}}

The number one is often
\href{https://www.vocabulary.com/dictionary/implicit}{implicit} in
mathematical notation. While we may write \(2x\) to denote
\(2\times x\), or two multiplied by \(x\), we \emph{do not} write
\(1x\), even if it is literally correct, because of convention. In
instances like this, the number one is implicit, and assumed to be
understood by those who know. If you happen to be one of those
\emph{not} in the know, here's your chance to join the other side.

When we write a fraction as \(\frac{3}{4}\) we mean the decimal \(0.75\)
and matters are clear. But all whole numbers are also fractions with the
denominator being \(1\). So, the fraction \(\frac{3}{1}\) is rarely
written in that form, even if syntactically correct, because usage
dictates that whole numbers are written to stand on their own, as \(3\),
in this case. Again, the \(1\) in the denominator is assumed to be
unobtrusively present:
\href{https://dictionary.cambridge.org/dictionary/english/out-of-sight-out-of-mind}{out
of sight but \emph{not} out of mind}.

When we write \(4^2\), spoken out as ``four squared'' we mean the number
obtained by multiplying \(4\) by itself. This nomenclature arose
because, if 4 was associated with the \emph{length} of, say, a piece of
string, the number ``four squared'' was used to denote the \emph{area}
of a square that had a side of length \(4\). So, \(4^2 = 4\times4 = 16\)

Likewise, the expression \(7^3\) or ``seven cubed'' denoted the volume
of a cube of side \(7\). Beyond the third dimension, this naming scheme
faded out, because we cannot percieve dimensions higher than three.

Therefore, \(6^4\) is spoken as ``six raised to the fourth (power)'' or
``six to the four''. In such statements, the number \(6\) is called the
\emph{base} and the number \(4\) is called the \emph{exponent}.

Following this logic, we might assert that \(5^1 = 5\) and that is
perfectly correct. But again, convention intrudes to say that we write
it simply as \(5\). \emph{Any number raised to the power of \(1\) equals
itself}.

The notation making \(1\) implicit in these scenarios reduces clutter
and simplifies notation. The absence of the implicit \(1\) might trouble
the heart of the sincere young mathematician, but familiarity with these
conventions will make for comfort in using them.

\hypertarget{multiplication}{%
\subsection{Multiplication}\label{multiplication}}

\hypertarget{division}{%
\subsection{Division}\label{division}}

\hypertarget{why-we-cannot-divide-by-zero}{%
\subsubsection{Why we cannot divide by
zero}\label{why-we-cannot-divide-by-zero}}

\hypertarget{exponentiation}{%
\subsection{Exponentiation}\label{exponentiation}}

Exponentiation may also be called \emph{taking powers}. It is a short
form for repeated multiplication by the \emph{same} number. For example,
if we multiply \(5\) by itself three times, we write it so:
\begin{equation}\protect\hypertarget{eq:exp}{}{
5\times5\times5 = 5^1\times5^1\times5^1 = 5^{(1+1+1)} = 5^{3} = 125
}\label{eq:exp}\end{equation} The number \(5\) is called the \emph{base}
and power \(3\) is called the \emph{exponent}. Notice that \(5^1 = 5\).
This is the first special attribute of \(1\) when we exponentiate.

\hypertarget{acknowledgements}{%
\subsection{Acknowledgements}\label{acknowledgements}}

\hypertarget{feedback}{%
\subsection{Feedback}\label{feedback}}

Please \href{mailto:feedback.swanlotus@gmail.com}{email me} your
comments and corrections.

\noindent A PDF version of this article is
\href{./the-two-most-important-number.pdf}{available for download here}:

\begin{small}

\begin{sffamily}

\url{https://swanlotus.netlify.app/blogs/the-two-most-important-numbers.pdf}

\end{sffamily}

\end{small}



\end{document}
