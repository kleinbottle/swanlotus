% Options for packages loaded elsewhere
\PassOptionsToPackage{unicode,linktoc=all}{hyperref}
\PassOptionsToPackage{hyphens}{url}
\PassOptionsToPackage{dvipsnames,svgnames,x11names}{xcolor}
%
\documentclass[
  a4paper,
]{article}
\usepackage{amsmath,amssymb}
\usepackage{iftex}
\ifPDFTeX
  \usepackage[T1]{fontenc}
  \usepackage[utf8]{inputenc}
  \usepackage{textcomp} % provide euro and other symbols
\else % if luatex or xetex
  \usepackage{unicode-math} % this also loads fontspec
  \defaultfontfeatures{Scale=MatchLowercase}
  \defaultfontfeatures[\rmfamily]{Ligatures=TeX,Scale=1}
\fi
\usepackage{lmodern}
\ifPDFTeX\else
  % xetex/luatex font selection
\fi
% Use upquote if available, for straight quotes in verbatim environments
\IfFileExists{upquote.sty}{\usepackage{upquote}}{}
\IfFileExists{microtype.sty}{% use microtype if available
  \usepackage[]{microtype}
  \UseMicrotypeSet[protrusion]{basicmath} % disable protrusion for tt fonts
}{}
\makeatletter
\@ifundefined{KOMAClassName}{% if non-KOMA class
  \IfFileExists{parskip.sty}{%
    \usepackage{parskip}
  }{% else
    \setlength{\parindent}{0pt}
    \setlength{\parskip}{6pt plus 2pt minus 1pt}}
}{% if KOMA class
  \KOMAoptions{parskip=half}}
\makeatother
\usepackage{xcolor}
\usepackage[margin=25mm]{geometry}
\usepackage{color}
\usepackage{fancyvrb}
\newcommand{\VerbBar}{|}
\newcommand{\VERB}{\Verb[commandchars=\\\{\}]}
\DefineVerbatimEnvironment{Highlighting}{Verbatim}{commandchars=\\\{\}}
% Add ',fontsize=\small' for more characters per line
\usepackage{framed}
\definecolor{shadecolor}{RGB}{48,48,48}
\newenvironment{Shaded}{\begin{snugshade}}{\end{snugshade}}
\newcommand{\AlertTok}[1]{\textcolor[rgb]{1.00,0.81,0.69}{#1}}
\newcommand{\AnnotationTok}[1]{\textcolor[rgb]{0.50,0.62,0.50}{\textbf{#1}}}
\newcommand{\AttributeTok}[1]{\textcolor[rgb]{0.80,0.80,0.80}{#1}}
\newcommand{\BaseNTok}[1]{\textcolor[rgb]{0.86,0.64,0.64}{#1}}
\newcommand{\BuiltInTok}[1]{\textcolor[rgb]{0.80,0.80,0.80}{#1}}
\newcommand{\CharTok}[1]{\textcolor[rgb]{0.86,0.64,0.64}{#1}}
\newcommand{\CommentTok}[1]{\textcolor[rgb]{0.50,0.62,0.50}{#1}}
\newcommand{\CommentVarTok}[1]{\textcolor[rgb]{0.50,0.62,0.50}{\textbf{#1}}}
\newcommand{\ConstantTok}[1]{\textcolor[rgb]{0.86,0.64,0.64}{\textbf{#1}}}
\newcommand{\ControlFlowTok}[1]{\textcolor[rgb]{0.94,0.87,0.69}{#1}}
\newcommand{\DataTypeTok}[1]{\textcolor[rgb]{0.87,0.87,0.75}{#1}}
\newcommand{\DecValTok}[1]{\textcolor[rgb]{0.86,0.86,0.80}{#1}}
\newcommand{\DocumentationTok}[1]{\textcolor[rgb]{0.50,0.62,0.50}{#1}}
\newcommand{\ErrorTok}[1]{\textcolor[rgb]{0.76,0.75,0.62}{#1}}
\newcommand{\ExtensionTok}[1]{\textcolor[rgb]{0.80,0.80,0.80}{#1}}
\newcommand{\FloatTok}[1]{\textcolor[rgb]{0.75,0.75,0.82}{#1}}
\newcommand{\FunctionTok}[1]{\textcolor[rgb]{0.94,0.94,0.56}{#1}}
\newcommand{\ImportTok}[1]{\textcolor[rgb]{0.80,0.80,0.80}{#1}}
\newcommand{\InformationTok}[1]{\textcolor[rgb]{0.50,0.62,0.50}{\textbf{#1}}}
\newcommand{\KeywordTok}[1]{\textcolor[rgb]{0.94,0.87,0.69}{#1}}
\newcommand{\NormalTok}[1]{\textcolor[rgb]{0.80,0.80,0.80}{#1}}
\newcommand{\OperatorTok}[1]{\textcolor[rgb]{0.94,0.94,0.82}{#1}}
\newcommand{\OtherTok}[1]{\textcolor[rgb]{0.94,0.94,0.56}{#1}}
\newcommand{\PreprocessorTok}[1]{\textcolor[rgb]{1.00,0.81,0.69}{\textbf{#1}}}
\newcommand{\RegionMarkerTok}[1]{\textcolor[rgb]{0.80,0.80,0.80}{#1}}
\newcommand{\SpecialCharTok}[1]{\textcolor[rgb]{0.86,0.64,0.64}{#1}}
\newcommand{\SpecialStringTok}[1]{\textcolor[rgb]{0.80,0.58,0.58}{#1}}
\newcommand{\StringTok}[1]{\textcolor[rgb]{0.80,0.58,0.58}{#1}}
\newcommand{\VariableTok}[1]{\textcolor[rgb]{0.80,0.80,0.80}{#1}}
\newcommand{\VerbatimStringTok}[1]{\textcolor[rgb]{0.80,0.58,0.58}{#1}}
\newcommand{\WarningTok}[1]{\textcolor[rgb]{0.50,0.62,0.50}{\textbf{#1}}}
\usepackage{longtable,booktabs,array}
\usepackage{calc} % for calculating minipage widths
% Correct order of tables after \paragraph or \subparagraph
\usepackage{etoolbox}
\makeatletter
\patchcmd\longtable{\par}{\if@noskipsec\mbox{}\fi\par}{}{}
\makeatother
% Allow footnotes in longtable head/foot
\IfFileExists{footnotehyper.sty}{\usepackage{footnotehyper}}{\usepackage{footnote}}
\makesavenoteenv{longtable}
\usepackage{graphicx}
\makeatletter
\def\maxwidth{\ifdim\Gin@nat@width>\linewidth\linewidth\else\Gin@nat@width\fi}
\def\maxheight{\ifdim\Gin@nat@height>\textheight\textheight\else\Gin@nat@height\fi}
\makeatother
% Scale images if necessary, so that they will not overflow the page
% margins by default, and it is still possible to overwrite the defaults
% using explicit options in \includegraphics[width, height, ...]{}
\setkeys{Gin}{width=\maxwidth,height=\maxheight,keepaspectratio}
% Set default figure placement to htbp
\makeatletter
\def\fps@figure{htbp}
\makeatother
\usepackage{svg}
\setlength{\emergencystretch}{3em} % prevent overfull lines
\providecommand{\tightlist}{%
  \setlength{\itemsep}{0pt}\setlength{\parskip}{0pt}}
\setcounter{secnumdepth}{-\maxdimen} % remove section numbering
\newlength{\cslhangindent}
\setlength{\cslhangindent}{1.5em}
\newlength{\csllabelwidth}
\setlength{\csllabelwidth}{3em}
\newlength{\cslentryspacingunit} % times entry-spacing
\setlength{\cslentryspacingunit}{\parskip}
\newenvironment{CSLReferences}[2] % #1 hanging-ident, #2 entry spacing
 {% don't indent paragraphs
  \setlength{\parindent}{0pt}
  % turn on hanging indent if param 1 is 1
  \ifodd #1
  \let\oldpar\par
  \def\par{\hangindent=\cslhangindent\oldpar}
  \fi
  % set entry spacing
  \setlength{\parskip}{#2\cslentryspacingunit}
 }%
 {}
\usepackage{calc}
\newcommand{\CSLBlock}[1]{#1\hfill\break}
\newcommand{\CSLLeftMargin}[1]{\parbox[t]{\csllabelwidth}{#1}}
\newcommand{\CSLRightInline}[1]{\parbox[t]{\linewidth - \csllabelwidth}{#1}\break}
\newcommand{\CSLIndent}[1]{\hspace{\cslhangindent}#1}
\ifLuaTeX
\usepackage[bidi=basic]{babel}
\else
\usepackage[bidi=default]{babel}
\fi
\babelprovide[main,import]{british}
% get rid of language-specific shorthands (see #6817):
\let\LanguageShortHands\languageshorthands
\def\languageshorthands#1{}
% $HOME/.pandoc/defaults/latex-header-includes.tex
% Common header includes for both lualatex and xelatex engines.
%
% Preliminaries
%
% \PassOptionsToPackage{rgb,dvipsnames,svgnames}{xcolor}
% \PassOptionsToPackage{main=british}{babel}
\PassOptionsToPackage{english}{selnolig}
\AtBeginEnvironment{quote}{\small}
\AtBeginEnvironment{quotation}{\small}
\AtBeginEnvironment{longtable}{\centering}
%
% Packages that are useful to include
%
\usepackage{graphicx}
\usepackage{subcaption}
\usepackage[inkscapeversion=1]{svg}
\usepackage[defaultlines=4,all]{nowidow}
\usepackage{etoolbox}
\usepackage{fontsize}
\usepackage{newunicodechar}
\usepackage{pdflscape}
\usepackage{fnpct}
\usepackage{parskip}
  \setlength{\parindent}{0pt}
\usepackage[style=american]{csquotes}
% \usepackage{setspace} Use the <fontname-plus.tex> files for setspace
%
\usepackage{hyperref} % cleveref must come AFTER hyperref
\usepackage[capitalize,noabbrev]{cleveref} % Must come after hyperref
% noto-plus.tex
% Font-setting header file for use with Pandoc Markdown
% to generate PDF via LuaLaTeX.
% The main font is Noto Serif.
% Other main fonts are also available in appropriately named file.
\usepackage{fontspec}
\usepackage{setspace}
\setstretch{1.3}
%
\defaultfontfeatures{Ligatures=TeX,Scale=MatchLowercase,Renderer=Node} % at the start always
%
% For English
% See also https://tex.stackexchange.com/questions/574047/lualatex-amsthm-polyglossia-charissil-error
% We use Node as Renderer for the Latin Font and Greek Font and HarfBuzz as renderer ofr Indic fonts.
%
\babelfont{rm}[Script=Latin,Scale=1]{NotoSerif}% Config is at $HOME/texmf/tex/latex/NotoSerif.fontspec
\babelfont{sf}[Script=Latin]{SourceSansPro}% Config is at $HOME/texmf/tex/latex/SourceSansPro.fontspec
\babelfont{tt}[Script=Latin]{FiraMono}% Config is at $HOME/texmf/tex/latex/FiraMono.fontspec
%
% Sanskrit, Tamil, and Greek fonts
%
\babelprovide[import, onchar=ids fonts]{sanskrit}
\babelprovide[import, onchar=ids fonts]{tamil}
\babelprovide[import, onchar=ids fonts]{greek}
%
\babelfont[sanskrit]{rm}[Scale=1.1,Renderer=HarfBuzz,Script=Devanagari]{NotoSerifDevanagari}
\babelfont[sanskrit]{sf}[Scale=1.1,Renderer=HarfBuzz,Script=Devanagari]{NotoSansDevanagari}
\babelfont[tamil]{rm}[Renderer=HarfBuzz,Script=Tamil]{NotoSerifTamil}
\babelfont[tamil]{sf}[Renderer=HarfBuzz,Script=Tamil]{NotoSansTamil}
\babelfont[greek]{rm}[Script=Greek]{GentiumBookPlus}
%
% Math font
%
\usepackage{unicode-math} % seems not to hurt % fallabck
\setmathfont[bold-style=TeX]{STIX Two Math}
\usepackage{amsmath}
\usepackage{esdiff} % for derivative symbols
%
%
% Other fonts
%
\newfontfamily{\emojifont}{Symbola}
%

\usepackage{titling}
\usepackage{fancyhdr}
    \pagestyle{fancy}
    \fancyhead{}
    \fancyfoot{}
    \renewcommand{\headrulewidth}{0.2pt}
    \renewcommand{\footrulewidth}{0.2pt}
    \fancyhead[LO,RE]{\scshape\thetitle}
    \fancyfoot[CO,CE]{\footnotesize Copyright © 2006\textendash\the\year, R (Chandra) Chandrasekhar}
    \fancyfoot[RE,RO]{\thepage}
\newfontfamily{\regulariconfont}{Font Awesome 6 Free Regular}[Color=Grey]
\newfontfamily{\solidiconfont}{Font Awesome 6 Free Solid}[Color=Grey]
\newfontfamily{\brandsiconfont}{Font Awesome 6 Brands}[Color=Grey]
%
% Direct input of Unicode code points
%
\newcommand{\faEnvelope}{\regulariconfont\ ^^^^f0e0\normalfont}
\newcommand{\faMobile}{\solidiconfont\ ^^^^f3cd\normalfont}
\newcommand{\faLinkedin}{\brandsiconfont\ ^^^^f0e1\normalfont}
\newcommand{\faGithub}{\brandsiconfont\ ^^^^f09b\normalfont}
\newcommand{\faAtom}{\solidiconfont\ ^^^^f5d2\normalfont}
\newcommand{\faPaperPlaneRegular}{\regulariconfont\ ^^^^f1d8\normalfont}
\newcommand{\faPaperPlaneSolid}{\solidiconfont\ ^^^^f1d8\normalfont}

%
% The block below is commented out because of Tofu glyphs in HTML
%
% \newcommand{\faEnvelope}{\regulariconfont\ \normalfont}
% \newcommand{\faMobile}{\solidiconfont\ \normalfont}
% \newcommand{\faLinkedin}{\brandsiconfont\ \normalfont}
% \newcommand{\faGithub}{\brandsiconfont\ \normalfont}
\ifLuaTeX
  \usepackage{selnolig}  % disable illegal ligatures
\fi
\IfFileExists{bookmark.sty}{\usepackage{bookmark}}{\usepackage{hyperref}}
\IfFileExists{xurl.sty}{\usepackage{xurl}}{} % add URL line breaks if available
\urlstyle{sf}
\hypersetup{
  pdftitle={A Foray into Rust: Euler One},
  pdfauthor={R (Chandra) Chandrasekhar},
  pdflang={en-GB},
  colorlinks=true,
  linkcolor={DarkOliveGreen},
  filecolor={Purple},
  citecolor={DarkKhaki},
  urlcolor={Maroon},
  pdfcreator={LaTeX via pandoc}}

\title{A Foray into \texttt{Rust}: Euler One}
\author{R (Chandra) Chandrasekhar}
\date{2021-07-31 | 2023-11-26}

\begin{document}
\maketitle

\thispagestyle{empty}


As a programmer, I am long in the tooth. I started out with
\href{https://fortran-lang.org/}{\texttt{FORTRAN}}, went on to
\href{https://www.forth.com/}{\texttt{Forth}}, and settled with
\href{https://www.iso.org/standard/74528.html}{\texttt{C}} through three
decades or more. Later, it was
\href{https://www.mathworks.com/}{\texttt{MATLAB}} and
\href{https://www.gnu.org/software/octave/index}{\texttt{Octave}} for
high level computing. For scripting, I used
\href{https://www.perl.org/}{\texttt{Perl}} or
\href{https://www.gnu.org/software/bash/}{\texttt{bash}}.
\href{https://www.python.org/}{\texttt{Python}}, the current darling of
programmers, is an
\href{http://www.finedictionary.com/bourn.html}{unknown bourne} to me.

So why did I choose \href{https://www.rust-lang.org/}{\texttt{Rust}} as
the new programming language to learn? Rust is \emph{the} emerging
programming language, developed at
\href{https://research.mozilla.org/rust/}{Mozilla}
{[}\protect\hyperlink{ref-rust}{1}{]}. It has been consistently voted
\href{https://insights.stackoverflow.com/survey/2020\#most-loved-dreaded-and-wanted}{the
most loved} programming language in Stack Overflow Developer Surveys
{[}\protect\hyperlink{ref-goulding2020}{2}{]}. End-users, such as
\href{https://www.nature.com/articles/d41586-020-03382-2}{scientists},
are turning to Rust when Python has proven inadequate for some reason
{[}\protect\hyperlink{ref-perkel2020}{3}{]}. And
\href{https://serokell.io/blog/rust-companies}{corporate users} include
\href{https://www.dropbox.com/}{Dropbox}, Mozilla,
\href{https://www.youtube.com/watch?v=NQBVUjdkLAA}{Microsoft},
\href{https://www.npmjs.com/}{npm}, etc.
{[}\protect\hyperlink{ref-dreimanis2020}{4}{]}.

But there are other, more personal, reasons as well. My previous bet on
the future was on \href{https://www.haskell.org/}{\texttt{Haskell}}. I
have tried many times to learn it, almost always giving up in despair,
because I was put off by the unfamiliar notation, and its corpus of
arcana, like
\href{en.wikipedia.org/wiki/Monad_(functional_programming)}{monads},
touted by the cognoscenti, as the way to tell the men from the boys.
Enough about the why. Now for the how.

I decided to start learning \texttt{Rust} by solving
\href{https://projecteuler.net/problem=1}{Project Euler Problem
One}---henceforth called \emph{Euler One}, \emph{the problem}, or
\emph{the question}---using Rust. This is a chronicle of my first
efforts, including false starts, errors, backtracks, etc.

\hypertarget{project-euler-problem-one}{%
\subsection{Project Euler Problem One}\label{project-euler-problem-one}}

The \href{https://projecteuler.net/problem=1}{statement of the problem}
is simple and pellucid:

\begin{quote}
\textbf{Multiples of 3 or 5}

If we list all the natural numbers below 10 that are multiples of 3 or
5, we get 3, 5, 6 and 9. The sum of these multiples is 23.

Find the sum of all the multiples of 3 \emph{or} 5 \emph{below} 1000.
{[}\emph{Emphasis} is mine{]}
\end{quote}

\hypertarget{algorithm-for-problem-solving}{%
\subsection{Algorithm for problem
solving}\label{algorithm-for-problem-solving}}

The algorithm for problem solving is
{[}\protect\hyperlink{ref-sas}{5}{]}:

\begin{enumerate}
\tightlist
\item
  Read the question carefully.
\item
  Understand the question correctly.
\item
  Answer the question precisely.
\end{enumerate}

The problem asks for \emph{all} the multiples of \(3\) \emph{or} \(5\)
\emph{below} \(1000\). I have \emph{emphasized} the words that require
careful understanding and thought. Care at this stage of acquaintance
with the problem staves off many a careless mistake by nipping it in the
bud.

\hypertarget{parsing-the-question}{%
\subsection{Parsing the question}\label{parsing-the-question}}

\hypertarget{the-word-or}{%
\subsubsection{The word ``or''}\label{the-word-or}}

I have emphasized three words in the problem definition: \emph{all},
\emph{or}, and \emph{below}. The first is obvious. Let us look at the
other two.

The phrase ``multiples of 3 \emph{or} 5'' may be interpreted in two
ways. If we think of it as an \emph{inclusive or}, then it means
``multiples of \(3\), multiples of \(5\), and multiples of both \(3\)
and \(5\)''.

If we think of it as an \emph{exclusive or}, then it means ``multiples
of \(3\), multiples of \(5\), but not multiples of both \(3\) and
\(5\)''.

Since the qualification of ``but not both'' is absent from the rubric,
we will assume an inclusive or, i.e., the first interpretation.

\hypertarget{the-word-below}{%
\subsubsection{The word ``below''}\label{the-word-below}}

The word ``below'' introduces the mathematical relation \(<\) as opposed
to \(\leq\). This means all multiples of three or five that are less
than \(1000\), excluding \(1000\).

The time spent in looking at the question through a magnifying glass is
time well spent, because it forces us to assume the mindset of the
author who carefully crafted the question. We thereby become acquainted
with the possibilities for pitfalls and potholes that could otherwise
\href{https://www.powerthesaurus.org/upend/synonyms}{upend} our efforts.

\hypertarget{initial-thoughts}{%
\subsection{Initial thoughts}\label{initial-thoughts}}

The multiples of \(3\) are those numbers, which when divided by \(3\),
leave a remainder of zero. Likewise the numbers which leave a remainder
of zero when divided by \(5\) are multiples of \(5\). This implies
\emph{integer arithmetic}, and that in turn, could mean we have to
\emph{declare} the type of numbers we are using. Floating point division
will never do for our problem. But anyway, division is problematic;
witness the caveat that the divisor may not be zero in the field of
rational numbers, \(\mathbb{Q}\).

In terms of division, the \texttt{\%} operator for integer division from
other programming languages suggests itself. But is division the most
natural way to identify the multiples of a number? Should it not be
multiplication instead? It is time to start thinking with a
\href{https://en.wikipedia.org/wiki/Shoshin}{beginner's mind}.

We also need a structure like an \emph{array} or \emph{list} where
numbers may be appended or inserted until the stopping condition is
reached. If we keep a running total, though, we do not need anything
else except three receptacles: one for the sum of multiples of three,
\(s_3\), another for the sum of multiples of five, \(s_5\), and one more
for the sum of multiples of \(15\), \(s_{15}\). Let us try the latter
option first, and leave arrays for a later refinement.

\hypertarget{setting-the-bounds}{%
\subsection{Setting the bounds}\label{setting-the-bounds}}

We know that \(1000 \div 3 = 333\) with a remainder of \(1\). The
largest multiple of \(3\) less than \(1000\) is therefore,
\(333 \times 3 = 999\). The number of multiples of \(3\), \(n_{3}\), we
will be dealing with is thus \(333\).

Likewise, \(1000 \div 5 = 200\) with a remainder of \(0\). Since
\(1000\) is a multiple of \(5\), we need the \emph{next lower} multiple
of \(5\) below \(1000\), which is \$1000 - 5 = \(995\). Now,
\(995 \div 5 = 199\); so \(n_{5} = 199\).

With \(15\), we have \(1000 \div 15 = 66\) with a remainder of \(10\).
So, \(66 \times 15 = 990\) is the upper bound, and the number of
multiples \(n_{15}\) is \(66\).

Because \(15\) is a multiple of \emph{both} \(3\) and \(5\), we need to
ensure that we do not add its multiples \emph{twice} in our summations.

\hypertarget{venn-diagram-representation}{%
\subsubsection{Venn diagram
representation}\label{venn-diagram-representation}}

Viewing a problem pictorially often helps us to grasp it better. In this
case, it is not a graph but a
\href{https://www.lucidchart.com/pages/tutorial/venn-diagram}{Venn
diagram} that helps. In \cref{fig:venn}, we use circles A and B to
represent the sets of multiples of \(3\) and \(5\) respectively. The two
circles overlap because there exist numbers that are multiples of both
\(3\) and \(5\): these are the multiples of \(15\).

\begin{figure}
\hypertarget{fig:venn}{%
\centering
\includesvg[width=0.7\textwidth,height=\textheight]{images/venn.svg}
\caption{Venn diagram relating multiples of \(3\), shown as set \(A\),
multiples of \(5\) as set \(B\), and multiples of \(15\) as their
intersection \(A\cap B\).}\label{fig:venn}
}
\end{figure}

We know from \href{https://en.wikipedia.org/wiki/Set_theory}{set theory}
that what we are after is \(A \cup B\) or the union of the sets \(A\)
and \(B\). Also, the number of elements in the sets are related by
\begin{equation}\protect\hypertarget{eq:venn}{}{
n(A \cup B) = n(A) + n(B) - n(A\cap B).
}\label{eq:venn}\end{equation} The expression \(n(A)\), for example,
denotes the number of (unique) elements in the set \(A\). \cref{eq:venn}
gives us a convenient way of counting the multiples of \(3\) or \(5\),
\emph{without double counting the multiples of \(15\)}.

\hypertarget{algorithm}{%
\subsection{Algorithm}\label{algorithm}}

The most direct algorithm to solve the problem in
\href{https://en.wikipedia.org/wiki/Pseudocode}{pseudocode} is:

\begin{enumerate}
\item
  Define \(s_3\) as the cumulative sum of the multiples of \(3\), and
  initialize it to \(0\).
\item
  Define \(s_5\) as the cumulative sum of the multiples of \(5\), and
  initialize it to \(0\).
\item
  Define \(s_{15}\) as the cumulative sum of multiples of \(15\), and
  initialize it to \(0\).
\item
  Loop through the natural numbers \(\mathbb{N}\) from \(1\) to \(333\),
  compute the multiples of \(3\), one at a time, and add it to \(s_3\).
\item
  Loop through the natural numbers \(\mathbb{N}\) from \(1\) to \(199\),
  compute the multiples of \(5\), one at a time, and add it to \(s_5\).
\item
  Loop through the natural numbers \(\mathbb{N}\) from \(1\) to \(66\),
  compute the multiples of \(15\), one at a time, and add it to
  \(s_{15}\).
\item
  Evaluate \((s_3 + s_5 - s_{15})\) and present it as the desired
  result. See \cref{eq:venn} for an explanation.
\end{enumerate}

\hypertarget{pseudocode}{%
\subsection{Pseudocode}\label{pseudocode}}

I envisage three independent \texttt{for} loops to achieve this. The
pseudocode could read:

\begin{Shaded}
\begin{Highlighting}[]
\NormalTok{s3 = s5 = s15 = 0 \# initialize variables}

\NormalTok{for n in [1:333]}
\NormalTok{do}
\NormalTok{  s3 = s3 + 3*n}
\NormalTok{done}

\NormalTok{for n i [1:199]}
\NormalTok{do}
\NormalTok{  s5 = s5 + 5*n}
\NormalTok{done}

\NormalTok{for n i [1:66]}
\NormalTok{do}
\NormalTok{  s15 = s15 + 15*n}
\NormalTok{done}

\NormalTok{print (s3 + s5 {-} s15)}
\end{Highlighting}
\end{Shaded}

We have implicitly assumed that the \texttt{for} loop increment is
\(1\). The mathematical convention for a closed interval is used above
to denote that \emph{both} the upper and lower limits are
\emph{inclusive}.

\hypertarget{first-attempt}{%
\subsection{First attempt}\label{first-attempt}}

Let us barge ahead using the syntax of \texttt{Rust} and see how the
above pseudo code fleshes out. It turns out that
\href{https://doc.rust-lang.org/reference/expressions/loop-expr.html}{\texttt{Rust}
supports five types of loop} and we need the one with the \texttt{for}
flavour, called the iterator loop.

There is also an example on that web page that is similar to our
problem. It uses a \texttt{for} loop, but the variable holding the sum
is initialized using the
\href{https://doc.rust-lang.org/std/keyword.mut.html}{\texttt{mut}
keyword}. Let us copy the code fragment and change it to suit our
purposes:

\begin{Shaded}
\begin{Highlighting}[]
\CommentTok{// Attempt Number 1}
    \KeywordTok{let} \KeywordTok{mut}\NormalTok{ s3 }\OperatorTok{=} \DecValTok{0}\OperatorTok{;}
    \KeywordTok{let} \KeywordTok{mut}\NormalTok{ s5 }\OperatorTok{=} \DecValTok{0}\OperatorTok{;}
    \KeywordTok{let} \KeywordTok{mut}\NormalTok{ s15 }\OperatorTok{=} \DecValTok{0}\OperatorTok{;}

    \ControlFlowTok{for}\NormalTok{ n }\KeywordTok{in} \DecValTok{1}\OperatorTok{..}\DecValTok{333} \OperatorTok{\{}
\NormalTok{      s3 }\OperatorTok{+=}\NormalTok{ n}\OperatorTok{*}\DecValTok{3}\OperatorTok{;}
    \OperatorTok{\}}

    \ControlFlowTok{for}\NormalTok{ n }\KeywordTok{in} \DecValTok{1}\OperatorTok{..}\DecValTok{199} \OperatorTok{\{}
\NormalTok{      s5 }\OperatorTok{+=}\NormalTok{ n}\OperatorTok{*}\DecValTok{5}\OperatorTok{;}
    \OperatorTok{\}}

    \ControlFlowTok{for}\NormalTok{ n }\KeywordTok{in} \DecValTok{1}\OperatorTok{..}\DecValTok{66} \OperatorTok{\{}
\NormalTok{      s15 }\OperatorTok{+=}\NormalTok{ n}\OperatorTok{*}\DecValTok{15}\OperatorTok{;}
    \OperatorTok{\}}

\NormalTok{    println(s3 }\OperatorTok{+}\NormalTok{ s5 }\OperatorTok{{-}}\NormalTok{ s15)}\OperatorTok{;}
\end{Highlighting}
\end{Shaded}

Not surprisingly, the above fragment contains numerous errors and would
not compile. So, I needed to backtrack to see an example of the
archetypal ``Hello World!'' program to get the
\href{https://doc.rust-lang.org/book/ch01-02-hello-world.html}{proper
invocatory syntax}. Languages like \texttt{C} and \texttt{Java} come
with some baggage that needs to be wrapped around the core code so that
it may be rendered into an executable program. \texttt{Rust} seems to
have borrowed this characteristic from them. Note the use of
\texttt{s3\ +=\ n*3;} which is shorthand for \texttt{s3\ =\ s3\ +\ n*3}.
The \texttt{+=} operator is available in \texttt{Rust}, but not always
in other languages.

\hypertarget{second-attempt}{%
\subsection{Second attempt}\label{second-attempt}}

My second attempt at the program, with proper indentation, is now:

\begin{Shaded}
\begin{Highlighting}[]
 \CommentTok{// Attempt Number 2}
\KeywordTok{fn}\NormalTok{ main() }\OperatorTok{\{}
    \KeywordTok{let} \KeywordTok{mut}\NormalTok{ s3 }\OperatorTok{=} \DecValTok{0}\OperatorTok{;}
    \KeywordTok{let} \KeywordTok{mut}\NormalTok{ s5 }\OperatorTok{=} \DecValTok{0}\OperatorTok{;}
    \KeywordTok{let} \KeywordTok{mut}\NormalTok{ s15 }\OperatorTok{=} \DecValTok{0}\OperatorTok{;}

    \ControlFlowTok{for}\NormalTok{ n }\KeywordTok{in} \DecValTok{1}\OperatorTok{..}\DecValTok{333} \OperatorTok{\{}
\NormalTok{        s3 }\OperatorTok{+=}\NormalTok{ n}\OperatorTok{*}\DecValTok{3}\OperatorTok{;}
    \OperatorTok{\}}

    \ControlFlowTok{for}\NormalTok{ n }\KeywordTok{in} \DecValTok{1}\OperatorTok{..}\DecValTok{199} \OperatorTok{\{}
\NormalTok{        s5 }\OperatorTok{+=}\NormalTok{ n}\OperatorTok{*}\DecValTok{5}\OperatorTok{;}
    \OperatorTok{\}}

    \ControlFlowTok{for}\NormalTok{ n }\KeywordTok{in} \DecValTok{1}\OperatorTok{..}\DecValTok{66} \OperatorTok{\{}
\NormalTok{        s15 }\OperatorTok{+=}\NormalTok{ n}\OperatorTok{*}\DecValTok{15}\OperatorTok{;}
    \OperatorTok{\}}

    \PreprocessorTok{println!}\NormalTok{(}\StringTok{"\{\}"}\OperatorTok{,}\NormalTok{ s3 }\OperatorTok{+}\NormalTok{ s5 }\OperatorTok{{-}}\NormalTok{ s15)}\OperatorTok{;}
\OperatorTok{\}}
\end{Highlighting}
\end{Shaded}

I have wrapped the whole code fragment with a \texttt{main()} function
just as in \texttt{C}. Moreover, I have learned that \texttt{println!}
is a macro rather than a function and that it is invoked as shown. This
has already disheartened me a bit because something too much like
\texttt{C} or \texttt{Java}---with a lot of clunky statements for simple
actions---is a step in the \emph{wrong} direction for an easier-to-use
programming language. Let us hope it does not rain pickaxes and shovels
when we compile the code!

This time, the code was compiled without a murmur. Upon execution, the
answer was \(232164\). Is it correct? Or have we tripped somewhere?

\hypertarget{result-with-octave}{%
\subsection{\texorpdfstring{Result with
\texttt{Octave}}{Result with Octave}}\label{result-with-octave}}

The easiest and laziest way to check the result was to use a naturally
vector-based language to verify the above result. I chose
\texttt{Octave} as it is freely available, and I know it somewhat.
Because the natural data structure in \texttt{Octave} is a vector or a
matrix, I could type out the whole sequence using the syntax
\texttt{{[}start:step:end{]}} and sum it up to get the three sums of
multiples. The code was so easy, that I could write it without reference
to paper:

\begin{Shaded}
\begin{Highlighting}[]
\FunctionTok{sum}\NormalTok{([}\FloatTok{3}\OperatorTok{:}\FloatTok{3}\OperatorTok{:}\FloatTok{999}\NormalTok{]) }\OperatorTok{+} \FunctionTok{sum}\NormalTok{([}\FloatTok{5}\OperatorTok{:}\FloatTok{5}\OperatorTok{:}\FloatTok{995}\NormalTok{]) }\OperatorTok{{-}} \FunctionTok{sum}\NormalTok{([}\FloatTok{15}\OperatorTok{:}\FloatTok{15}\OperatorTok{:}\FloatTok{990}\NormalTok{])}
\end{Highlighting}
\end{Shaded}

and this gave a result of \(233168\). Ouch! it differs from the result
using \texttt{Rust}. I must also say that, though laconic,
\texttt{Octave} got the job done with very little fuss or fanfare.
Vectorized code is both more powerful and simpler to understand and
maintain. The best language for someone working with vectors is one that
supports them natively.

We must now make a third, ``repair and maintenance'' attempt with the
\texttt{rust} code.

\hypertarget{troubleshooting}{%
\subsection{Troubleshooting}\label{troubleshooting}}

The \texttt{rust} program is so simple that the most likely error must
lie with the limits in the \texttt{for} loop. Indeed, an experienced
programmer would have seen it at once.

Programming languages are notoriously inconsistent on two fronts:

\begin{enumerate}
\def\labelenumi{\alph{enumi}.}
\tightlist
\item
  Whether they start their indexing with \(0\) or with \(1\); and
\item
  Whether their index ranges are on closed intervals \([a, b]\), or
  semi-closed intervals \([a, b)\), or \((a, b]\), or open intervals
  \((a, b).\)
\end{enumerate}

One would have thought that common sense would impel language designers
to adopt uniform conventions on these two issues. Unfortunately the
authors of programming languages have rather fiercely held philosophical
notions, and a divide persists. Thus each foray into a new language must
be cautiously done with these two factors in mind.

In our case, we need to hark back to the
\href{https://doc.rust-lang.org/reference/expressions/range-expr.html}{definition
of the \texttt{..} range} operator in \texttt{Rust}. The expression
\texttt{start..end} means that the index variable \texttt{i} lies in a
semi-closed interval: \texttt{start\ \textless{}=\ i\ \textless{}\ end}.
The \texttt{end} parameters in each case need to be increased by one in
our program. Our third attempt is shown below:

\hypertarget{third-attempt}{%
\subsection{Third attempt}\label{third-attempt}}

\begin{Shaded}
\begin{Highlighting}[]
 \CommentTok{// Attempt Number 3}
\KeywordTok{fn}\NormalTok{ main() }\OperatorTok{\{}
    \KeywordTok{let} \KeywordTok{mut}\NormalTok{ s3 }\OperatorTok{=} \DecValTok{0}\OperatorTok{;}
    \KeywordTok{let} \KeywordTok{mut}\NormalTok{ s5 }\OperatorTok{=} \DecValTok{0}\OperatorTok{;}
    \KeywordTok{let} \KeywordTok{mut}\NormalTok{ s15 }\OperatorTok{=} \DecValTok{0}\OperatorTok{;}

    \ControlFlowTok{for}\NormalTok{ n }\KeywordTok{in} \DecValTok{1}\OperatorTok{..}\DecValTok{334} \OperatorTok{\{}
\NormalTok{        s3 }\OperatorTok{+=}\NormalTok{ n}\OperatorTok{*}\DecValTok{3}\OperatorTok{;}
    \OperatorTok{\}}

    \ControlFlowTok{for}\NormalTok{ n }\KeywordTok{in} \DecValTok{1}\OperatorTok{..}\DecValTok{200} \OperatorTok{\{}
\NormalTok{        s5 }\OperatorTok{+=}\NormalTok{ n}\OperatorTok{*}\DecValTok{5}\OperatorTok{;}
    \OperatorTok{\}}

    \ControlFlowTok{for}\NormalTok{ n }\KeywordTok{in} \DecValTok{1}\OperatorTok{..}\DecValTok{67} \OperatorTok{\{}
\NormalTok{        s15 }\OperatorTok{+=}\NormalTok{ n}\OperatorTok{*}\DecValTok{15}\OperatorTok{;}
    \OperatorTok{\}}

    \PreprocessorTok{println!}\NormalTok{(}\StringTok{"\{\}"}\OperatorTok{,}\NormalTok{ s3 }\OperatorTok{+}\NormalTok{ s5 }\OperatorTok{{-}}\NormalTok{ s15)}\OperatorTok{;}
\OperatorTok{\}}
\end{Highlighting}
\end{Shaded}

This again complied incident-free and the result that popped out was
\(233168\). Bingo! It is the same as what \texttt{Octave} gave us. That
is a reassuring feeling. The real arbiter of truth, though, is
mathematics. What does it say?

\hypertarget{the-gold-standard}{%
\subsection{The Gold Standard}\label{the-gold-standard}}

We are fortunate that in this case, the mathematics is both simple and
well known. We are dealing with the sums of three
\href{https://en.wikipedia.org/wiki/Arithmetic_progression\#Sum}{arithmetic
progressions (AP)}. The \emph{first term} in an AP is usually denoted
\(a\) and the \emph{common difference} is denoted by \(d\). The number
of terms is usually \(n\). The \emph{last term} is
\(a_n = a + (n - 1)d\), and the sum to \(n\) terms is
\begin{equation}\protect\hypertarget{eq:apsum}{}{
a + a + d + a + 2d + a + 3d + \cdots + a + (n - 1)d = \frac{n}{2}\left( a + a_n \right)
}\label{eq:apsum}\end{equation} Using this formula, for the multiples of
\(3\), we have \(a = 3\), \(n = 333\) and \(a_n=999\), giving us \[
s_3 = \frac{333}{2}\left( 3 + 999\right) = 166833.
\] Likewise, for the multiples of \(5\), we have \(a = 5\), \(n = 199\)
and \(a_n = 995\), yielding \[
s_5 = \frac{199}{2}\left( 5 + 995\right) = 99500.
\] Finally, the sum of multiples of \(15\) is given by \[
s_{15} = \frac{66}{2}\left( 15 + 990\right) = 33165.
\]

The required sum, \(s\) is therefore \[
s= s_3 + s_5 - s_{15} = 166833 + 99500 -33165 = 233168.
\] So, we have indeed got the correct result!

\hypertarget{vectorizing}{%
\subsection{Vectorizing}\label{vectorizing}}

The single-line \texttt{Octave} program made the solution seem laughably
easy. Why? Because the standard data structure in \texttt{Octave} is a
vector or a matrix. In the context of \texttt{Rust}, we may pose these
questions:

\begin{enumerate}
\item
  Does \texttt{Rust} have a ready implementation of vectors that may be
  called upon?
\item
  Would such an implementation be faster? Less error prone? Easier to
  visualize and troubleshoot?
\end{enumerate}

I had a little peep at the possibilities with \texttt{Rust} and realized
that being a
\href{https://en.wikipedia.org/wiki/Comparison_of_multi-paradigm_programming_languages}{multi-paradigm
language}, \texttt{Rust} provides many possibilities to accomplish the
same task. And the choices available will overwhelm a Rust-neophyte like
me. Moreover, once the simplicity of scalars is left behind, the
knowledge curve with vectors is rather steep. So, vectorizing must
promise returns commensurate with the learning effort. I will leave
vectors in \texttt{Rust} for another day.

\hypertarget{fizzbuzz}{%
\subsection{FizzBuzz}\label{fizzbuzz}}

The \href{https://leetcode.com/problems/fizz-buzz/}{FizzBuzz coding
problem} is a natural successor to EulerOne. The original problem, used
in early school to teach multiplication, is stated for coding below:

\begin{quote}
For every integer from \(1\) to \(n\), print \emph{Fizz} if it is
divisible by \(3\), \emph{Buzz} if it is divisible by \(5\), and
\emph{FizzBuzz} if it is divisible by \(15\). Otherwise, do nothing.
{[}For our purposes, we may set an upper limit as \(n < 1000\).{]}
\end{quote}

This is a favourite coding-interview problem because it is simple enough
to reveal the thought processes of the candidate who wrote the program.
Note that we are asked to \emph{sort} the numbers into \emph{four}
groups.

Vectors and set intersections are the easiest way to achieve this, but
\texttt{Rust} presents a steep climb in knowledge acquisition before
even meagre results start trickling in.

With Euler One, we have already computed the sums of multiples of \(3\),
\(5\), and \(15\), which are less than \(1000\). But we did not retain
the multiples themselves as separate entities.

\hypertarget{octave-implementation-of-fizzbuzz}{%
\subsubsection{\texorpdfstring{\texttt{Octave} implementation of
\texttt{FizzBuzz}}{Octave implementation of FizzBuzz}}\label{octave-implementation-of-fizzbuzz}}

In \texttt{Octave}, the implementation of \texttt{FizzBuzz} is starkly
simple. The availability of the
\href{https://en.wikipedia.org/wiki/Complement_(set_theory)\#Relative_complement}{set
difference} as an operation gives us a ready-made solution as shown
below. Of course, I have not printed the output, but the vectors named
\texttt{fizz}, \texttt{buzz} and \texttt{fizzbuzz} contain the numbers
whose elements are associated with these responses.\footnote{If a
  particular number does not reside in any of these three vectors, there
  is no response.} This is more a ``proof-of-concept'' demonstration,
rather than a proper solution, because the logic associating the
response with the number is missing.

\begin{Shaded}
\begin{Highlighting}[]
\CommentTok{\% FizzBuzz}
\NormalTok{threes }\OperatorTok{=}\NormalTok{ [}\FloatTok{3}\OperatorTok{:}\FloatTok{3}\OperatorTok{:}\FloatTok{999}\NormalTok{]}\OperatorTok{;}
\NormalTok{fives }\OperatorTok{=}\NormalTok{ [}\FloatTok{5}\OperatorTok{:}\FloatTok{5}\OperatorTok{:}\FloatTok{995}\NormalTok{]}\OperatorTok{;}
\NormalTok{fifteens }\OperatorTok{=}\NormalTok{ [}\FloatTok{15}\OperatorTok{:}\FloatTok{15}\OperatorTok{:}\FloatTok{990}\NormalTok{]}\OperatorTok{;}
\NormalTok{fizz }\OperatorTok{=} \FunctionTok{setdiff}\NormalTok{ (threes}\OperatorTok{,}\NormalTok{ fifteens)}\OperatorTok{;}
\NormalTok{buzz }\OperatorTok{=} \FunctionTok{setdiff}\NormalTok{ (fives}\OperatorTok{,}\NormalTok{ fifteens)}\OperatorTok{;}
\NormalTok{fizzbuzz }\OperatorTok{=}\NormalTok{ fifteens}\OperatorTok{;}
\end{Highlighting}
\end{Shaded}

\hypertarget{closing-thoughts}{%
\subsection{Closing thoughts}\label{closing-thoughts}}

To learn \texttt{Rust} requires fortitude of mind and heart. It is not
for the timid. It is no swimming-pool language; it plumbs the ocean
deeps. Its power must lie in its apparent versatility. I do not feel any
heart-tug to learn it when \texttt{Ocatve}, like Aladdin's Lamp, is
there to fulfil my programming wishes. But for those who are
professional programmers, I think that missing out on \texttt{Rust}
might be like missing out on the main course in a meal.

\hypertarget{afterword}{%
\subsection{Afterword}\label{afterword}}

After I had written this blog, I came across a
\href{https://iambryanhaney.medium.com/another-unreasonable-deep-dive-into-project-euler-problem-1-51a3a841ad67\#:~:text=The\%20Problem,0\%20modulo\%203\%20or\%205}{fascinating
blog on Euler One} {[}\protect\hyperlink{ref-haney2020}{6}{]}.

He has made the very valid point that whenever we are faced with
products of a constant \(k\) with sums of the first \(n\) numbers, we
may use the formula for the sum of the first \(n\) numbers in closed
form to give us the required product frugally as
\(k\frac{n(n + 1)}{2}\). In this way, we multiply only twice instead of
\(n\) times.\footnote{That is the power of the distributive law.}

I urge you to read the blog to stretch your mental muscles, while at the
same time developing an appreciation for the beauty of mathematics. Try
your own hand at analyzing the seemingly simple Euler One problem and
see whether it gives you insights that you did not have before.

\hypertarget{caveat-lector-or-reader-beware-or-disclaimer}{%
\subsubsection{Caveat Lector! or Reader Beware! or
Disclaimer}\label{caveat-lector-or-reader-beware-or-disclaimer}}

I am learning \texttt{Rust}. What I have written here represents my
efforts at learning. \emph{The code here is not mature, idiomatic
\texttt{Rust} code and should not be construed as such. Do not take it
as an example of how to code in \texttt{Rust}.} Experienced
``Rustaceans'' who find errors are requested to
\href{mailto:feedback.swanlotus@gmail.com}{email me} with their
corrections. \emojifont {😐}\normalfont

\hypertarget{feedback}{%
\subsection{Feedback}\label{feedback}}

Please \href{mailto:feedback.swanlotus@gmail.com}{email me} your
comments and corrections.

\noindent A PDF version of this article is
\href{./rust-euler-one.pdf}{available for download here}:

\begin{small}

\begin{sffamily}

\url{https://swanlotus.netlify.app/blogs/rust-euler-one.pdf}

\end{sffamily}

\end{small}

\hypertarget{bibliography}{%
\section*{References}\label{bibliography}}
\addcontentsline{toc}{section}{References}

\hypertarget{refs}{}
\begin{CSLReferences}{0}{0}
\leavevmode\vadjust pre{\hypertarget{ref-rust}{}}%
\CSLLeftMargin{{[}1{]} }%
\CSLRightInline{---. {Rust}. Retrieved 1 August 2021 from
\url{https://research.mozilla.org/rust/}}

\leavevmode\vadjust pre{\hypertarget{ref-goulding2020}{}}%
\CSLLeftMargin{{[}2{]} }%
\CSLRightInline{Jake Goulding. 2020. {What is Rust and why is it so
popular?} Retrieved 1 August 2021 from
\url{https://stackoverflow.blog/2020/01/20/what-is-rust-and-why-is-it-so-popular/}}

\leavevmode\vadjust pre{\hypertarget{ref-perkel2020}{}}%
\CSLLeftMargin{{[}3{]} }%
\CSLRightInline{Jeffrey M Perkel. 2020. {Why scientists are turning to
Rust}. Despite having a steep learning curve, the programming language
offers speed and safety. Retrieved 1 August 2021 from
\url{https://www.nature.com/articles/d41586-020-03382-2}}

\leavevmode\vadjust pre{\hypertarget{ref-dreimanis2020}{}}%
\CSLLeftMargin{{[}4{]} }%
\CSLRightInline{Gints Dreimanis. 2020. {9 Companies That Use Rust in
Production}. Retrieved 1 August 2021 from
\url{https://serokell.io/blog/rust-companies}}

\leavevmode\vadjust pre{\hypertarget{ref-sas}{}}%
\CSLLeftMargin{{[}5{]} }%
\CSLRightInline{R (Chandra) Chandrasekhar. 2023. {Secrets of Academic
Success}. {Timeless Principles for Lifelong Learning}. Retrieved 24
November 2023 from
\url{https://swanlotus.netlify.app/sas-manuscript/SAS-partial.pdf}}

\leavevmode\vadjust pre{\hypertarget{ref-haney2020}{}}%
\CSLLeftMargin{{[}6{]} }%
\CSLRightInline{Bryan Haney. 2020. {Another Unreasonably Deep Dive into
Project Euler Problem 1}. Retrieved 25 November 2023 from
\url{https://iambryanhaney.medium.com/another-unreasonable-deep-dive-into-project-euler-problem-1-51a3a841ad67\#:~:text=The\%20Problem,0\%20modulo\%203\%20or\%205}}

\end{CSLReferences}



\end{document}
