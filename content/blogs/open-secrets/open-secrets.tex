% Options for packages loaded elsewhere
\PassOptionsToPackage{unicode,linktoc=all}{hyperref}
\PassOptionsToPackage{hyphens}{url}
\PassOptionsToPackage{dvipsnames,svgnames,x11names}{xcolor}
%
\documentclass[
  a4paper,
]{article}
\usepackage{amsmath,amssymb}
\usepackage{iftex}
\ifPDFTeX
  \usepackage[T1]{fontenc}
  \usepackage[utf8]{inputenc}
  \usepackage{textcomp} % provide euro and other symbols
\else % if luatex or xetex
  \usepackage{unicode-math} % this also loads fontspec
  \defaultfontfeatures{Scale=MatchLowercase}
  \defaultfontfeatures[\rmfamily]{Ligatures=TeX,Scale=1}
\fi
\usepackage{lmodern}
\ifPDFTeX\else
  % xetex/luatex font selection
\fi
% Use upquote if available, for straight quotes in verbatim environments
\IfFileExists{upquote.sty}{\usepackage{upquote}}{}
\IfFileExists{microtype.sty}{% use microtype if available
  \usepackage[]{microtype}
  \UseMicrotypeSet[protrusion]{basicmath} % disable protrusion for tt fonts
}{}
\makeatletter
\@ifundefined{KOMAClassName}{% if non-KOMA class
  \IfFileExists{parskip.sty}{%
    \usepackage{parskip}
  }{% else
    \setlength{\parindent}{0pt}
    \setlength{\parskip}{6pt plus 2pt minus 1pt}}
}{% if KOMA class
  \KOMAoptions{parskip=half}}
\makeatother
\usepackage{xcolor}
\usepackage[margin=25mm]{geometry}
\usepackage{color}
\usepackage{fancyvrb}
\newcommand{\VerbBar}{|}
\newcommand{\VERB}{\Verb[commandchars=\\\{\}]}
\DefineVerbatimEnvironment{Highlighting}{Verbatim}{commandchars=\\\{\}}
% Add ',fontsize=\small' for more characters per line
\usepackage{framed}
\definecolor{shadecolor}{RGB}{48,48,48}
\newenvironment{Shaded}{\begin{snugshade}}{\end{snugshade}}
\newcommand{\AlertTok}[1]{\textcolor[rgb]{1.00,0.81,0.69}{#1}}
\newcommand{\AnnotationTok}[1]{\textcolor[rgb]{0.50,0.62,0.50}{\textbf{#1}}}
\newcommand{\AttributeTok}[1]{\textcolor[rgb]{0.80,0.80,0.80}{#1}}
\newcommand{\BaseNTok}[1]{\textcolor[rgb]{0.86,0.64,0.64}{#1}}
\newcommand{\BuiltInTok}[1]{\textcolor[rgb]{0.80,0.80,0.80}{#1}}
\newcommand{\CharTok}[1]{\textcolor[rgb]{0.86,0.64,0.64}{#1}}
\newcommand{\CommentTok}[1]{\textcolor[rgb]{0.50,0.62,0.50}{#1}}
\newcommand{\CommentVarTok}[1]{\textcolor[rgb]{0.50,0.62,0.50}{\textbf{#1}}}
\newcommand{\ConstantTok}[1]{\textcolor[rgb]{0.86,0.64,0.64}{\textbf{#1}}}
\newcommand{\ControlFlowTok}[1]{\textcolor[rgb]{0.94,0.87,0.69}{#1}}
\newcommand{\DataTypeTok}[1]{\textcolor[rgb]{0.87,0.87,0.75}{#1}}
\newcommand{\DecValTok}[1]{\textcolor[rgb]{0.86,0.86,0.80}{#1}}
\newcommand{\DocumentationTok}[1]{\textcolor[rgb]{0.50,0.62,0.50}{#1}}
\newcommand{\ErrorTok}[1]{\textcolor[rgb]{0.76,0.75,0.62}{#1}}
\newcommand{\ExtensionTok}[1]{\textcolor[rgb]{0.80,0.80,0.80}{#1}}
\newcommand{\FloatTok}[1]{\textcolor[rgb]{0.75,0.75,0.82}{#1}}
\newcommand{\FunctionTok}[1]{\textcolor[rgb]{0.94,0.94,0.56}{#1}}
\newcommand{\ImportTok}[1]{\textcolor[rgb]{0.80,0.80,0.80}{#1}}
\newcommand{\InformationTok}[1]{\textcolor[rgb]{0.50,0.62,0.50}{\textbf{#1}}}
\newcommand{\KeywordTok}[1]{\textcolor[rgb]{0.94,0.87,0.69}{#1}}
\newcommand{\NormalTok}[1]{\textcolor[rgb]{0.80,0.80,0.80}{#1}}
\newcommand{\OperatorTok}[1]{\textcolor[rgb]{0.94,0.94,0.82}{#1}}
\newcommand{\OtherTok}[1]{\textcolor[rgb]{0.94,0.94,0.56}{#1}}
\newcommand{\PreprocessorTok}[1]{\textcolor[rgb]{1.00,0.81,0.69}{\textbf{#1}}}
\newcommand{\RegionMarkerTok}[1]{\textcolor[rgb]{0.80,0.80,0.80}{#1}}
\newcommand{\SpecialCharTok}[1]{\textcolor[rgb]{0.86,0.64,0.64}{#1}}
\newcommand{\SpecialStringTok}[1]{\textcolor[rgb]{0.80,0.58,0.58}{#1}}
\newcommand{\StringTok}[1]{\textcolor[rgb]{0.80,0.58,0.58}{#1}}
\newcommand{\VariableTok}[1]{\textcolor[rgb]{0.80,0.80,0.80}{#1}}
\newcommand{\VerbatimStringTok}[1]{\textcolor[rgb]{0.80,0.58,0.58}{#1}}
\newcommand{\WarningTok}[1]{\textcolor[rgb]{0.50,0.62,0.50}{\textbf{#1}}}
\usepackage{longtable,booktabs,array}
\usepackage{calc} % for calculating minipage widths
% Correct order of tables after \paragraph or \subparagraph
\usepackage{etoolbox}
\makeatletter
\patchcmd\longtable{\par}{\if@noskipsec\mbox{}\fi\par}{}{}
\makeatother
% Allow footnotes in longtable head/foot
\IfFileExists{footnotehyper.sty}{\usepackage{footnotehyper}}{\usepackage{footnote}}
\makesavenoteenv{longtable}
\setlength{\emergencystretch}{3em} % prevent overfull lines
\providecommand{\tightlist}{%
  \setlength{\itemsep}{0pt}\setlength{\parskip}{0pt}}
\setcounter{secnumdepth}{-\maxdimen} % remove section numbering
\newlength{\cslhangindent}
\setlength{\cslhangindent}{1.5em}
\newlength{\csllabelwidth}
\setlength{\csllabelwidth}{3em}
\newlength{\cslentryspacingunit} % times entry-spacing
\setlength{\cslentryspacingunit}{\parskip}
\newenvironment{CSLReferences}[2] % #1 hanging-ident, #2 entry spacing
 {% don't indent paragraphs
  \setlength{\parindent}{0pt}
  % turn on hanging indent if param 1 is 1
  \ifodd #1
  \let\oldpar\par
  \def\par{\hangindent=\cslhangindent\oldpar}
  \fi
  % set entry spacing
  \setlength{\parskip}{#2\cslentryspacingunit}
 }%
 {}
\usepackage{calc}
\newcommand{\CSLBlock}[1]{#1\hfill\break}
\newcommand{\CSLLeftMargin}[1]{\parbox[t]{\csllabelwidth}{#1}}
\newcommand{\CSLRightInline}[1]{\parbox[t]{\linewidth - \csllabelwidth}{#1}\break}
\newcommand{\CSLIndent}[1]{\hspace{\cslhangindent}#1}
\ifLuaTeX
\usepackage[bidi=basic]{babel}
\else
\usepackage[bidi=default]{babel}
\fi
\babelprovide[main,import]{british}
% get rid of language-specific shorthands (see #6817):
\let\LanguageShortHands\languageshorthands
\def\languageshorthands#1{}
% $HOME/.pandoc/defaults/latex-header-includes.tex
% Common header includes for both lualatex and xelatex engines.
%
% Preliminaries
%
% \PassOptionsToPackage{rgb,dvipsnames,svgnames}{xcolor}
% \PassOptionsToPackage{main=british}{babel}
\PassOptionsToPackage{english}{selnolig}
\AtBeginEnvironment{quote}{\small}
\AtBeginEnvironment{quotation}{\small}
\AtBeginEnvironment{longtable}{\centering}
%
% Packages that are useful to include
%
\usepackage{graphicx}
\usepackage{subcaption}
\usepackage[inkscapeversion=1]{svg}
\usepackage[defaultlines=4,all]{nowidow}
\usepackage{etoolbox}
\usepackage{fontsize}
\usepackage{newunicodechar}
\usepackage{pdflscape}
\usepackage{fnpct}
\usepackage{parskip}
  \setlength{\parindent}{0pt}
\usepackage[style=american]{csquotes}
% \usepackage{setspace} Use the <fontname-plus.tex> files for setspace
%
\usepackage{hyperref} % cleveref must come AFTER hyperref
\usepackage[capitalize,noabbrev]{cleveref} % Must come after hyperref
% \usepackage{longdivision}
% noto-plus.tex
% Font-setting header file for use with Pandoc Markdown
% to generate PDF via LuaLaTeX.
% The main font is Noto Serif.
% Other main fonts are also available in appropriately named file.
\usepackage{fontspec}
\usepackage{setspace}
\setstretch{1.3}
%
\defaultfontfeatures{Ligatures=TeX,Scale=MatchLowercase,Renderer=Node} % at the start always
%
% For English
% See also https://tex.stackexchange.com/questions/574047/lualatex-amsthm-polyglossia-charissil-error
% We use Node as Renderer for the Latin Font and Greek Font and HarfBuzz as renderer ofr Indic fonts.
%
\babelfont{rm}[Script=Latin,Scale=1]{NotoSerif}% Config is at $HOME/texmf/tex/latex/NotoSerif.fontspec
\babelfont{sf}[Script=Latin]{SourceSansPro}% Config is at $HOME/texmf/tex/latex/SourceSansPro.fontspec
\babelfont{tt}[Script=Latin]{FiraMono}% Config is at $HOME/texmf/tex/latex/FiraMono.fontspec
%
% Sanskrit, Tamil, and Greek fonts
%
\babelprovide[import, onchar=ids fonts]{sanskrit}
\babelprovide[import, onchar=ids fonts]{tamil}
\babelprovide[import, onchar=ids fonts]{greek}
%
\babelfont[sanskrit]{rm}[Scale=1.1,Renderer=HarfBuzz,Script=Devanagari]{NotoSerifDevanagari}
\babelfont[sanskrit]{sf}[Scale=1.1,Renderer=HarfBuzz,Script=Devanagari]{NotoSansDevanagari}
\babelfont[tamil]{rm}[Renderer=HarfBuzz,Script=Tamil]{NotoSerifTamil}
\babelfont[tamil]{sf}[Renderer=HarfBuzz,Script=Tamil]{NotoSansTamil}
\babelfont[greek]{rm}[Script=Greek]{GentiumBookPlus}
%
% Math font
%
\usepackage{unicode-math} % seems not to hurt % fallabck
\setmathfont[bold-style=TeX]{STIX Two Math}
\usepackage{amsmath}
\usepackage{esdiff} % for derivative symbols
% \renewcommand{\mathbf}{\symbf}
%
%
% Other fonts
%
\newfontfamily{\emojifont}{Symbola}
%

\usepackage{titling}
\usepackage{fancyhdr}
    \pagestyle{fancy}
    \fancyhead{}
    \fancyfoot{}
    \renewcommand{\headrulewidth}{0.2pt}
    \renewcommand{\footrulewidth}{0.2pt}
    \fancyhead[LO,RE]{\scshape\thetitle}
    \fancyfoot[CO,CE]{\footnotesize Copyright © 2006\textendash\the\year, R (Chandra) Chandrasekhar}
    \fancyfoot[RE,RO]{\thepage}
%
\usepackage{newunicodechar}
\newunicodechar{√}{\textsf{√}}
\newfontfamily{\regulariconfont}{Font Awesome 6 Free Regular}[Color=Grey]
\newfontfamily{\solidiconfont}{Font Awesome 6 Free Solid}[Color=Grey]
\newfontfamily{\brandsiconfont}{Font Awesome 6 Brands}[Color=Grey]
%
% Direct input of Unicode code points
%
\newcommand{\faEnvelope}{\regulariconfont\ ^^^^f0e0\normalfont}
\newcommand{\faMobile}{\solidiconfont\ ^^^^f3cd\normalfont}
\newcommand{\faLinkedin}{\brandsiconfont\ ^^^^f0e1\normalfont}
\newcommand{\faGithub}{\brandsiconfont\ ^^^^f09b\normalfont}
\newcommand{\faAtom}{\solidiconfont\ ^^^^f5d2\normalfont}
\newcommand{\faPaperPlaneRegular}{\regulariconfont\ ^^^^f1d8\normalfont}
\newcommand{\faPaperPlaneSolid}{\solidiconfont\ ^^^^f1d8\normalfont}

%
% The block below is commented out because of Tofu glyphs in HTML
%
% \newcommand{\faEnvelope}{\regulariconfont\ \normalfont}
% \newcommand{\faMobile}{\solidiconfont\ \normalfont}
% \newcommand{\faLinkedin}{\brandsiconfont\ \normalfont}
% \newcommand{\faGithub}{\brandsiconfont\ \normalfont}
\ifLuaTeX
  \usepackage{selnolig}  % disable illegal ligatures
\fi
\IfFileExists{bookmark.sty}{\usepackage{bookmark}}{\usepackage{hyperref}}
\IfFileExists{xurl.sty}{\usepackage{xurl}}{} % add URL line breaks if available
\urlstyle{sf}
\hypersetup{
  pdftitle={Open Secrets: √2, π, e},
  pdfauthor={R (Chandra) Chandrasekhar},
  pdflang={en-GB},
  colorlinks=true,
  linkcolor={DarkOliveGreen},
  filecolor={Purple},
  citecolor={DarkKhaki},
  urlcolor={Maroon},
  pdfcreator={LaTeX via pandoc}}

\title{Open Secrets: √2, π, e}
\author{R (Chandra) Chandrasekhar}
\date{2023-03-21 | 2023-12-09}

\begin{document}
\maketitle

\thispagestyle{empty}


\hypertarget{i-made-a-wager}{%
\subsection{``I made a wager''}\label{i-made-a-wager}}

``I made a wager'', my friend Solus ``Sol'' Simkin opened up the other
day. We were enjoying the warmth of a spring afternoon, with just the
right balance of light and shade, just the right balance of heat and
cold, and a humidity that could have descended from Heaven. I might have
slipped into the delicious oblivion of sleep, were it not for the fact
that I cherished hearing Sol, even more than I desired sleep. So, I was
all agog to hear him regale yet another tale from his intellectual
adventures.

``Tell me about your wager,'' I responded.

``I think you know of my paternal cousin, once removed, Hieronymus
Septimus Simkin, whom I affectionately call Seven. Well, he is an expert
in matters Microsoft, and is no mean programmer himself. Seven and I
were holidaying once at
\href{https://www.nationalparks.uk/park/loch-lomond-the-trossachs/}{Loch
Lomond} in Scotland.

``We were drinking in the beauty of scenic Nature when a thought struck
me like a thunderbolt. Nature was varied and variegated in a way that
defied monotony. Perhaps, you will remember that
\href{https://en.wikipedia.org/wiki/Leopold_Kronecker}{Leopold
Kronecker} was reputed to have said \emph{`Die ganzen Zahlen hat der
liebe Gott gemacht, alles andere ist Menschenwerk'}, meaning that `God
made the integers, all else is the work of man'
{[}\protect\hyperlink{ref-kronecker}{1}{]}. That instant, I was
convinced that Kronecker was as wrong as wrong could be.

``I mentioned this to Seven, but he laughed it off. A little miffed, I
told Seven that I would convince him that Kronecker was wrong, and was
prepared to bet on it. There are many open secrets of mathematics all
around us. All of them cry out that he was wrong.''

Seven---never one to withdraw from an engagement with
probability---said, ``You are on. I \emph{know} that
\href{https://cloud.google.com/learn/what-is-encryption}{encryption of
secrets} is grounded in the integers. Integers guard our secrets. So,
you are bound to lose.''

``And so began a discussion on the open secrets of mathematics that took
us into the realms of the integers, the rationals, the irrationals,
\(\pi\), \(e\), \(\phi\), etc.,'' continued Sol.

\hypertarget{the-integers-have-their-place}{%
\subsection{The integers have their
place}\label{the-integers-have-their-place}}

Sol told me that he started off with the integers that Kronecker had so
exalted. ``The integers are fundamental because all mathematics begins
with counting. The quantitative fields are all founded on the natural
numbers we count with. And
\href{https://swanlotus.netlify.app/blogs/the-two-most-important-numbers-zero-and-one}{zero
and one are the two most important integers}---that I grant you,'' he
had told Seven. ``But we cannot stop with integers and exclude
everything else.''

\hypertarget{the-square-and-the-circle}{%
\subsection{The square and the circle}\label{the-square-and-the-circle}}

Sol had told Seven that the square is \emph{the} four-sided regular
polygon. If we take a side length equal to one unit, by the theorem of
Pythagoras, we know that its diagonal has a length equal to
\(\sqrt{1^2 + 1^2} = \sqrt{2}\) units. And there are proofs aplenty on
the Web that this number is in no way an integer. Indeed, it is not even
the ratio of two integers. How could something as basic as the diagonal
of a square cause the first chink in Kronecker's armour?

Moving from the finite to the infinite, the circle may be viewed as the
limiting case of a regular polygon of \(n\) sides as \(n \to \infty\).
And if we tried to find out how many radii would fit into the
circumference of a circle, we do not get an integer, or even an exact
fraction, but rather a number that sits between \(6\) and \(7\), having
decimal places without end, namely, \(2\pi \approx 6.283185307\). And
that number is not an integer by a country mile.

``The natural numbers, the integers, and the rationals---all of these
come under Kronecker's integers, but where do we stash \(\sqrt{2}\) and
\(2\pi\) amongst them?'' Sol asked Seven.

He was met with bemused silence.

\hypertarget{how-about-the-number-e}{%
\subsection{\texorpdfstring{How about the number
\(e\)?}{How about the number e?}}\label{how-about-the-number-e}}

Encouraged that he had stupefied Seven right at the start, Sol mounted
his next hobby horse, and expounded on \(e\).

``The number \(e\) is probably \emph{the} most important number after
\(0\) and \(1\). And do you know what it is? It is both
\href{https://mathworld.wolfram.com/IrrationalNumber.html}{irrational}
and
\href{https://en.wikipedia.org/wiki/Transcendental_number}{transcendental}.
If you differentiate or integrate, you will find that the exponential
function \(\exp(x) = e^x\) is an eigenfunction of each operation. If you
look into Nature, \(e\) holds the pride of place in the
\href{https://www.khanacademy.org/math/statistics-probability/modeling-distributions-of-data/normal-distributions-library/a/normal-distributions-review}{normal
distribution}. If you are into
\href{https://www.cns.nyu.edu/~david/handouts/linear-systems/linear-systems.html}{linear
system theory} you cannot escape \(e\).

But what exactly is the value of \(e\)? Again it cannot be confined like
an integer: \(e \approx 2.718281828\) again in a never ending decimal
sequence. This number pervades all of Nature and yet it cannot be
bottled into a finite number of digits! Where are the legions of
integers to duel with this puny expeditionary force of three numbers?
\emph{It appears that Nature prefers the non-integers!}''

``Very poetic and ably said,'' I nodded in appreciation.

\hypertarget{open-secrets}{%
\subsection{Open secrets}\label{open-secrets}}

``Helen Keller is reputed to have exclaimed, when she felt the warm glow
of a wood-fire, that it was the release of sunbeams that had been
trapped long ago in the wood. Her statement is remarkably perceptive,
poetic, and precise,'' Sol continued.

``Unlike ancient sunlight trapped in wood, \(\sqrt{2}\), \(\pi\), and
\(e\), cannot be caged in a finite box. These three numbers---that
pervade Nature---have decimal forms that clearly announce that they are
\emph{not} integers. Their value defies finite expression; only with
symbols may we do them justice.

``Do you know why they are open secrets? They are public, staring at us
from every square, circle, and signal, and yet, their full form is never
revealed. They cannot be contained except in infinity. To know the next
decimal place of \(\sqrt{2}\), or \(\pi\), or \(e\), one needs to
\emph{compute it} using some formula. Or one could look up a table. But
there is no \emph{knowing} that sought after next decimal place, as we
know \(\frac{1}{2} = 0.5\), with as many zeros stacked at the end as we
wish. That sort of closed form is not baked into nature. She prefers the
indescribable exactitude of numbers like \(e\).'' Sol had told Seven.

The rest of Sol's dialogue with Seven was intricately mathematical. I
have recorded it here, not as a dialogue, but as logical
exposition---complete with references---for the benefit of the casual
reader.

\hypertarget{the-square-root-of-two}{%
\subsection{The square root of two}\label{the-square-root-of-two}}

Of the triad we first consider, \(\sqrt{2}\). It is the most within our
grasp. It embodies a number, rather than a symbol, in its expression. It
is the diagonal of a unit square. And we know that its square root must
lie between \(1\) and \(1.5\), whose square is \(2.25\). It may be
evaluated painstakingly using algorithms from the
age-before-calculators. So, let us look at that first.

\hypertarget{manual-extraction-of-2}{%
\subsubsection{Manual extraction of √2}\label{manual-extraction-of-2}}

The manual extraction of square roots is a form of longdivision that is
both tedious and error-prone. The algorithm uses that fact that the
factor \(2\) figures in any square, witness:
\((x + a) = x^2 + 2x +a^2\). So, this particular long division makes use
of this fact at each step in the ``longdivision'' that is done. To see
the end result and the working, please see
\href{https://www.cuemath.com/algebra/square-root-of-2/}{this}
{[}\protect\hyperlink{ref-cuemathsqrt}{2}{]}. For a deeper explanation,
\href{https://www.cantorsparadise.com/the-square-root-algorithm-f97ab5c29d6d}{read
this blog} {[}\protect\hyperlink{ref-ujjwalsingh2021}{3}{]}.

\hypertarget{continued-fractions}{%
\subsection{Continued Fractions}\label{continued-fractions}}

There are basically two ways of representing real numbers: decimals, and
continued fractions
{[}\protect\hyperlink{ref-niven1991}{4}--\protect\hyperlink{ref-simoson2019}{6}{]}.
The latter representation is fascinating as it reveals patterns that are
absent in the

Practically, every irrational, when pressed to computational use, is
really a rational approximation to the irrational, to an accuracy that
serves the purpose. In that sense, Kronecker was not far from the truth.
But the full glory of \(\sqrt{2}\), or \(\pi\), or \(e\) can only be
encapsulated by the symbols we use for them. Every other, rational
expression is but a costumed appearance, not the true persona.

I would like to demonstrate here another method of evaluating
\(\sqrt{2}\), using
\href{https://en.wikipedia.org/wiki/Continued_fraction}{continued
fractions}. The method might seem like sleight of hand, but it is
well-founded, and it is also and example of how integers are used to
tame the irrationals.

Continued fraction are curious mathematical entities that have
surprising properties. They are an alternative representation rational
number representation of real numbers. No finite continued fraction can
equate to an irrational number. But a never-ending continued fraction
expansion can indeed represent an irrational number.

Rational numbers may be represented as continued fractions in \emph{two}
ways {[}{]}. But irrational numbers may

\hypertarget{continued-fraction-expansion-of-2}{%
\subsubsection{Continued fraction expansion of
√2}\label{continued-fraction-expansion-of-2}}

The number \(\sqrt{2}\) is particularly amenable to a simply derived
continued fraction expansion. Consider:
\begin{equation}\protect\hypertarget{eq:sqrt2}{}{
\begin{aligned}
\sqrt{2} &= \sqrt{2} &\text{ [add and subtract $1$ on the RHS]}\\
&= 1 + \sqrt{2} - 1 &\text{ [multiply second term on RHS by $\frac{\sqrt{2}+1}{\sqrt{2}+1} = 1$}]\\
&= 1 + \frac{(\sqrt{2} - 1)(\sqrt{2} + 1)}{\sqrt{2} + 1} &\text{ [difference of two squares]}\\
&= 1 + \frac{1}{1 + \textcolor{Maroon}{\sqrt{2}}}\\
\end{aligned}
}\label{eq:sqrt2}\end{equation} Since the LHS in \cref{eq:sqrt2} is
\(\sqrt{2}\), we may substitute the entire RHS in place of the term
\(\textcolor{Maroon}{\sqrt{2}}\) on the RHS. So doing, we get the
following descending staircase of continued fractions:
\begin{equation}\protect\hypertarget{eq:sqrt2cfinfty}{}{
\begin{aligned}
\sqrt{2} &= 1 + \frac{1}{1 + \textcolor{Maroon}{\sqrt{2}}}\\
&= 1 + \cfrac{1}{1 + \textcolor{Maroon}{1 + \cfrac{1}{1+\sqrt{2}}}}\\
&= 1 + \cfrac{1}{2 + \cfrac{1}{1 + \sqrt{2}}} &\text{ [and recursively substituting for $\sqrt{2}$ again]}\\
&= 1 + \cfrac{1}{2 + \cfrac{1}{1 + 1 + \cfrac{1}{1 + \sqrt{2}}}}\\
&= 1 + \cfrac{1}{2 + \cfrac{1}{2 + \cfrac{1}{1 + \sqrt{2}}}}\\
&= 1 + \cfrac{1}{2 + \cfrac{1}{2 + \cfrac{1}{2 + \cfrac{1}{1+\sqrt{2}}}}}\\
&\hskip 100pt\ddots\\ % Care!
\end{aligned}
}\label{eq:sqrt2cfinfty}\end{equation} The \emph{congruents} or
\emph{approximants} from a continued fraction are partial sums that we
may accumulate as approximations to the irrational number, in our case,
that we seek to represent. Unfurling the continued fractions into
partial sums is a tricky exercise. There are also recurrence relations
for them. In our particular case, we ignore the
\(\frac{1}{1 + \sqrt{2}}\) terms that occur in the \emph{denominator} of
\cref{eq:sqrt2cfinfty} but count the numerator terms to get a sequence
of fractions.

In this way, we start off with \(1\), followed by
\(1 + \frac{1}{2} = \frac{3}{2}\). Working our way down, we encounter
\(1 + \frac{1}{2 + \frac{1}{2}} = 1+\frac{1}{\frac{5}{2}} = 1 + \frac{2}{5} = \frac{7}{5}\).
The next convergent after this, when simplified, is
\(1 + \frac{1}{2+\frac{2}{5}} = 1 + \frac{5}{12} = \frac{17}{12}\).

Sol said that working out these fractions could be a form of torture
unless you are particularly fond of or adept at computing them. He
himself did not relish such hand computations but preferred to program
to get a solution. As it turns out, he was able to get a sequence of the
first eight successive convergents from the \texttt{Julia} code below:

\begin{Shaded}
\begin{Highlighting}[]
\ImportTok{using} \BuiltInTok{Pkg}
\BuiltInTok{Pkg}\NormalTok{.}\FunctionTok{add}\NormalTok{(}\StringTok{"RealContinuedFractions"}\NormalTok{)}
\CommentTok{\#}
\CommentTok{\# Use the above only for the first time.}
\CommentTok{\#}
\ImportTok{using} \BuiltInTok{RealContinuedFractions}
\FunctionTok{convergents}\NormalTok{(}\FunctionTok{contfrac}\NormalTok{(}\FunctionTok{sqrt}\NormalTok{(}\FloatTok{2}\NormalTok{), }\FloatTok{15}\NormalTok{))}
\end{Highlighting}
\end{Shaded}

which gave the following results:

\begin{Shaded}
\begin{Highlighting}[]
\FloatTok{15}\OperatorTok{{-}}\NormalTok{element }\DataTypeTok{Vector}\NormalTok{\{}\DataTypeTok{Rational}\NormalTok{\{}\DataTypeTok{Int64}\NormalTok{\}\}}\OperatorTok{:}
      \FloatTok{1}\OperatorTok{//}\FloatTok{1}
      \FloatTok{3}\OperatorTok{//}\FloatTok{2}
      \FloatTok{7}\OperatorTok{//}\FloatTok{5}
     \FloatTok{17}\OperatorTok{//}\FloatTok{12}
     \FloatTok{41}\OperatorTok{//}\FloatTok{29}
     \FloatTok{99}\OperatorTok{//}\FloatTok{70}
    \FloatTok{239}\OperatorTok{//}\FloatTok{169}
    \FloatTok{577}\OperatorTok{//}\FloatTok{408}
   \FloatTok{1393}\OperatorTok{//}\FloatTok{985}
   \FloatTok{3363}\OperatorTok{//}\FloatTok{2378}
   \FloatTok{8119}\OperatorTok{//}\FloatTok{5741}
  \FloatTok{19601}\OperatorTok{//}\FloatTok{13860}
  \FloatTok{47321}\OperatorTok{//}\FloatTok{33461}
 \FloatTok{114243}\OperatorTok{//}\FloatTok{80782}
 \FloatTok{275807}\OperatorTok{//}\FloatTok{195025}
\end{Highlighting}
\end{Shaded}

The rational fractions above are tabulated with their decimal versions
to provide an idea of how the convergents do indeed converge to the
``benchmark'' decimal value of \(\sqrt{2}\) as available on a
\texttt{Julia}
\href{https://en.wikipedia.org/wiki/Read\%E2\%80\%93eval\%E2\%80\%93print_loop}{REPL},
which is shown below:

\begin{Shaded}
\begin{Highlighting}[]
\FunctionTok{sqrt}\NormalTok{(}\FunctionTok{big}\NormalTok{(}\FloatTok{2}\NormalTok{))}
\FloatTok{1.414213562373095048801688724209698078569671875376948073176679737990732478462102}
\end{Highlighting}
\end{Shaded}

SEE Wolfram Alpha for repeating digits

\hypertarget{tbl:sqrt2convergents}{}
\begin{longtable}[]{@{}
  >{\centering\arraybackslash}p{(\columnwidth - 4\tabcolsep) * \real{0.2340}}
  >{\raggedright\arraybackslash}p{(\columnwidth - 4\tabcolsep) * \real{0.2979}}
  >{\raggedleft\arraybackslash}p{(\columnwidth - 4\tabcolsep) * \real{0.4681}}@{}}
\caption{\label{tbl:sqrt2convergents}The first fifteen convergents for
\(\sqrt{2}\).}\tabularnewline
\toprule\noalign{}
\begin{minipage}[b]{\linewidth}\centering
Convergent
\end{minipage} & \begin{minipage}[b]{\linewidth}\raggedright
Decimal Value
\end{minipage} & \begin{minipage}[b]{\linewidth}\raggedleft
Period
\end{minipage} \\
\midrule\noalign{}
\endfirsthead
\toprule\noalign{}
\begin{minipage}[b]{\linewidth}\centering
Convergent
\end{minipage} & \begin{minipage}[b]{\linewidth}\raggedright
Decimal Value
\end{minipage} & \begin{minipage}[b]{\linewidth}\raggedleft
Period
\end{minipage} \\
\midrule\noalign{}
\endhead
\bottomrule\noalign{}
\endlastfoot
\(\frac{1}{1}\) & \(1.0\) & \(0\) \\
\(\frac{3}{2}\) & \(1.5\) & \(0\) \\
\(\frac{7}{5}\) & \(1.4\) & \(0\) \\
\(\frac{17}{12}\) & \(1.41\overline{6}\) & \(1\) \\
\(\frac{41}{29}\) & \(1.\overline{4137931034482758620689655172}\) &
\(28\) \\
\(\frac{99}{70}\) & \(1.4\overline{142857}\) & \(6\) \\
\(\frac{239}{169}\) &
\(1.\overline{414201183431952662721893491124260355029585798816568047337278106508875739644970}\)
& \(78\) \\
\(\frac{577}{408}\) & \(1.414\overline{2156862745098039}\) & \(16\) \\
\(\frac{1393}{985}\) & \(1.41421319796954314...\) & \(98\) \\
\(\frac{3363}{2378}\) & \(1.4142136248948696\) & \(140\) \\
\(\frac{8119}{5741}\) &
\(1.414213551646054778387906480929814279079437255859375...\) &
\(5740\) \\
\(\frac{19601}{13860}\) & \(1.41\overline{421356}\) & \(6\) \\
\(\frac{47321}{33461}\) &
\(1.414213562057320405784821559791453182697296142578125...\) &
\(4780\) \\
\(\frac{114243}{80782}\) &
\(1.4142135624272734024906538585328414745859226065212547349657101829...\)
& \(546\) \\
\(\frac{275807}{195025}\) &
\(1.4142135623637994701340403480571694672107696533203125...\) &
\(1876\) \\
\end{longtable}

The ``benchmark'' decimal avlue of \(\sqrt{2}\) as available on a
\texttt{Julia}
\href{https://en.wikipedia.org/wiki/Read\%E2\%80\%93eval\%E2\%80\%93print_loop}{REPL}
is shown below. There is agreement at best to four decimal places.

\begin{Shaded}
\begin{Highlighting}[]
\FunctionTok{sqrt}\NormalTok{(}\FunctionTok{big}\NormalTok{(}\FloatTok{2}\NormalTok{))}
\FloatTok{1.414213562373095048801688724209698078569671875376948073176679737990732478462102}
\end{Highlighting}
\end{Shaded}

The results are tabulated for comparison with the decimal value of
\(\sqrt{2}\) which, to 15 decimal places, is \(1.414213562373095\).
\(1.4142135623730951454746218587388284504413604736328125\) Julia
\(1.41421356237309504880168872420969807856967187537694807317667973799073247846210\)
Julia Big Int
\$1.414213562373095048801688724209698078569671875376948073176679737990732478462102
Julia Big Int
\$1.4142135623730950488016887242096980785696718753769480731766797379907324784621070388503875343276415727350138462309122970249248360\ldots{}
Wolfram Alpha \$1.4142135623730951454746218587388284504413604736328125
\#\# Where the rationals and the irrationals meet

``Infinity is where the rationals and irrationals meet,'' Sol had
continues in his discussion with Seven. ``And as far as I know, infinity
s not an integer. What it is, I do not precisely know.'' Take \[
\] https://r-knott.surrey.ac.uk/Fibonacci/cfINTRO.html\#section3.1
https://xlinux.nist.gov/dads/HTML/squareRoot.html
https://math.stackexchange.com/questions/2916718/calculating-the-square-root-of-2
https://medium.com/not-zero/how-to-calculate-square-roots-by-hand-21a78b6da9ae
https://www.cantorsparadise.com/the-square-root-algorithm-f97ab5c29d6d
https://nebusresearch.wordpress.com/2014/10/17/how-richard-feynman-got-from-the-square-root-of-2-to-e/
https://medium.com/i-math/how-to-find-square-roots-by-hand-f3f7cadf94bb
https://en.wikipedia.org/wiki/List\_of\_formulae\_involving\_\%CF\%80

https://mathworld.wolfram.com/PiFormulas.html

https://math.stackexchange.com/questions/2153619/where-do-mathematicians-get-inspiration-for-pi-formulas

Synesthtetic people can see grayness blurring a beautiful landscape if
there is wrong digit in the digit in the decimal expression for pi.

\hypertarget{where-the-rationals-and-the-irrationals-meet}{%
\subsection{Where the rationals and the irrationals
meet}\label{where-the-rationals-and-the-irrationals-meet}}

\hypertarget{what-is-the-next-digit}{%
\subsection{What is the next digit?}\label{what-is-the-next-digit}}

\hypertarget{fibonacci-has-closed-form-but}{%
\subsection{Fibonacci has closed form but
\ldots{}}\label{fibonacci-has-closed-form-but}}

\hypertarget{continues-fractions-pi}{%
\subsection{Continues fractions pi}\label{continues-fractions-pi}}

\hypertarget{how-to-construct-a-rational-from-two-irrationals}{%
\subsection{How to construct a rational from two
irrationals}\label{how-to-construct-a-rational-from-two-irrationals}}

\hypertarget{acknowledgements}{%
\subsection{Acknowledgements}\label{acknowledgements}}

\hypertarget{feedback}{%
\subsection{Feedback}\label{feedback}}

Please \href{mailto:feedback.swanlotus@gmail.com}{email me} your
comments and corrections.

\noindent A PDF version of this article is
\href{./open-secrets.pdf}{available for download here}:

\begin{small}

\begin{sffamily}

\url{https://swanlotus.netlify.app/blogs/open-secrets.pdf}

\end{sffamily}

\end{small}

https://math.stackexchange.com/questions/586008/is-a-decimal-with-a-predictable-pattern-a-rational-number

https://math.stackexchange.com/questions/61937/how-can-i-prove-that-all-rational-numbers-are-either-terminating-decimal-or-repe

https://math.stackexchange.com/questions/1759007/which-real-numbers-have-2-possible-decimal-representations

https://en.wikipedia.org/wiki/Champernowne\_constant

https://math.stackexchange.com/questions/1259073/rational-irrational-numbers

https://en.wikipedia.org/wiki/Liouville\_number

https://en.wikipedia.org/wiki/Transcendental\_number

https://www.khanacademy.org/math/algebra/x2f8bb11595b61c86:irrational-numbers/x2f8bb11595b61c86:sums-and-products-of-rational-and-irrational-numbers/v/sums-and-products-of-irrational-numbers

https://www.google.com/search?q=example+of+a+\%2B+b+\%3D+r+where+a+is+irratioonal\%2C+b+is+irrational+and+r+is+rational

https://www.khanacademy.org/math/algebra/x2f8bb11595b61c86:irrational-numbers/x2f8bb11595b61c86:sums-and-products-of-rational-and-irrational-numbers/v/sums-and-products-of-irrational-numbers

https://www.youtube.com/watch?v=RpDWHlFKHy4

https://www.numberempire.com/3363

https://math.stackexchange.com/questions/730349/convergents-of-square-root-of-2

https://www.youtube.com/watch?v=E4b-k\_Dug\_E

https://www.youtube.com/watch?v=CaasbfdJdJg

https://www.youtube.com/watch?v=zCFF1l7NzVQ

https://math.stackexchange.com/questions/716944/how-to-find-continued-fraction-of-pi

https://perl.plover.com/classes/cftalk/INFO/gosper.html

https://tex.stackexchange.com/questions/334917/box-around-continued-fraction

https://www.quora.com/Why-can-some-irrational-numbers-be-expressed-as-continued-infinite-fractions

https://www.quora.com/What-is-the-difference-between-using-continued-fractions-to-represent-rationals-and-irrationals-Is-continued-fraction-unique

Euler and Lagrange proved that periodic continued fractions represent
quadratic irrational numbers. https://qr.ae/pKUeyc

https://www.quora.com/What-is-the-difference-between-using-continued-fractions-to-represent-rationals-and-irrationals-Is-continued-fraction-unique

https://math.stackexchange.com/questions/1349073/how-to-find-out-the-number-of-repeating-digits-of-a-rational-number-in-decimal-f

https://math.stackexchange.com/questions/140583/compute-the-period-of-a-decimal-number-a-priori

https://proofwiki.org/wiki/Continued\_Fraction\_Expansion\_of\_Irrational\_Square\_Root/Examples/2
Compute the period of a decimal number a priori

Are the numerator and the denominator of a convergent of a continued
fraction always coprime?
https://math.stackexchange.com/questions/1493902/are-the-numerator-and-the-denominator-of-a-convergent-of-a-continued-fraction-al

https://math.stackexchange.com/questions/1493902/are-the-numerator-and-the-denominator-of-a-convergent-of-a-continued-fraction-al

\hypertarget{bibliography}{%
\section*{References}\label{bibliography}}
\addcontentsline{toc}{section}{References}

\hypertarget{refs}{}
\begin{CSLReferences}{0}{0}
\leavevmode\vadjust pre{\hypertarget{ref-kronecker}{}}%
\CSLLeftMargin{{[}1{]} }%
\CSLRightInline{Wikipedia contributors. 2023. {Leopold
Kronecker---Wikipedia, The Free Encyclopedia}. Retrieved 6 December 2023
from \url{https://en.wikipedia.org/wiki/Leopold_Kronecker}}

\leavevmode\vadjust pre{\hypertarget{ref-cuemathsqrt}{}}%
\CSLLeftMargin{{[}2{]} }%
\CSLRightInline{---. 2023. {Square Root of 2}. Retrieved 9 December 2023
from \url{https://www.cuemath.com/algebra/square-root-of-2/}}

\leavevmode\vadjust pre{\hypertarget{ref-ujjwalsingh2021}{}}%
\CSLLeftMargin{{[}3{]} }%
\CSLRightInline{Ujjwal Singh. 2021. {The Square Root Algorithm}.
Retrieved 8 December 2023
from\href{\%0A\%20\%20\%20\%20\%20\%20\%20\%20\%20\%20\%20https://www.cantorsparadise.com/the-square-root-algorithm-f97ab5c29d6d\%0A\%20\%20\%20\%20\%20\%20\%20\%20\%20\%20\%20}{
https://www.cantorsparadise.com/the-square-root-algorithm-f97ab5c29d6d
}}

\leavevmode\vadjust pre{\hypertarget{ref-niven1991}{}}%
\CSLLeftMargin{{[}4{]} }%
\CSLRightInline{Ian Niven, Herbert S Zuckerman, and Hugh L Montgomery.
1991. \emph{{An Introduction to the Theory of Numbers}} (5th ed.). John
Wiley \& Sons.}

\leavevmode\vadjust pre{\hypertarget{ref-davenport2008}{}}%
\CSLLeftMargin{{[}5{]} }%
\CSLRightInline{Harold Davenport and James H Davenport. 2008. \emph{{The
Higher Arithmetic}: {An Introduction to the Theory of Numbers}} (8th
ed.). Cambridge University Press.}

\leavevmode\vadjust pre{\hypertarget{ref-simoson2019}{}}%
\CSLLeftMargin{{[}6{]} }%
\CSLRightInline{Andrew J Simoson. 2019. \emph{{Exploring Continued
Fractions}: {From the Integers to Solar Eclipses}}. MAA Press/American
Mathematical Society.}

\end{CSLReferences}



\end{document}
