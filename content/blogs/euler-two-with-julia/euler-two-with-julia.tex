% Options for packages loaded elsewhere
\PassOptionsToPackage{unicode,linktoc=all}{hyperref}
\PassOptionsToPackage{hyphens}{url}
\PassOptionsToPackage{dvipsnames,svgnames,x11names}{xcolor}
%
\documentclass[
  a4paper,
]{article}
\usepackage{amsmath,amssymb}
\usepackage{iftex}
\ifPDFTeX
  \usepackage[T1]{fontenc}
  \usepackage[utf8]{inputenc}
  \usepackage{textcomp} % provide euro and other symbols
\else % if luatex or xetex
  \usepackage{unicode-math} % this also loads fontspec
  \defaultfontfeatures{Scale=MatchLowercase}
  \defaultfontfeatures[\rmfamily]{Ligatures=TeX,Scale=1}
\fi
\usepackage{lmodern}
\ifPDFTeX\else
  % xetex/luatex font selection
\fi
% Use upquote if available, for straight quotes in verbatim environments
\IfFileExists{upquote.sty}{\usepackage{upquote}}{}
\IfFileExists{microtype.sty}{% use microtype if available
  \usepackage[]{microtype}
  \UseMicrotypeSet[protrusion]{basicmath} % disable protrusion for tt fonts
}{}
\makeatletter
\@ifundefined{KOMAClassName}{% if non-KOMA class
  \IfFileExists{parskip.sty}{%
    \usepackage{parskip}
  }{% else
    \setlength{\parindent}{0pt}
    \setlength{\parskip}{6pt plus 2pt minus 1pt}}
}{% if KOMA class
  \KOMAoptions{parskip=half}}
\makeatother
\usepackage{xcolor}
\usepackage[margin=25mm]{geometry}
\usepackage{color}
\usepackage{fancyvrb}
\newcommand{\VerbBar}{|}
\newcommand{\VERB}{\Verb[commandchars=\\\{\}]}
\DefineVerbatimEnvironment{Highlighting}{Verbatim}{commandchars=\\\{\}}
% Add ',fontsize=\small' for more characters per line
\usepackage{framed}
\definecolor{shadecolor}{RGB}{48,48,48}
\newenvironment{Shaded}{\begin{snugshade}}{\end{snugshade}}
\newcommand{\AlertTok}[1]{\textcolor[rgb]{1.00,0.81,0.69}{#1}}
\newcommand{\AnnotationTok}[1]{\textcolor[rgb]{0.50,0.62,0.50}{\textbf{#1}}}
\newcommand{\AttributeTok}[1]{\textcolor[rgb]{0.80,0.80,0.80}{#1}}
\newcommand{\BaseNTok}[1]{\textcolor[rgb]{0.86,0.64,0.64}{#1}}
\newcommand{\BuiltInTok}[1]{\textcolor[rgb]{0.80,0.80,0.80}{#1}}
\newcommand{\CharTok}[1]{\textcolor[rgb]{0.86,0.64,0.64}{#1}}
\newcommand{\CommentTok}[1]{\textcolor[rgb]{0.50,0.62,0.50}{#1}}
\newcommand{\CommentVarTok}[1]{\textcolor[rgb]{0.50,0.62,0.50}{\textbf{#1}}}
\newcommand{\ConstantTok}[1]{\textcolor[rgb]{0.86,0.64,0.64}{\textbf{#1}}}
\newcommand{\ControlFlowTok}[1]{\textcolor[rgb]{0.94,0.87,0.69}{#1}}
\newcommand{\DataTypeTok}[1]{\textcolor[rgb]{0.87,0.87,0.75}{#1}}
\newcommand{\DecValTok}[1]{\textcolor[rgb]{0.86,0.86,0.80}{#1}}
\newcommand{\DocumentationTok}[1]{\textcolor[rgb]{0.50,0.62,0.50}{#1}}
\newcommand{\ErrorTok}[1]{\textcolor[rgb]{0.76,0.75,0.62}{#1}}
\newcommand{\ExtensionTok}[1]{\textcolor[rgb]{0.80,0.80,0.80}{#1}}
\newcommand{\FloatTok}[1]{\textcolor[rgb]{0.75,0.75,0.82}{#1}}
\newcommand{\FunctionTok}[1]{\textcolor[rgb]{0.94,0.94,0.56}{#1}}
\newcommand{\ImportTok}[1]{\textcolor[rgb]{0.80,0.80,0.80}{#1}}
\newcommand{\InformationTok}[1]{\textcolor[rgb]{0.50,0.62,0.50}{\textbf{#1}}}
\newcommand{\KeywordTok}[1]{\textcolor[rgb]{0.94,0.87,0.69}{#1}}
\newcommand{\NormalTok}[1]{\textcolor[rgb]{0.80,0.80,0.80}{#1}}
\newcommand{\OperatorTok}[1]{\textcolor[rgb]{0.94,0.94,0.82}{#1}}
\newcommand{\OtherTok}[1]{\textcolor[rgb]{0.94,0.94,0.56}{#1}}
\newcommand{\PreprocessorTok}[1]{\textcolor[rgb]{1.00,0.81,0.69}{\textbf{#1}}}
\newcommand{\RegionMarkerTok}[1]{\textcolor[rgb]{0.80,0.80,0.80}{#1}}
\newcommand{\SpecialCharTok}[1]{\textcolor[rgb]{0.86,0.64,0.64}{#1}}
\newcommand{\SpecialStringTok}[1]{\textcolor[rgb]{0.80,0.58,0.58}{#1}}
\newcommand{\StringTok}[1]{\textcolor[rgb]{0.80,0.58,0.58}{#1}}
\newcommand{\VariableTok}[1]{\textcolor[rgb]{0.80,0.80,0.80}{#1}}
\newcommand{\VerbatimStringTok}[1]{\textcolor[rgb]{0.80,0.58,0.58}{#1}}
\newcommand{\WarningTok}[1]{\textcolor[rgb]{0.50,0.62,0.50}{\textbf{#1}}}
\usepackage{longtable,booktabs,array}
\usepackage{calc} % for calculating minipage widths
% Correct order of tables after \paragraph or \subparagraph
\usepackage{etoolbox}
\makeatletter
\patchcmd\longtable{\par}{\if@noskipsec\mbox{}\fi\par}{}{}
\makeatother
% Allow footnotes in longtable head/foot
\IfFileExists{footnotehyper.sty}{\usepackage{footnotehyper}}{\usepackage{footnote}}
\makesavenoteenv{longtable}
\setlength{\emergencystretch}{3em} % prevent overfull lines
\providecommand{\tightlist}{%
  \setlength{\itemsep}{0pt}\setlength{\parskip}{0pt}}
\setcounter{secnumdepth}{-\maxdimen} % remove section numbering
\newlength{\cslhangindent}
\setlength{\cslhangindent}{1.5em}
\newlength{\csllabelwidth}
\setlength{\csllabelwidth}{3em}
\newlength{\cslentryspacingunit} % times entry-spacing
\setlength{\cslentryspacingunit}{\parskip}
\newenvironment{CSLReferences}[2] % #1 hanging-ident, #2 entry spacing
 {% don't indent paragraphs
  \setlength{\parindent}{0pt}
  % turn on hanging indent if param 1 is 1
  \ifodd #1
  \let\oldpar\par
  \def\par{\hangindent=\cslhangindent\oldpar}
  \fi
  % set entry spacing
  \setlength{\parskip}{#2\cslentryspacingunit}
 }%
 {}
\usepackage{calc}
\newcommand{\CSLBlock}[1]{#1\hfill\break}
\newcommand{\CSLLeftMargin}[1]{\parbox[t]{\csllabelwidth}{#1}}
\newcommand{\CSLRightInline}[1]{\parbox[t]{\linewidth - \csllabelwidth}{#1}\break}
\newcommand{\CSLIndent}[1]{\hspace{\cslhangindent}#1}
\ifLuaTeX
\usepackage[bidi=basic]{babel}
\else
\usepackage[bidi=default]{babel}
\fi
\babelprovide[main,import]{british}
% get rid of language-specific shorthands (see #6817):
\let\LanguageShortHands\languageshorthands
\def\languageshorthands#1{}
% $HOME/.pandoc/defaults/latex-header-includes.tex
% Common header includes for both lualatex and xelatex engines.
%
% Preliminaries
%
% \PassOptionsToPackage{rgb,dvipsnames,svgnames}{xcolor}
% \PassOptionsToPackage{main=british}{babel}
\PassOptionsToPackage{english}{selnolig}
\AtBeginEnvironment{quote}{\small}
\AtBeginEnvironment{quotation}{\small}
\AtBeginEnvironment{longtable}{\centering}
%
% Packages that are useful to include
%
\usepackage{graphicx}
\usepackage{subcaption}
\usepackage[inkscapeversion=1]{svg}
\usepackage[defaultlines=4,all]{nowidow}
\usepackage{etoolbox}
\usepackage{fontsize}
\usepackage{newunicodechar}
\usepackage{pdflscape}
\usepackage{fnpct}
\usepackage{parskip}
  \setlength{\parindent}{0pt}
\usepackage[style=american]{csquotes}
% \usepackage{setspace} Use the <fontname-plus.tex> files for setspace
%
\usepackage{hyperref} % cleveref must come AFTER hyperref
\usepackage[capitalize,noabbrev]{cleveref} % Must come after hyperref
% noto-plus.tex
% Font-setting header file for use with Pandoc Markdown
% to generate PDF via LuaLaTeX.
% The main font is Noto Serif.
% Other main fonts are also available in appropriately named file.
\usepackage{fontspec}
\usepackage{setspace}
\setstretch{1.3}
%
\defaultfontfeatures{Ligatures=TeX,Scale=MatchLowercase,Renderer=Node} % at the start always
%
% For English
% See also https://tex.stackexchange.com/questions/574047/lualatex-amsthm-polyglossia-charissil-error
% We use Node as Renderer for the Latin Font and Greek Font and HarfBuzz as renderer ofr Indic fonts.
%
\babelfont{rm}[Script=Latin,Scale=1]{NotoSerif}% Config is at $HOME/texmf/tex/latex/NotoSerif.fontspec
\babelfont{sf}[Script=Latin]{SourceSansPro}% Config is at $HOME/texmf/tex/latex/SourceSansPro.fontspec
\babelfont{tt}[Script=Latin]{FiraMono}% Config is at $HOME/texmf/tex/latex/FiraMono.fontspec
%
% Sanskrit, Tamil, and Greek fonts
%
\babelprovide[import, onchar=ids fonts]{sanskrit}
\babelprovide[import, onchar=ids fonts]{tamil}
\babelprovide[import, onchar=ids fonts]{greek}
%
\babelfont[sanskrit]{rm}[Scale=1.1,Renderer=HarfBuzz,Script=Devanagari]{NotoSerifDevanagari}
\babelfont[sanskrit]{sf}[Scale=1.1,Renderer=HarfBuzz,Script=Devanagari]{NotoSansDevanagari}
\babelfont[tamil]{rm}[Renderer=HarfBuzz,Script=Tamil]{NotoSerifTamil}
\babelfont[tamil]{sf}[Renderer=HarfBuzz,Script=Tamil]{NotoSansTamil}
\babelfont[greek]{rm}[Script=Greek]{GentiumBookPlus}
%
% Math font
%
\usepackage{unicode-math} % seems not to hurt % fallabck
\setmathfont[bold-style=TeX]{STIX Two Math}
\usepackage{amsmath}
\usepackage{esdiff} % for derivative symbols
%
%
% Other fonts
%
\newfontfamily{\emojifont}{Symbola}
%

\usepackage{titling}
\usepackage{fancyhdr}
    \pagestyle{fancy}
    \fancyhead{}
    \fancyfoot{}
    \renewcommand{\headrulewidth}{0.2pt}
    \renewcommand{\footrulewidth}{0.2pt}
    \fancyhead[LO,RE]{\scshape\thetitle}
    \fancyfoot[CO,CE]{\footnotesize Copyright © 2006\textendash\the\year, R (Chandra) Chandrasekhar}
    \fancyfoot[RE,RO]{\thepage}
\newfontfamily{\regulariconfont}{Font Awesome 6 Free Regular}[Color=Grey]
\newfontfamily{\solidiconfont}{Font Awesome 6 Free Solid}[Color=Grey]
\newfontfamily{\brandsiconfont}{Font Awesome 6 Brands}[Color=Grey]
%
% Direct input of Unicode code points
%
\newcommand{\faEnvelope}{\regulariconfont\ ^^^^f0e0\normalfont}
\newcommand{\faMobile}{\solidiconfont\ ^^^^f3cd\normalfont}
\newcommand{\faLinkedin}{\brandsiconfont\ ^^^^f0e1\normalfont}
\newcommand{\faGithub}{\brandsiconfont\ ^^^^f09b\normalfont}
\newcommand{\faAtom}{\solidiconfont\ ^^^^f5d2\normalfont}
\newcommand{\faPaperPlaneRegular}{\regulariconfont\ ^^^^f1d8\normalfont}
\newcommand{\faPaperPlaneSolid}{\solidiconfont\ ^^^^f1d8\normalfont}

%
% The block below is commented out because of Tofu glyphs in HTML
%
% \newcommand{\faEnvelope}{\regulariconfont\ \normalfont}
% \newcommand{\faMobile}{\solidiconfont\ \normalfont}
% \newcommand{\faLinkedin}{\brandsiconfont\ \normalfont}
% \newcommand{\faGithub}{\brandsiconfont\ \normalfont}
\ifLuaTeX
  \usepackage{selnolig}  % disable illegal ligatures
\fi
\IfFileExists{bookmark.sty}{\usepackage{bookmark}}{\usepackage{hyperref}}
\IfFileExists{xurl.sty}{\usepackage{xurl}}{} % add URL line breaks if available
\urlstyle{sf}
\hypersetup{
  pdftitle={Euler Two with Julia},
  pdfauthor={R (Chandra) Chandrasekhar},
  pdflang={en-GB},
  colorlinks=true,
  linkcolor={DarkOliveGreen},
  filecolor={Purple},
  citecolor={DarkKhaki},
  urlcolor={Maroon},
  pdfcreator={LaTeX via pandoc}}

\title{Euler Two with \texttt{Julia}}
\author{R (Chandra) Chandrasekhar}
\date{2023-11-27 | 2023-12-03}

\begin{document}
\maketitle

\thispagestyle{empty}


In a
\href{https://swanlotus.netlify.app/blogs/a-foray-into-rust-euler-one}{recent
blog} I chronicled my efforts at solving the
\href{https://projecteuler.net/problem=1}{Project Euler Problem 1} using
the programming language \texttt{Rust}. At the conclusion of the blog, I
was less enthusiastic about \texttt{Rust} than I was at the start.

Was there another relatively new programming language that was better
suited to my temperament and that held promise to become mainstream? I
thought of \href{https://julialang.org/}{\texttt{Julia}} which has an
impressive pedigree of academics from reputed institutions behind its
invention.

I also happened to watch recently
\href{https://www.youtube.com/watch?v=rgcwvrGGFww}{a talk} by an
academic from India,
\href{https://www.cmi.ac.in/people/fac-profile.php?id=sourish}{Professor
Sourish Das}, belonging to \href{https://www.cmi.ac.in/}{the reputed
Chennai Mathematical Institute}, in which he pitched for Julia as
\emph{the} programming language of the future for the field of
\href{https://en.wikipedia.org/wiki/Data_science}{Data Science}. The
good professor compared Julia with current mainstream languages used in
Data Science, and explained with convincing facts why he was batting for
\texttt{Julia} to be the leader in a few years.

\hypertarget{solving-project-euler-problem-one}{%
\subsection{Solving Project Euler Problem
One}\label{solving-project-euler-problem-one}}

To compare apples with apples, I thought I would first try to solve
Euler Problem 1 using \texttt{Julia}.

The first thing I looked for was vectorization and the use of the simple
\texttt{start:step:end} syntax that was familiar to me from
\href{https://octave.org/}{Octave}. I searched for vectors and ranges
and landed up
\href{https://docs.juliahub.com/CalculusWithJulia/AZHbv/0.0.5/precalc/ranges.html}{here}.

From there, it was almost trivial to solve Euler One with this one-liner
in \texttt{Julia}:

\begin{Shaded}
\begin{Highlighting}[]
\FunctionTok{sum}\NormalTok{(}\FloatTok{3}\OperatorTok{:}\FloatTok{3}\OperatorTok{:}\FloatTok{999}\NormalTok{) }\OperatorTok{+} \FunctionTok{sum}\NormalTok{(}\FloatTok{5}\OperatorTok{:}\FloatTok{5}\OperatorTok{:}\FloatTok{999}\NormalTok{) }\OperatorTok{{-}} \FunctionTok{sum}\NormalTok{(}\FloatTok{15}\OperatorTok{:}\FloatTok{15}\OperatorTok{:}\FloatTok{999}\NormalTok{)}
\end{Highlighting}
\end{Shaded}

Note that we have the same \texttt{end} value for the three vectors
because of the words \texttt{below\ 1000} in the problem. We did not
need to separately compute the number of terms in each case. And the
fact that ranges in vectors are inclusive at both ends, all make for a
simple and succinct solution. The answer, of course, is \texttt{233168}
as before.

With Euler One out of the way, we are now free to analyze and solve
Euler Two using \texttt{Julia}.

\hypertarget{statement-of-problem-euler-two}{%
\subsection{Statement of problem: Euler
Two}\label{statement-of-problem-euler-two}}

The \href{https://projecteuler.net/problem=2}{Euler Project Problem 2}
is stated as follows:

\begin{quote}
Each new term in the Fibonacci sequence is generated by adding the
previous two terms. By starting with \(1\) and \(2\), the first \(10\)
terms will be:
\end{quote}

\begin{quote}
\(1, 2, 3, 5, 8, 13, 21, 34, 55, 89, \ldots\)
\end{quote}

\begin{quote}
By considering the terms in the Fibonacci sequence whose values do not
exceed four million, find the sum of the even-valued terms.
\end{quote}

\hypertarget{parsing-the-problem}{%
\subsubsection{Parsing the problem}\label{parsing-the-problem}}

The problem states ``By starting with 1 and 2,'' indicating that the
starting point is not universally accepted as \(1\) and \(2\).

Indeed, the \href{https://oeis.org/A000045}{Fibonacci sequence} at
\href{https://oeis.org/wiki/Main_Page}{OEIS} starts off like this:
\begin{equation}\protect\hypertarget{eq:fibonacci}{}{
0, 1, 1, 2, 3, 5, 8, 13, 21, 34, 55, 89, 144, 233, 377, \ldots
}\label{eq:fibonacci}\end{equation} I prefer this latter sequence for
aesthetic reasons and will use it in this problem.

The second important phrase asks for the sum of the ``even-valued
terms''. It is important to understand the meaning of ``even'' here. The
\emph{even-numbered} terms are those occupying even \emph{positions} in
the sequence. The \emph{even-valued} terms are those whose \emph{values}
are even. The two need not be the same, and are not the same in this
case.

The third important phrase is ``whose values do not exceed four
million''. A number less than \(x\) does not exceed \(x\). Equally, a
number that equals \(x\) also does not exceed \(x\). So, the
mathematical condition is ``\(\leq 4,000,000\)''. Moreover, the stopping
condition refers to the \emph{whole Fibonacci sequence.} Let this upper
bound be called: \begin{equation}\protect\hypertarget{eq:stopping}{}{
F_{m} = F_{\max} \leq 4,000,000.
}\label{eq:stopping}\end{equation} We do not know the value of either
\(m\) or \(F_{\max}\) at present.

\hypertarget{recurrence-relation-for-the-fibonacci-sequence}{%
\subsection{Recurrence relation for the Fibonacci
sequence}\label{recurrence-relation-for-the-fibonacci-sequence}}

The recurrence relation for the Fibonacci sequence is:
\begin{equation}\protect\hypertarget{eq:recurrence}{}{
\begin{aligned}
F_{1} &= 0\\
F_{2} &= 1\\
F_{3} &= F_{2} + F_{1} = 1\\
F_{4} &= F_{3} + F_{2} = 1 + 1 = 2\\
F_{n} &= F_{n-1} + F_{n-2} \mbox{ for } n \in \mathbb{N} \mbox{ and } n > 2
\end{aligned}
}\label{eq:recurrence}\end{equation}

While there \emph{is} an explicit formula for the
\(n\)\textsuperscript{th} Fibonacci number, called
\href{https://artofproblemsolving.com/wiki/index.php/Binet\%27s_Formula}{Binet's
formula}, its use involves the irrational, algebraic number \(\sqrt5\),
and programs using it will suffer from
\href{https://www.cs.drexel.edu/~popyack/Courses/CSP/Fa17/extras/Rounding/index.html}{rounding
errors}. However, this does not preclude methods based on Binet's
formula, provided they are used knowledgeably.

\hypertarget{the-even-valued-fibonacci-subsequence}{%
\subsection{The even-valued Fibonacci
subsequence}\label{the-even-valued-fibonacci-subsequence}}

If we look at \cref{eq:fibonacci}, we will notice that, assuming zero is
even, the even terms are: \[
0, 2, 8, 34, 144, \ldots
\] and their position in the sequence is \[
1, 4, 7, 10, 13, \ldots
\] spaced at every \emph{three} terms apart. Note that the indices of
the even-valued Fibonacci subsequence actually form an arithmetic
sequence with \(a = 1\) and \(d = 3\) and \(n\)\textsuperscript{th} term
\(1 + (n - 1)3 = 3n - 2\). If we use this sequence to filter out the
Fibonacci sequence and sum it, we will be done. This is one approach. We
will explore other approaches later.

So, the sum we are after, assuming that \cref{eq:recurrence} holds, is
\begin{equation}\protect\hypertarget{eq:even-sum}{}{
\sum_{k=1}^{m}F_{3k-2}.
}\label{eq:even-sum}\end{equation} where \(m\) is the index of the
largest even-valued Fibonacci number that does not exceed 4,000,000. Let
us call this term \(F_{\max}\). We will consider this subsequence in its
own right \protect\hyperlink{a-second-look-at-the-problem}{toward the
end of this blog}.

\hypertarget{small-steps-toward-the-solution}{%
\subsection{Small steps toward the
solution}\label{small-steps-toward-the-solution}}

Because the syntax of \texttt{Julia} is new to me, I will start with
trivial scripts that almost single-step toward the solution.

\hypertarget{append-the-third-fibonacci-number-to-the-array}{%
\subsubsection{Append the third Fibonacci number to the
array}\label{append-the-third-fibonacci-number-to-the-array}}

Let us concatenate the third Fibonacci number to the first two.
Obviously, we need a one-dimensional array , or vector, \texttt{F} to
hold the Fibonacci numbers.

The first two elements of \texttt{F} are pre-defined. So, only the third
element must be defined by the recurrence relation \cref{eq:recurrence},
and added to the \emph{end} of the array, or \emph{appended} to it. The
\href{first-three.jl}{code to do this} is:

\begin{Shaded}
\begin{Highlighting}[]
\CommentTok{\# Append the third Fibonacci number to the Fibonacci array}
\NormalTok{F }\OperatorTok{=}\NormalTok{ [}\FloatTok{0}\NormalTok{, }\FloatTok{1}\NormalTok{];}
\FunctionTok{push!}\NormalTok{(F, F[}\FloatTok{1}\NormalTok{] }\OperatorTok{+}\NormalTok{ F[}\FloatTok{2}\NormalTok{]);}
\FunctionTok{println}\NormalTok{(F);}
\end{Highlighting}
\end{Shaded}

It works and gives us \texttt{{[}0,\ 1,\ 1{]}}. So far so good.

\hypertarget{first-twenty-elements-of-the-fibonacci-sequence}{%
\subsubsection{First twenty elements of the Fibonacci
Sequence}\label{first-twenty-elements-of-the-fibonacci-sequence}}

Because we know that there are twenty elements beforehand, our task is
easier and may be \href{auxiliary/first-twenty.jl}{accomplished by a
\texttt{for} loop}.

\begin{Shaded}
\begin{Highlighting}[]
\CommentTok{\# Generate the first twenty Fibonacci numbers}
\NormalTok{F }\OperatorTok{=}\NormalTok{ [}\FloatTok{0}\NormalTok{, }\FloatTok{1}\NormalTok{];}
\ControlFlowTok{for}\NormalTok{ i }\KeywordTok{in}\NormalTok{ (}\FloatTok{3}\OperatorTok{:}\FloatTok{20}\NormalTok{)}
  \FunctionTok{push!}\NormalTok{(F, F[i}\OperatorTok{{-}}\FloatTok{1}\NormalTok{] }\OperatorTok{+}\NormalTok{ F[i}\OperatorTok{{-}}\FloatTok{2}\NormalTok{]); }\CommentTok{\# append to array}
\ControlFlowTok{end}
\FunctionTok{println}\NormalTok{(F);}
\end{Highlighting}
\end{Shaded}

This gives the first twenty Fibonacci numbers as
\texttt{{[}0,\ 1,\ 1,\ 2,\ 3,\ 5,\ 8,\ 13,\ 21,\ 34,\ 55,\ 89,\ 144,\ 233,\ 377,\ 610,\ 987,\ 1597,\ 2584,\ 4181{]}}.
Array indices in \texttt{Julia} begin with \(1\), and the \texttt{{[}}
and the \texttt{{]}} represent the array delimiters. You can download
the script \href{auxiliary/\%7Dfirst-twenty.jl}{here}.

So, we are good. Notice that we did not even need to define
\texttt{F{[}i{]}\ =\ F{[}i-1{]}\ +\ F{[}i-2{]}} but could simply invoke
the right hand side (RHS) of the recurrence relation and append it. Thus
far, syntax in \texttt{Julia} has tracked the mathematical expression
very closely.

\hypertarget{how-to-stop-when-we-need-to}{%
\subsubsection{How to stop when we need
to?}\label{how-to-stop-when-we-need-to}}

The difficulty is that while we know from \cref{eq:stopping} the upper
bound that should not be exceeded, we know neither the value of the
largest Fibonacci number, \(F_{\max}\), at which we must stop, nor its
index \(m\). We therefore need to allocate an array whose dimensions are
not known in advance. This also precludes the use of the \texttt{for}
iterator for the same reason.

\texttt{Julia} also offers a \texttt{while} loop. As with \texttt{while}
loops in all languages, it must be used with care, because of the
following possible undesirable outcomes, if used erroneously:

\begin{enumerate}
\def\labelenumi{\alph{enumi}.}
\tightlist
\item
  It does not execute even once.
\item
  It stops at one more than the expected condition.
\item
  It stops at one less than the expected condition.
\item
  It loops infinitely.
\end{enumerate}

From what I know so far, the program should preferably use a
\texttt{while} loop that goes on forever and is forcibly terminated when
a known stopping condition is encountered within its body.

Following this approach, here is \href{auxiliary/until-4m.jl}{my
program} to output all the Fibonacci terms which do not exceed
\(4,000,000\).

\begin{Shaded}
\begin{Highlighting}[]
\CommentTok{\# Generate the Fibonacci numbers which do not exceed 4 million}
\KeywordTok{const}\NormalTok{ MAX }\OperatorTok{=} \FloatTok{4000000}\NormalTok{;}
\NormalTok{F }\OperatorTok{=}\NormalTok{ [}\FloatTok{0}\NormalTok{, }\FloatTok{1}\NormalTok{, }\ConstantTok{undef}\NormalTok{]}
\NormalTok{i }\OperatorTok{=} \FloatTok{3} \CommentTok{\# Array already has three elements}
\CommentTok{\#}
\ControlFlowTok{while} \ConstantTok{true}
  \KeywordTok{global}\NormalTok{ i}
\NormalTok{  F[i] }\OperatorTok{=}\NormalTok{ F[i}\OperatorTok{{-}}\FloatTok{1}\NormalTok{] }\OperatorTok{+}\NormalTok{ F[i}\OperatorTok{{-}}\FloatTok{2}\NormalTok{]}
  \CommentTok{\# println("$(i)" , " ", "$(F[i])") \# for troubleshooting}
\NormalTok{  (F[i] }\OperatorTok{+}\NormalTok{ F[i}\OperatorTok{{-}}\FloatTok{1}\NormalTok{]) }\OperatorTok{\textless{}=}\NormalTok{ MAX }\OperatorTok{||} \ControlFlowTok{break}
  \FunctionTok{push!}\NormalTok{(F, F[i]) }\CommentTok{\# append to array}
\NormalTok{  i }\OperatorTok{+=} \FloatTok{1}
\ControlFlowTok{end}
\CommentTok{\#}
\FunctionTok{println}\NormalTok{(i)}
\FunctionTok{println}\NormalTok{(F)}
\end{Highlighting}
\end{Shaded}

\hypertarget{commentary-on-the-program}{%
\subsubsection{Commentary on the
program}\label{commentary-on-the-program}}

This program was written with very little knowledge of \texttt{Julia}.
\emph{It uses the wrong approach, but gives the right answer. It should
not be used as an example to write code in \texttt{Julia}.}

Nevertheless, I will go through the above program one line at a time. We
first define the constant \(4,000,000\) using an uppercase name, as is
prevalent in \texttt{C} as well. Note that we may terminate a line with
a semi-colon, or leave it out, as we please.

The array \texttt{F} is of \texttt{type} \texttt{Any} because we have
not assigned a type to it. While type assignment might matter in other
situations, not assigning it now is simpler for our purposes.

We are within our rights in allocating values to the first two elements
of \texttt{F} because the sequence cannot fire up otherwise.

Note that the third element of \texttt{F} is \texttt{undef} , i.e, it is
left undefined. We need to \emph{reserve a place} for the third element
because we will be evaluating it at the top of the \texttt{while} loop,
and we could run into an \texttt{out\ of\ bounds} error otherwise.

Array indices start at \texttt{1} in \texttt{Julia} and we need
\texttt{i} to be \texttt{3} at the start of the \texttt{while} loop. The
value of \texttt{i} cannot, however, be passed to the \texttt{while}
loop unless we declare it global. I later found out
\href{https://discourse.julialang.org/t/help-with-project-euler-2-undef-inits-printing-multiplication-by-juxtaposition-and-more/106930}{from
the user-community} that declaring variables \texttt{global} is a strict
no-no.

We then progress to the code that actually embodies the recurrence
relation in \cref{eq:recurrence}:
\texttt{F{[}i{]}\ =\ F{[}i-1{]}\ +\ F{[}i-2{]}}. The line after this was
used for troubleshooting and may be uncommented for that purpose if
desired.

The statement
\texttt{(F{[}i{]}\ +\ F{[}i-1{]})\ \textless{}=\ MAX\ \textbar{}\textbar{}\ break}
is the condition that operates the \texttt{while}. To see why it takes
this form, assume that we have the second largest Fibonacci number to be
\(2.5\) million and the next number to be \(3.5\) million. The next
Fibonacci number will therefore be \(6\) million. The loop will surely
stop when it encounters \(6\) million. But will it stop at \(3.5\)
million which is still within the stipulated bound?

Therefore, we need to \emph{evaluate the next Fibonacci number} and test
it against the stopping condition. If it is within bounds, we append it
to the array with \texttt{push!(F,\ F{[}i{]})} and then increment the
array index \texttt{i}. Otherwise, we will abruptly \texttt{break} and
exit the program.

At the end, we print out the last \texttt{i} value, which is \texttt{34}
and the complete Fibonacci sequence whose values do not exceed
\(4,000,000\). The largest permissible Fibonacci number, obeying this
condition, is \texttt{3524578}. The full sequence is:

\texttt{{[}0,\ 1,\ 1,\ 2,\ 3,\ 5,\ 8,\ 13,\ 21,\ 34,\ 55,\ 89,\ 144,\ 233,\ 377,\ 610,\ 987,\ 1597,\ 2584,\ 4181,\ 6765,\ 10946,\ 17711,\ 28657,\ 46368,\ 75025,\ 121393,\ 196418,\ 317811,\ 514229,\ 832040,\ 1346269,\ 2178309,\ 3524578{]}}.
It appears prefixed by the word \texttt{Any} which is an artifact of the
language and should be ignored.

We may now append a second script to sum the even numbers in this
sequence.

\hypertarget{sum-of-even-valued-fibonacci-numbers-not-exceeding-4-million}{%
\subsubsection{\texorpdfstring{Sum of even-valued Fibonacci numbers not
exceeding \(4\)
million}{Sum of even-valued Fibonacci numbers not exceeding 4 million}}\label{sum-of-even-valued-fibonacci-numbers-not-exceeding-4-million}}

We are a stone's throw away from solving the set problem. All we need do
is to sum the even numbers in the array \texttt{F}. The indices of these
are \(1, 4, 7,\) etc. So we \emph{slice} \texttt{F} as shown below to
get the even sequence \texttt{E}. The sum of \texttt{E} is our desired
result.

\begin{Shaded}
\begin{Highlighting}[]
\CommentTok{\#}
\CommentTok{\# Sum of even Fibonacci numbers in F}
\CommentTok{\#}
\NormalTok{E }\OperatorTok{=}\NormalTok{ F[}\FloatTok{1}\OperatorTok{:}\FloatTok{3}\OperatorTok{:}\KeywordTok{end}\NormalTok{]}
\FunctionTok{println}\NormalTok{(E)}
\FunctionTok{println}\NormalTok{(}\FunctionTok{sum}\NormalTok{(E)) }\CommentTok{\# This is what we want}
\end{Highlighting}
\end{Shaded}

The answer to the problem is \texttt{4613732}. For completeness the full
program may be \href{auxiliary/even-sum.jl}{downloaded here}.

\hypertarget{things-i-got-wrong-with-the-above-script}{%
\subsection{Things I got wrong with the above
script}\label{things-i-got-wrong-with-the-above-script}}

Since I am just learning \texttt{Julia}, and had some doubts, I posted a
\href{https://discourse.julialang.org/t/is-it-possible-to-suppres-the-printing-of-any-before-an-array-is-printed/106930}{question}
at the community forum where experienced \texttt{Julia} programmers
respond to queries. Their replies were very definite on what to do and
what not to do, and also revealed a diversity of approaches to solve the
problem.

The experts were in one voice saying that \texttt{global} variables
\emph{should not} be used. They advised me to define a function and use
it instead. Indeed, they even gave me examples of code to solve Euler
Two.

I have slightly modified one of the solutions I received, and list it
below. It uses functions to circumvent the need for \texttt{global}
variables. Please treat my previous solution as a mistake, and consider
the code fragment below as one of many proper solutions.

\begin{Shaded}
\begin{Highlighting}[]
\KeywordTok{function} \FunctionTok{myfib}\NormalTok{(maxval)}
\NormalTok{    val }\OperatorTok{=} \FloatTok{1}
\NormalTok{    F }\OperatorTok{=}\NormalTok{ [}\FloatTok{0}\NormalTok{, val] }\CommentTok{\# assign values to the first two Fibonacci terms}
    \ControlFlowTok{while}\NormalTok{ val }\OperatorTok{\textless{}=}\NormalTok{ maxval}
        \FunctionTok{push!}\NormalTok{(F, val) }\CommentTok{\# the third term is also 1}
\NormalTok{        val }\OperatorTok{=}\NormalTok{ F[}\KeywordTok{end}\NormalTok{] }\OperatorTok{+}\NormalTok{ F[}\KeywordTok{end}\OperatorTok{{-}}\FloatTok{1}\NormalTok{] }\CommentTok{\# the recurrence relation; same as v += F[end{-}1]}
    \ControlFlowTok{end}
    \ControlFlowTok{return}\NormalTok{ F }\CommentTok{\# return the whole array}
\KeywordTok{end}

\CommentTok{\#}
\CommentTok{\# Obtain the Fibonacci sequence for all values not exceeding 4 million}
\CommentTok{\#}
\NormalTok{A }\OperatorTok{=} \FunctionTok{myfib}\NormalTok{(}\FloatTok{4000000}\NormalTok{)}
\FunctionTok{println}\NormalTok{(A, }\StringTok{" "}\NormalTok{, }\FunctionTok{length}\NormalTok{(A)) }\CommentTok{\# The length gives the index of the largest admissible Fibonacci number}

\CommentTok{\#}
\CommentTok{\# Extract and sum the even{-}valued terms}
\CommentTok{\#}
\NormalTok{E }\OperatorTok{=}\NormalTok{ A[}\FloatTok{1}\OperatorTok{:}\FloatTok{3}\OperatorTok{:}\KeywordTok{end}\NormalTok{]}
\FunctionTok{println}\NormalTok{(}\FunctionTok{sum}\NormalTok{(E))}
\end{Highlighting}
\end{Shaded}

The sum of the even-valued terms in the Fibonacci sequence not exceeding
four million is \texttt{4613732}. This file may be accessed
\href{auxiliary/even-f-sum.jl}{here}.

\hypertarget{a-second-look-at-the-problem}{%
\subsection{A second look at the
problem}\label{a-second-look-at-the-problem}}

In our first pass, we stuck very closely to the Fibonacci sequence,
evaluated it in full until the prescribed limit, and then obtained the
even-valued sum. From a programming point 0f view, this may be deemed
``naive'' and ``wasteful'' in a way because:

\begin{enumerate}
\item
  We evaluate the \emph{entire} sequence and throw away more than half
  the terms. It would be more efficient to obtain a \emph{recurrence
  relation for the even-valued subsequence} and use that instead.
\item
  After assembling the array, we sum it to get a \emph{single} number.
  We are aggregating a vector into a scalar. Would it not be better to
  \emph{sum as we go along} and keep accumulating the result in a single
  scalar, rather than maintaining a vector that we sum at the end?
\end{enumerate}

These observations are generic keys to efficient programming, and could
result in more frugal programs in terms of memory and execution times.
This ``philosophy'' will help us write better programs. The rest of this
blog explores this approach.

\hypertarget{the-even-valued-fibonacci-subsequence-1}{%
\subsubsection{The even-valued Fibonacci
subsequence}\label{the-even-valued-fibonacci-subsequence-1}}

Let us call the \emph{even-valued subsequence of the Fibonacci sequence}
the \emph{even sequence} for short, remembering that it is not the
\emph{even terms} but the \emph{even values} that we are after.

Two important factors will guide our efforts at getting a recurrence
relation for the even sequence:

\begin{enumerate}
\item
  The original recurrence relation applies to the Fibonacci sequence.
\item
  The even sequence is predictably distributed in the Fibonacci
  sequence, with even values every three places.
\end{enumerate}

This means that we should be able to get a recurrence relation by
writing the relationships in terms of the even sequence alone, starting
from the full Fibonacci sequence. The even-valued terms are three
positions apart. So, we span seven terms from index \(k\) to index
\((k+6)\) to get the recurrence relation:
\begin{equation}\protect\hypertarget{eq:even-fibonacci-recurrence}{}{
\begin{aligned}
F_{k+6} &= F_{k+5} + F_{k+4} \mbox{ (expand RHS solely in terms of $F_{k+3}$ and $F_{k}$)}\\
&= F_{k+4} + F_{k+3} + F_{k+3} + F_{k+2}\\
&= F_{k+3} + F_{k+2} + F_{k+3} + F_{k+3} + F_{k+2} \mbox{ (expand $F_{k+2} = F_{k+1} + F_{k}$)}\\
&= 3F_{k+3} + F_{k+2} + F_{k+1} + F_{k} \mbox{ (since $F_{k+3} = F_{k+2} + F_{k+1}$)}\\
&= 3F_{k+3} + F_{k+3} + F_{k}\\
&= 4F_{k+3} + F_{k}\\
\end{aligned}
}\label{eq:even-fibonacci-recurrence}\end{equation} If we write
\cref{eq:even-fibonacci-recurrence} in terms of the even sequence,
\(E_{n}\), noting that indices \(k\), \((k+3)\), and \((k+6)\) represent
its successive terms, which we shall call \(n\), \((n+1)\), and
\((n+2)\), we get from \cref{eq:even-fibonacci-recurrence},
\begin{equation}\protect\hypertarget{eq:even-recurrence}{}{
E_{n+2} = 4E_{n+1} +E_{n}
}\label{eq:even-recurrence}\end{equation} We need to initialize
\(E_{1} = 0\) and \(E_{2} = 2\). Thenceforth, we may generate the whole
even sequence. Let us write this using \texttt{Julia} syntax.

\hypertarget{summing-the-even-valued-sequence-in-julia}{%
\subsubsection{\texorpdfstring{Summing the even-valued sequence in
\texttt{Julia}}{Summing the even-valued sequence in Julia}}\label{summing-the-even-valued-sequence-in-julia}}

Three variables are used in the recurrence relation. A fourth is needed
for the sum. So, with four variables and some judicious code, we should
be able to solve the problem directly, efficiently, and fast.

\begin{Shaded}
\begin{Highlighting}[]
\KeywordTok{function} \FunctionTok{fibonacci\_even\_sum}\NormalTok{(maxval)}
\NormalTok{  current, previous }\OperatorTok{=} \FloatTok{2}\NormalTok{, }\FloatTok{0}
\NormalTok{  sum }\OperatorTok{=}\NormalTok{ current }\OperatorTok{+}\NormalTok{ previous}
\NormalTok{  next }\OperatorTok{=} \FloatTok{4}\NormalTok{current }\OperatorTok{+}\NormalTok{ previous }\CommentTok{\# recurrence relation for even{-}valued Fibonacci numbers}
  \ControlFlowTok{while}\NormalTok{ (next }\OperatorTok{\textless{}=}\NormalTok{ maxval)}
\NormalTok{    sum }\OperatorTok{+=}\NormalTok{ next}
\NormalTok{    current, previous }\OperatorTok{=}\NormalTok{ next, current}
\NormalTok{    next }\OperatorTok{=} \FloatTok{4}\NormalTok{current }\OperatorTok{+}\NormalTok{ previous }\CommentTok{\# recurrence relation for even{-}valued Fibonacci numbers}
  \ControlFlowTok{end}
  \ControlFlowTok{return}\NormalTok{ sum}
\KeywordTok{end}
\end{Highlighting}
\end{Shaded}

Observe the following:

\begin{enumerate}
\item
  The assignment \texttt{current,\ previous\ =\ 2,\ 0} is idiom for
  \texttt{(current,\ previous)\ =\ (2,\ 0)} and is the syntax for
  \href{https://discourse.julialang.org/t/correct-way-to-do-multiple-variable-assignments/30242}{\texttt{tuple}}
  assignment in \texttt{Julia}. One could either use the parentheses or
  leave them out altogether.
\item
  The expression \texttt{4current} is a
  \href{https://docs.julialang.org/en/v1/manual/integers-and-floating-point-numbers/\#man-numeric-literal-coefficients}{shorthand}
  for \texttt{4*current} and is yet another aspect of the language that
  makes its syntax mathematics-friendly and nimble.
\item
  The \texttt{while} condition need not have parentheses, although I
  have shown it here with them.
\item
  Lines need not end with a \texttt{;} and the return value need not be
  wrapped in parentheses. Code written without syntactic clutter like
  this is easier on the eye and also simpler to decode to reveal the
  underlying algorithm.
\end{enumerate}

The function must be called separately from its definition and the value
of the sum should be printed out. Therefore, if we append the following
two lines to the above function, we would get our final result:

\begin{Shaded}
\begin{Highlighting}[]
\NormalTok{evensum }\OperatorTok{=} \FunctionTok{fibonacci\_even\_sum}\NormalTok{(}\FloatTok{4e6}\NormalTok{) }\CommentTok{\# 4 million}
\FunctionTok{println}\NormalTok{(evensum)}
\end{Highlighting}
\end{Shaded}

which gives \texttt{4613732} as before, and all is well.

\hypertarget{final-assessment}{%
\subsection{Final assessment}\label{final-assessment}}

The \texttt{Julia} programming language is refreshingly original in its
syntax and allows the programmer to solve the problem in very many ways.

In the case of Euler Project Problem 2, I found out that I ran into
trouble, mostly because I was running foul of doing things the ``right
way''. The language gently nudges one to think again before coding. It
coaxes rather than coerces the programmer to adopt efficient and safe
coding practices.

The existence of a knowledgeable user-community who were ready to help,
and who could illuminate the problem from different angles, made
learning \texttt{Julia} enjoyable, educational, and enriching. It is a
language that I will spend time learning properly, and use in the
future.

\hypertarget{caveat-lector-or-reader-beware-or-disclaimer}{%
\subsubsection{Caveat Lector! or Reader Beware! or
Disclaimer}\label{caveat-lector-or-reader-beware-or-disclaimer}}

I am new to \texttt{Julia}, and what I have written here represents my
efforts at learning---errors and all. Experienced ``Julians''\footnote{\texttt{Rust}
  pogrammers call themselves ``Rustaceans''. \texttt{Python} programmers
  call themselves ``Pythonistas''. I propose that \texttt{Julia}
  programmers call themselves ``Julians''.} who find errors are
requested to \href{mailto:feedback.swanlotus@gmail.com}{email me} with
their corrections. \emojifont {☹️} \normalfont

\hypertarget{acknowledgements}{%
\subsection{Acknowledgements}\label{acknowledgements}}

The \texttt{Julia}-user community is knowledgeable, courteous, and
helpful, and they are very enthusiastic about their programming
language. The website to ask for help is
\href{https://discourse.julialang.org/}{Julia Programming Language - A
forum for users and developers}.

When I encountered difficulties with my code, I sought the community's
help in
\href{https://discourse.julialang.org/t/help-with-project-euler-2-undef-inits-printing-multiplication-by-juxtaposition-and-more/106930}{this
thread} {[}\protect\hyperlink{ref-julia-help}{1}{]}. The wealth of
information given there is enough to keep one busy for quite a while,
learning different ways to \emph{solve} the same problem and also
different ways to \emph{think} about problems in general.

\hypertarget{feedback}{%
\subsection{Feedback}\label{feedback}}

Please \href{mailto:feedback.swanlotus@gmail.com}{email me} your
comments and corrections.

\noindent A PDF version of this article is
\href{auxiliary/euler-two-with-julia.pdf}{available for download here}:

\begin{small}

\begin{sffamily}

\url{https://swanlotus.netlify.app/blogs/euler-two-with-julia.pdf}

\end{sffamily}

\end{small}

\hypertarget{bibliography}{%
\section*{References}\label{bibliography}}
\addcontentsline{toc}{section}{References}

\hypertarget{refs}{}
\begin{CSLReferences}{0}{0}
\leavevmode\vadjust pre{\hypertarget{ref-julia-help}{}}%
\CSLLeftMargin{{[}1{]} }%
\CSLRightInline{Various. 2023. {Help with Project Euler \#2\: undef
inits, printing, multiplication by juxtaposition, and more}. Retrieved 2
December 2023 from
\url{https://discourse.julialang.org/t/help-with-project-euler-2-undef-inits-printing-multiplication-by-juxtaposition-and-more/106930}}

\end{CSLReferences}



\end{document}
