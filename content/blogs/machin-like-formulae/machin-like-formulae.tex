% Options for packages loaded elsewhere
\PassOptionsToPackage{unicode,linktoc=all}{hyperref}
\PassOptionsToPackage{hyphens}{url}
\PassOptionsToPackage{dvipsnames,svgnames,x11names}{xcolor}
%
\documentclass[
  a4paper,
]{article}
\usepackage{amsmath,amssymb}
\usepackage{iftex}
\ifPDFTeX
  \usepackage[T1]{fontenc}
  \usepackage[utf8]{inputenc}
  \usepackage{textcomp} % provide euro and other symbols
\else % if luatex or xetex
  \usepackage{unicode-math} % this also loads fontspec
  \defaultfontfeatures{Scale=MatchLowercase}
  \defaultfontfeatures[\rmfamily]{Ligatures=TeX,Scale=1}
\fi
\usepackage{lmodern}
\ifPDFTeX\else
  % xetex/luatex font selection
\fi
% Use upquote if available, for straight quotes in verbatim environments
\IfFileExists{upquote.sty}{\usepackage{upquote}}{}
\IfFileExists{microtype.sty}{% use microtype if available
  \usepackage[]{microtype}
  \UseMicrotypeSet[protrusion]{basicmath} % disable protrusion for tt fonts
}{}
\makeatletter
\@ifundefined{KOMAClassName}{% if non-KOMA class
  \IfFileExists{parskip.sty}{%
    \usepackage{parskip}
  }{% else
    \setlength{\parindent}{0pt}
    \setlength{\parskip}{6pt plus 2pt minus 1pt}}
}{% if KOMA class
  \KOMAoptions{parskip=half}}
\makeatother
\usepackage{xcolor}
\usepackage[margin=25mm]{geometry}
\usepackage{longtable,booktabs,array}
\usepackage{calc} % for calculating minipage widths
% Correct order of tables after \paragraph or \subparagraph
\usepackage{etoolbox}
\makeatletter
\patchcmd\longtable{\par}{\if@noskipsec\mbox{}\fi\par}{}{}
\makeatother
% Allow footnotes in longtable head/foot
\IfFileExists{footnotehyper.sty}{\usepackage{footnotehyper}}{\usepackage{footnote}}
\makesavenoteenv{longtable}
\setlength{\emergencystretch}{3em} % prevent overfull lines
\providecommand{\tightlist}{%
  \setlength{\itemsep}{0pt}\setlength{\parskip}{0pt}}
\setcounter{secnumdepth}{-\maxdimen} % remove section numbering
% definitions for citeproc citations
\NewDocumentCommand\citeproctext{}{}
\NewDocumentCommand\citeproc{mm}{%
  \begingroup\def\citeproctext{#2}\cite{#1}\endgroup}
\makeatletter
 % allow citations to break across lines
 \let\@cite@ofmt\@firstofone
 % avoid brackets around text for \cite:
 \def\@biblabel#1{}
 \def\@cite#1#2{{#1\if@tempswa , #2\fi}}
\makeatother
\newlength{\cslhangindent}
\setlength{\cslhangindent}{1.5em}
\newlength{\csllabelwidth}
\setlength{\csllabelwidth}{3em}
\newenvironment{CSLReferences}[2] % #1 hanging-indent, #2 entry-spacing
 {\begin{list}{}{%
  \setlength{\itemindent}{0pt}
  \setlength{\leftmargin}{0pt}
  \setlength{\parsep}{0pt}
  % turn on hanging indent if param 1 is 1
  \ifodd #1
   \setlength{\leftmargin}{\cslhangindent}
   \setlength{\itemindent}{-1\cslhangindent}
  \fi
  % set entry spacing
  \setlength{\itemsep}{#2\baselineskip}}}
 {\end{list}}
\usepackage{calc}
\newcommand{\CSLBlock}[1]{\hfill\break#1\hfill\break}
\newcommand{\CSLLeftMargin}[1]{\parbox[t]{\csllabelwidth}{\strut#1\strut}}
\newcommand{\CSLRightInline}[1]{\parbox[t]{\linewidth - \csllabelwidth}{\strut#1\strut}}
\newcommand{\CSLIndent}[1]{\hspace{\cslhangindent}#1}
\ifLuaTeX
\usepackage[bidi=basic]{babel}
\else
\usepackage[bidi=default]{babel}
\fi
\babelprovide[main,import]{british}
% get rid of language-specific shorthands (see #6817):
\let\LanguageShortHands\languageshorthands
\def\languageshorthands#1{}
% $HOME/.pandoc/defaults/latex-header-includes.tex
% Common header includes for both lualatex and xelatex engines.
%
% Preliminaries
%
% \PassOptionsToPackage{rgb,dvipsnames,svgnames}{xcolor}
% \PassOptionsToPackage{main=british}{babel}
\PassOptionsToPackage{english}{selnolig}
\AtBeginEnvironment{quote}{\small}
\AtBeginEnvironment{quotation}{\small}
\AtBeginEnvironment{longtable}{\centering}
%
% Packages that are useful to include
%
\usepackage{graphicx}
\usepackage{subcaption}
\usepackage[inkscapeversion=auto]{svg}
\usepackage[defaultlines=4,all]{nowidow}
\usepackage{etoolbox}
\usepackage{fontsize}
\usepackage{newunicodechar}
\usepackage{pdflscape}
\usepackage{fnpct}
\usepackage{parskip}
  \setlength{\parindent}{0pt}
\usepackage[style=american]{csquotes}
% \usepackage{setspace} Use the <fontname-plus.tex> files for setspace
%
\usepackage{hyperref} % cleveref must come AFTER hyperref
\usepackage[capitalize,noabbrev]{cleveref} % Must come after hyperref
\let\longdivision\relax
\usepackage{longdivision}
\newcommand{\dd}{\ensuremath{mathrm d}}
%
% Assume that amsmath is already loaded via \usepackage{amsmath}
% in the standard LaTeX template foe Pandoc
% 
\DeclareMathOperator{\sech}{sech}
\DeclareMathOperator{\csch}{csch}
\DeclareMathOperator{\arcsec}{arcsec}
\DeclareMathOperator{\arccot}{arccot}
\DeclareMathOperator{\arccsc}{arccsc}
\DeclareMathOperator{\arccosh}{arccosh}
\DeclareMathOperator{\arcsinh}{arcsinh}
\DeclareMathOperator{\arctanh}{arctanh}
\DeclareMathOperator{\arcsech}{arcsech}
\DeclareMathOperator{\arccsch}{arccsch}
\DeclareMathOperator{\arccoth}{arccoth} 
% noto-plus.tex
% Font-setting header file for use with Pandoc Markdown
% to generate PDF via LuaLaTeX.
% The main font is Noto Serif.
% Other main fonts are also available in appropriately named file.
\usepackage{fontspec}
\usepackage{setspace}
\setstretch{1.3}
%
\defaultfontfeatures{Ligatures=TeX,Scale=MatchLowercase,Renderer=Node} % at the start always
%
% For English
% See also https://tex.stackexchange.com/questions/574047/lualatex-amsthm-polyglossia-charissil-error
% We use Node as Renderer for the Latin Font and Greek Font and HarfBuzz as renderer ofr Indic fonts.
%
\babelfont{rm}[Script=Latin,Scale=1]{NotoSerif}% Config is at $HOME/texmf/tex/latex/NotoSerif.fontspec
\babelfont{sf}[Script=Latin]{SourceSansPro}% Config is at $HOME/texmf/tex/latex/SourceSansPro.fontspec
\babelfont{tt}[Script=Latin]{FiraMono}% Config is at $HOME/texmf/tex/latex/FiraMono.fontspec
%
% Sanskrit, Tamil, and Greek fonts
%
\babelprovide[import, onchar=ids fonts]{sanskrit}
\babelprovide[import, onchar=ids fonts]{tamil}
\babelprovide[import, onchar=ids fonts]{greek}
%
\babelfont[sanskrit]{rm}[Scale=1.1,Renderer=HarfBuzz,Script=Devanagari]{NotoSerifDevanagari}
\babelfont[sanskrit]{sf}[Scale=1.1,Renderer=HarfBuzz,Script=Devanagari]{NotoSansDevanagari}
\babelfont[tamil]{rm}[Renderer=HarfBuzz,Script=Tamil]{NotoSerifTamil}
\babelfont[tamil]{sf}[Renderer=HarfBuzz,Script=Tamil]{NotoSansTamil}
\babelfont[greek]{rm}[Script=Greek]{GentiumBookPlus}
%
% Math font
%
\usepackage{unicode-math} % seems not to hurt % fallabck
\setmathfont[bold-style=TeX]{STIX Two Math}
\usepackage{amsmath}
\usepackage{esdiff} % for derivative symbols
% \renewcommand{\mathbf}{\symbf}
%
%
% Other fonts
%
\newfontfamily{\emojifont}{Symbola}
%

\usepackage{titling}
\usepackage{fancyhdr}
    \pagestyle{fancy}
    \fancyhead{}
    \fancyfoot{}
    \renewcommand{\headrulewidth}{0.2pt}
    \renewcommand{\footrulewidth}{0.2pt}
    \fancyhead[LO,RE]{\scshape\thetitle}
    \fancyfoot[CO,CE]{\footnotesize Copyright © 2006\textendash\the\year, R (Chandra) Chandrasekhar}
    \fancyfoot[RE,RO]{\thepage}
%
\usepackage{newunicodechar}
\newunicodechar{√}{\textsf{√}}
\usepackage {caption}
    \captionsetup{font={sf,stretch=1.4}}
\ifLuaTeX
  \usepackage{selnolig}  % disable illegal ligatures
\fi
\IfFileExists{bookmark.sty}{\usepackage{bookmark}}{\usepackage{hyperref}}
\IfFileExists{xurl.sty}{\usepackage{xurl}}{} % add URL line breaks if available
\urlstyle{sf}
\hypersetup{
  pdftitle={Machin-like formulae},
  pdfauthor={R (Chandra) Chandrasekhar},
  pdflang={en-GB},
  colorlinks=true,
  linkcolor={DarkGreen},
  filecolor={Purple},
  citecolor={Teal},
  urlcolor={Maroon},
  pdfcreator={LaTeX via pandoc}}

\title{Machin-like formulae}
\author{R (Chandra) Chandrasekhar}
\date{2024-11-30 | 2024-07-25}

\begin{document}
\maketitle

\thispagestyle{empty}


\subsection{Detour: What does arctan
mean?}\label{detour-what-does-arctan-mean}

We know from high school that the isosceles right-angled triangle and
the 30/60/90 right triangle give rise to these identities, where angles
are expressed in radians:
\begin{equation}\phantomsection\label{eq:arctan}{
\begin{aligned}
\tan\frac{\pi}{3} &= \sqrt{3} &\implies &\arctan\sqrt{3} &= \frac{\pi}{3}\\
\tan\frac{\pi}{4} &= 1 &\implies &\arctan 1 &= \frac{\pi}{4}\\
\tan\frac{\pi}{6} &= \frac{\sqrt{3}}{3} &\implies &\arctan\frac{\sqrt{3}}{3} &= \frac{\pi}{6}\\
\end{aligned}
}\end{equation} Note in \cref{eq:arctan} that we have an irrational
angle \(\frac{\pi}{4}\) giving rise to the rational tangent \(1\). The
other commonly known tangents have both angles and values as irrational.
This is why the angle \(\frac{\pi}{4}\) is so special in algorithms
involving arctangents.

It is my suspicion that the the prefix \emph{arc} is applied to the
tangent to denote the arc or angle corresponding to a tangent. Recall
that the angle in radians is proportional to arc length:
\(\theta = \frac{s}{r}\) where \(\theta\) is the angle, \(s\) the length
of arc subtending the angle, and \(r\) the radius.\footnote{See
  \href{https://swanlotus.netlify.app/blogs/a-tale-of-two-measures-degrees-and-radians}{``A
  tale of two measures: degrees and radians''}.}

In formulae for computing \(\pi\) efficiently and accurately,
mathematicians have been on the lookout for \emph{linear combinations of
rational arctangents that sum to a multiple of \(\frac{\pi}{4}\)}. Once
this guiding principle has been grasped, we will be better equipped to
assess different formulae that have been developed for evaluating
\(\pi\) better, especially those based on \(\arctan\).

\subsubsection{Rational points on the unit
circles}\label{rational-points-on-the-unit-circles}

\subsubsection{Rational fractions of
π}\label{rational-fractions-of-ux3c0}

\subsubsection{How to choose the intersection
set?}\label{how-to-choose-the-intersection-set}

\subsection{Sums of angles}\label{sums-of-angles}

The \(\arctan\) function in \cref{eq:madgregleib} holds the key to a
more solid understanding of what is happening in infinite series
involving \(\pi\). The expression \(\arctan 1\) refers to the
\emph{angle} whose tangent is \(1\), with the implicit understanding
that the angle lies in the interval \([-\frac{\pi}{2}, \frac{\pi}{2}]\).

The next great breakthrough occurred when the single angle \(\arctan 1\)
could be split into sums or differences of other angles. You might think
that adding more terms to the computation would increase computation
time and lower accuracy. But if the sum is \emph{judicously contrived}
with numbers that are either large and whose terms decay rather fast, or
whose powers are easily computed, then accurate and speedy computation
by hand become feasible. And the whole subject of \emph{Machin-like}
formulae--the Holy Grail---of the \href{}{Pi-Chasers} is simply the
quest for parsimony in calculation coupled with accuracy in result.

And it all amounts to splitting an angle, \(\frac{\pi}{4}\) to be
precise, into smaller fragments to our advantage.

The three \(\arctan\) arguments in \cref{eq:machin-formula} are all
rational. If we substitute the arguments in the RHS of
\cref{eq:machin-formula} into the variable in \cref{eq:madgregleib}, we
get: \begin{equation}\phantomsection\label{eq:machin-pi-series}{
\begin{aligned}
\pi &= 16\left[\frac{1}{1\cdot5^1} - \frac{1}{3\cdot5^3} + \frac{1}{5\cdot5^5} - \frac{1}{7\cdot5^7} + \dots \right]\\
&+ 4\left[\frac{1}{1\cdot239^1} - \frac{1}{3\cdot239^3} + \frac{1}{5\cdot239^5} - \frac{1}{7\cdot239^7} + \dots \right]
\end{aligned}
}\end{equation} We have deliberately used a notation that brings out the
pattern: a term raised to the power one is explicitly shown as so.

Observe that \(\left[{\frac{1}{p}}\right]^n = \frac{1}{p^n}\). Having a
unit numerator and large integers as denominators assists in computation
because fewer terms have to be evaluated for a good estimate.

\begin{enumerate}
\tightlist
\item
  Substituting
\end{enumerate}

Rational arguments for acrtan

Sum and difference formula, where we seek rational numbers with large
denominators

Experimental for 10 by 10 square grid

``Prime factorization of Gaussian integers'' as the basis for further
derivations.

Examples

\href{https://en.wikipedia.org/wiki/John_Machin}{John Machin} followed
in the footsteps of the Madhava-Gregory-Leibniz series, but he used the
difference in the arctangents of \emph{two} values to arrive at a more
rapidly convergent series for \(\pi\). To better understand his method,
let us recall that if \(\tan A = \frac{a_1}{b_1}\) and
\(\tan B = \frac{a_2}{b_2}\), then
{[}\citeproc{ref-libre-inv-trig-deriv}{1}{]}: \[
\begin{aligned}
\tan(A + B) &= \frac{\tan A + \tan B}{1 - \tan A\tan B}\\
&= \frac{\frac{a_{1}}{b_{1}} + \frac{a_{2}}{b_{2}}}{1 - \frac{a_{1}a_{2}}{b_{1}b_{2}}}\\
&= \frac{a_{1}b_{2} + a_{2}b_{1}}{b_{1}b_{2} - a_{1}a_{2}}\\
\end{aligned}
\] Notice that \begin{equation}\phantomsection\label{eq:machin-arc}{
\begin{aligned}
\arctan\tan(A+B) &= (A + B) \mbox { which implies}\\
\arctan\frac{a_1}{b_1}  + \arctan\frac{a_2}{b_2} &= \arctan\left[\frac{a_{1}b_{2} + a_{2}b_{1}}{b_{1}b_{2} - a_{1}a_{2}}\right]\\
\end{aligned}
}\end{equation}

Suppose we set \(a_{1} = a_{2} = 1\), then, \cref{eq:machin-arc} we get
these sum and difference formulae:
\begin{equation}\phantomsection\label{eq:sum-diff-arct}{
\begin{aligned}
\arctan\frac{1}{b_1}  + \arctan\frac{1}{b_2} &= \arctan\left[\frac{b_{1} + b_{2}}
{b_{1}b_{2} - 1}\right]\\
\arctan\frac{1}{b_1}  - \arctan\frac{1}{b_2} &= \arctan\left[\frac{b_{1} - b_{2}}
{b_{1}b_{2} + 1}\right]
\end{aligned}
}\end{equation} To get \(\pi\) correct to ten decimal places, we need to
evaluate only X partial sums when using the Machin formula.

\cref{eq:machin-arc} is at the root of the Machin Formula
{[}\citeproc{ref-machin-like-wiki}{2}{]}:

But what made Machin use these particular numbers in
\cref{eq:machin-formula}? The answer to this vital question will take us
a little far afield into the factorization of
\href{https://en.wikipedia.org/wiki/Gaussian_integer}{Gaussian Integers}
and related ideas. Those interested in the details should consult
\href{https://www2.oberlin.edu/faculty/jcalcut/gausspi.pdf}{this
dedicated paper} {[}\citeproc{ref-calcut2009}{3}{]} or refer to these
discussions
{[}\citeproc{ref-mse-machin-one}{4},\citeproc{ref-mse-machin-two}{5}{]}.

The Machin formula's claim to fame is that it converges faster than the
abysmally slow Madhava-Gregory-Leibniz series. Indeed it inspired
formulae that were the basis for approximating \(\pi\) more accurately
for a century or two.

\subsection{Newton's approach to estimating
π}\label{newtons-approach-to-estimating-ux3c0}

When he needed to estimate \(\pi\) accurately, Newton extended his own
pathbreaking binomial theorem to develop the binomial power series. For
a fascinating account of how this happened, Read {[}this online
article{[}(https://www.quantamagazine.org/how-isaac-newton-discovered-the-binomial-power-series-20220831/)
{[}\citeproc{ref-strogatz-newton-2022}{6}{]}.

\subsection{Gauss's contribution}\label{gausss-contribution}

Arithmetic-Geometric Mean AGM

\subsection{Ramanujan and the
Chudnovskys}\label{ramanujan-and-the-chudnovskys}

\subsection{Acknowledgements}\label{acknowledgements}

Wolfram Alpha for several results.

\subsection{Feedback}\label{feedback}

Please \href{mailto:feedback.swanlotus@gmail.com}{email me} your
comments and corrections.

\section*{References}\label{bibliography}
\addcontentsline{toc}{section}{References}

\phantomsection\label{refs}
\begin{CSLReferences}{0}{0}
\bibitem[\citeproctext]{ref-libre-inv-trig-deriv}
\CSLLeftMargin{{[}1{]} }%
\CSLRightInline{Gilbert Strang and Edwin ``Jed'' Herman and et al.
{Derivatives of Inverse Trig Functions}. Mathematics LibreTexts.
Retrieved 1 August 2024 from
\url{https://math.libretexts.org/Courses/Monroe_Community_College/MTH_210_Calculus_I_(Professor_Dean)/Chapter_3:_Derivatives/3._10:_Derivatives_of_Inverse_Trig_Functions}}

\bibitem[\citeproctext]{ref-machin-like-wiki}
\CSLLeftMargin{{[}2{]} }%
\CSLRightInline{Wikipedia contributors. 2024. {Machin-like formula}.
{Wikipedia, The Free Encyclopedia}. Retrieved 3 August 2024 from
\url{https://en.wikipedia.org/wiki/Machin-like_formula}}

\bibitem[\citeproctext]{ref-calcut2009}
\CSLLeftMargin{{[}3{]} }%
\CSLRightInline{Jack S Calcut. 2009. {Gaussian Integers and Arctangent
Identities for π}. \emph{{The American Mathematical Monthly}} 116,
(2009), 515--530. Retrieved 3 August 2024 from
\url{https://www2.oberlin.edu/faculty/jcalcut/gausspi.pdf}}

\bibitem[\citeproctext]{ref-mse-machin-one}
\CSLLeftMargin{{[}4{]} }%
\CSLRightInline{Various. 2011. {Machin's formula and cousins}.
Mathematics stack exchange. Retrieved from
\url{https://math.stackexchange.com/questions/44595/machins-formula-and-cousins}}

\bibitem[\citeproctext]{ref-mse-machin-two}
\CSLLeftMargin{{[}5{]} }%
\CSLRightInline{Various. 2017. {How are the Machin Pi formulas found?}.
{Mathematics Stack Exchange}. Retrieved 3 August 2024 from
\url{https://math.stackexchange.com/questions/2209111/how-are-the-machin-pi-formulas-found}}

\bibitem[\citeproctext]{ref-strogatz-newton-2022}
\CSLLeftMargin{{[}6{]} }%
\CSLRightInline{Steven Strogatz. 2022. {How Isaac Newton Discovered the
Binomial Power Series}. {Quanta Magazine}. Retrieved 5 August 2024 from
\url{https://www.quantamagazine.org/how-isaac-newton-discovered-the-binomial-power-series-20220831/}}

\end{CSLReferences}



\end{document}
