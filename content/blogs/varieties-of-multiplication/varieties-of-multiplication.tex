% Options for packages loaded elsewhere
\PassOptionsToPackage{unicode,linktoc=all}{hyperref}
\PassOptionsToPackage{hyphens}{url}
\PassOptionsToPackage{dvipsnames,svgnames,x11names}{xcolor}
%
\documentclass[
  a4paper,
]{article}
\usepackage{amsmath,amssymb}
\usepackage{iftex}
\ifPDFTeX
  \usepackage[T1]{fontenc}
  \usepackage[utf8]{inputenc}
  \usepackage{textcomp} % provide euro and other symbols
\else % if luatex or xetex
  \usepackage{unicode-math} % this also loads fontspec
  \defaultfontfeatures{Scale=MatchLowercase}
  \defaultfontfeatures[\rmfamily]{Ligatures=TeX,Scale=1}
\fi
\usepackage{lmodern}
\ifPDFTeX\else
  % xetex/luatex font selection
\fi
% Use upquote if available, for straight quotes in verbatim environments
\IfFileExists{upquote.sty}{\usepackage{upquote}}{}
\IfFileExists{microtype.sty}{% use microtype if available
  \usepackage[]{microtype}
  \UseMicrotypeSet[protrusion]{basicmath} % disable protrusion for tt fonts
}{}
\makeatletter
\@ifundefined{KOMAClassName}{% if non-KOMA class
  \IfFileExists{parskip.sty}{%
    \usepackage{parskip}
  }{% else
    \setlength{\parindent}{0pt}
    \setlength{\parskip}{6pt plus 2pt minus 1pt}}
}{% if KOMA class
  \KOMAoptions{parskip=half}}
\makeatother
\usepackage{xcolor}
\usepackage[margin=25mm]{geometry}
\usepackage{longtable,booktabs,array}
\usepackage{calc} % for calculating minipage widths
% Correct order of tables after \paragraph or \subparagraph
\usepackage{etoolbox}
\makeatletter
\patchcmd\longtable{\par}{\if@noskipsec\mbox{}\fi\par}{}{}
\makeatother
% Allow footnotes in longtable head/foot
\IfFileExists{footnotehyper.sty}{\usepackage{footnotehyper}}{\usepackage{footnote}}
\makesavenoteenv{longtable}
\usepackage{graphicx}
\makeatletter
\def\maxwidth{\ifdim\Gin@nat@width>\linewidth\linewidth\else\Gin@nat@width\fi}
\def\maxheight{\ifdim\Gin@nat@height>\textheight\textheight\else\Gin@nat@height\fi}
\makeatother
% Scale images if necessary, so that they will not overflow the page
% margins by default, and it is still possible to overwrite the defaults
% using explicit options in \includegraphics[width, height, ...]{}
\setkeys{Gin}{width=\maxwidth,height=\maxheight,keepaspectratio}
% Set default figure placement to htbp
\makeatletter
\def\fps@figure{htbp}
\makeatother
\usepackage{svg}
\setlength{\emergencystretch}{3em} % prevent overfull lines
\providecommand{\tightlist}{%
  \setlength{\itemsep}{0pt}\setlength{\parskip}{0pt}}
\setcounter{secnumdepth}{-\maxdimen} % remove section numbering
\newlength{\cslhangindent}
\setlength{\cslhangindent}{1.5em}
\newlength{\csllabelwidth}
\setlength{\csllabelwidth}{3em}
\newlength{\cslentryspacingunit} % times entry-spacing
\setlength{\cslentryspacingunit}{\parskip}
\newenvironment{CSLReferences}[2] % #1 hanging-ident, #2 entry spacing
 {% don't indent paragraphs
  \setlength{\parindent}{0pt}
  % turn on hanging indent if param 1 is 1
  \ifodd #1
  \let\oldpar\par
  \def\par{\hangindent=\cslhangindent\oldpar}
  \fi
  % set entry spacing
  \setlength{\parskip}{#2\cslentryspacingunit}
 }%
 {}
\usepackage{calc}
\newcommand{\CSLBlock}[1]{#1\hfill\break}
\newcommand{\CSLLeftMargin}[1]{\parbox[t]{\csllabelwidth}{#1}}
\newcommand{\CSLRightInline}[1]{\parbox[t]{\linewidth - \csllabelwidth}{#1}\break}
\newcommand{\CSLIndent}[1]{\hspace{\cslhangindent}#1}
\ifLuaTeX
\usepackage[bidi=basic]{babel}
\else
\usepackage[bidi=default]{babel}
\fi
\babelprovide[main,import]{british}
% get rid of language-specific shorthands (see #6817):
\let\LanguageShortHands\languageshorthands
\def\languageshorthands#1{}
% $HOME/.pandoc/defaults/latex-header-includes.tex
% Common header includes for both lualatex and xelatex engines.
%
% Preliminaries
%
% \PassOptionsToPackage{rgb,dvipsnames,svgnames}{xcolor}
% \PassOptionsToPackage{main=british}{babel}
\PassOptionsToPackage{english}{selnolig}
\AtBeginEnvironment{quote}{\small}
\AtBeginEnvironment{quotation}{\small}
\AtBeginEnvironment{longtable}{\centering}
%
% Packages that are useful to include
%
\usepackage{graphicx}
\usepackage{subcaption}
\usepackage[inkscapeversion=1]{svg}
\usepackage[defaultlines=4,all]{nowidow}
\usepackage{etoolbox}
\usepackage{fontsize}
\usepackage{newunicodechar}
\usepackage{pdflscape}
\usepackage{fnpct}
\usepackage{parskip}
  \setlength{\parindent}{0pt}
\usepackage[style=american]{csquotes}
% \usepackage{setspace} Use the <fontname-plus.tex> files for setspace
%
\usepackage{hyperref} % cleveref must come AFTER hyperref
\usepackage[capitalize,noabbrev]{cleveref} % Must come after hyperref
% noto-plus.tex
% Font-setting header file for use with Pandoc Markdown
% to generate PDF via LuaLaTeX.
% The main font is Noto Serif.
% Other main fonts are also available in appropriately named file.
\usepackage{fontspec}
\usepackage{setspace}
\setstretch{1.3}
%
\defaultfontfeatures{Ligatures=TeX,Scale=MatchLowercase,Renderer=Node} % at the start always
%
% For English
% See also https://tex.stackexchange.com/questions/574047/lualatex-amsthm-polyglossia-charissil-error
% We use Node as Renderer for the Latin Font and Greek Font and HarfBuzz as renderer ofr Indic fonts.
%
\babelfont{rm}[Script=Latin,Scale=1]{NotoSerif}% Config is at $HOME/texmf/tex/latex/NotoSerif.fontspec
%
\babelfont{sf}[Script=Latin]{SourceSansPro}% Config is at $HOME/texmf/tex/latex/SourceSansPro.fontspec
%
\babelfont{tt}[Script=Latin]{FiraMono}% Config is at $HOME/texmf/tex/latex/FiraMono.fontspec
%
% Sanskrit, Tamil, and Greek fonts
%
\babelprovide[import, onchar=ids fonts]{sanskrit}
\babelprovide[import, onchar=ids fonts]{tamil}
\babelprovide[import, onchar=ids fonts]{greek}
%
\babelfont[sanskrit]{rm}[Scale=1.1,Renderer=HarfBuzz,Script=Devanagari]{NotoSerifDevanagari}
\babelfont[sanskrit]{sf}[Scale=1.1,Renderer=HarfBuzz,Script=Devanagari]{NotoSansDevanagari}
\babelfont[tamil]{rm}[Renderer=HarfBuzz,Script=Tamil]{NotoSerifTamil}
\babelfont[tamil]{sf}[Renderer=HarfBuzz,Script=Tamil]{NotoSansTamil}
\babelfont[greek]{rm}[Script=Greek]{GentiumBookPlus}
%
% Math font
%
\usepackage{unicode-math} % seems not to hurt % fallabck
\setmathfont[bold-style=TeX]{STIX Two Math}
\usepackage{amsmath}
\usepackage{esdiff} % for derivative symbols
%
%
% Other fonts
%
\newfontfamily{\emojifont}{Symbola}
%

\usepackage{titling}
\usepackage{fancyhdr}
    \pagestyle{fancy}
    \fancyhead{}
    \fancyfoot{}
    \renewcommand{\headrulewidth}{0.2pt}
    \renewcommand{\footrulewidth}{0.2pt}
    \fancyhead[LO,RE]{\scshape\thetitle}
    \fancyfoot[CO,CE]{\footnotesize Copyright © 2006\textendash\the\year, R (Chandra) Chandrasekhar}
    \fancyfoot[RE,RO]{\thepage}
\newfontfamily{\regulariconfont}{Font Awesome 6 Free Regular}[Color=Grey]
\newfontfamily{\solidiconfont}{Font Awesome 6 Free Solid}[Color=Grey]
\newfontfamily{\brandsiconfont}{Font Awesome 6 Brands}[Color=Grey]
%
% Direct input of Unicode code points
%
\newcommand{\faEnvelope}{\regulariconfont\ ^^^^f0e0\normalfont}
\newcommand{\faMobile}{\solidiconfont\ ^^^^f3cd\normalfont}
\newcommand{\faLinkedin}{\brandsiconfont\ ^^^^f0e1\normalfont}
\newcommand{\faGithub}{\brandsiconfont\ ^^^^f09b\normalfont}
\newcommand{\faAtom}{\solidiconfont\ ^^^^f5d2\normalfont}
\newcommand{\faPaperPlaneRegular}{\regulariconfont\ ^^^^f1d8\normalfont}
\newcommand{\faPaperPlaneSolid}{\solidiconfont\ ^^^^f1d8\normalfont}

%
% The block below is commented out because of Tofu glyphs in HTML
%
% \newcommand{\faEnvelope}{\regulariconfont\ \normalfont}
% \newcommand{\faMobile}{\solidiconfont\ \normalfont}
% \newcommand{\faLinkedin}{\brandsiconfont\ \normalfont}
% \newcommand{\faGithub}{\brandsiconfont\ \normalfont}
\ifLuaTeX
  \usepackage{selnolig}  % disable illegal ligatures
\fi
\IfFileExists{bookmark.sty}{\usepackage{bookmark}}{\usepackage{hyperref}}
\IfFileExists{xurl.sty}{\usepackage{xurl}}{} % add URL line breaks if available
\urlstyle{sf}
\hypersetup{
  pdftitle={Varieties of Multiplication},
  pdfauthor={R (Chandra) Chandrasekhar},
  pdflang={en-GB},
  colorlinks=true,
  linkcolor={DarkOliveGreen},
  filecolor={Purple},
  citecolor={DarkKhaki},
  urlcolor={Maroon},
  pdfcreator={LaTeX via pandoc}}

\title{Varieties of Multiplication}
\author{R (Chandra) Chandrasekhar}
\date{2012-05-14 | 2023-11-10}

\begin{document}
\maketitle

\thispagestyle{empty}


\begin{quote}
This was my first blog---written in 2012---under the
\href{https://swanlotus.netlify.app/tag/mathematical-musings}{\emph{Mathemtical
Musings}} tag. The intention was to re-visit topics in mathematics that
trigger concern or disquiet in the earnest student of the subject. My
hope was that ideas that appeared puzzling or forbidding at first sight
could be coaxed to become friendly and helpful. Unhurried explanations
and a different perspective would underpin the approach. I have
retained, substantially unchanged, what I first wrote, to maintain the
freshness, flavour, and vintage of the original blog, even if it is a
little rough around the edges.
\end{quote}

\hypertarget{prologue}{%
\subsection{Prologue}\label{prologue}}

This blog is experimental in three ways.

First, this is my maiden attempt to display mathematics on a web page.
It might look simple, but believe me, it is no mean task. Thanks to the
concerted efforts of many generous people, I am using
\href{https://www.mathjax.org/}{MathJax} to render the mathematics via
\href{https://pandoc.org/}{Pandoc}, and its flavour of
\href{https://garrettgman.github.io/rmarkdown/authoring_pandoc_markdown.html}{Markdown}.

The second experimental feature is what I have called ``slicing the
orange of knowledge with a different cut'' in my book
\href{https://swanlotus.netlify.app/sas}{\emph{Secrets of Academic
Success}}. The idea of multiplication runs like a thread through much of
mathematics, from the most elementary stages of counting to what
constitutes cutting edge research. Unfortunately, in the way mathematics
is taught at present, multiplication is bundled with each stage of
mathematics and viewed separately as an operation in that context.

By concentrating on the single unifying \emph{idea} of multiplication,
and viewing it across the whole of mathematics, we are indeed ``slicing
the orange of knowledge with a different cut''. Even if you have not
encountered some of the varieties of multiplication mentioned here, this
exposure will help you grasp those varieties better when you do
encounter them. Please \href{mailto:feedback.swanlotus@gmail.com}{send
me feedback} on whether this approach works for you.

Third, this is an \emph{extremely long} blog. In fact, I call it a
\href{https://www.vocabulary.com/dictionary/slog}{\emph{slog}}
\emojifont {😉} \normalfont. It took me some weeks to write it. So, take
your time reading it. It is unlikely that you will finish it in one
sitting. \emph{Read it in parts at your own pace}. After having read it
once, cast your eyes and mind over the whole to get an overall view of
the main ideas.

I thought of splitting the blog into three or four manageable parts, but
decided against it because I wanted the evolution of multiplication as
an idea to be left whole in a single post.
\href{mailto:feedback.swanlotus@gmail.com}{Tell me} if it put you to
sleep \emojifont {😄} \normalfont.

With that out of the way, let us begin. I want to look at some of the
varieties of multiplication that mathematicians have developed over
time. It is a survey that will serve as a pinhole through which we can
view how a single, simple mathematical idea has been expanded and
elaborated into uses far beyond its historical moorings.

\hypertarget{multiplication-as-a-binary-operation}{%
\subsection{Multiplication as a binary
operation}\label{multiplication-as-a-binary-operation}}

Consistency is valued more in mathematics than in other disciplines.
\emph{The idea is not to upset the apple cart but to expand it}.
Definitions, conventions, rules, facts, and fallacies---once
established---are usually above dispute, and do not vary with time or
place. So, let us start by defining some terms.

Multiplication is a \emph{binary} operation: it is something that we do
with \emph{two} mathematical objects, whatever they might be. Usually,
the two are \emph{similar} objects or at least \emph{compatible}
objects. The whole numbers are an example. We can and do multiply two
whole numbers.

\hypertarget{multiplication-as-repeated-addition}{%
\subsection{Multiplication as repeated
addition}\label{multiplication-as-repeated-addition}}

Practically and historically, multiplication arose as an arithmetic
convenience for repeated addition. If we add the number \(3\) four
times, we have \[
3 + 3 + 3 + 3 = (3 + 3) + (3 + 3) = 6 + 6 = 12
\] The reason for adding \(3\) in \emph{pairs}, as shown above, is that
addition is a binary operation, just like multiplication. Using the
shorthand of multiplication, we write this as \(4 \times 3 = 12\).
\emph{So, multiplication is a shorthand for repeated addition.}

When we see the arithmetic expression \(4 \times 3\), we say ``four
times three'' in English. Or, we could equally well say ``four threes'',
as I was taught at school, which is less ambiguous and much clearer.
Think of four lots of three being added together like we have seen
above: \begin{equation}\protect\hypertarget{eq:multiplier}{}{
\begin{array}{c c c r}
4 &\times &3 &= 3 + 3 + 3 + 3 = (3 + 3) + (3 + 3) = 6 + 6 = 12\\
\uparrow & & \uparrow & \uparrow\\
\mbox{multiplier} & & \mbox{multiplicand} & \mbox{product}\\
\end{array}
}\label{eq:multiplier}\end{equation} The number \(4\) is the
\href{https://www.thefreedictionary.com/multiplier}{multiplier} and the
number \(3\) is the
\href{https://mathworld.wolfram.com/Multiplicand.html}{multiplicand}.
This is the standard definition.

We say that the ``something'' which is repeatedly added, is the
\emph{multiplicand.} The number of times that ``something'' is added is
the \emph{multiplier.} And the result of this operation is the
\emph{product.} Thus far we are in perfect harmony with accepted usage.

\hypertarget{commutativity-and-multiplication}{%
\subsection{Commutativity and
multiplication}\label{commutativity-and-multiplication}}

Multiplication of numbers is \href{}{commutative}, i.e., the multiplier
and multiplicand can change roles without affecting the result.
\begin{equation}\protect\hypertarget{eq:comm}{}{
4 \times 3 = 3 \times 4 = 12.
}\label{eq:comm}\end{equation} Note that at the very left of
\cref{eq:comm}, \(4\) is the multiplier, and \(3\) the multiplicand,
whereas in the middle of \cref{eq:comm}, \(3\) is the multiplier and
\(4\) the multiplicand. To labour the point,
\begin{equation}\protect\hypertarget{eq:multiplicand}{}{
\begin{array}{c c c r}
3 &\times &4 &= 4 + 4 + 4 = (4 +4) + 4 = 4 + (4 + 4) = 12\\
\uparrow & & \uparrow & \uparrow\\
\mbox{multiplier} & & \mbox{multiplicand} & \mbox{product}\\
\end{array}
}\label{eq:multiplicand}\end{equation} Although the two numbers have
changed their names and roles from \cref{eq:multiplier} to
\cref{eq:multiplicand}, the result is the same because the
multiplication of numbers is commutative.

To accommodate our
\href{https://www.thefreedictionary.com/zeitgeist}{zeitgeist}, the
distinction between multiplier and multiplicand is fading away, in
favour of the symmetrical and neutral term \emph{factor}. The result of
multiplying two \emph{factors} is still the \emph{product}, as before.

\hypertarget{rectangular-numbers}{%
\subsection{Rectangular numbers}\label{rectangular-numbers}}

Historically, stones were used to count. And the stones representing any
number may be arranged in geometric shapes, like lines, triangles,
rectangles, and so on. This gives us a geometrical or pictorial
representation of numbers.

All numbers which are the products of two whole numbers, neither of
which is one, may be expressed as \emph{rectangular numbers}. The
symbolic operation \(4 \times 3\) may be shown pictorially as a series
of \(12\) icons arranged four across and three high. Because we may swap
the order of multiplication, we may also show it as \(12\) rectangles
three across and four high. As we have seen, multiplication of numbers
is commutative. The image below shows this equivalence.

\begin{figure}
\hypertarget{fig:four-by-three}{%
\centering
\includesvg[width=0.5\textwidth,height=\textheight]{images/four-by-three.svg}
\caption{The multiplier is the number of rows, and the multiplicand is
the number of columns.}\label{fig:four-by-three}
}
\end{figure}

\hypertarget{factorization-is-not-unique}{%
\subsection{Factorization is not
unique}\label{factorization-is-not-unique}}

There is a subtle but important point to grasp here. The product \(12\)
is called a \emph{composite} number and its \emph{factors} in the
illustrated representation are \(4\) and \(3\). \emph{But this
factorization is not unique.} We could have equally correctly claimed
that \(2 \times 6 = 12\) leading to a different factorization. While we
may assert that both \(4 \times 3\) and \(2 \times 6\) lead to the same
unique composite number as the product, we cannot reverse the process
and claim uniqueness of factors for any particular composite number.

\begin{figure}
\hypertarget{fig:six-by-two}{%
\centering
\includesvg[width=0.7\textwidth,height=\textheight]{images/six-by-two.svg}
\caption{Twelve is a composite number. It may be factorized as
\(2 \times 6\), \(4 \times 3\), or as the trivial
\(1 \times 12\).}\label{fig:six-by-two}
}
\end{figure}

\hypertarget{square-numbers}{%
\subsection{Square numbers}\label{square-numbers}}

A square is a special case of a rectangle whose length and width are
equal. When we write \(3 \times 3 = 9\), and we arrange the resulting
nine squares in a rectangle, we get a \emph{square number} of three
squares by three squares. This is why we say that we are \emph{squaring}
a number when we multiply it by itself.

\begin{figure}
\hypertarget{fig:three-square}{%
\centering
\includesvg[width=0.2\textwidth,height=\textheight]{images/three-square.svg}
\caption{Nine is a square number. Because \(3 \times 3 = 9\), it may be
so portrayed geometrically.}\label{fig:three-square}
}
\end{figure}

One could carry this analogy further and move into three dimensions to
represent a number like \(3 \times 3 \times 3 = 27\) with small cubes
arranged in a large \emph{cube.}\footnote{Try this with toy blocks to
  convince yourself of its truth.} This is why we say we are
\emph{cubing} a number when we multiply it by itself twice.

\hypertarget{prime-numbers}{%
\subsection{Prime numbers}\label{prime-numbers}}

A number which cannot be expressed as the product of two numbers other
than one and itself is called a \emph{prime number}. Prime numbers can
only be arranged in a line, never in a rectangle. Seven is a prime
number as illustrated below.

\begin{figure}
\hypertarget{fig:seven}{%
\centering
\includesvg[width=0.45\textwidth,height=\textheight]{images/seven.svg}
\caption{Seven is a prime number. Its seven icons cannot be re-arranged
in a rectangle.}\label{fig:seven}
}
\end{figure}

Try to rearrange the seven icons into a rectangle to convince yourself
that it is not possible and that seven is prime. Experimenting like this
will help you better understand what testing for primality entails.

Prime numbers are like building blocks that may be used to build larger
numbers by multiplication.

\hypertarget{prime-factorization-is-unique}{%
\subsection{Prime factorization is
unique}\label{prime-factorization-is-unique}}

Let us look at the number \(12\) again, breaking it down this time into
its \emph{prime factors} so: \(12 = 2 \times 2 \times 3\). There are two
instances of the number \(2\) and one instance of the number \(3\).
These numbers cannot be decomposed any further because they are prime.
If we disregard the order of arrangement of these prime factors, i.e.,
we do not distinguish between \(2 \times 2 \times 3\) and
\(2 \times 3 \times 2\) and so on, we may assert that \emph{the prime
factors of a composite number are unique}. This statement is known as
the
\href{https://en.wikipedia.org/wiki/Fundamental_theorem_of_arithmetic}{Fundamental
Theorem of Arithmetic}. It is also sometimes called the Prime
Factorization Theorem.

\hypertarget{symbols-for-multiplication}{%
\subsection{Symbols for
multiplication}\label{symbols-for-multiplication}}

If you are sharp-eyed, you might have come across the multiplication of
two negative numbers by enclosing each of them in parentheses: (). The
same symbols are also used to define associativity and distributivity
later in this blog. We now look at the chequered history of how the
notation for multiplication has changed with time, need, and context.

\hypertarget{times-sign}{%
\subsubsection{Times sign}\label{times-sign}}

The symbol for multiplication that we first learn is the rotated plus
sign ``\(+\)'' that looks like \(\times\). It is called the
``multiplication sign'' and is usually read as ``times'', as we have
already seen. It serves reasonably well even when we outgrow the whole
numbers and move onto fractions.

\hypertarget{parentheses}{%
\subsubsection{Parentheses}\label{parentheses}}

Parentheses, written in pairs as (), have traditionally denoted
\emph{precedence} during evaluation of an expression. Division and
multiplication are evaluated \emph{before} subtraction and addition.
This order may be altered by including terms in parentheses, which are
accorded highest priority during evaluation. So,
\(5 \times 4 + 1 = 20 + 1 = 21\), whereas
\(5 \times (4 + 1) = 5 \times 5 = 25\).

When our discourse embraces negative quantities, in order to avoid
ambiguity, we need something to enclose both the negative sign, \(-\),
and the number to which it applies. The expression \(5 \times -4\) is
ambiguous and therefore never written so when we actually mean
\(5 \times (-4)\). This notation led to two pairs of parentheses
\emph{without any explicit multiplication symbol in between} to denote
the multiplication of the two enclosed numbers thus:
\((5)(-4) = 5 \times (-4) = -20\).

\hypertarget{juxtaposition-without-any-symbol}{%
\subsubsection{Juxtaposition without any
symbol}\label{juxtaposition-without-any-symbol}}

The archetypal symbol for the unknown in algebra is \(x\), which looks a
little too much like the multiplication symbol \(\times\), especially
when handwritten.

To avoid confusion, a convention was adopted that when two
\emph{algebraic variables,} representing unknown quantities, were
written next to each other or \emph{juxtaposed,} it indicated the
multiplication of the two quantities. There is no intervening symbol
between the two variables.

Thus, \(x \times y = (x)(y) = xy\). Note that this convention is for
algebraic variables only. We cannot use this convention with digits
representing numbers because of \emph{place value} in our decimal
system. The number \(45\) does \emph{not} represent the product of \(4\)
and \(5\) but actually means \(40 + 5 = 45\).

\hypertarget{centred-dot}{%
\subsubsection{Centred dot}\label{centred-dot}}

As more exotic objects entered the mathematical collection, yet another
symbol was devised to show (at least one form of) multiplication. The
vertically centred dot \(\cdot\) was used to indicate one of the several
products of \emph{vectors,} which we shall discuss
\protect\hyperlink{dot-or-scalar-product}{later} in this blog. Again,
the dot is not useful in the context of digits because it could be
confused with a decimal point.

So, there is both a rationale and a mathematical context for the
introduction of each symbol for multiplication, according to time, need,
and circumstance.

\hypertarget{asterisk}{%
\subsubsection{Asterisk}\label{asterisk}}

The latter half of the twentieth century saw the introduction of yet
another symbol for multiplication, this time for use in programming
languages. Because the
\href{https://datatracker.ietf.org/doc/html/rfc20}{ASCII character set,}
devised during the era of
\href{https://en.wikipedia.org/wiki/Teleprinter}{teleprinters,} did not
include the symbol \(\times\) for multiplication, another available
symbol had to be chosen for multiplication. The winner was the
\href{https://en.wikipedia.org/wiki/Asterisk}{asterisk}, denoted by *.
Note that even with the symbol *, multiplying \(3\) by \(-4\) still
requires one to type \(3 * (-4)\) to avoid ambiguity.

Repeated multiplication---or
\protect\hyperlink{exponentiation}{exponentiation}---is usually
represented by a double asterisk ** in most computing languages,
although a caret \^{} assumes this function in some languages.

\hypertarget{laws-of-arithmetic}{%
\subsection{Laws of arithmetic}\label{laws-of-arithmetic}}

We now return to the assertion, made at the start of this blog, that
multiplication is a \emph{binary} operation: something that happens
between \emph{two} mathematical objects. You might object that we can
and do multiply three numbers. For example, \(2 \times 3 \times 5 = 30\)
and no one would doubt the veracity of that assertion. Why then is
multiplication classified as a binary operation and how might it be
reconciled with what we know about the multiplication of three or more
numbers?

Early mathematics was eminently practical, concerned with computing
areas and volumes, profit and loss, and so on. In the course of time,
mathematicians started to systematize their body of knowledge by
introducing logic and rigour into their subject. They wanted to move
beyond ad hoc methodology and construct an intellectual edifice that was
stable, durable, and generalizable.

\hypertarget{the-real-numbers-as-a-field}{%
\subsection{The real numbers as a
field}\label{the-real-numbers-as-a-field}}

Some of the most unpleasant experiences of school mathematics are the
sudden and unexpected changes that intrude into the familiar arithmetic
of primary school. Division, fractions, negative numbers, multiplication
by zero, product of two negatives being positive, etc. are a few
examples. When rule upon unanticipated rule is foisted on the young
student, with no rhyme or reason, the student gets overwhelmed and
develops a distaste for mathematics or even a reflex fear of it. This
need not be so.

One way out is a quick but easy introduction to some ideas of abstract
algebra which lay the foundation for all subsequent mathematics. This
way, all the rules are bunched together as unquestioned assumptions or
axioms. Then, based on those assumptions, we build a mathematical
edifice that is logical, consistent, and extensible. Mathematics will
then be changed from a mysterious game with ever-changing rules into a
trustworthy friend who can be relied upon.

The numbers we use every day are drawn from a
\href{https://www.cuemath.com/algebra/sets/}{set} called the
\href{https://mathworld.wolfram.com/RealNumber.html}{real numbers}
denoted by \(\mathbb{R}\). Numbers such as \(0\), \(1\), \(-200\),
\(50004\), \(\frac{1}{2}\), \(-\frac{3}{4}\), \(0.3333\cdots\),
\(-0.5\), \(\sqrt{2}\), \(\pi\), and countless others belong to this
grab bag set.

The set \(\mathbb{R}\) comes bundled with \emph{two} binary operations:
addition, denoted by \(+\), and multiplication denoted by \(\times\) or
by other means as outlined
\protect\hyperlink{symbols-for-multiplication-across-time-and-need}{above}.
One property that makes real numbers so useful is that any addition or
multiplication of real numbers results in another real number. This is
called \emph{closure} and is an important aspect of real arithmetic. In
addition, the reals also exhibit other familiar behaviours---with
fancy-sounding names---which are tabulated below, using arbitrary real
numbers, \(a, b, c\).

\begin{small}

\hypertarget{tbl:axioms}{}
\begin{longtable}[]{@{}
  >{\raggedright\arraybackslash}p{(\columnwidth - 4\tabcolsep) * \real{0.2083}}
  >{\raggedright\arraybackslash}p{(\columnwidth - 4\tabcolsep) * \real{0.3611}}
  >{\raggedright\arraybackslash}p{(\columnwidth - 4\tabcolsep) * \real{0.3611}}@{}}
\caption{\label{tbl:axioms}Axioms for the real numbers}\tabularnewline
\toprule\noalign{}
\begin{minipage}[b]{\linewidth}\raggedright
Property
\end{minipage} & \begin{minipage}[b]{\linewidth}\raggedright
Addition
\end{minipage} & \begin{minipage}[b]{\linewidth}\raggedright
Multiplication
\end{minipage} \\
\midrule\noalign{}
\endfirsthead
\toprule\noalign{}
\begin{minipage}[b]{\linewidth}\raggedright
Property
\end{minipage} & \begin{minipage}[b]{\linewidth}\raggedright
Addition
\end{minipage} & \begin{minipage}[b]{\linewidth}\raggedright
Multiplication
\end{minipage} \\
\midrule\noalign{}
\endhead
\bottomrule\noalign{}
\endlastfoot
Associativity & \((a+b)+c=a+(b+c)\) & \((ab)c = a(bc)\) \\
Commutativity & \(a+b=b+a\) & \(ab=ba\) \\
Identity & \(a+0=a=0+a\) & \(a·1=a=1·a\) \\
Inverse & \(a+(-a)=0=(-a)+a\) & \(aa^{-1}=1=a^{-1}a\) for \(a \ne 0\) \\
\end{longtable}

\end{small}

\hfill\break
For the record, formal definitions for the above terms are available on
the Web from reputable sites whose links are listed below:

\begin{enumerate}
\item
  \href{https://mathworld.wolfram.com/Associative.html}{Associativity}.
\item
  \href{https://mathworld.wolfram.com/Commutative.html}{Commutativity}.
\item
  \href{https://en.wikipedia.org/wiki/Distributive_property}{Distributivity}.
  Multiplication is distributive over addition. For completeness, we
  define \[
  \begin{array}{l  l}
  \mbox{Left Distributivity} & a(b+c)=ab+ac\\
  \mbox{Right Distributivity} & (a+b)c=ac+bc\\
  \end{array}
  \] In our case, both conditions hold, and we may simply say that
  \emph{multiplication is distributive over addition for the reals}.
\end{enumerate}

A mathematical object consisting of a set with two binary operations
having the above properties is called a
\href{https://en.wikipedia.org/wiki/Field_(mathematics)}{field}. The
real numbers constitute a field.

\hypertarget{associativity-of-multiplication}{%
\subsection{Associativity of
multiplication}\label{associativity-of-multiplication}}

\emph{Because multiplication is binary, we can only multiply two numbers
at any one time.} If we need to multiply together three or more numbers,
we have to multiply two of them first to get a single product before we
can multiply it with next number, and so on.

The associative law simply states that when we multiply three numbers,
it does not matter which two of the three we multiply first; the result
will always be the same. It is this property that allows us to write
something like \(2 \times 3 \times 5\) or \(abc\) and still make
sense---because it denotes something unique---even though we know that
multiplication is a binary operation.

In addition to the three properties of associativity, commutativity, and
distributivity, the real numbers we use every day have an additive
identity and inverse in \cref{tbl:axioms}. These are considered next.

\hypertarget{the-additive-identity-and-inverse-in-mathbbr}{%
\subsection{\texorpdfstring{The additive identity and inverse in
\(\mathbb{R}\)}{The additive identity and inverse in \textbackslash mathbb\{R\}}}\label{the-additive-identity-and-inverse-in-mathbbr}}

The additive identity in \(\mathbb{R}\) is \(0\). This means that for
any \(a \in \mathbb{R}\), \(a + 0 = 0 + a = a\). The
\href{https://en.wikipedia.org/wiki/Additive_inverse}{additive inverse}
of \(a\) is \((-a)\) and this means \(a + (-a) = (-a) + a = 0\).

\emph{When we add together a number and its inverse, we get the additive
identity.}

Another way of understanding the additive inverse is to look at it
geometrically as a reflection in a double-sided mirror placed
perpendicular to the real line at the position of \(0\). Why at zero?
\emph{Because zero is its own additive inverse. The reflection of zero
in the mirror gives us back zero}.

\begin{figure}
\hypertarget{fig:mirror-at-zero}{%
\centering
\includesvg[width=0.9\textwidth,height=\textheight]{images/mirror-at-zero.svg}
\caption{Geometric interpretation of the additive
inverse.}\label{fig:mirror-at-zero}
}
\end{figure}

With reference to \cref{fig:mirror-at-zero}, the mirror is the
silver-colored line placed at zero. The irrational number
\(a = \sqrt{2}\), represented by the red dot, is reflected in the red
silvered side of the mirror to give the blue dot, which is
\(-a = -\sqrt{2}\). A second reflection of \(-a = -\sqrt{2}\) on the
blue silvered side gives us back the original number \(a = \sqrt{2}\).
This means that \emph{the additive inverse of the additive inverse gives
us back the original number}.

\begin{figure}
\hypertarget{fig:additive-inverse}{%
\centering
\includesvg[width=0.7\textwidth,height=\textheight]{images/additive-inverse.svg}
\caption{A graphical construction to obtain the additive
inverse.}\label{fig:additive-inverse}
}
\end{figure}

At the risk of expounding the obvious, let us take another look at a
pictorial representation of how to obtain the additive inverse of a
number \(x \in \mathbb{R}\). This is illustrated graphically in
\cref{fig:additive-inverse}. Every ordered pair of coordinates lying on
the straight line \(y = -x\) represents a number and its additive
inverse. So, if \(x = a\), \(y = -a\), and the ordered pair \((a, -a)\)
represents a number and its additive inverse. Moreover, the ordered pair
\((-a, a)\) also lies on this line and again represents a number and its
additive inverse, this time for \(x = -a\) and \(y = a\). This view
might be helpful for those who think in terms of pictures rather than
symbols.

An even more concrete algorithm to obtain the additive inverse is now
given. Suppose we want the additive inverse of \(1.5\). We draw a
vertical line from \(1.5\) on the \(x\)-axis to cut the line \(y = x\)
and then extend the line horizontally until it cuts the \(y\)-axis at
\(-1.5\). This last number is the additive inverse. It is clear from
\cref{fig:additive-inverse} that the \emph{only} point that is invariant
to this operation is \(0\) which maps to itself. \emph{The additive
inverse of the additive identity is itself}.

We show later in \cref{eq:mult-minus-one} that multiplying \(a\) by
\((-1)\) also gives us its additive inverse.

\hypertarget{the-multiplicative-identity-and-inverse-in-mathbbr}{%
\subsection{\texorpdfstring{The multiplicative identity and inverse in
\(\mathbb{R}\)}{The multiplicative identity and inverse in \textbackslash mathbb\{R\}}}\label{the-multiplicative-identity-and-inverse-in-mathbbr}}

We next consider the multiplicative identity and inverse in
\(\mathbb{R}\) which are also shown in \cref{tbl:axioms}.

The
\href{https://mathworld.wolfram.com/MultiplicativeInverse.html}{multiplicative
inverse} for arbitrary \(a \in \mathbb{R}\) is defined as
\(\frac{1}{a} \mbox{ for } a \ne 0\). Why the exclusion of zero? Let us
look for a graphical reason.

If we plot \(x\) against its multiplicative inverse \(\frac{1}{x}\), we
would get a \href{https://en.wikipedia.org/wiki/Hyperbola}{rectangular
hyperbola}, as shown in \cref{fig:hyperbola}.\footnote{We use the
  familiar \(x\) instead of \(a\) in the graphical context.} We have
coloured the two ``arms'' of the hyperbola in the first and third
quadrants in blue and red respectively, although they are part of the
same curve.

A construction to get the multiplicative inverse of some point \(a\)
(not equal to zero) is to locate the point with the \(x\) coordinate
\(a\) on the curve and to find out its \(y\) coordinate. Every ordered
pair \((a, \frac{1}{a})\) therefore represents a number and its
multiplicative inverse, with the proviso that \(a \ne 0\). Indeed, we
cannot ever get to \(a = 0\) on a graph of the hyperbola.

Note the following insights from \cref{fig:hyperbola}:

\begin{enumerate}
\def\labelenumi{\alph{enumi}.}
\item
  The function becomes unbounded as \(x\) approaches \(0\). This happens
  both from the positive and negative sides. Symbolically,
  \(\lim_{x \to 0^{+}}\frac{1}{x}=\infty\) and equally,
  \(\lim_{x \to 0^{-}}\frac{1}{x}=-\infty\).\footnote{There is no limit
    as \(x \to 0\).} \emph{This is why \(0\) has no multiplicative
  inverse}.
\item
  The multiplicative inverse has the same sign as the original number,
  since the hyperbola has two ``arms''.
\item
  There are only two values of \(x\) for which the multiplicative
  inverse is also the original number. They are at \(x = 1\) and
  \(x = -1\). This is because the line \(y = x\) intersects the
  hyperbola at two points: \((1, 1)\) and \((-1, -1)\).
\end{enumerate}

\begin{figure}
\hypertarget{fig:hyperbola}{%
\centering
\includesvg[width=0.95\textwidth,height=\textheight]{images/hyperbola.svg}
\caption{The multiplicative inverse in \(\mathbb{R}\) plotted as
\(y = \frac{1}{x}\).}\label{fig:hyperbola}
}
\end{figure}

\hypertarget{where-have-subtraction-and-division-disappeared}{%
\subsection{Where have subtraction and division
disappeared?}\label{where-have-subtraction-and-division-disappeared}}

If you are wondering where subtraction and division have disappeared,
they are hiding in plain sight. Subtracting \(b\) from \(a\) amounts to
adding the additive inverse of \(b\) to \(a\). So, \[
a - b = a + (-b)
\] Likewise, dividing \(a\) by \(b \ne 0\) amounts to multiplying \(a\)
by the multiplicative inverse of \(b\) which equals \(\frac{1}{b}\): \[
a \div b = a \times \frac{1}{b} = \frac{a}{b} \mbox{ for } b \ne 0.
\]

\hypertarget{multiplying-any-number-by-zero-always-gives-zero}{%
\subsubsection{Multiplying any number by zero always gives
zero}\label{multiplying-any-number-by-zero-always-gives-zero}}

Recall from \cref{tbl:axioms} that \(0\) is the unique \emph{additive}
identity. Zero's claim to notoriety is that it does \emph{not} have a
\emph{multiplicative inverse}. And this is because any number
\emph{multiplied} by zero gives zero.

Consider an arbitrary number \(a \in \mathbb{R}\). Recall also that
because \(0\) is the additive identity, \(0 + 0 = 0\). We may then claim
that \begin{equation}\protect\hypertarget{eq:zero-prod-zero}{}{
\begin{aligned}
a \cdot 0 &= a \cdot (0 + 0) & \mbox{ $0$ is the identity for addition}\\
a \cdot 0 &= a \cdot 0 + a \cdot 0 & \mbox { distributive law; subtract $a \cdot 0$}\\
0 &= a \cdot 0 &\\
\end{aligned}
}\label{eq:zero-prod-zero}\end{equation} Note that if we take \(a = 0\),
the product of zero with itself is also zero---something which might be
dismissed as trivially obvious, but which could be deceptively difficult
to prove.

We have carefully tiptoed our way to justify each step with one of the
field axioms. This is the power of the axiomatic approach. There is no
``wasted'' axiom; there is no ``missing'' axiom. The set of axioms are
the minimum necessary for the numbers to rest on a stable foundation.

This \emph{minimal sufficiency} is at the heart of why mathematics is so
strong. It has no extra fat. There is also no deficiency. It is frugal
but sufficient. We will encounter this same idea in the statement that a
\emph{basis is a minimal spanning set in a vector space}.\footnote{Which
  is the subject for another blog.} Any other algebraic structures, like
the complex numbers \(\mathbb{C}\), that obey the same axioms as
\(\mathbb{R}\), also have the same properties.

If you feel that \cref{eq:zero-prod-zero} is too much sleight of hand,
and you would like something more concrete, you can always console
yourself with the convenient but specific example of \(5 \times 0\).
Here \(5\) is the multiplier and \(0\) the multiplicand. So, we may
write: \[
\begin{aligned}
5 \times 0 &= 0 + 0 + 0 + 0 + 0; & \mbox{ multiplication is repeated addition}\\
&= (0 + 0) + (0 + 0) + 0; & \mbox{ addition is a binary operation}\\
&= (0 + 0) + 0; & \mbox{ $0$ is the additive inverse}\\
&= 0. & \mbox{ $0$ is the additive inverse}\\
\end{aligned}
\] And the choice of a whole number made it easy to see the
multiplication as repeated addition. What about \(0 \times 5\)? Because
multiplication is commutative, we may assert that \(0 \times 5\) is also
\(0\), without doing any additional work. I hope you are getting to see
mathematics as treasure hunt with clues and short cuts that lead to an
exciting finale. If you are interested in pursuing these ideas further,
please read
\href{https://medium.com/swlh/why-a-0-0-and-other-proofs-of-the-obvious-da52dd0caefb}{this
excellent article online}
{[}\protect\hyperlink{ref-chodnicki2020}{1}{]}.

\hypertarget{product-of-a-positive-and-a-negative-number}{%
\subsubsection{Product of a positive and a negative
number}\label{product-of-a-positive-and-a-negative-number}}

Negative numbers arose when loans had to be given and taken. They also
find use in describing the depth of an ocean trench as being so much
\emph{below} sea level. Other applications arise naturally with
temperatures below the freezing point or with electric charges of a
negative type, etc.

The signs of products featuring negative numbers are not intuitively
comprehensible. So, we have to rely on the axioms to guide us to
consistent results. What is the sign of the product of a positive and a
negative number? To find out, we first prove that \emph{multiplying a
number by the additive inverse of another number gives the additive
inverse of their product}, i.e.,
\begin{equation}\protect\hypertarget{eq:a-minus-b-eq-minus-ab}{}{
a(-b) = -(ab)
}\label{eq:a-minus-b-eq-minus-ab}\end{equation} To prove
\cref{eq:a-minus-b-eq-minus-ab}, we use the fact that zero multiplied by
any number is zero, as shown in \cref{eq:zero-prod-zero}.

\begin{equation}\protect\hypertarget{eq:minusprod-prodminus}{}{
\begin{aligned}
a \cdot 0 &= a(b + (-b)) & \mbox{ additive inverse}\\
0 &= ab + a(-b) & \mbox{ distributive law; add $-(ab)$}\\
-(ab) &= a(-b) &\\
\end{aligned}
}\label{eq:minusprod-prodminus}\end{equation}

If we now assume that \(a\) and \(b\) are both positive, then \(ab\) is
positive, and \((-b)\) is negative. The product of \(a\) and \((-b)\) is
the negative number \(-(ab)\). So, \emph{the product of a positive and a
negative number is a negative number}. Conversely, a negative number may
be split into the product of a positive number and a negative number.

One interesting corollary from \cref{eq:a-minus-b-eq-minus-ab} and
\cref{eq:minusprod-prodminus} is:
\begin{equation}\protect\hypertarget{eq:mult-minus-one}{}{
\begin{aligned}
(-1)(a) &= -(1 \cdot a).\\
&= -a\\
\end{aligned}
}\label{eq:mult-minus-one}\end{equation} This means that \emph{the
additive inverse of a number is obtained by multiplying the original
number by \((-1)\)}. This we also already know from
\cref{fig:additive-inverse}.

These slow but careful explanations might seem contrived, but they
provide guardrails against falling off when future mathematical objects
are encountered. And it is a whole lot more satisfying than hand-waving
or saying ``Just take it on faith.''

\hypertarget{why-is-the-product-of-two-negative-numbers-positive}{%
\subsubsection{Why is the product of two negative numbers
positive?}\label{why-is-the-product-of-two-negative-numbers-positive}}

Let us use the field axioms to navigate our way to this result as well.

Let \(a > 0, b > 0 \in \mathbb{R}\). Then, both \((-a)\) and \((-b)\)
are negative. Note that \(ab\) is positive and its additive inverse is
\(-(ab) = (-a)(b) = (a)(-b)\) which is negative from our previous
result.

The scheme we have followed so far is to add something to the object of
interest and use the axioms to prove that the sum is zero. The result we
are after will then pop out. Let us apply that method again, using the
final result fron \cref{eq:minusprod-prodminus}: \[
\begin{aligned}
-(ab) + (-a)(-b) &= (a)(-b) + (-a)(-b); & \mbox{ result from previous section}\\
&= (a + (-a))(-b) & \mbox{ distributive law}\\
&= (0)(-b) & \mbox{ additive inverse}\\
&= 0. & \mbox{ zero mutiplied by anything is zero}\\
\end{aligned}
\] The upshot is that we have \(-(ab) + (-a)(-b) = 0\) which means that
they are additive inverses. Since the additive inverse of \(-(ab)\) is
\(ab\), we conclude that \((-a)(-b) = ab\) implying that the product of
two negative numbers is a positive number. With the axioms to guide us,
we can move with the sure-footedness of a mountain goat as we scale the
heights and depths of each proof.

I have dealt with the arithmetic of fractions and negative numbers in my
freely downloadable chapter ``Arithmetic Revisited'' from
\href{https://swanlotus.netlify.app/sas-manuscript/SAS-partial.pdf}{\emph{Secrets
of Academic Success}}. I urge you to read it if you feel the need.

\hypertarget{exponentiation}{%
\section{Exponentiation}\label{exponentiation}}

Just as multiplication with whole numbers is repeated addition,
exponentiation is repeated multiplication. A new notation is used to
indicate repeated multiplication. We denote it by a \emph{superscript}
indicating how many times the number is multiplied:
\(5 \times 5 \times 5 = 5^{3}\), and so on. The number \(5\) is called
the \emph{base} and \(3\) the \emph{exponent}. We call the number
\(5^{3}\) as ``five cubed'' for reasons already explained, or as ``five
(raised) to the (power of) three''.

What is the exponent in the number written plainly as \(5?\) The base is
\(5\) but the exponent is \emph{implicit} or understood but not written.
If we take \(5 = 5^{1}\), the
\(5 \times 5 \times 5 = 5^{1} \times 5^{1} \times 5^{1} = 5^{3}\). When
we multiply numbers with the same base, we can simply add the exponents.
And this fact underlies a very powerful computational
device---logarithms.

\hypertarget{logarithms-multiplying-by-adding}{%
\section{Logarithms: multiplying by
adding}\label{logarithms-multiplying-by-adding}}

We may reduce multiplication to addition if we focused on the exponents
of a common base. This is exactly what was done by an eccentric Scottish
laird called
\href{http://www-history.mcs.st-andrews.ac.uk/Biographies/Napier.html}{John
Napier} whose labours have made all our lives much less tedious. The
books of logarithmic tables, affectionately called ``log books'' when I
was at school, along with the \href{http://sliderulemuseum.com/}{slide
rule} were the mainstay of physicists and engineers before the advent of
the electronic calculator in the mid-1970s. And they all relied on
Napier's scheme of reducing multiplication to addition.

Multiplying by adding is simple. Suppose we want to multiply \(2\) by
\(3\). We follow this algorithm:

\begin{enumerate}
\item
  Express the number \(2\) as a power of \(10\):
  \(2 \approx 10^{0.30103}\).
\item
  Express the number \(3\) as a power of \(10\):
  \(3 \approx 10^{0.47712}\).
\item
  Add the two powers or exponents: \(0.30103 + 0.47712 = 0.77815\).
\item
  Find out what number equals \(10^{0.77815}\). In our case,
  \(10^{0.77815} \approx 5.99998\).
\end{enumerate}

``Aha!'' you might say. ``But the answer is not exactly \(6\) which is
the correct answer.'' You are right. With logarithms and the limited
number of digits of precision, we can at best obtain a very good
approximation to a computation.

If you had to compute \(23.589 \times 459.1213\) what would you do? You
would run to a calculator and punch in the digits to get \(10830.212\)
with very little effort today. But if you lived in a period without
electronic calculators, you would be very glad that logarithms existed,
and you would happy to chirp out the answer as \(10830\), thanks to
Napier.

\hypertarget{square-roots}{%
\section{Square roots}\label{square-roots}}

Taking a square root is a form of \emph{exponentiation}. If I gave you a
number like \(9\) and asked you to find a number such that when it was
\emph{added} to itself, you would get \(9\), you would almost
instinctively divide \(9\) by \(2\) to give me \(4.5\) and indeed
\(4.5 + 4.5 = 9\).

If instead, I asked you to find that number, which when
\emph{multiplied} by itself gives \(9\), how would you go about solving
it? We know that in this case, the answer is \(3\) because
\(3 \times 3 = 9\). To approach it systematically, we need a symbol for
the operation, which when performed on \(9\) gives us \(3\). For
historical reasons it was called \emph{taking the square root} and the
symbol \(\surd\) is a stylized letter ``r'' for ``radix'' meaning
\href{https://math.stackexchange.com/questions/809799/why-the-name-square-root}{``root''}
in Latin {[}\protect\hyperlink{ref-squareroot}{2}{]}. So, we write
\(\sqrt{9} = 3\).

But is that the whole story? Recall that the product of two negative
numbers is positive. So, \((-3)(-3) = 9\) as well. So, which is the
``true'' or ``correct'' square root? By convention, we associate the
positive square root with the symbol \(\surd\). So, without any
ambiguity, \(\sqrt{9} = 3\) and \(-\sqrt{9} = -3\).

\hypertarget{complex-numbers}{%
\section{Complex numbers}\label{complex-numbers}}

Talking about square roots and negative numbers, can we ever take the
square root of a negative number? Do such numbers exist? If so, where?
And are they useful?

The squares of real numbers give rise to only two possibilities. The
square of zero is zero. The square of any non-zero real number, whether
positive or negative, is always positive, as we have just seen. So, the
possibility of a negative number being the square of a real number does
not ever arise.

But we do encounter a different situation when solving an algebraic
equation like \(x^{2} + 4 = 0\). When such equations were first
encountered, mathematicians simply said that they had \emph{no
solutions.} This is true even when you initially encounter them at
school today. \emph{This equation has no solution in the field of real
numbers.}

In the course of time, these pesky numbers---whose square is a negative
number---kept popping up insistently in unlikely places. They were then
reluctantly assigned mathematical existence with the somewhat pejorative
term \emph{imaginary numbers.} They were after all \emph{not real
numbers!}

In the course of time, a sandwich number was invented which was composed
of the sum of a real number and an imaginary number. This number was
called a \emph{complex number.} It is the first of several mathematical
objects we will encounter in this blog that are different from the real
numbers of everyday life.

The set of complex numbers is denoted by \(\mathbb{C}\) and is defined
as \[
\mathbb{C} = \{a+bi: a, b, \in \mathbb{R} \mbox{ and } i^2 = -1\}
\] In plain English, this means that the set of complex numbers, denoted
by \(\mathbb{C}\), is defined as all numbers of the form \(a + bi\)
where \(a\) and \(b\) are real numbers and \(i\), is called the
\emph{imaginary unit} and is the positive of the two roots of the
equation \(z^2 + 1 = 0.\)\footnote{Because \(i\) is often associated
  with current, electrical engineers often use the symbol \(j\) instead.}

But there is a little asymmetry in the expression \(a + bi\). While
\(b\) is multiplied by \(i\), the \(a\) stands alone. Or does it? We
could always write \(a(1) + bi\) to restore symmetry into the
expression. Then \(1\) is the unit for the real part and \(i\) is the
unit for the imaginary part. I find this latter way of writing the
complex number very reassuring because it displays the symmetry that I
was looking for. In the interests of cutting through the clutter though,
the \((1)\) is dropped by convention, and we write \(a + bi\).

\hypertarget{multiplication-of-complex-numbers}{%
\section{Multiplication of complex
numbers}\label{multiplication-of-complex-numbers}}

When we multiply complex numbers, we are really performing
multiplication on \emph{two pairs} of real numbers with the imaginary
unit sandwiched in between. Because of the property \(i^2 = -1\), the
rules of multiplication with terms involving \(i\) must be modified to
honour this equation. Let us see how, with a step-by-step example: \[
\begin{aligned}
(2 + 3i)(4 - 5i) & = 2(4) + 2(-5i) + (3i)(4) + (3i)(-5i)\\
& = 8 - 10i + 12i + (-15)(i^2)\\
& = 8 + 2i + (-15)(-1)\\
& = 23 + 2i.
\end{aligned}
\] Note that we use \(i^2 = -1\) and group like terms, but otherwise
proceed as normal. If we were to generalize this to a pair of complex
numbers, we get \[(a + bi)(c + di) = (ac - bd) + (ad + bc)i.\]

Although the rules of multiplication of complex numbers differ from
those of real numbers, the \emph{field axioms} still hold. This is the
purpose behind the development of abstract algebra. The principle is DRY
(Don't Repeat Yourself) or, equivalently, do not re-invent the wheel.

\hypertarget{polar-form-of-complex-numbers}{%
\subsection{Polar form of complex
numbers}\label{polar-form-of-complex-numbers}}

We have resorted to the unit circle to unravel the meaning of the
tangent ratio and function in a
\href{https://swanlotus.netlify.app/blogs/a-tale-of-two-measures-degrees-and-radians}{previous
blog}.

We now take recourse to the same unit circle to better understand the
multiplication of complex numbers. Let us press the \(xy\) coordinate
plane to represent complex numbers of the form \(a + ib\) where the real
part \(a\) is the \(x\) coordinate and the imaginary part \(ib\) is the
\(y\) coordinate. This use of the coordinate plane is referred to as an
\href{https://en.wikipedia.org/wiki/Complex_plane\#Argand_diagram}{Argand
diagram}. The \(x\) axis is labelled \(\Re\) to represent the real axis
and the \(y\) axis is labelled \(\Im\) to represent the real coefficient
of the imaginary part, shown in \cref{fig:complex-unit-circle}.

\begin{figure}
\hypertarget{fig:complex-unit-circle}{%
\centering
\includesvg[width=0.85\textwidth,height=\textheight]{images/complex-unit-circle.svg}
\caption{The complex plane with an arbitrary point \(z = a + ib\),
represented by the point \((a, ib)\) on the unit circle. The same number
may also be expressed as \(re^{i\theta}\) where
\(r=1\).}\label{fig:complex-unit-circle}
}
\end{figure}

The formula \begin{equation}\protect\hypertarget{eq:euler}{}{
e^{i\theta} = \cos\theta + i \sin\theta
}\label{eq:euler}\end{equation} is called the
\href{https://en.wikipedia.org/wiki/Euler\%27s_formula}{Euler Formula}.
This particular equation has been called
\href{https://www.feynmanlectures.caltech.edu/I_22.html}{``the
unification of algebra and geometry''} by the legendary physicist,
Richard Feynman {[}\protect\hyperlink{ref-feynman22}{3}{]}. Do read the
transcript of his lecture to better appreciate what exactly he means by
the above quote.

Referring to \cref{fig:complex-unit-circle}, we may say: \[
\begin{aligned}
a &= r\cos\theta\\
ib &= ir\sin\theta\\
r &= \sqrt{a^2 + b^2} \mbox{ since $\cos^2\theta + \sin^2\theta = 1$ }\\
\end{aligned}
\] \(r\) is called the \emph{modulus} and \(\theta\) the \emph{argument}
form of the complex number. Alternatively, it may be called the
\emph{polar} representation of the complex number.

What exactly is the advantage of all this jiggery-pokery? It makes
multiplication easier because exponents are added when the numbers they
represent are multiplied.

\begin{figure}
\hypertarget{fig:complex-polar}{%
\centering
\includesvg[width=1\textwidth,height=\textheight]{images/complex-polar.svg}
\caption{The multiplication of two complex numbers is more easily
performed when they are expressed in polar form. The \(x\) coordinate is
the real part and the \(y\) coordinate the imaginary
part.}\label{fig:complex-polar}
}
\end{figure}

If we have two complex numbers \(z_{1} = r_{1}e^{i\theta}\) and
\(z_{2} = r_{2}e^{i\phi}\), the modulus of their product is simply the
product of their moduli, and the argument is simply the sum of their
arguments, as illustrated in \cref{fig:polarmult}. This means:
\begin{equation}\protect\hypertarget{eq:polarmult}{}{
\begin{aligned}
z_{1}z_{2} &= r_{1}e^{i\theta} \cdot r_{2}e^{i\phi}\\
&= r_{3}e^{i\psi} \mbox{ where }\\
r_{3} &= r_{1}r_{2} \mbox{ and }\\
\psi &= \theta + \phi.
\end{aligned}
}\label{eq:polarmult}\end{equation}

\hypertarget{consistency-between-real-and-complex-multiplication}{%
\subsection{Consistency between real and complex
multiplication}\label{consistency-between-real-and-complex-multiplication}}

What happens if we used the rule for complex multiplication above but
set the imaginary parts to zero so that complex multiplication reverts
to the multiplication of two real numbers? Do we get consistent results?
Let us try it by substituting \(b = d = 0\) in the above case. We then
have: \[
\begin{aligned}
(a + 0i)(c + 0i) & = [ac - (0)(0)] + [a(0) + (0)c]i\\
& = ac.
\end{aligned}
\] It is a great relief that the result is as expected. The
multiplication of two complex numbers gives us the same result as the
earlier definition for multiplication of two real numbers when we set to
zero the imaginary parts of the two complex numbers. The definitions of
real and complex multiplication are therefore consistent.

It is a primary requirement in mathematics that when we extend the
definition of an operation on a simpler object to encompass a more
complicated mathematical object, the new definition should revert to the
accepted definition for the simpler object when the complicated object
reverts into the simpler object. This is the point about consistency
that I made at the start of this blog.

\hypertarget{complex-numbers-as-ordered-pairs}{%
\subsection{Complex numbers as ordered
pairs}\label{complex-numbers-as-ordered-pairs}}

Arithmetic operations on complex numbers result only in complex numbers
and do not give rise to new types of numbers. And all complex numbers
consist of two parts: a real part and an imaginary part.

Hence, if we develop a new notation for this two-part number for
purposes of arithmetic, we may dispense altogether with the symbol
\(i\), which after all, seems somewhat out of place in a numerical
context.

The simplest notation is to represent complex numbers as
\href{http://www.mathsisfun.com/definitions/ordered-pair.html}{\emph{ordered
pairs}} of real numbers with the convention that the first number is the
real part and the second number is the imaginary part. Once we have done
this, we need to re-define all the arithmetic operations for these
ordered pairs.

We may thus use the ordered pair \((a, b)\) to denote the complex number
\(a + bi\) and likewise, \((c, d)\) for the complex number \(c + di\).

Drawing upon our previous results, we may then \emph{define}
multiplication for this ordered pair as being \[
(a, b)\cdot(c, d) = (ac - bd, ad + bc).
\] Note that order matters here: the first number of the ordered pair is
the real part of the complex number and the second is the imaginary
part. We cannot swap them willy nilly. TO REMOVE \[
  \begin{bmatrix}
  8\\
  4\\
  \end{bmatrix}
  +
  \begin{bmatrix}
  2\\
  6\\
  \end{bmatrix}
  =
  \begin{bmatrix}
  10\\
  10\\
  \end{bmatrix}
\]

\hypertarget{vectors}{%
\section{Vectors}\label{vectors}}

Ordered pairs lead rather neatly to the idea of
{[}\emph{vectors}.{]}{[}https://en.wikipedia.org/wiki/Euclidean\_vector{]}\footnote{Properly
  called Euclidean vectors in our context.} Indeed, there is more than a
passing resemblance between complex numbers and two-dimensional vectors.

Both may be represented by ordered pairs of real numbers and the rules
for addition and subtraction of these ordered pairs are identical.
Moreover, they may both be represented as points on a
\href{https://en.wikipedia.org/wiki/Cartesian_coordinate_system}{Cartesian
plane}. Vectors are the second new mathematical object, after complex
numbers, that we are meeting in this blog.

\begin{figure}
\hypertarget{fig:two-vectors}{%
\centering
\includesvg[width=0.25\textwidth,height=\textheight]{images/two-vectors.svg}
\caption{Two vectors shown as directed line segments with different
lengths and directions.}\label{fig:two-vectors}
}
\end{figure}

A \emph{vector} is traditionally defined as a quantity having two
attributes: \emph{magnitude} and \emph{direction.} A simple everyday
example is the wind which has both a speed and direction, and may
therefore be represented by a vector. Indeed, if you have already
encountered vectors, you might think of them exclusively as
\emph{directed line segments} or lines of specific lengths with arrow
tips as shown in \cref{fig:two-vectors}.

\hypertarget{addition-and-subtraction-of-vectors-geometric-viewpoint}{%
\subsection{Addition and subtraction of vectors: geometric
viewpoint}\label{addition-and-subtraction-of-vectors-geometric-viewpoint}}

How do we add and subtract these geometric entities, let alone multiply
and divide them? If you have done physics at high school, you will know
that we use something called the \emph{parallelogram law.}

We draw a pair of two-dimensional vectors so that both originate from
the same point. We then complete the parallelogram formed by these two
vectors by drawing in the other two sides. The \emph{sum} of the two
vectors, or their \emph{resultant,} is the diagonal of the parallelogram
that starts at the same origin as the two vectors. This is something
best grasped from a picture: see \cref{fig:vector-sum}.

\begin{figure}
\hypertarget{fig:vector-sum}{%
\centering
\includesvg[width=0.7\textwidth,height=\textheight]{images/vector-sum.svg}
\caption{The parallelogram law for the addition of two
vectors.}\label{fig:vector-sum}
}
\end{figure}

The origin of the Cartesian plane is labelled \(O\). We have three named
vectors and three ordered pairs indicate the positions of their arrow
tips:

\begin{enumerate}
\tightlist
\item
  \(\symbf{u}\) : the vector from \(O\) to the point \((8, 4);\)
\item
  \(\symbf{v}\) : the vector from \(O\) to the point \((2, 6);\) and
\item
  \(\symbf{w}\) : the vector from \(O\) to the point \((10, 10)\).
\end{enumerate}

The dotted grey lines indicate the two sides of the parallelogram that
we draw to close the figure. The vector \(\symbf{w}\) is the diagonal
that represents the sum of \(\symbf{u}\) and \(\symbf{v}\). This sum is
written as \[
\symbf{u} + \symbf{v} = \symbf{w}
\] If you look at the illustration carefully, you would note that if we
added the first co-ordinate of \(\symbf{u}\) and the first co-ordinate
of \(\symbf{v}\) we get the first co-ordinate of \(\symbf{w}\). And
likewise for the second co-ordinate.

So, we may represent the addition of \(\symbf{u}\) and \(\symbf{v}\) as
\emph{row} vectors, so \[
(8, 4) + (2, 6) = ((8+2), (4+6)) = (10, 10)
\] Likewise, we could treat the vectors as \emph{column} vectors, and
write, equally correctly, as: \[
\begin{bmatrix}8\\4\end{bmatrix}
+
\begin{bmatrix}2\\6\\\end{bmatrix}
=
\begin{bmatrix}10\\10\\\end{bmatrix}
\] This is no accident. Nor is it a special case. It works for all
parallelograms.\footnote{I leave it to you to convince yourself of this.
  (Hint: use graph paper, draw the co-ordinate axes, and use algebraic
  variables for the co-ordinates of \(\symbf{u}\), \(\symbf{v}\), and
  \(\symbf{w}\).)}

The parallelogram law is a geometric statement of what happens when we
add two ordered pairs the way we would two complex numbers: \[
(a, b) + (c, d) = (a+c, b+d)
\] We may therefore dispense with line segments and arrow tips and
abstract vectors as ordered pairs even as we rescued complex numbers
from explicit association with \(i\).

\hypertarget{addition-and-subtraction-of-vectors-algebraic-viewpoint}{%
\subsection{Addition and subtraction of vectors: algebraic
viewpoint}\label{addition-and-subtraction-of-vectors-algebraic-viewpoint}}

We may identify two-dimensional vectors \emph{uniquely} by an ordered
pair representing their co-ordinates on the Cartesian plane. This is the
algebraic viewpoint. It is less cumbersome and more powerful as we have
already seen from the addition of two vectors.

Subtraction is equally simple. We may add the \emph{additive inverse} of
each component of the vector being subtracted to get: \[
(8, 4) + (-(2), -(6)) = (8, 4) + (-2, -6) = (6, -2)
\] Geometrically, this is tantamount to reversing the second vector and
adding it to the first. Of course, plain old subtraction also works,
provided we allow for negative numbers: \[
(8, 4) - (2, 6) = (6, -2)
\] Both results are the same and again underline the consistency that
runs through mathematics like a golden thread. \emojifont {😄}
\normalfont

\hypertarget{vectors-as-algebraic-entities}{%
\subsection{Vectors as algebraic
entities}\label{vectors-as-algebraic-entities}}

We have just seen that two-dimensional vectors may be represented by
ordered pairs on the Cartesian plane. This representation might be
extrapolated to include vectors of dimensions greater than two.
Obviously, we would then be moving from ordered pairs to ordered
triples, etc.\footnote{An ordered triple, for example, would live in our
  familiar three-dimensional space.}

To generalize, we may think of vectors as a list of ``numbers in a slim
teabag'' where their order matters. Formally, an \(n\)-dimensional
vector is an \emph{ordered list} of \(n\) numbers written within
enclosing parentheses or brackets and treated as a \emph{single entity.}
This is also called an
\href{https://en.wikipedia.org/wiki/Tuple}{\emph{ordered n-tuple}}. Each
individual number is called a \emph{component} of the vector.

The components of a vector may be written within parentheses or
brackets. They may be arranged vertically as a \emph{column vector} of a
single column and \(n\) rows or horizontally as a \emph{row vector} of a
single row and \(n\) columns. A column vector may be \emph{transposed}
into a row vector. Transposition is indicated by the superscript \(T\).
For example, with \(n = 4\) \[
\begin{aligned}
\begin{bmatrix}a\\b\\c\\d\\\end{bmatrix}^{T} & = \begin{bmatrix}a&b&c&d\\\end{bmatrix}\\
\begin{bmatrix}a&b&c&d\\\end{bmatrix}^{T} & = \begin{bmatrix}a\\b\\c\\d\\\end{bmatrix}\\
\end{aligned}
\] Note that there are no commas separating the elements of a row vector
unlike in ordered pairs. Also, we could just as well have used
parentheses as brackets.

It is conventional to assume that an arbitrary vector is a column
vector. Row vectors are then the transposes of the column vectors. This
is the convention we will follow.

\hypertarget{row-column-nomenclature}{%
\subsection{Row-column nomenclature}\label{row-column-nomenclature}}

The size of a vector is denoted by writing down the number of rows
followed by a \(\times\) sign followed by the number of columns. Thus a
column vector with four rows is a \emph{\(4 \times 1\) column vector}
while a row vector with four columns is a \emph{\(1 \times 4\) row
vector.}

By definition, a vector must have at least one dimension equal to \(1\).
When \emph{both} dimensions are equal to one, the vector degenerates
into a single component which, in the context of vectors, is called a
\href{https://en.wikipedia.org/wiki/Scalar_\%28mathematics\%29}{\emph{scalar}}
to distinguish it from a vector.

\hypertarget{addition-and-subtraction-of-vectors}{%
\subsection{Addition and subtraction of
vectors}\label{addition-and-subtraction-of-vectors}}

Addition and subtraction for the ordered n-tuples representing two
vectors may be defined as the addition or subtraction of their
\emph{respective components.}

Just to free ourselves from geometrical thinking about vectors, let us
add two \emph{four-dimensional} vectors whose components are given by
algebraic variables representing real numbers. \[
\begin{bmatrix}a\\b\\c\\d\\\end{bmatrix} + \begin{bmatrix}p\\q\\r\\s\\\end{bmatrix} = \begin{bmatrix}a+p\\b+q\\c+r\\d+s\\\end{bmatrix}
\]

Subtraction is equally ``commonsensical'': \[
\begin{bmatrix}a\\b\\c\\d\\\end{bmatrix} - \begin{bmatrix}p\\q\\r\\s\\\end{bmatrix} = \begin{bmatrix}a-p\\b-q\\c-r\\d-s\\\end{bmatrix}
\]

These vector sums and differences would be difficult to visualize
geometrically, but they are trivially routine algebraically.

\hypertarget{multiplication-of-vectors}{%
\section{Multiplication of vectors}\label{multiplication-of-vectors}}

It is easy to think of the addition or subtraction of vectors, say in
the context of ``wind speed'' and ``air speed'' of an aircraft. But what
does the multiplication of vectors consist of and what meaning could we
extract from the operation?

Vector multiplication is a strange, many-headed beast. It is important
to know what it is and what it is not. Here is a quick run down:

\begin{enumerate}
\def\labelenumi{\arabic{enumi}.}
\item
  Vector multiplication is different from real number multiplication.
\item
  Vector multiplication is different from complex number multiplication.
\item
  There are several varieties of vector multiplication, some of which
  give us scalars, others vectors, and still others matrices:\footnote{Ignore
    any unfamiliar terms for now.}

  \begin{enumerate}
  \def\labelenumii{(\alph{enumii})}
  \tightlist
  \item
    multiplication of a vector by a scalar to yield a vector
  \item
    \href{http://mathworld.wolfram.com/DotProduct.html}{dot product} or
    scalar product or inner product of two vectors to yield a scalar
  \item
    {[}cross product{]}{[}cross{]} of two vectors to yield a third
    vector orthogonal to the other two
  \item
    \href{http://en.wikipedia.org/wiki/Tensor_product}{tensor product}
    or \href{http://en.wikipedia.org/wiki/Outer_product}{outer product}
    of two vectors to yield a matrix
  \end{enumerate}
\end{enumerate}

\begin{enumerate}
\def\labelenumi{\arabic{enumi}.}
\setcounter{enumi}{3}
\item
  Each variety of vector product was devised as an operation because it
  is useful and has a ready meaning in a particular context.
\item
  When it comes to multiplication, vectors reveal their nature as a
  class of mathematical object quite different from real or complex
  numbers.
\end{enumerate}

Let us consider each type of multiplication in turn.

\hypertarget{multiplication-by-a-scalar}{%
\subsection{Multiplication by a
scalar}\label{multiplication-by-a-scalar}}

Multiplication of a vector by a scalar is the easiest to understand. In
this operation, we see the original arithmetic definition of real
multiplication at play. We are magnifying or diminishing the magnitude
of the vector by multiplying it with a scalar, while the direction of
the vector is either reversed or remains unchanged.

If we have a vector \(\symbf{u}\) and we multiply it by a scalar \(k\)
the result is the vector \(k\symbf{u}\). This may be easily visualized
geometrically. If we associate \(\symbf{u}\) with an arbitrary
four-dimensional column vector, we may write \[
\symbf{u} = \begin{bmatrix}a\\b\\c\\d\\\end{bmatrix}
\]

Multiplication of \(\symbf{u}\) by a real scalar \(k\) gives us \[
k\symbf{u} = k\begin{bmatrix}a\\b\\c\\d\\\end{bmatrix} = \begin{bmatrix}ka\\kb\\kc\\kd\\\end{bmatrix}
\]

When we multiply a vector by a scalar \(k\), each component of the
vector is individually multiplied by \(k\). The magnitude of the
resulting vector takes on a different meanings depending on the value
and sign of \(k\):

\begin{enumerate}
\item
  \(k = 1\): the vector is unchanged in magnitude and direction.
\item
  \(k = -1\): the vector is unchanged in magnitude but reversed in
  direction.
\item
  \(k < -1\): the vector is enlarged in magnitude and reversed in
  direction.
\item
  \(k > 1\): the vector is enlarged in magnitude and unchanged in
  direction.
\item
  \(-1 < k < 0\): the vector is diminished in magnitude and reversed in
  direction.
\item
  \(0 < k < 1\): the vector is diminished in magnitude and unchanged in
  direction.
\item
  \(k = 0\): the vector has zero magnitude and its direction is
  undefined.
\end{enumerate}

While this might seem quite a mouthful, it is really quite simple:

\begin{itemize}
\tightlist
\item
  a negative \(k\) reverses the direction;
\item
  a positive \(k\) keeps the direction unchanged;
\item
  value of \(k\) that lies between \(-1\) and \(0\) or between \(0\) and
  \(1\) diminishes the magnitude; and
\item
  a value of \(k\) less than \(-1\) or greater than \(+1\) increases the
  magnitude of the vector.
\end{itemize}

\emph{The special cases pertaining to \(k = \pm 1\) and \(k = 0\)
underscore the paramount importance of \(0\) and \(1\) in the whole of
mathematics. They are keystone numbers}.

Did you pick up the fact that after uncoupling geometry and vectors, we
finally resorted to geometry when talking about the meaning of scalar
multiplication? This dual viewpoint runs through much of mathematical
thinking.

Scalar division of vectors by \(k \ne 0\) is really multiplication by
\(\frac{1}{k}\) and is therefore not considered separately.

If we view scalar multiplication as a black box, it takes in one n-tuple
and gives out another n-tuple. Like the merchant in \emph{Aladdin and
the Wonderful Lamp,} we are simply exchanging old vectors for new. There
is no difference in \emph{kind} between the input and output
mathematical objects.

\hypertarget{the-zero-vector-is-not-the-number-zero}{%
\subsection{The zero vector is not the number
zero}\label{the-zero-vector-is-not-the-number-zero}}

Distinguish carefully between the \emph{real number \(0\)} and the
\emph{n-dimensional zero vector} which results from scalar
multiplication with \(k = 0\). The latter is an n-tuple of zeros and
does \emph{not} equal the real number zero. For example, with \(n = 4\),
the zero vector is: \[
\begin{bmatrix}0\\0\\0\\0\\\end{bmatrix} \neq 0
\]

As and when new mathematical objects are invented (or discovered?) new
definitions for the equivalents of zero and one for these new objects
may also be necessary. The \(n\)-dimensional zero vector is unique as
shown above.

The four-dimensional column vector with all entries equal to \(1\)
exists and is, of course, unique: \[
\begin{bmatrix}1\\1\\1\\1\\\end{bmatrix} \ne 1
\] But it is not special enough to merit its own name, as it does not
function as the vector analogue to the number \(1\). But it does have
its uses in data analysis and might get its own name in the future!
{[}\protect\hyperlink{ref-honner2022}{4}{]}

We now turn to other varieties of multiplication that may be applied to
vectors.

\hypertarget{dot-or-scalar-product}{%
\subsection{Dot or scalar product}\label{dot-or-scalar-product}}

The centred dot \(\cdot\) as a symbol for multiplication makes its
appearance here. The first departure from conventional multiplication
was with complex numbers. The
\href{https://en.wikipedia.org/wiki/Dot_product}{\emph{dot product,}}
also called the \emph{scalar product,} or \emph{inner
product}\footnote{An inner product is something more general, of which
  the dot product is a special case.} is the next variety of
unconventional multiplication.

\hypertarget{existence-of-the-dot-product}{%
\subsection{Existence of the dot
product}\label{existence-of-the-dot-product}}

\emph{The dot product is defined only between vectors of the same
dimension.} This is important to grasp. When we deal with real numbers,
the multiplicand, multiplier, and product are all real numbers. They are
mathematical objects of the \emph{same} kind. So, we may afford to be a
little careless in multiplying them together without performing any
check.

We cannot afford to be equally lackadaisical with vectors. We have to
respect the fact that they are not numbers per se, but a different type
of mathematical object. A product of some sort might not exist between
any two arbitrary vectors.

\hypertarget{example-of-dot-product}{%
\subsection{Example of dot product}\label{example-of-dot-product}}

It is helpful to begin with a concrete example. Let
\(\symbf{u}^{T} = \begin{bmatrix}1&2&3&4\\\end{bmatrix}\) and let
\(\symbf{v}^{T} = \begin{bmatrix}5&6&7&8\\\end{bmatrix}\). Their dot
product is written as: \[
\begin{aligned}
\symbf{u}^{T}\cdot\symbf{v} &= \begin{bmatrix}1&2&3&4\\\end{bmatrix} \cdot \begin{bmatrix} 5\\6\\7\\8\\ \end{bmatrix}\\
&= (1)(5) + (2)(6) + (3)(7) + (4)(8)\\
&= 5 + 12 + 21 + 32\\
&= 70
\end{aligned}
\]

If you look at the dot product carefully, you will see the following:

\begin{enumerate}
\item
  the first component of \(\symbf{u}\) is multiplied by the first
  component of \(\symbf{v}\) and likewise for the other components;
\item
  the individual products are then added together; and
\item
  the sum is the dot or scalar product.
\end{enumerate}

It is now clear why the two vectors must have the same dimensions. If
not, we will run out of either multiplier or multiplicand for pairwise
multiplication.

The result, being a sum of products, is a single number, or scalar,
explaining the name \emph{scalar product} for this operation. We prefer
the term dot product to avoid confusion with multiplication by a scalar.

It is easy to verify by direct evaluation that the dot product is
{[}commutative{]}{[}commutative{]} and therefore symmetrical. Verify if
you please that
\(\symbf{u}^{T}\cdot\symbf{v} = \symbf{v}^{T}\cdot\symbf{u}\) for the
above case.

Why did we need to write the dot product as being between a row vector
and a column vector? One reason is that the product of a column vector
with a row vector is actually a different type of multiplication which
we will meet
\protect\hyperlink{outer-product-of-two-vectorscircledtimes}{later}.
Another reason is that the row-column product mirrors
\protect\hyperlink{example-of-matrix-multiplication}{matrix
multiplication} as explained later.

\hypertarget{general-case-and-formula}{%
\subsubsection{General case and
formula}\label{general-case-and-formula}}

These results for the dot product may be generalized by taking
\(\symbf{u}\) and \(\symbf{v}\) to be arbitrary \(n\)-dimensional
vectors whose components may be referred to by the \emph{subscripts}
\(1\) to \(n\).

It is conventional to write the vector itself in boldface as
\(\symbf{u}\) or with an arrow on top as \(\vec{u}\) (or using some
other mark of distinction when written by hand or printed) whereas the
individual components are always written normally. They are after all
scalars. \[
\begin{aligned}
\symbf{u}^{T} &= \begin{bmatrix}u_{1} & u_{2} &  \cdots & u_{n}\\\end{bmatrix}\\
\symbf{v} &= \begin{bmatrix}v_{1} \\ v_{2} \\ \vdots \\ v_{n}\\\end{bmatrix}\\
\symbf{u}^{T}\cdot\symbf{v} &= u_{1}v_{1} + u_{2}v_{2} + \cdots + u_{n}v_{n}\\
&= \sum_{i=1}^{n} u_{i}v_{i}
\end{aligned}
\]

Observe that the vector \(\symbf{u}^{T}\) is a \(1 \times n\) vector
whereas \(\symbf{v}\) is an \(n \times 1\) vector. Their dot
product---between a \(1 \times n\) vector and an \(n \times 1\)
vector---yields a \(1 \times 1\) ``vector'' which is really a scalar. In
a manner of speaking, we may ``cancel out'' the two inner dimensions
\(n\) to get the dimension of the product as being \(1 \times 1\). This
mnemonic will prove useful later on as well.

\hypertarget{consistency-with-real-multiplication}{%
\subsubsection{Consistency with real
multiplication}\label{consistency-with-real-multiplication}}

What happens if our two vectors degenerate into scalars having single
components \(u_{1}\) and \(v_{1}\)? The dot product then collapses into
plain multiplication between two numbers and equals \(u_{1}v_{1}\),
entirely concordant with the product of two numbers. Consistency rules
once again!

\hypertarget{geometric-viewpoint}{%
\subsubsection{Geometric viewpoint}\label{geometric-viewpoint}}

What does the dot product mean? What does it signify given that vectors
originated as physical abstractions? We need to put on our ``geometric
glasses'' and view the dot product geometrically. We will need a little
bit of trigonometry on the way.\footnote{Which you might have to take on
  trust if it is unfamiliar or forgotten.}

\begin{figure}
\hypertarget{fig:projection}{%
\centering
\includesvg[width=0.6\textwidth,height=\textheight]{images/projection.svg}
\caption{Vector components as values of projections on the orthogonal
axes \(x\) and \(y\) for a two-dimensional vector
\(\symbf{u}\).}\label{fig:projection}
}
\end{figure}

Let us consider a two-dimensional vector \(\symbf{u}\) from the origin
\(O\) to a point \(U\) at \((u_{x}, u_{y})\) on the Cartesian plane. Let
\(OU\) make an angle \(\alpha\) with the positive \(x\) axis as shown.
Then, we have:

\begin{enumerate}
\def\labelenumi{\arabic{enumi}.}
\item
  By the Theorem of Pythagoras, the magnitude of the vector
  \(\symbf{u}\), denoted by \(\lVert\symbf{u}\rVert\), is given by
  \(\sqrt{u_{x}^2 + u_{y}^2}\). The symbol \(\lVert\mbox{}\rVert\)
  denoting a pair of double vertical lines represents the
  \href{https://mathworld.wolfram.com/Norm.html}{norm} or magnitude of
  the vector written within it.
\item
  The magnitudes of the
  \href{https://en.wikipedia.org/wiki/Vector_projection}{projections} of
  \(\symbf{u}\) in the directions of the \(x\) and \(y\) axes are
  respectively \[
  \begin{aligned}
  u_{x} &= \lVert\symbf{u}\rVert\cos\alpha\\
  u_{y} &= \lVert\symbf{u}\rVert\sin\alpha\\
  &= \lVert\symbf{u}\rVert\cos(90°-\alpha)\\
  \end{aligned}
  \]
\end{enumerate}

The magnitude of the projection of a vector in a \emph{particular}
direction is equal the magnitude of the vector multiplied by the
\emph{cosine} of the angle made by the vector with \emph{that}
direction.

We could make similar claims for a vector \(\symbf{v}\) at
\((v_{x}, v_{y})\) that makes an angle \(\beta\) with the positive \(x\)
axis.

Let the angle between the two vectors be denoted by
\(\alpha - \beta = \theta\). The dot product of the two vectors may then
be written as: \[
\begin{aligned}
\symbf{u} \cdot \symbf{v} &= u_{x}v_{x} + u_{y}v_{y}\\
&= \lVert\symbf{u}\rVert\cos\alpha\lVert\symbf{v}\rVert\cos\beta + \lVert\symbf{u}\rVert\sin\alpha\lVert\symbf{v}\rVert\sin\beta\\
&= \lVert\symbf{u}\rVert\lVert\symbf{v}\rVert(\cos\alpha\cos\beta + \sin\alpha\sin\beta)\\
&= \lVert\symbf{u}\rVert\lVert\symbf{v}\rVert\cos(\alpha - \beta)\\
&= \lVert\symbf{u}\rVert\lVert\symbf{v}\rVert\cos\theta\\
\end{aligned}
\]

\begin{figure}
\hypertarget{fig:dot-product}{%
\centering
\includesvg[width=0.65\textwidth,height=\textheight]{images/dot-product.svg}
\caption{Dot product of two vectors, \(\symbf{u}\) and
\(\symbf{v}\).}\label{fig:dot-product}
}
\end{figure}

The dot product of two vectors is equal to the product of their
magnitudes and the cosine of the angle between them. It is a scalar.

\hypertarget{applications-of-the-dot-product}{%
\subsubsection{Applications of the dot
product}\label{applications-of-the-dot-product}}

Unlike straightforward multiplication of real or complex numbers, the
dot product seems a little contrived. Why is it so defined? And is it
useful?

The answers to both questions lie in the practical utility of the dot
product. Vectors are used to represent forces, displacements, momenta,
and a whole host of other abstractions that are the bread and butter of
physics. And the dot product neatly dovetails with a recurring pattern
of relationships in physics where two vectors give rise to a scalar in a
multiplicative fashion.

For example,
\href{https://en.wikipedia.org/wiki/Work_\%28physics\%29}{mechanical
work} \(W\) is a scalar defined as the dot product of the vector
representing force \(\symbf{F}\) and the vector representing
displacement \(\symbf{s}\) through the equation
\(W = \symbf{F}\cdot\symbf{s}\). Alternatively, work is the projection
of the force in the direction of the displacement, multiplied by the
displacement. Both definitions are equivalent.

The dot product is succinct, precise, notationally crisp, and
practically useful. That is why it has been defined and that is why it
still exists.

\hypertarget{the-cosine-and-sine-functions}{%
\subsubsection{The cosine and sine
functions}\label{the-cosine-and-sine-functions}}

The cosine and sine of an angle are
\href{https://en.wikipedia.org/wiki/Trigonometric_Ratios}{trigonometric
ratios} from right-angled triangles that were later expanded in scope to
become
\href{https://mathworld.wolfram.com/Trigonometry.html}{mathematical
functions}. Graphs of \(\cos x\) and \(\sin x\) against \(x\), in the
``unitless unit'' called
\href{https://en.wikipedia.org/wiki/Radian}{radians}, are shown
below.\footnote{See my blog
  \href{https://swanlotus.netlify.app/blogs/a-tale-of-two-measures-degrees-and-radians}{\emph{A
  tale of two measures: degrees and radians}} if you are unfamiliar with
  radians.}

For now, we only need to focus on these facts, bearing in mind that
\(x\) is in radians:\footnote{If radians bother you, keep in mind that
  \(0°\) equals \(0\) radians and that \(\frac{\pi}{2}\) radians equals
  \(90°\).}

\begin{enumerate}
\item
  The values of \(\cos x\) and \(\sin x\) lie only between \(-1\) and
  \(1\).
\item
  \(\cos 0 = 1\) and \(\cos(\frac{\pi}{2}) = 0\).
\item
  \(\sin 0 = 0\) and \(\sin(\frac{\pi}{2}) = 1\).
\item
  \(\sin x = \cos(\frac{\pi}{2} - x)\).
\item
  \(\cos x = \sin (\frac{\pi}{2} - x)\).
\end{enumerate}

\begin{figure}
\hypertarget{fig:cosx-sinx}{%
\centering
\includesvg[width=0.9\textwidth,height=\textheight]{images/cosx-sinx.svg}
\caption{Graphs of \(\cos x\) and \(\sin x\) for the domain
\([-\pi, \pi]\) in radians or \([-180, 180]\) in degrees. The \(x\) axis
is labelled in radians.}\label{fig:cosx-sinx}
}
\end{figure}

\hypertarget{oddness-and-evenness}{%
\subsubsection{Oddness and evenness}\label{oddness-and-evenness}}

Observe from these graphs that \(\cos x\) is an
\href{https://mathworld.wolfram.com/EvenFunction.html}{even function}
that is symmetrical about the \(y\) axis whereas \(\sin x\) is an
\href{https://mathworld.wolfram.com/OddFunction.html}{odd
function}.\footnote{These properties make the dot product a
  \href{https://mathworld.wolfram.com/Commutative.html}{commutative}
  operation and the cross product an
  \href{https://mathworld.wolfram.com/Anticommutative.html}{anti-commutative}
  operation.}

Although not apparent from the graphs, both the cosine and sine
functions are
\href{https://mathworld.wolfram.com/PeriodicFunction.html}{periodic} and
repeat themselves every \(360°\) or \(2\pi\) radians for all values of
the independent variable.

\hypertarget{orthogonality}{%
\subsubsection{Orthogonality}\label{orthogonality}}

To recapitulate, the cosine of an angle is \(1\) at zero degrees and
\(0\) at ninety degrees. Therefore if two vectors are
\href{https://mathworld.wolfram.com/Orthogonal.html}{orthogonal}---or
perpendicular, or at ninety degrees---to each other, their dot product
is zero.

For example, a force vector \(\symbf{F}\) has no effect perpendicular to
the direction in which it acts, and this is because its \emph{component}
or \href{https://en.wikipedia.org/wiki/Vector_projection}{projection} in
that direction is \(\lVert\symbf{F}\rVert\cos 90° = 0\).

The dot product therefore measures the \emph{degree of alignment} or
\emph{similarity} between two vectors. When the angle between them is
zero degrees, this alignment is at its greatest. When the vectors are
orthogonal, each vector has no component in the direction of the other;
so they are \emph{independent.} When the two vectors make an angle
greater than \(90°\) the sine of their angle is negative. The two
vectors act in opposition when they are at an angle of \(180°\) with
each other.

Orthogonality---and the independence of vectors it implies---is a very
powerful property that finds application daily whenever we talk over the
telephone or download a compressed image from the Web.

The idea of projecting some mathematical object onto another and the
idea of one mathematical object being orthogonal to another are both
fundamental to many areas of mathematics and are well worth keeping in
mind.

We now move on to the next type of vector product.

\hypertarget{cross-product}{%
\subsection{Cross product}\label{cross-product}}

The third type of vector product is the
\href{https://mathworld.wolfram.com/CrossProduct.html}{\emph{cross
product}.} Because the dot product gave a \emph{scalar} result that
involved the \emph{cosine} function, you might ask tongue in cheek,
whether the cross product produces a \emph{vector} result that involves
the \emph{sine} function in its definition. And facetious or not, you
are actually right. \emojifont {😄} \normalfont

The cross product is a vector and it does involve the sine of the angle
between the two vectors. In addition, just as in the dot product,
orthogonality peeps at us again through the cross product.

We will consider three-dimensional vectors. Any pair of
three-dimensional vectors \(\symbf{u}\) and \(\symbf{v}\) between them
define a two-dimensional plane. Just think of two rulers arranged in any
orientation on a flat table to visualize and understand why.

The result of a cross product is orthogonal to the two vectors giving
rise to it. There are \emph{two} directions orthogonal to the plane.
Think of the flat table again. An arrow at right angles to the table
coming \emph{out} of it and pointing \emph{upwards} is in one direction.
Now reverse the direction of the arrow so that it goes \emph{into} the
table pointing \emph{downwards.} This is the other orthogonal direction.
They both lie along the same straight line but are oriented in opposite
directions.

We are now ready to define the cross product as \[
\begin{aligned}
\symbf{w} &= \symbf{u} \times \symbf{v}\\
&\triangleq (\lVert\symbf{u}\rVert \lVert\symbf{v}\rVert \sin \theta) \thinspace \symbf{n}
\end{aligned}
\]

\begin{figure}
\hypertarget{fig:cross-product}{%
\centering
\includesvg[width=0.65\textwidth,height=\textheight]{images/cross-product.svg}
\caption{The cross product of two vectors, \(\symbf{u}\) and
\(\symbf{v}\) is given by \(\symbf{w}\). The direction of the angle
\(\theta\) is from \(\symbf{u}\) to \(\symbf{v}\). By the right-hand
rule, \(\symbf{w}\) is positive in the direction of its
arrow.}\label{fig:cross-product}
}
\end{figure}

Both the vectors \(\symbf{u}\) and \(\symbf{v}\) point outward from the
same origin \(O\). The cross product vector \(\symbf{w}\) is
perpendicular or orthogonal to \emph{both} \(\symbf{u}\) and
\(\symbf{v}\) and again points outward from the same origin \(O\).

The expression
\((\lVert\symbf{u}\rVert \lVert\symbf{v}\rVert \sin \theta)\) is a
scalar. \(\symbf{n}\) is a
\href{http://mathworld.wolfram.com/UnitVector.html}{\emph{unit vector}}
perpendicular to both \(\symbf{u}\) and \(\symbf{v}\) with a magnitude
of one. Its direction can only be one of two as we have seen. To
determine which, we use a convention called the
\href{http://mathworld.wolfram.com/Right-HandRule.html}{right-hand rule}
or \href{http://en.wikipedia.org/wiki/Right-hand_rule}{right hand
corkscrew rule.}

Imagine that you are rotating a corkscrew starting at \(\symbf{u}\) and
moving toward \(\symbf{v}\). The direction in which the corkscrew
advances is the positive direction for the unit vector
\(\symbf{n}\).\footnote{This is a convenient mathematical convention
  which is also in accord with actual physical situations.}

Since the corkscrew would then move \emph{upwards,} that is the
direction of both \(\symbf{w}\) and \(\symbf{n}\). The sole purpose of
\(\symbf{n}\) is to indicate the direction of \(\symbf{w}\) as
determined by the right-hand rule. The letter \(\symbf{n}\) is used to
indicate that it is \emph{normal} or \emph{perpendicular} to the plane.
The only purpose of \(\symbf{n}\) is to denote the \emph{direction} of
\(\symbf{w}\). Being a \emph{unit vector}, the magnitude of
\(\symbf{n}\) is \(\lVert\symbf n\rVert = 1\).

In the cross product, we have just met the \(\times\) sign for
multiplication again, but so far afield from its original use and
meaning that it is almost unrecognizable except for form. Many
mathematical terms and symbols are reused in different contexts with
completely different meanings.

\hypertarget{anti-commutativity}{%
\subsection{Anti-commutativity}\label{anti-commutativity}}

If we were to compute \(\symbf{v} \times \symbf{u}\) we would turn the
right-handed corkscrew from \(\symbf{v}\) to \(\symbf{u}\) and the
cross-product vector would then point \emph{downwards.} Its magnitude
would however be the same as before. We therefore write: \[
\symbf{v} \times \symbf{u} = - (\symbf{u} \times \symbf{v})
\] The cross product is said to be
\href{https://mathworld.wolfram.com/Anticommutative.html}{anti-commutative}.
This means that if we reverse the order of the operands, there is a
change in the sign of the result. In contrast, the dot product is
commutative. So, each variety of vector multiplication has its own
well-delineated properties.

\hypertarget{applications-of-the-cross-product}{%
\subsection{Applications of the cross
product}\label{applications-of-the-cross-product}}

Like the dot product, the cross product owes its ubiquity to its
usefulness in physics. For example, the
\href{https://en.wikipedia.org/wiki/Torque}{torque} vector
\(\symbf{\tau}\)\footnote{Pronounced tau.} is defined by
\(\symbf{\tau} = \symbf{r}\times\symbf{F}\). Torque is what makes the
wheels of a car rotate. Visit
\href{https://en.wikipedia.org/wiki/Torque}{this page} to see an
animation and explanation of what the term ``torque'' means. The cross
product also simplifies the mathematical description of the laws of
electromagnetism.

\hypertarget{outer-product-of-two-vectors}{%
\section{Outer product of two
vectors}\label{outer-product-of-two-vectors}}

The outer product is the last of the four varieties of multiplication
for vectors that we will consider here.

Recall that the dot product is defined as the scalar resulting from the
multiplication of a \(1\times n\) row vector with an \(n \times 1\)
column vector. The result was a \(1 \times 1\) ``vector'' which is
really a scalar. The two ``inner dimensions'' \(n\) cancel out.

What happens if we swap the order and start multiplying an
\(n \times 1\) column vector with a \(1\times n\) row vector? Would the
two ``inner dimensions,'' both equal to \(1\), cancel out? And would we
get an \(n \times n\) ``vector'' of some sort?

Indeed we do. And the resulting product is a different mathematical
object called a
\href{http://en.wikipedia.org/wiki/Matrix_\%28mathematics\%29}{matrix.}
This is an example of a mathematical operation involving two known
mathematical objects whose result gives rise to a new kind of
mathematical object which then acquires a life and personality of its
own.

This type of multiplication is called an
\href{http://en.wikipedia.org/wiki/Outer_product}{outer product,} in
contrast to the dot product which is a type of
\href{http://planetmath.org/encyclopedia/InnerProduct.html}{inner
product.} It is also sometimes called a
\href{http://en.wikipedia.org/wiki/Tensor_product}{tensor product} in
honour of the fact that we are ascending a hierarchy in
\href{http://en.wikipedia.org/wiki/Linear_algebra}{linear algebra} that
starts with scalars and moves on to vectors and then to matrices and on
to
\href{http://en.wikipedia.org/wiki/Tensor_\%28intrinsic_definition\%29}{tensors}
with progressive generalizations at each step.

\hypertarget{outer-product-symbol-and-example}{%
\subsection{Outer product: symbol and
example}\label{outer-product-symbol-and-example}}

The symbol for the outer product is \(\otimes\) which is a ``circled
times'' sign.

As already presaged, the outer product results from the multiplication
of an \(m \times 1\) column vector and a \(1 \times n\) row vector to
give an \(m \times n\) matrix. As with the dot product, the ``inner
dimensions'' of the two vectors, both equal to one here, ``cancel out''
in a manner of speaking, to yield a matrix of dimension \(m \times n\).

In contrast to the dot product, however, the two vectors may have
different numbers of elements. This is why the resulting matrix is not
necessarily a square matrix with equal numbers of rows and columns, but
rather has \(m\) rows and \(n\) columns. The values of \(m\) and \(n\)
may be equal but are not required to be so.

Here is an example that will help you decipher how the outer product is
computed: \[
\begin{bmatrix}
1\\2\\3\\4\\
\end{bmatrix}
\otimes
\begin{bmatrix}
5 & 6 & 7\\
\end{bmatrix}
= %
\begin{bmatrix}
5 & 6 & 7\\
10 & 12 & 14\\
15 & 18 & 21\\
20 & 24 & 28\\
\end{bmatrix}
\] Each element of the outer product matrix is the product of a pair of
real numbers. What is new here is their position in the product matrix
and their meaning in this context.

\hypertarget{the-outer-product-is-non-commutative}{%
\subsection{The outer product is
non-commutative}\label{the-outer-product-is-non-commutative}}

The outer product is not commutative. To see why, consider \(\symbf{u}\)
as an \(m \times 1\) column vector and \(\symbf{v}^{T}\) as a
\(1 \times n\) row vector. Then, \(\symbf{u} \otimes \symbf{v}^{T}\)
gives an \(m \times n\) matrix. But \(\symbf{v} \otimes \symbf{u}^{T}\)
gives an \(n \times m\) matrix. The two are obviously not constrained to
be equal.

\hypertarget{applications-of-the-outer-product}{%
\subsection{Applications of the outer
product}\label{applications-of-the-outer-product}}

The outer product finds application in fields like physics, electrical
engineering, and statistics. Whether application precedes or follows the
original mathematical development, whenever a new mathematical object
persists, it is almost always due to its usefulness for some purpose or
other.

The outer product of two vectors leads to matrices and the
\emph{multiplication of matrices} is yet another variety of
multiplication. It is the last we will consider in this blog.

\hypertarget{matrices}{%
\subsection{Matrices}\label{matrices}}

We have already seen that a matrix consists of a lot more ``numbers in a
teabag'' in which order is respected. An \(m \times n\) matrix is an
array of \(m\) rows and \(n\) columns of numbers. A row vector is just a
\href{http://en.wikipedia.org/wiki/Degenerate_\%28mathematics\%29}{degenerate}
matrix with one row, i.e., \(m = 1\), whereas a column vector is a
degenerate matrix with a single column, i.e., \(n = 1\). The plural of
\emph{matrix} is \emph{matrices.}

Hark back to complex numbers and remember how the real numbers are
merely complex numbers whose imaginary parts are zero. We hear a similar
refrain with matrices, vectors, and scalars. Vectors are matrices with
one column or one row. A matrix with a single column and row is a
scalar.

Each time a mathematical object is generalized, we will see a previously
defined object appearing as a degenerate case of the new object. This
provides a link between the new and the old and also ensures that
consistency is maintained in this evolutionary spiral.

It is customary to refer to a matrix by an uppercase letter. The
individual numbers, or \emph{elements,} of a matrix are usually denoted
by a lowercase letter and given double subscripts denoting their
position in the matrix.

The element \(a_{ij}\) in a matrix \(A\) is the element occupying the
\(i\)th row and the \(j\)th column in the matrix. For the subscript, the
row number is written first followed by the column number. If all this
seems too abstract, here is a concrete example of a \(2 \times 3\)
matrix: \[
A = \left[
\begin{matrix}
1 & 3 & 5\\
2 & 4 & 6\\
\end{matrix}
\right]
\] where we have \(a_{12} = 3\) and \(a_{21} = 2.\)

\hypertarget{applications-of-matrices}{%
\subsection{Applications of matrices}\label{applications-of-matrices}}

Matrices arose naturally from the study and solution of
\href{http://mathworld.wolfram.com/LinearSystemofEquations.html}{systems
of linear equations.} They are also useful in succinctly embodying
geometric transformations of points in the two-dimensional Cartesian
plane. They are profoundly useful in electrical engineering, physics,
economics, and many other fields.

Indeed, if one considers matrices as a class of mathematical object,
what we do with them and the meanings we assign to these actions are
largely limited only by our imagination and the mathematical consistency
of the results. This is how new mathematics is built up from the old,
and constantly expanded in scope, variety, and application.

\hypertarget{matrix-multiplication}{%
\section{Matrix multiplication}\label{matrix-multiplication}}

The product of matrix \(A\) with matrix \(B\) is denoted by \(AB\) with
no intervening symbol. If matrix \(A\) has dimensions \(m \times n\),
and matrix \(B\) has dimensions \(p \times q\), and we wish to multiply
them \emph{in that order,} we first need to ensure that the ``inner
dimensions,'' \(n\) and \(p\) in this case, are indeed equal. If
\(n = p\) the two matrices \(A\) and \(B\) can be multiplied together to
yield the \(m \times q\) matrix \(AB\), and the two matrices are said to
be \href{http://en.wikipedia.org/wiki/Conformable_matrix}{conformable.}
Otherwise, the product \(AB\) does not exist. Conversely, if
\(n \neq p\) but \(q = m\), the matrix product \(BA\) exists and the
result is a \(p \times n\) matrix.

Any two real numbers may be multiplied together, but the product of any
two matrices need not necessarily be defined. As the mathematical
objects that we deal with become increasingly complex, additional
constraints often apply to operations on them.

\hypertarget{example-of-matrix-multiplication}{%
\subsection{Example of matrix
multiplication}\label{example-of-matrix-multiplication}}

\begin{figure}
\hypertarget{fig:matrixmult}{%
\centering
\includesvg[width=0.6\textwidth,height=\textheight]{images/matrixmult.svg}
\caption{How the product of two matrices is computed. Like coloured
elements are multiplied together, and those products summed, to give the
single result shown in black. The algorithm is repeated until
exhaustion.}\label{fig:matrixmult}
}
\end{figure}

Here is an example of matrix multiplication. We group a whole row on the
left matrix and multiply it element-wise with a whole column on the
right matrix and add all the products. In this case, we compute
\((1)(7) + (2)(9) + (3)(1) = 7 + 18 + 3 = 28\).

This is reminiscent of the dot product. Indeed, matrix multiplication
may be viewed as a generalization of the dot product for matrices and
the dot product as a degenerate case of matrix multiplication in which
the left matrix is a row vector and the right matrix is a column vector.

\hypertarget{non-commutativity}{%
\subsection{Non-commutativity}\label{non-commutativity}}

For any two matrices \(A\) and \(B\), the matrix product \(AB\) exists
only if the matrices are conformable, i.e., the number of columns in
\(A\) equals the number of rows in \(B\). Likewise, the product \(BA\)
exists only if the number of columns in \(B\) equals the number of rows
in \(A\).

If both matrices are square and of the same dimensions, is their
multiplication commutative? In other words, does \(AB = BA\)?

In mathematics a single exception falsifies the rule. Let us consider
the following simple example: \[
\begin{aligned}
A = \left[
\begin{matrix}
1 & 3\\
2 & 4\\
\end{matrix}
\right]
&\quad
B = \left[
\begin{matrix}
5 & 7\\
6 & 8\\
\end{matrix}
\right]
\\
AB = \left[
\begin{matrix}
23 & 31\\
34 & 46\\
\end{matrix}
\right]
&\quad
BA = \left[
\begin{matrix}
19 & 43\\
22 & 50\\
\end{matrix}
\right]
\end{aligned}
\] Clearly matrix multiplication is not commutative. In any matrix
product, the matrix on the left \emph{pre-multiplies} the matrix on the
right. Conversely, the matrix on the right \emph{post-multiplies} the
matrix on the left.

\hypertarget{geometric-effects-of-matrix-multiplication-2d-case}{%
\subsection{Geometric effects of matrix multiplication: 2D
case}\label{geometric-effects-of-matrix-multiplication-2d-case}}

A \(2\times 2\) matrix may be interpreted as a geometric transformation
of points on the Cartesian plane. Indeed, this is often how you might
have first encountered matrices as mathematical objects. Suppose we wish
to reflect an arbitrary point \((a, b)\) using the \(y\) axis as a
mirror. With a little visualization, you will agree that image point is
\((-a, b)\).

How might a matrix accomplish this? If we \emph{post-multiply} a matrix
by a vector, we will get another vector. We need to transform \(a\) to
\(-a\) while leaving \(b\) unchanged. A little thought or tinkering with
matrices will show that required matrix is as shown below: \[
\begin{bmatrix}
-1 & 0\\
0 & 1\\
\end{bmatrix}
\begin{bmatrix}
a\\b\\
\end{bmatrix}
= %
\begin{bmatrix}
-a\\b\\
\end{bmatrix}
\] Because \(a\) and \(b\) are arbitrary, we may associate a
\href{http://planetmath.org/encyclopedia/DerivationOf2DReflectionMatrix.html}{reflection}
across the \(y\) axis with the above matrix. Likewise, it may be shown
that a counter-clockwise rotation about the origin \(O\) through an
angle \(\theta\) with the positive \(x\)-axis is associated with the
matrix \[
\begin{bmatrix}
\cos\theta & -\sin\theta\\
\sin\theta & \cos\theta\\
\end{bmatrix}
\] We can then chain together such
\href{http://en.wikipedia.org/wiki/Transformation_matrix}{transformations}
by multiplying the relevant matrices in the correct order. Those who
author video games use concepts such as these to program their games.

Pay attention to the interplay between the symbolic and the pictorial,
between the algebraic and the geometric aspects of the one operation. If
you develop the ability to maintain this ``dual vision'' as you study
mathematics, it will be helpful for your own unfolding understanding. A
strange algebraic object correctly used might work geometric miracles
right under your nose, and vice versa.

And that completes my survey of varieties of multiplication. I do not
know if you are heaving a sigh of relief but I certainly am! We have
only scratched the surface here. There are many more varieties of
multiplication and each serves a purpose. You will discover them in the
course of your studies.

\hypertarget{summary}{%
\section{Summary}\label{summary}}

This blog has been a journey through mathematics tracking multiplication
as the single theme.

Multiplication happens between two mathematical objects to yield a
third. In this survey, we have encountered \emph{four} different
mathematical objects:

\begin{enumerate}
\tightlist
\item
  Real numbers
\item
  Complex numbers
\item
  Vectors
\item
  Matrices
\end{enumerate}

The way the multiplication is accomplished as well as its meaning differ
with context. We have met \emph{seven} different varieties of
multiplication here:

\begin{enumerate}
\tightlist
\item
  Real multiplication

  \begin{itemize}
  \tightlist
  \item
    product is real
  \item
    commutative
  \end{itemize}
\item
  Complex multiplication

  \begin{itemize}
  \tightlist
  \item
    product is complex
  \item
    commutative
  \end{itemize}
\item
  Multiplication of a vector by a scalar

  \begin{itemize}
  \tightlist
  \item
    product is a vector
  \item
    magnitude and direction depend on value of scalar
  \end{itemize}
\item
  Dot product of two vectors

  \begin{itemize}
  \tightlist
  \item
    product is a scalar
  \item
    commutative
  \item
    measures ``similarity'' or ``alignment'' between the two vectors
  \item
    involves cosine of angle between the two vectors
  \end{itemize}
\item
  Cross product of two vectors

  \begin{itemize}
  \tightlist
  \item
    product is a third vector orthogonal to the two vectors
  \item
    anti-commutative
  \item
    involves the sine of the angle between the two vectors
  \end{itemize}
\item
  Outer product of two vectors

  \begin{itemize}
  \tightlist
  \item
    product is a matrix
  \item
    not commutative
  \end{itemize}
\item
  Matrix product

  \begin{itemize}
  \tightlist
  \item
    product is another matrix
  \item
    not commutative
  \end{itemize}
\end{enumerate}

We have made glancing acquaintance with logarithms and how they
transform multiplication into addition. We have also skimmed over the
trigonometric functions, given their place in the theory of vectors.

If you carry away nothing else from this blog than a few
\emph{qualitative ideas,} they should include some of these:

\begin{enumerate}
\item
  Multiplication is a binary operation: it takes place between two
  compatible mathematical objects.
\item
  Mathematical objects are more varied than animals in a zoo. Each has
  its own nature, diet, habitat etc. Apart from the real numbers, we
  have encountered complex numbers, vectors, and matrices here.
\item
  Multiplication is commutative for the real and complex numbers but not
  for necessarily for vectors or matrices.
\item
  The meaning of a product has evolved a long way from the original
  ``three lots of four'' in the context of whole numbers. The product of
  a multiplication might yield an object that is quite different from
  the multiplicand and multiplier. We have seen scalars popping out of
  dot products of two vectors and matrices issuing from the outer
  product of two vectors.
\item
  The ideas of zero and one, of symmetry, of commutativity, of
  consistency of definitions, of projections, and of orthogonality, are
  worth remembering because they pervade much if not all of mathematics.
\end{enumerate}

May the product be with you!

\hypertarget{to-explore-further}{%
\subsection{To explore further}\label{to-explore-further}}

Nahin's books

Feynman's lectures etc.

\hypertarget{afterword}{%
\section{Afterword}\label{afterword}}

This blog started off as something that promised to be short and fizzy,
tangy and piquant. But it soon became a little like hot treacle: too hot
to swallow and too sticky to spit out. It transmogrified into a
jumboblog. If you have stuck with me this far, I applaud and thank you.

The thought crossed my mind that I could split this blog into several
sub-blogs. But I soon gave up that idea because the connectedness of the
thread will be lost in the segmentation. So, here you have the whole hog
and the whole blog.

Mathematics is like a pastry puff: only the layers never seem to end and
neither does the puff! I needed to cap the well at some point, and
matrices seemed as good a place to stop as any. The pleasures of many
other types of multiplication await your future explorations!
\emojifont {😄} \normalfont

As an independent scholar, I work in isolation without the benefits of a
university environment or consultation with peers. So, an error of fact
or fancy is all the more likely in what I write. If you are
mathematically inclined, and have spotted any mistakes here, please let
me know.

If you wish to read this blog as a single PDF document you may get it
from the \href{http://swanlotus.org/downloads}{downloads page.}

\begin{center}\rule{0.5\linewidth}{0.5pt}\end{center}

\hypertarget{bibliography}{%
\section*{References}\label{bibliography}}
\addcontentsline{toc}{section}{References}

\hypertarget{refs}{}
\begin{CSLReferences}{0}{0}
\leavevmode\vadjust pre{\hypertarget{ref-chodnicki2020}{}}%
\CSLLeftMargin{{[}1{]} }%
\CSLRightInline{Slawomir Chodnicki. 2020. {Why a·0=0 and Other Proofs of
{`the Obvious'}}. {How {`obvious'} becomes {`interesting'} when we go
back to first principles}. Retrieved 12 November 2023 from
\url{https://medium.com/swlh/why-a-0-0-and-other-proofs-of-the-obvious-da52dd0caefb}}

\leavevmode\vadjust pre{\hypertarget{ref-squareroot}{}}%
\CSLLeftMargin{{[}2{]} }%
\CSLRightInline{Various. 2014. {Why the name "square root"?} Retrieved
15 November 2023 from
\url{https://math.stackexchange.com/questions/809799/why-the-name-square-root}}

\leavevmode\vadjust pre{\hypertarget{ref-feynman22}{}}%
\CSLLeftMargin{{[}3{]} }%
\CSLRightInline{Richard P Feynman. 1963. Algebra. Retrieved 16 November
2023 from \url{https://www.feynmanlectures.caltech.edu/I_22.html}}

\leavevmode\vadjust pre{\hypertarget{ref-honner2022}{}}%
\CSLLeftMargin{{[}4{]} }%
\CSLRightInline{Patrick Honner aka MrHonner. 2022. {The All 1s Vector}.
Retrieved 18 November 2023 from
\url{https://mrhonner.com/archives/21303}}

\end{CSLReferences}



\end{document}
