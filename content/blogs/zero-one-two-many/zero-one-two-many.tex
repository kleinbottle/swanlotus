% Options for packages loaded elsewhere
\PassOptionsToPackage{unicode,linktoc=all}{hyperref}
\PassOptionsToPackage{hyphens}{url}
\PassOptionsToPackage{dvipsnames,svgnames,x11names}{xcolor}
%
\documentclass[
  a4paper,
]{article}
\usepackage{amsmath,amssymb}
\usepackage{iftex}
\ifPDFTeX
  \usepackage[T1]{fontenc}
  \usepackage[utf8]{inputenc}
  \usepackage{textcomp} % provide euro and other symbols
\else % if luatex or xetex
  \usepackage{unicode-math} % this also loads fontspec
  \defaultfontfeatures{Scale=MatchLowercase}
  \defaultfontfeatures[\rmfamily]{Ligatures=TeX,Scale=1}
\fi
\usepackage{lmodern}
\ifPDFTeX\else
  % xetex/luatex font selection
\fi
% Use upquote if available, for straight quotes in verbatim environments
\IfFileExists{upquote.sty}{\usepackage{upquote}}{}
\IfFileExists{microtype.sty}{% use microtype if available
  \usepackage[]{microtype}
  \UseMicrotypeSet[protrusion]{basicmath} % disable protrusion for tt fonts
}{}
\makeatletter
\@ifundefined{KOMAClassName}{% if non-KOMA class
  \IfFileExists{parskip.sty}{%
    \usepackage{parskip}
  }{% else
    \setlength{\parindent}{0pt}
    \setlength{\parskip}{6pt plus 2pt minus 1pt}}
}{% if KOMA class
  \KOMAoptions{parskip=half}}
\makeatother
\usepackage{xcolor}
\usepackage[margin=25mm]{geometry}
\usepackage{longtable,booktabs,array}
\usepackage{calc} % for calculating minipage widths
% Correct order of tables after \paragraph or \subparagraph
\usepackage{etoolbox}
\makeatletter
\patchcmd\longtable{\par}{\if@noskipsec\mbox{}\fi\par}{}{}
\makeatother
% Allow footnotes in longtable head/foot
\IfFileExists{footnotehyper.sty}{\usepackage{footnotehyper}}{\usepackage{footnote}}
\makesavenoteenv{longtable}
\setlength{\emergencystretch}{3em} % prevent overfull lines
\providecommand{\tightlist}{%
  \setlength{\itemsep}{0pt}\setlength{\parskip}{0pt}}
\setcounter{secnumdepth}{-\maxdimen} % remove section numbering
\newlength{\cslhangindent}
\setlength{\cslhangindent}{1.5em}
\newlength{\csllabelwidth}
\setlength{\csllabelwidth}{3em}
\newlength{\cslentryspacingunit} % times entry-spacing
\setlength{\cslentryspacingunit}{\parskip}
\newenvironment{CSLReferences}[2] % #1 hanging-ident, #2 entry spacing
 {% don't indent paragraphs
  \setlength{\parindent}{0pt}
  % turn on hanging indent if param 1 is 1
  \ifodd #1
  \let\oldpar\par
  \def\par{\hangindent=\cslhangindent\oldpar}
  \fi
  % set entry spacing
  \setlength{\parskip}{#2\cslentryspacingunit}
 }%
 {}
\usepackage{calc}
\newcommand{\CSLBlock}[1]{#1\hfill\break}
\newcommand{\CSLLeftMargin}[1]{\parbox[t]{\csllabelwidth}{#1}}
\newcommand{\CSLRightInline}[1]{\parbox[t]{\linewidth - \csllabelwidth}{#1}\break}
\newcommand{\CSLIndent}[1]{\hspace{\cslhangindent}#1}
\ifLuaTeX
\usepackage[bidi=basic]{babel}
\else
\usepackage[bidi=default]{babel}
\fi
\babelprovide[main,import]{british}
% get rid of language-specific shorthands (see #6817):
\let\LanguageShortHands\languageshorthands
\def\languageshorthands#1{}
% $HOME/.pandoc/defaults/latex-header-includes.tex
% Common header includes for both lualatex and xelatex engines.
%
% Preliminaries
%
% \PassOptionsToPackage{rgb,dvipsnames,svgnames}{xcolor}
% \PassOptionsToPackage{main=british}{babel}
\PassOptionsToPackage{english}{selnolig}
\AtBeginEnvironment{quote}{\small}
\AtBeginEnvironment{quotation}{\small}
\AtBeginEnvironment{longtable}{\centering}
%
% Packages that are useful to include
%
\usepackage{graphicx}
\usepackage{subcaption}
\usepackage[inkscapeversion=1]{svg}
\usepackage[defaultlines=4,all]{nowidow}
\usepackage{etoolbox}
\usepackage{fontsize}
\usepackage{newunicodechar}
\usepackage{pdflscape}
\usepackage{fnpct}
\usepackage{parskip}
  \setlength{\parindent}{0pt}
\usepackage[style=american]{csquotes}
% \usepackage{setspace} Use the <fontname-plus.tex> files for setspace
%
\usepackage{esdiff} % for derivative symbols
\usepackage{amsmath}
\usepackage{hyperref} % cleveref must come AFTER hyperref
\usepackage[capitalize,noabbrev]{cleveref} % Must come after hyperref
% noto-plus.tex
% Font-setting header file for use with Pandoc Markdown
% to generate PDF via LuaLaTeX.
% The main font is Noto Serif.
% Other main fonts are also available in appropriately named file.
\usepackage{fontspec}
\usepackage{setspace}
\setstretch{1.3}
%
\defaultfontfeatures{Ligatures=TeX,Scale=MatchLowercase,Renderer=Node} % at the start always
%
% For English
% See also https://tex.stackexchange.com/questions/574047/lualatex-amsthm-polyglossia-charissil-error
% We use Node as Renderer for the Latin Font and Greek Font and HarfBuzz as renderer ofr Indic fonts.
%
\babelfont{rm}[Script=Latin,Scale=1]{NotoSerif}% Config is at $HOME/texmf/tex/latex/NotoSerif.fontspec
%
\babelfont{sf}[Script=Latin]{SourceSansPro}% Config is at $HOME/texmf/tex/latex/SourceSansPro.fontspec
%
\babelfont{tt}[Script=Latin]{FiraMono}% Config is at $HOME/texmf/tex/latex/FiraMono.fontspec
%
% Sanskrit, Tamil, and Greek fonts
%
\babelprovide[import, onchar=ids fonts]{sanskrit}
\babelprovide[import, onchar=ids fonts]{tamil}
\babelprovide[import, onchar=ids fonts]{greek}
%
\babelfont[sanskrit]{rm}[Scale=1.1,Renderer=HarfBuzz,Script=Devanagari]{NotoSerifDevanagari}
\babelfont[sanskrit]{sf}[Scale=1.1,Renderer=HarfBuzz,Script=Devanagari]{NotoSansDevanagari}
\babelfont[tamil]{rm}[Renderer=HarfBuzz,Script=Tamil]{NotoSerifTamil}
\babelfont[tamil]{sf}[Renderer=HarfBuzz,Script=Tamil]{NotoSansTamil}
\babelfont[greek]{rm}[Script=Greek]{GentiumBookPlus}
%
% Math font
%
\usepackage{unicode-math} % seems not to hurt % fallabck
\setmathfont[bold-style=TeX]{STIX Two Math}
%
%
% Other fonts
%
\newfontfamily{\emojifont}{Symbola}
%

\usepackage{titling}
\usepackage{fancyhdr}
    \pagestyle{fancy}
    \fancyhead{}
    \fancyfoot{}
    \renewcommand{\headrulewidth}{0.2pt}
    \renewcommand{\footrulewidth}{0.2pt}
    \fancyhead[LO,RE]{\scshape\thetitle}
    \fancyfoot[CO,CE]{\footnotesize Copyright © 2006\textendash\the\year, R (Chandra) Chandrasekhar}
    \fancyfoot[RE,RO]{\thepage}
\newfontfamily{\regulariconfont}{Font Awesome 6 Free Regular}[Color=Grey]
\newfontfamily{\solidiconfont}{Font Awesome 6 Free Solid}[Color=Grey]
\newfontfamily{\brandsiconfont}{Font Awesome 6 Brands}[Color=Grey]
%
% Direct input of Unicode code points
%
\newcommand{\faEnvelope}{\regulariconfont\ ^^^^f0e0\normalfont}
\newcommand{\faMobile}{\solidiconfont\ ^^^^f3cd\normalfont}
\newcommand{\faLinkedin}{\brandsiconfont\ ^^^^f0e1\normalfont}
\newcommand{\faGithub}{\brandsiconfont\ ^^^^f09b\normalfont}
\newcommand{\faAtom}{\solidiconfont\ ^^^^f5d2\normalfont}
\newcommand{\faPaperPlaneRegular}{\regulariconfont\ ^^^^f1d8\normalfont}
\newcommand{\faPaperPlaneSolid}{\solidiconfont\ ^^^^f1d8\normalfont}

%
% The block below is commented out because of Tofu glyphs in HTML
%
% \newcommand{\faEnvelope}{\regulariconfont\ \normalfont}
% \newcommand{\faMobile}{\solidiconfont\ \normalfont}
% \newcommand{\faLinkedin}{\brandsiconfont\ \normalfont}
% \newcommand{\faGithub}{\brandsiconfont\ \normalfont}
\ifLuaTeX
  \usepackage{selnolig}  % disable illegal ligatures
\fi
\IfFileExists{bookmark.sty}{\usepackage{bookmark}}{\usepackage{hyperref}}
\IfFileExists{xurl.sty}{\usepackage{xurl}}{} % add URL line breaks if available
\urlstyle{sf}
\hypersetup{
  pdftitle={Zero, One, Two, Many},
  pdfauthor={R (Chandra) Chandrasekhar},
  pdflang={en-GB},
  colorlinks=true,
  linkcolor={DarkOliveGreen},
  filecolor={Purple},
  citecolor={DarkKhaki},
  urlcolor={Maroon},
  pdfcreator={LaTeX via pandoc}}

\title{Zero, One, Two, Many}
\author{R (Chandra) Chandrasekhar}
\date{2007-12-28 | 2023-10-30}

\begin{document}
\maketitle

\thispagestyle{empty}


\begin{flushright}

\begin{footnotesize}

It's been said that programming has only three nice\\
numbers: zero, one, and however many you please.\\
\textsc{Tom Christiansen and Nathan Torkington}
{[}\protect\hyperlink{ref-perlcookbook2003}{1}{]}

\end{footnotesize}

\end{flushright}

\hfill\break
In the context of computer programming
{[}\protect\hyperlink{ref-perlcookbook2003}{1}{]}, there are only three
numbers worth being concerned about: zero, one, and many. And if you are
into the \href{https://en.wiktionary.org/wiki/arcana}{arcana} of
\href{https://developer.mozilla.org/en-US/docs/Web/JavaScript/Guide/Regular_expressions}{``regular
expressions''} and
\href{https://www.google.com/search?q=pattern+matching}{``pattern
matching''}---which is something done
\href{https://www.dictionary.com/browse/implicitly}{implicitly} every
time you do a Google search, or look for a book at an online
bookstore---that is very
\href{https://www.merriam-webster.com/dictionary/sage}{sage} advice. But
how serviceable is this
\href{https://www.dictionary.com/browse/dictum}{dictum} in everyday
life?

\hypertarget{a-real-life-scenario}{%
\subsection{A real-life scenario}\label{a-real-life-scenario}}

Let us assume that you are living alone, but have decided to invite
friends for a
\href{https://www.britannica.com/topic/Halloween}{Halloween} party. You
had at first assumed that the full
\href{https://www.vocabulary.com/dictionary/cohort}{cohort} of fifty or
so friends to whom you sent invitations were all going to turn up, but
had later revised that estimate to half that number. You shop and
eagerly prepare sufficient food and drink for the festive occasion.
Exhausted but elated with anticipation, you finally finish the chores,
sit back, heave a sigh of relief, and wait for your friends.

As the clock ticks
\href{https://www.thefreedictionary.com/inexorably}{inexorably} toward
the appointed hour, you watch the door but there are no early birds. You
stay patient and console yourself that perhaps they will turn up just on
time. The awaited hour comes and goes and still no one has shown up.

Meanwhile, you get a
\href{https://dictionary.cambridge.org/dictionary/english/flurry}{flurry}
of emails, text messages, and phone calls from your friends, who all
mysteriously claim, that at the last minute some unforeseen circumstance
prevented their presence. They convey their apologies for not turning
up. A full hour after the scheduled time for dinner, you reluctantly
conclude that nobody is going to turn up after all, and decide to
\href{https://www.merriam-webster.com/dictionary/assuage}{assuage} your
sorrow by
\href{https://www.oxfordlearnersdictionaries.com/definition/english/tuck-in}{tucking
in}.

\hypertarget{zero}{%
\subsection{Zero}\label{zero}}

And there you have it: the number zero. It precisely equals the number
of your expected guests who turned up. It is catastrophic, as on this
occasion, or when you have scored zero marks in an examination, or zero
goals in a soccer match. A zero bank balance can drive you up the wall,
or worse. Having zero cavities when you visit the dentist, however, is
cause for celebration. But is zero really a number?

\hypertarget{face-value-and-place-value}{%
\subsubsection{Face value and place
value}\label{face-value-and-place-value}}

The \emph{face value} of a digit is its single-digit value taken in
isolation. In this scheme of things, the digit 5 is greater in value
than the digit 0.

But the decimal system of numbers also brought with it the notion of
\emph{place value} in which the position of a digit in a number
determines its value. For example, in the number 45, the digit 4
represents forty and the digit 5 represents five, giving us an implicit
sum of forty plus five, which we call forty five. In this case, the
digit 4 has a face value less than 5 but a place value greater than 5.

Likewise, in the number 105, we have one hundred, zero tens and five
units. The digits 1 and 0 both have a place value greater than the digit
5. Ponder on this: would we ever be able to denote the number one
hundred and five without the digit 0? People struggled with this problem
for ages before the subtle idea of zero gave them a way out.

Without zero as a place holder, we would not be able to perform
arithmetic unambiguously and efficiently. We could never write numbers
solely using digits if we had not taken recourse to zero and its
beguiling fullness in the midst of emptiness. So, zero, although it is
nothing, is paradoxically, something to be reckoned with. And something
very powerful at that. It has been called
\href{https://www.amazon.in/Nothing-That-Natural-History-Zero/dp/0195142373}{\emph{The
Nothing That Is}} {[}\protect\hyperlink{ref-zero2000}{2}{]}.

\hypertarget{one}{%
\subsection{One}\label{one}}

Think of yourself: the one and only lonely consumer of food at your
festive banquet. Left to
\href{https://www.thefreedictionary.com/sup}{sup} alone and without the
cheer of \href{https://www.dictionary.com/browse/convivial}{convivial}
company, \emph{you} nevertheless count. You---the subject---are always
there, and therefore, the number one has a subjective significance that
rivals only that of zero. This is why \emph{I} is called the \emph{first
person pronoun}. Without the first person, everything else is vacuous.

Each of us has only \emph{one} body. And even if we had a dozen houses
spread across five continents, we can occupy only \emph{one} at any one
time. The number one imposes some fundamental constraints on our lives.

\hypertarget{the-magic-of-successor-numbers}{%
\subsubsection{The magic of successor
numbers}\label{the-magic-of-successor-numbers}}

But one has another magical, mathematical property. If you add one to
one, you get two, which is the \emph{successor} to one. If you add one
to two, you get three---the successor to two---and so on. So, any number
that you care to name can be generated by adding one to itself that many
times, but one! For example, to get \emph{four}, you need to add one to
itself \emph{three} times.

If that sounds too complicated for you, you can equally say that any
number can be generated by adding one to zero that number of times. For
instance, \emph{four} is one added to zero \emph{four} times. Note,
however, that the bedrock for this magic is the number one. Adding zero
to itself does us no good in creating new numbers. One needs to add one
to zero before the brewing begins.

So, both zero and one are powerful numbers. They hold within themselves
the whole menagerie of numbers. Indeed, our modern civilization, built
as it is on computers and the
\href{https://www.britannica.com/science/binary-number-system}{binary
number system}, has the numbers zero and one at its technological
foundations. But are there any other powerful and important numbers
besides these two?

\hypertarget{two}{%
\subsection{Two}\label{two}}

I nominate the number two. The saying,
\href{https://dictionary.cambridge.org/dictionary/english/it-takes-two-to-tango}{``It
takes two to tango''} takes on a new meaning in the worlds of chemistry,
physics, and biology.

\hypertarget{two-in-chemistry}{%
\subsubsection{Two in Chemistry}\label{two-in-chemistry}}

In chemistry, elements react with each other to try to attain
configurations in which there are \emph{eight} electrons or completed
octets. And eight is two cubed. This quest to make eight, either by
sharing or tearing electrons is what gives elements their reactivity. It
is what drives poisonous chlorine and combustible sodium to form
physiologically benign sodium chloride or common salt. And even the
lightest of the rare gases, helium, has \emph{two} electrons in its
solitary atomic state. Evidence again of the power of two.

\hypertarget{two-in-physics}{%
\subsubsection{Two in Physics}\label{two-in-physics}}

Physics relies on repeatable measurements. And all measurements rely on
comparisons. When you weigh one kilogram of beans, you are really
comparing the weight of the beans with a weight that is known to be one
kilogram. If you are using a commercial scale, which relies on spring
tension, you are really comparing the tension or compression on the
spring produced by the beans with the tension or compression from the
known one kilogram weight. And this is as true with time, distance, and
anything else you care to measure, as it is with weight. \emph{All
measurements are comparisons, and for that we need two}.

\hypertarget{two-in-biology}{%
\subsubsection{Two in Biology}\label{two-in-biology}}

In biology, there is the famous
\href{https://www.genome.gov/genetics-glossary/Double-Helix}{double
helix}---DNA or the molecule of life is known to be like a spiral
staircase or twisted ladder. But why two? Why not one? Or zero? Or
three? Very likely because Mother Nature knew, before us, that all
comparisons need two. And if you want to make a copy of something, you
need to compare the copy with the original. Since life is really a
process of copying, with elegant and random variations thrown in
(assuming that you are not totally allergic to the
\href{https://www.livescience.com/474-controversy-evolution-works.html}{Theory
of Evolution}) then, two rules again.

Another possible reason for life's affinity with two is the insurance
provided by redundancy. We have two kidneys, two ears, two eyes, etc.,
primarily because of aesthetics, balance, and design elegance, but
secondarily as a backup in case one fails. Someone who has lost hearing
in one ear or sight in one eye can still hear or see. They might have
lost the ability to localize sound or perceive depth, but the sensory
channels are still functional.

The number two---and its powers like 4, 8, 16 etc.---are also at the
heart of
\href{https://www.nature.com/scitable/definition/mitosis-cell-division-47/}{mitosis}.
Apart from conception where two cells---sperm and ovum---become one
embryo or zygote, in all subsequent cellular proliferation, they divide
in two.

\hypertarget{to-sum-up}{%
\subsection{To sum up}\label{to-sum-up}}

Zero and one are mathematically profound. The number two comes into its
own with matter, measurement, and life. But what about the other
numbers? What about three, or thirty-seven, or three hundred and thirty
million? They might each be interesting in their own right and have some
endearing virtue or peculiarity, but these other numbers are not
\href{https://www.thefreedictionary.com/lynchpin}{lynchpins}. The
\href{https://www.dictionary.com/browse/appellation}{appellation}
``many'' will do for them.

Moreover, not giving these other numbers special distinction is
something we have done many times already. In the grammars for most
languages, only two numbers are recognized: singular and plural, one and
many. The automatic matching of singular nouns with singular verbs, and
plural nouns with plural verbs, is a chore to automate with reports
generated by computing languages. And that is because the number one is
special. Beyond it, we only think of many. But bear in mind that ancient
languages like Greek and Sanskrit had three numbers in grammar:
singular, dual, and plural. And that, I think is giving the numbers one
and two their fair share of distinction. And oh! To speak nothing of
nothing will not do, so, naught or nought, must be there too!

Vive la zero, one, two, many!

\hypertarget{note-to-the-reader}{%
\subsection{Note to the Reader}\label{note-to-the-reader}}

This blog first appeared on 28 December 2007 as a \emph{Daily Dose}---an
informal essay that I circulated each day on a private email list. I
have retrieved and refreshed it to allow its reincarnation as an online
blog now. Its purpose is to inform and educate. Accordingly, \emph{some}
words that are likely to be unfamiliar to the reader are shown in blue
and hyperlinked to entries on the Web that will explain their meaning.
The rest are left for the reader to explore. \emojifont {😉} \normalfont

\hypertarget{feedback}{%
\subsection{Feedback}\label{feedback}}

Please \href{mailto:feedback.swanlotus@gmail.com}{email me} your
comments and corrections.

\noindent A PDF version of this article is
\href{./zero-one-two-many.pdf}{available for download here}:

\begin{small}

\begin{sffamily}

\url{https://swanlotus.netlify.app/blogs/zero-one-two-many.pdf}

\end{sffamily}

\end{small}

\hypertarget{bibliography}{%
\section*{References}\label{bibliography}}
\addcontentsline{toc}{section}{References}

\hypertarget{refs}{}
\begin{CSLReferences}{0}{0}
\leavevmode\vadjust pre{\hypertarget{ref-perlcookbook2003}{}}%
\CSLLeftMargin{{[}1{]} }%
\CSLRightInline{Tom Christiansen and Nathan Torkington. 2003.
\emph{{Perl Cookbook}} (2nd ed.). O'Reilly, Sebastopol, CA, USA.}

\leavevmode\vadjust pre{\hypertarget{ref-zero2000}{}}%
\CSLLeftMargin{{[}2{]} }%
\CSLRightInline{Robert Kaplan. 2000. \emph{{The Nothing That Is: A
Natural History of Zero}}. Oxford University Press.}

\end{CSLReferences}



\end{document}
