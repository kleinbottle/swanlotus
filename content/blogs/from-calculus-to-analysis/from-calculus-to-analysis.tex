% Options for packages loaded elsewhere
\PassOptionsToPackage{unicode,linktoc=all}{hyperref}
\PassOptionsToPackage{hyphens}{url}
\PassOptionsToPackage{dvipsnames,svgnames,x11names}{xcolor}
%
\documentclass[
  a4paper,
]{article}
\usepackage{amsmath,amssymb}
\usepackage{iftex}
\ifPDFTeX
  \usepackage[T1]{fontenc}
  \usepackage[utf8]{inputenc}
  \usepackage{textcomp} % provide euro and other symbols
\else % if luatex or xetex
  \usepackage{unicode-math} % this also loads fontspec
  \defaultfontfeatures{Scale=MatchLowercase}
  \defaultfontfeatures[\rmfamily]{Ligatures=TeX,Scale=1}
\fi
\usepackage{lmodern}
\ifPDFTeX\else
  % xetex/luatex font selection
\fi
% Use upquote if available, for straight quotes in verbatim environments
\IfFileExists{upquote.sty}{\usepackage{upquote}}{}
\IfFileExists{microtype.sty}{% use microtype if available
  \usepackage[]{microtype}
  \UseMicrotypeSet[protrusion]{basicmath} % disable protrusion for tt fonts
}{}
\makeatletter
\@ifundefined{KOMAClassName}{% if non-KOMA class
  \IfFileExists{parskip.sty}{%
    \usepackage{parskip}
  }{% else
    \setlength{\parindent}{0pt}
    \setlength{\parskip}{6pt plus 2pt minus 1pt}}
}{% if KOMA class
  \KOMAoptions{parskip=half}}
\makeatother
\usepackage{xcolor}
\usepackage[margin=25mm]{geometry}
\usepackage{longtable,booktabs,array}
\usepackage{calc} % for calculating minipage widths
% Correct order of tables after \paragraph or \subparagraph
\usepackage{etoolbox}
\makeatletter
\patchcmd\longtable{\par}{\if@noskipsec\mbox{}\fi\par}{}{}
\makeatother
% Allow footnotes in longtable head/foot
\IfFileExists{footnotehyper.sty}{\usepackage{footnotehyper}}{\usepackage{footnote}}
\makesavenoteenv{longtable}
\setlength{\emergencystretch}{3em} % prevent overfull lines
\providecommand{\tightlist}{%
  \setlength{\itemsep}{0pt}\setlength{\parskip}{0pt}}
\setcounter{secnumdepth}{-\maxdimen} % remove section numbering
% definitions for citeproc citations
\NewDocumentCommand\citeproctext{}{}
\NewDocumentCommand\citeproc{mm}{%
  \begingroup\def\citeproctext{#2}\cite{#1}\endgroup}
\makeatletter
 % allow citations to break across lines
 \let\@cite@ofmt\@firstofone
 % avoid brackets around text for \cite:
 \def\@biblabel#1{}
 \def\@cite#1#2{{#1\if@tempswa , #2\fi}}
\makeatother
\newlength{\cslhangindent}
\setlength{\cslhangindent}{1.5em}
\newlength{\csllabelwidth}
\setlength{\csllabelwidth}{3em}
\newenvironment{CSLReferences}[2] % #1 hanging-indent, #2 entry-spacing
 {\begin{list}{}{%
  \setlength{\itemindent}{0pt}
  \setlength{\leftmargin}{0pt}
  \setlength{\parsep}{0pt}
  % turn on hanging indent if param 1 is 1
  \ifodd #1
   \setlength{\leftmargin}{\cslhangindent}
   \setlength{\itemindent}{-1\cslhangindent}
  \fi
  % set entry spacing
  \setlength{\itemsep}{#2\baselineskip}}}
 {\end{list}}
\usepackage{calc}
\newcommand{\CSLBlock}[1]{\hfill\break\parbox[t]{\linewidth}{\strut\ignorespaces#1\strut}}
\newcommand{\CSLLeftMargin}[1]{\parbox[t]{\csllabelwidth}{\strut#1\strut}}
\newcommand{\CSLRightInline}[1]{\parbox[t]{\linewidth - \csllabelwidth}{\strut#1\strut}}
\newcommand{\CSLIndent}[1]{\hspace{\cslhangindent}#1}
\ifLuaTeX
\usepackage[bidi=basic]{babel}
\else
\usepackage[bidi=default]{babel}
\fi
\babelprovide[main,import]{british}
% get rid of language-specific shorthands (see #6817):
\let\LanguageShortHands\languageshorthands
\def\languageshorthands#1{}
% $HOME/.pandoc/defaults/latex-header-includes.tex
% Common header includes for both lualatex and xelatex engines.
%
% Preliminaries
%
% \PassOptionsToPackage{rgb,dvipsnames,svgnames}{xcolor}
% \PassOptionsToPackage{main=british}{babel}
\PassOptionsToPackage{english}{selnolig}
\AtBeginEnvironment{quote}{\small}
\AtBeginEnvironment{quotation}{\small}
\AtBeginEnvironment{longtable}{\centering}
%
% Packages that are useful to include
%
\usepackage{graphicx}
\usepackage{subcaption}
\usepackage[inkscapeversion=1]{svg}
\usepackage[defaultlines=4,all]{nowidow}
\usepackage{etoolbox}
\usepackage{fontsize}
\usepackage{newunicodechar}
\usepackage{pdflscape}
\usepackage{fnpct}
\usepackage{parskip}
  \setlength{\parindent}{0pt}
\usepackage[style=american]{csquotes}
% \usepackage{setspace} Use the <fontname-plus.tex> files for setspace
%
\usepackage{hyperref} % cleveref must come AFTER hyperref
\usepackage[capitalize,noabbrev]{cleveref} % Must come after hyperref
\let\longdivision\relax
\usepackage{longdivision}
% noto-plus.tex
% Font-setting header file for use with Pandoc Markdown
% to generate PDF via LuaLaTeX.
% The main font is Noto Serif.
% Other main fonts are also available in appropriately named file.
\usepackage{fontspec}
\usepackage{setspace}
\setstretch{1.3}
%
\defaultfontfeatures{Ligatures=TeX,Scale=MatchLowercase,Renderer=Node} % at the start always
%
% For English
% See also https://tex.stackexchange.com/questions/574047/lualatex-amsthm-polyglossia-charissil-error
% We use Node as Renderer for the Latin Font and Greek Font and HarfBuzz as renderer ofr Indic fonts.
%
\babelfont{rm}[Script=Latin,Scale=1]{NotoSerif}% Config is at $HOME/texmf/tex/latex/NotoSerif.fontspec
\babelfont{sf}[Script=Latin]{SourceSansPro}% Config is at $HOME/texmf/tex/latex/SourceSansPro.fontspec
\babelfont{tt}[Script=Latin]{FiraMono}% Config is at $HOME/texmf/tex/latex/FiraMono.fontspec
%
% Sanskrit, Tamil, and Greek fonts
%
\babelprovide[import, onchar=ids fonts]{sanskrit}
\babelprovide[import, onchar=ids fonts]{tamil}
\babelprovide[import, onchar=ids fonts]{greek}
%
\babelfont[sanskrit]{rm}[Scale=1.1,Renderer=HarfBuzz,Script=Devanagari]{NotoSerifDevanagari}
\babelfont[sanskrit]{sf}[Scale=1.1,Renderer=HarfBuzz,Script=Devanagari]{NotoSansDevanagari}
\babelfont[tamil]{rm}[Renderer=HarfBuzz,Script=Tamil]{NotoSerifTamil}
\babelfont[tamil]{sf}[Renderer=HarfBuzz,Script=Tamil]{NotoSansTamil}
\babelfont[greek]{rm}[Script=Greek]{GentiumBookPlus}
%
% Math font
%
\usepackage{unicode-math} % seems not to hurt % fallabck
\setmathfont[bold-style=TeX]{STIX Two Math}
\usepackage{amsmath}
\usepackage{esdiff} % for derivative symbols
% \renewcommand{\mathbf}{\symbf}
%
%
% Other fonts
%
\newfontfamily{\emojifont}{Symbola}
%

\usepackage{titling}
\usepackage{fancyhdr}
    \pagestyle{fancy}
    \fancyhead{}
    \fancyfoot{}
    \renewcommand{\headrulewidth}{0.2pt}
    \renewcommand{\footrulewidth}{0.2pt}
    \fancyhead[LO,RE]{\scshape\thetitle}
    \fancyfoot[CO,CE]{\footnotesize Copyright © 2006\textendash\the\year, R (Chandra) Chandrasekhar}
    \fancyfoot[RE,RO]{\thepage}
%
\usepackage{newunicodechar}
\newunicodechar{√}{\textsf{√}}
\ifLuaTeX
  \usepackage{selnolig}  % disable illegal ligatures
\fi
\usepackage{bookmark}
\IfFileExists{xurl.sty}{\usepackage{xurl}}{} % add URL line breaks if available
\urlstyle{sf}
\hypersetup{
  pdftitle={From Calculus to Analysis},
  pdfauthor={R (Chandra) Chandrasekhar},
  pdflang={en-GB},
  colorlinks=true,
  linkcolor={DarkOliveGreen},
  filecolor={Purple},
  citecolor={DarkKhaki},
  urlcolor={Maroon},
  pdfcreator={LaTeX via pandoc}}

\title{From Calculus to Analysis}
\author{R (Chandra) Chandrasekhar}
\date{2024-03-24 | 2024-05-15}

\begin{document}
\maketitle

\thispagestyle{empty}


\begin{quote}
Portions of this blog have been taken from the chapter entitled
``Mathematics at University'' from my book,
\href{https://swanlotus.netlify.app/sas}{\emph{Secrets of Academic
Success}}, which is
\href{https://swanlotus.netlify.app/sas-manuscript/SAS-partial.pdf}{available
as a free PDF download}. Since that chapter was written, I have gained
greater understanding on why calculus at high school needed to morph
into analysis at university. Accordingly, the material has been
augmented, but kept simple enough to be accessible to a high school
student just entering university.

I am not a professional mathematician, and make no claim to rigour in
this blog. Rather, I hope to demystify analysis from the grip of symbols
by explaining its
\href{https://www.thefreedictionary.com/raison+d+etre}{\emph{raison
d'être}} in plain English. In the process, I hope analysis appears less
forbidding and more friendly.
\end{quote}

\subsection{The transition from high school to
university}\label{the-transition-from-high-school-to-university}

At high school you were taught how to integrate and differentiate. You
were exposed to all sorts of tricks and special techniques to compute
specific integrals, especially those involving inverse trigonometric
functions, and so on. If you revelled in mastering and applying such
techniques, you might find that what succeeds high school
\emph{calculus}, is a horse of an entirely different colour, called
\emph{analysis}, at university.

You might even be alarmed that rather than have to solve routine
problems for the value of some quantity, or simply to work through and
demonstrate a fact in a straightforward fashion, you are now required to
prove theorems: something that requires a different mindset and skill
set.

\subsection{Why the change?}\label{why-the-change}

High school calculus appeals to intuition and the visual sense, through
\emph{geometric} ideas like slopes and areas. Words like ``approaches'',
``tends to'', etc., signify motion and resonate with our sense of space
and time.

Analysis, on the other hand, is logically precise and uses
\emph{arithmetic} as the basis for deriving results. Intuition has given
way to logical precision, and pictures have yielded to symbols. The
implicit scaffolding of familiar ideas like space and time---borrowed
from our everyday experience---has been replaced with the clinical
precision of inequalities and universal and existential quantifiers.

The logically indefensible infinitesimals of calculus have given way to
the rigorously justifiable limits and infinite sums of analysis. Numbers
alone provide the foundation, and this required the idea of a number
itself to be strengthened as an abstraction beyond question or doubt.

All this takes some getting used to. We have moved from an innocent
nature-hewn cave to a fabricated apartment block that reaches for the
sky. Questions such as, ``Is this change really necessary?'' and ``If it
ain't broke, why fix it?'' arise in consequence.

But the sober truth is that high school calculus \emph{is} ``broke''. It
is fun-filled, but nevertheless, a convenient fiction by mathematical
standards. Two centuries passed between the discovery of the calculus as
a magical computing machine and the recognition of the need to fix it so
that it would work robustly in all circumstances.

In this sense, the progression from calculus to analysis is similar to
the accretion of zero and the negative numbers to the natural numbers,
or the introduction of imaginary numbers to account for roots of certain
polynomials. All such changes were resisted at first, just like a new
pair of shoes that initially pinch, but time and usage have borne
testament to the wisdom behind the change. The somewhat painful
transition from calculus to analysis will prove to be the same.

\subsection{Geometry's fall from
favour}\label{geometrys-fall-from-favour}

Mathematics texts at high school level are generously and often
colorfully illustrated, especially when dealing with geometry. But, if
you look at any university-level analysis textbook, colour would have
taken heel, and pictures will be the exception rather than the norm. For
those who think in pictures, this will come as a letdown, accompanied by
the puzzling question, ``Why?''

Geometric intuition, upon which Greek mathematics rested securely for
well nigh two centuries, was not infallible. The development of new
\href{https://en.wikipedia.org/wiki/Non-Euclidean_geometry}{non-Euclidean
geometries} in the nineteenth century robbed mathematics of its innocent
certitude in geometric foundations. And the resulting revolution, called
the \href{https://en.wikipedia.org/wiki/Erlangen_program}{Erlangen
program}, made algebra the basis of classifying geometry itself.

Our everyday experience is rooted in our sense of space and time. So, I
will attempt to illustrate the ideas of analysis using pictures wherever
I can, so that the alien syntax and symbology of analysis is buttressed
by the reassuring presence of equivalent pictures. Unfortunately, many
introductory analysis books do not generously make this student-friendly
concession, if at all.

\subsection{Holes in the real number line: irrational
numbers}\label{holes-in-the-real-number-line-irrational-numbers}

What do irrational numbers have to do with analysis? Quite a lot,
really.

Real analysis lives on the real number line. From the time of the
Pythagoreans, the irrational numbers have caused mathematicians' hearts
to skip a beat. ``Are irrational numbers really numbers? If so, where do
they live?''

The Greeks circumvented this by limiting the irrational numbers to
geometric contexts. The ratio of the circumference of a unit circle to
its diameter is \(\pi\). The diagonal of a unit square is \(\sqrt{2}\).
``Let them live geometrically as \emph{lengths}, but let us remain
silent about their existence elsewhere.'' It was a dichotomy between
\emph{counting} and \emph{measurement} or between the \emph{discrete}
and the \emph{continuous}.

Even today, the sets for the \emph{natural numbers}, the
\emph{integers}, the \emph{rational numbers} all have their own symbols,
\(\mathbb{N}\), \(\mathbb{Z}\), \(\mathbb{Q}\), respectively, but not
the irrational numbers. That should be clue enough to indicate that the
irrationals were more than a handful for mathematicians to contend with,
especially if they were unsure where they rightfully belonged.

The rational numbers, when expressed as decimals, either have a finite
decimal representation, like \(\frac{1}{2} = 0.5\)\footnote{To muddy
  matters even more, \(\frac{1}{2} = 0.4999\dots\) is an equally a valid
  representation!}, or an infinitely recurring decimal representation
like \(\frac{1}{3} = 0.333\dots\).

The irrational number \(\sqrt{2}\)---when expressed as a
decimal---neither terminates nor recurs without end. It simply goes on
and on and on. So, where exactly does it sit on the real line? This
troubling question went unanswered until
\href{https://plato.stanford.edu/entries/dedekind-foundations/}{Richard
Dedekind} skilfully introduced the
\href{https://www.britannica.com/science/Dedekind-cut}{Dedekind cut} to
legitimately accommodate the irrationals as first-class citizens of the
real number line, fully capable of undergoing all the arithmetic
operations of the rational numbers.

The inclusion of the irrational numbers into the fold of real numbers
along with the rational numbers \(\mathbb{Q}\),
\href{https://en.wikipedia.org/wiki/Completeness_of_the_real_numbers}{completes}
the set of real numbers, \(\mathbb{R}\), and helps lay the foundation
for rigour in analysis.

\subsection{Ordered Archimedean field}\label{ordered-archimedean-field}

Two properties of real numbers are highlighted as being essential for
real analysis:

\begin{enumerate}
\def\labelenumi{\alph{enumi}.}
\item
  The \href{https://en.wikipedia.org/wiki/Law_of_trichotomy}{trichotomy}
  of the real numbers, which states that every real number is either
  positive, zero, or negative. Given any two real numbers, \(x\) and
  \(y\), \emph{only one} of the following three statements is true:
  \(x < y\), or \(x = y\), or \(x > y\).
\item
  The \href{https://planetmath.org/archimedeanproperty}{Archimedean
  property} which forbids infinities and infinitesimals from being real
  numbers. It states that:

  \begin{enumerate}
  \def\labelenumii{(\roman{enumii})}
  \tightlist
  \item
    For \emph{any} real number \(x\), there is a natural number \(n\)
    such that \(n > x\).\footnote{\(\forall x \in \mathbb{R}, \exists n \in \mathbb{N} \text{ such that } n > x.\)}
  \item
    For \emph{any positive} real number \(y\), there is a natural number
    \(n\) such that \(\frac{1}{n} < y\).\footnote{\(\forall y \in \mathbb{R}, y > 0, \exists n \in \mathbb{N} \text{ such that } \frac{1}{n} < y\).}
  \end{enumerate}
\end{enumerate}

\subsection{Inequalities and
distances}\label{inequalities-and-distances}

Calculus problems are generally concerned with \emph{evaluating} some
quantity and are therefore centred around \emph{equalities}.

Analysis, on the other hand, relies heavily on logical statements about
quantities whose exact values may not be known, but about whom
statements of \emph{relative size} need to be made. This is where
\emph{inequalities} enter the discourse.\footnote{You would have already
  seen from above how inequalities are used to define an Archimedean
  field.}

This change in emphasis can be disconcerting. Students seeking to
familiarize themselves with manipulations of inequalities should consult
entry-level texts dealing with this subject
{[}\citeproc{ref-kazarinoff-1961}{1}--\citeproc{ref-alsina-nelsen-2009}{3}{]}.

When the derivative was defined geometrically as the value of the slope
of a tangent to the graph of the function at a particular point, the
geometrical relationship alone sufficed to encapsulate the definition
and to help compute the value. XXX FIG

However, when the derivative is defined as a
\href{https://en.wikipedia.org/wiki/Limit_(mathematics)}{\emph{limit}},
using arithmetic instead of geometry, we do not have \emph{intuitive}
markers to guide us. And such markers are serviceable, but not
infallible. Therefore, we need some symbols and operations to contend
with ideas like the ``closeness'' of two points, etc.

This is where we encounter two mathematical devices quintessential to
analysis and much of modern mathematics:

\begin{enumerate}
\def\labelenumi{\alph{enumi}.}
\item
  Inequalities: by understanding and manipulating the symbols \(>\) and
  \(<\), we can work toward watertight definitions that serve to
  constrain what we are talking about. See \href{}{Spider in a matchbox}
  later on.
\item
  Distance: by using a sensible definition of the distance between two
  points, we may measure their ``closeness''. Geometry and intuition
  have been replaced by arithmetic and precision.

  If there are two \emph{distinct} real numbers \(x\) and \(y\),
  \(|x - y|\) defines the positive number that quantifies their
  separation or distance. Since we are on the real line, the
  \href{https://mathworld.wolfram.com/AbsoluteValue.html}{\emph{absolute
  value function}} is used for this purpose. XXX FIG
\end{enumerate}

\subsection{Limits: conceptually and
rigorously}\label{limits-conceptually-and-rigorously}

One interesting feature of limits for functions is that \emph{a limit at
a point may exist even if the function is not defined at that particular
point}. The definition of a limit therefore requires closeness to the
point but \emph{not equality} to the point. This ``arbitrarily close''
condition is couched thus:

\begin{quote}
``For arbitrary \(\varepsilon > 0\) and \(| x - y | < \varepsilon\)''.
\end{quote}

This incantation will become so familiar that it will become
unremarkable when you devote time and practice to analysis. And notice
the usefulness of inequalities here.

\subsection{\texorpdfstring{Approximations to \(\sqrt{2}\) as a Cauchy
sequence}{Approximations to \textbackslash sqrt\{2\} as a Cauchy sequence}}\label{approximations-to-sqrt2-as-a-cauchy-sequence}

\href{https://en.wikipedia.org/wiki/Augustin-Louis_Cauchy}{Augustin
Cauchy}

\subsection{Irrational numbers as
reals}\label{irrational-numbers-as-reals}

Irrationals were never properly integrated into the real numbers until
That is not a story we will visit today, except to say that
strengthening the foundations of the real numbers was a necessary
prerequisite that laid the foundations of analysis. Numbers such as
\(\pi\) and \(\sqrt{2}\) arising from the simplest plane geometry can
now be accommodated legitimately as first-class citizens of the real
number line.

\subsection{The reals as an ordered Archimedean
field}\label{the-reals-as-an-ordered-archimedean-field}

\subsection{Nested sequences in the real line and Matrioshka
dolls}\label{nested-sequences-in-the-real-line-and-matrioshka-dolls}

\subsection{Spider in a matchbox}\label{spider-in-a-matchbox}

\subsection{Why epsilon before delta}\label{why-epsilon-before-delta}

\subsection{Examples of non-limits:
1/x}\label{examples-of-non-limits-1x}

\subsection{Limit, continuity, smoothness,
differentiability}\label{limit-continuity-smoothness-differentiability}

\subsection{Nested intervals and Matrioshka dolls: {[}also
wavelets{]}}\label{nested-intervals-and-matrioshka-dolls-also-wavelets}

\subsection{Infinite processes and
infinities}\label{infinite-processes-and-infinities}

\subsection{\texorpdfstring{The idea of \emph{closeness} in arithmetic
terms:
\(| a_n - a_m | < \varepsilon\)}{The idea of closeness in arithmetic terms: \textbar{} a\_n - a\_m \textbar{} \textless{} \textbackslash varepsilon}}\label{the-idea-of-closeness-in-arithmetic-terms-a_n---a_m-varepsilon}

\begin{enumerate}
\tightlist
\item
  Real numbers: holes in the number line
\item
  Infinite Processes and Infinities
\item
  No geometry
\item
  No algebra
\item
  But arithmetic
\item
  Squeezing something from below and above: Archimedes
\item
  Archimedean Field
\item
  Epsilon-Delta matchbox to confine the Limit spider
\item
  Continuity
\item
  Differentiability
\item
  But most books: no pictures
\item
  Pathological functions
\item
  Fourier Series: how can a curvy function produce a square wave?
\item
  My selection of books
\item
  Extract from chapter
\item
  Order in the field of real numbers; Is the Complex field ordered?
\item
  Modulus function to define length
\item
  Approximation rather than Computation
\item
  Why Differentiation need s to be re-defined
\item
  Why Integration needs to be redefined
\item
  What happens at Infinity? Is the equals sign really valid?
\end{enumerate}

\subsection{Acknowledgements}\label{acknowledgements}

\subsection{Feedback}\label{feedback}

Please \href{mailto:feedback.swanlotus@gmail.com}{email me} your
comments and corrections.

\noindent A PDF version of this article is
\href{./from-calculus-to-analysis.pdf}{available for download here}:

\begin{small}

\begin{sffamily}

\url{https://swanlotus.netlify.app/blogs/from-calculus-to-analysi.pdf}

\end{sffamily}

\end{small}

https://mercedesbernard.com/blog/recursion-and-nesting-dolls/

https://www.studysmarter.co.uk/explanations/math/pure-maths/cauchy-sequence/

https://math.libretexts.org/Bookshelves/Analysis/Mathematical\_Analysis\_(Zakon)/03:\_Vector\_Spaces\_and\_Metric\_Spaces/3.13:\_Cauchy\_Sequences.\_Completeness

https://math.stackexchange.com/questions/471615/archimedean-property-concept

\section*{References}\label{bibliography}
\addcontentsline{toc}{section}{References}

\phantomsection\label{refs}
\begin{CSLReferences}{0}{0}
\bibitem[\citeproctext]{ref-kazarinoff-1961}
\CSLLeftMargin{{[}1{]} }%
\CSLRightInline{Nicholas D Kazarinoff. 1961. \emph{Geometric
inequalities}. The Mathematical Association of America.}

\bibitem[\citeproctext]{ref-beckenbach-bellman-1962}
\CSLLeftMargin{{[}2{]} }%
\CSLRightInline{Edwin Beckenbach and Richard Bellman. 1962. \emph{An
introduction to inequalities}. The Mathematical Association of America.}

\bibitem[\citeproctext]{ref-alsina-nelsen-2009}
\CSLLeftMargin{{[}3{]} }%
\CSLRightInline{Claudi Alsina and Roger B Nelsen. 2009. \emph{When less
is more: Visualizing basic inequalities}. The Mathematical Association
of America.}

\end{CSLReferences}



\end{document}
