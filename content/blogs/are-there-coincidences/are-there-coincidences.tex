% Options for packages loaded elsewhere
\PassOptionsToPackage{unicode,linktoc=all}{hyperref}
\PassOptionsToPackage{hyphens}{url}
\PassOptionsToPackage{dvipsnames,svgnames,x11names}{xcolor}
%
\documentclass[
  a4paper,
]{article}
\usepackage{amsmath,amssymb}
\usepackage{iftex}
\ifPDFTeX
  \usepackage[T1]{fontenc}
  \usepackage[utf8]{inputenc}
  \usepackage{textcomp} % provide euro and other symbols
\else % if luatex or xetex
  \usepackage{unicode-math} % this also loads fontspec
  \defaultfontfeatures{Scale=MatchLowercase}
  \defaultfontfeatures[\rmfamily]{Ligatures=TeX,Scale=1}
\fi
\usepackage{lmodern}
\ifPDFTeX\else
  % xetex/luatex font selection
\fi
% Use upquote if available, for straight quotes in verbatim environments
\IfFileExists{upquote.sty}{\usepackage{upquote}}{}
\IfFileExists{microtype.sty}{% use microtype if available
  \usepackage[]{microtype}
  \UseMicrotypeSet[protrusion]{basicmath} % disable protrusion for tt fonts
}{}
\makeatletter
\@ifundefined{KOMAClassName}{% if non-KOMA class
  \IfFileExists{parskip.sty}{%
    \usepackage{parskip}
  }{% else
    \setlength{\parindent}{0pt}
    \setlength{\parskip}{6pt plus 2pt minus 1pt}}
}{% if KOMA class
  \KOMAoptions{parskip=half}}
\makeatother
\usepackage{xcolor}
\usepackage[margin=25mm]{geometry}
\usepackage{longtable,booktabs,array}
\usepackage{calc} % for calculating minipage widths
% Correct order of tables after \paragraph or \subparagraph
\usepackage{etoolbox}
\makeatletter
\patchcmd\longtable{\par}{\if@noskipsec\mbox{}\fi\par}{}{}
\makeatother
% Allow footnotes in longtable head/foot
\IfFileExists{footnotehyper.sty}{\usepackage{footnotehyper}}{\usepackage{footnote}}
\makesavenoteenv{longtable}
\setlength{\emergencystretch}{3em} % prevent overfull lines
\providecommand{\tightlist}{%
  \setlength{\itemsep}{0pt}\setlength{\parskip}{0pt}}
\setcounter{secnumdepth}{-\maxdimen} % remove section numbering
\ifLuaTeX
\usepackage[bidi=basic]{babel}
\else
\usepackage[bidi=default]{babel}
\fi
\babelprovide[main,import]{british}
% get rid of language-specific shorthands (see #6817):
\let\LanguageShortHands\languageshorthands
\def\languageshorthands#1{}
% $HOME/.pandoc/defaults/latex-header-includes.tex
% Common header includes for both lualatex and xelatex engines.
%
% Preliminaries
%
% \PassOptionsToPackage{rgb,dvipsnames,svgnames}{xcolor}
% \PassOptionsToPackage{main=british}{babel}
\PassOptionsToPackage{english}{selnolig}
\AtBeginEnvironment{quote}{\small}
\AtBeginEnvironment{quotation}{\small}
\AtBeginEnvironment{longtable}{\centering}
%
% Packages that are useful to include
%
\usepackage{graphicx}
\usepackage{subcaption}
\usepackage[inkscapeversion=auto]{svg}
\usepackage{nowidow}
\usepackage{etoolbox}
\usepackage{fontsize}
\usepackage{newunicodechar}
\usepackage{pdflscape}
\usepackage{fnpct}
\usepackage{parskip}
  \setlength{\parindent}{0pt}
\usepackage[style=american]{csquotes}
% \usepackage{setspace} Use the <fontname-plus.tex> files for setspace
%
\usepackage{hyperref} % cleveref must come AFTER hyperref
\usepackage[capitalize,noabbrev]{cleveref} % Must come after hyperref
\let\longdivision\relax
\usepackage{longdivision}
\newcommand{\dd}{\ensuremath{mathrm d}}
%
% Assume that amsmath is already loaded via \usepackage{amsmath}
% in the standard LaTeX template foe Pandoc
% 
\DeclareMathOperator{\sech}{sech}
\DeclareMathOperator{\csch}{csch}
\DeclareMathOperator{\arcsec}{arcsec}
\DeclareMathOperator{\arccot}{arccot}
\DeclareMathOperator{\arccsc}{arccsc}
\DeclareMathOperator{\arccosh}{arccosh}
\DeclareMathOperator{\arcsinh}{arcsinh}
\DeclareMathOperator{\arctanh}{arctanh}
\DeclareMathOperator{\arcsech}{arcsech}
\DeclareMathOperator{\arccsch}{arccsch}
\DeclareMathOperator{\arccoth}{arccoth} 
% noto-plus.tex
% Font-setting header file for use with Pandoc Markdown
% to generate PDF via LuaLaTeX.
% The main font is Noto Serif.
% Other main fonts are also available in appropriately named file.
\usepackage{fontspec}
\usepackage{setspace}
\setstretch{1.3}
%
\defaultfontfeatures{Ligatures=TeX,Scale=MatchLowercase,Renderer=Node} % at the start always
%
% For English
% See also https://tex.stackexchange.com/questions/574047/lualatex-amsthm-polyglossia-charissil-error
% We use Node as Renderer for the Latin Font and Greek Font and HarfBuzz as renderer ofr Indic fonts.
%
\babelfont{rm}[Script=Latin,Scale=1]{NotoSerif}% Config is at $HOME/texmf/tex/latex/NotoSerif.fontspec
\babelfont{sf}[Script=Latin]{SourceSansPro}% Config is at $HOME/texmf/tex/latex/SourceSansPro.fontspec
\babelfont{tt}[Script=Latin]{FiraMono}% Config is at $HOME/texmf/tex/latex/FiraMono.fontspec
%
% Sanskrit, Tamil, and Greek fonts
%
\babelprovide[import, onchar=ids fonts]{sanskrit}
\babelprovide[import, onchar=ids fonts]{tamil}
\babelprovide[import, onchar=ids fonts]{greek}
%
\babelfont[sanskrit]{rm}[Scale=1.1,Renderer=HarfBuzz,Script=Devanagari]{NotoSerifDevanagari}
\babelfont[sanskrit]{sf}[Scale=1.1,Renderer=HarfBuzz,Script=Devanagari]{NotoSansDevanagari}
\babelfont[tamil]{rm}[Renderer=HarfBuzz,Script=Tamil]{NotoSerifTamil}
\babelfont[tamil]{sf}[Renderer=HarfBuzz,Script=Tamil]{NotoSansTamil}
\babelfont[greek]{rm}[Script=Greek]{GentiumBookPlus}
%
% Math font
%
\usepackage{unicode-math} % seems not to hurt % fallabck
\setmathfont[bold-style=TeX]{STIX Two Math}
\usepackage{amsmath}
\usepackage{esdiff} % for derivative symbols
% \renewcommand{\mathbf}{\symbf}
%
%
% Other fonts
%
\newfontfamily{\emojifont}{Symbola}
%

\usepackage{titling}
\usepackage{fancyhdr}
    \pagestyle{fancy}
    \fancyhead{}
    \fancyfoot{}
    \renewcommand{\headrulewidth}{0.2pt}
    \renewcommand{\footrulewidth}{0.2pt}
    \fancyhead[LO,RE]{\scshape\thetitle}
    \fancyfoot[CO,CE]{\footnotesize Copyright © 2006\textendash\the\year, R (Chandra) Chandrasekhar}
    \fancyfoot[RE,RO]{\thepage}
%
\usepackage{newunicodechar}
\newunicodechar{√}{\textsf{√}}
\usepackage {caption}
    \captionsetup{font={sf,stretch=1.4}}
\ifLuaTeX
  \usepackage{selnolig}  % disable illegal ligatures
\fi
\IfFileExists{bookmark.sty}{\usepackage{bookmark}}{\usepackage{hyperref}}
\IfFileExists{xurl.sty}{\usepackage{xurl}}{} % add URL line breaks if available
\urlstyle{sf}
\hypersetup{
  pdftitle={Are There Coincidences?},
  pdfauthor={R (Chandra) Chandrasekhar},
  pdflang={en-GB},
  colorlinks=true,
  linkcolor={DarkGreen},
  filecolor={Purple},
  citecolor={Teal},
  urlcolor={Maroon},
  pdfcreator={LaTeX via pandoc}}

\title{Are There Coincidences?}
\author{R (Chandra) Chandrasekhar}
\date{2006-08-22 | 2025-01-23}

\begin{document}
\maketitle

\thispagestyle{empty}


\subsection{Two or more perfectly timed
events}\label{two-or-more-perfectly-timed-events}

\href{https://www.merriam-webster.com/dictionary/serendipity}{Serendipity},
\href{https://dictionary.cambridge.org/dictionary/english/fortuitousness}{fortuitousness},
\href{https://www.collinsdictionary.com/dictionary/english/happenstance}{happenstance},
good fortune, happy coincidence,
\href{https://www.vocabulary.com/dictionary/synchronicity}{synchronicity}.
These diverse expressions describe two or more desirable events that
happened together, but were not arranged beforehand. They seemed to
happen on their own, by accident. Yet they were so appropriate and
well-timed as to appear to be pre-arranged. We term this happy
confluence a coincidence: co: meaning together; incidence: happening,
i.e., a conjunction of two or more apparently unrelated events that was
not planned, but perfect when they happened together.

\subsection{Is there a science behind
coincidences}\label{is-there-a-science-behind-coincidences}

Are there coincidences? We have all had experiences of thinking about
someone we have not seen for a long time, when suddenly the doorbell
rings and there is that person standing at the door. Or perhaps he or
she sent you a letter. Or better still sent you an email or spoke to you
by VoIP on your PC. How do these unarranged incidents take place? Are
they really the outcome of vanishingly small joint probabilities? Or are
they an uncanny result of some deeper force at work in our lives and in
the larger universe?

This was the sort of philosophical conundrum that grated on Einstein's
deeper-than-scientific sensibilities when he pondered the probabilistic
and utilitarian interpretation given to the Quantum Theory by Niels Bohr
and his disciples. Einstein famously said that he did not believe that
God played dice with the world. To which the other retorted, ``Stop
telling God what to do''.

So it was that Einstein postulated an exceedingly interesting experiment
called the EPR (Einstein Podolsky Rosen) experiment. In that thought
experiment, two photons that have been born together from the
annihilation (mutual dematerialization) of a positron (positively
charged electron) and an electron go their merry ways. Because they are
twins, though, their destinies should be linked (or correlated)
according to the Quantum Theory. This means that if you take the
electron that has gone in one direction and tweak its ear so to speak,
the other electron will yell ``Ouch'', again in a manner of speaking.
What is more, this will happen \emph{instantaneously}. Not at the speed
of light, but at once. However, according to Einstein, nothing can
travel faster than light. So, he said that the Quantum Theory must be
wrong.

But you know what? The outcome Einstein predicted actually happens. Over
the last twenty years or so, teams or experimental physicists have
proved that this sort of effect does indeed happen: it is a form of
``quantum spookiness''. The technical label for this is
``entanglement''. It is a consequence of the Quantum Theory. A physicist
called Bell proved this mathematically about twenty five years after the
EPR experiment was first suggested. The results of his paper was stated
simply as ``Reality is non-local''. What exactly does this mean?

Well it only apples to elementary particles that have been twinned. But
this quantum spookiness means that such particles are interconnected
regardless of how far, how long, and where they go. As usual,
laboratories are now outdoing each other in trying to apply this effect
to encode messages, etc. If you want to know more, read Amir Aczel's
eminently accessible book ``Entanglement''
\url{http://tinyurl.com/oqgd4}

But ponder the deeper philosophical implications of this experiment.
What does it mean on a larger scale or deeper level? It raises the
possibility that there may be a ``something'' that interconnects twinned
particles. Can that something also connect each and every one of us? Who
knows? Reality seems stranger than the imagination.

So what does this imply for coincidences? It may very well mean that
there are no coincidences. The law of karma and its inevitability may
one day be a consequence of one of the strangest results thrown up by
science. That interconnecting something may hold the key to some of the
great mysteries of life.

\subsection{Acknowledgements}\label{acknowledgements}

\subsection{Feedback}\label{feedback}

Please \href{mailto:feedback.swanlotus@gmail.com}{email me} your
comments and corrections.

\noindent A PDF version of this article is
\href{./are-there-coincidences.pdf}{available for download here}:

\begin{sffamily}

\url{https://swanlotus.netlify.app/blogs/are-there-coincidences.pdf}

\end{sffamily}

\section{Are There Coincidences?}\label{are-there-coincidences}

RCS 22 Aug 06



\end{document}
