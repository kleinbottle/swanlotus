% Options for packages loaded elsewhere
\PassOptionsToPackage{unicode,linktoc=all}{hyperref}
\PassOptionsToPackage{hyphens}{url}
\PassOptionsToPackage{dvipsnames,svgnames,x11names}{xcolor}
%
\documentclass[
  a4paper,
]{article}
\usepackage{amsmath,amssymb}
\usepackage{iftex}
\ifPDFTeX
  \usepackage[T1]{fontenc}
  \usepackage[utf8]{inputenc}
  \usepackage{textcomp} % provide euro and other symbols
\else % if luatex or xetex
  \usepackage{unicode-math} % this also loads fontspec
  \defaultfontfeatures{Scale=MatchLowercase}
  \defaultfontfeatures[\rmfamily]{Ligatures=TeX,Scale=1}
\fi
\usepackage{lmodern}
\ifPDFTeX\else
  % xetex/luatex font selection
\fi
% Use upquote if available, for straight quotes in verbatim environments
\IfFileExists{upquote.sty}{\usepackage{upquote}}{}
\IfFileExists{microtype.sty}{% use microtype if available
  \usepackage[]{microtype}
  \UseMicrotypeSet[protrusion]{basicmath} % disable protrusion for tt fonts
}{}
\makeatletter
\@ifundefined{KOMAClassName}{% if non-KOMA class
  \IfFileExists{parskip.sty}{%
    \usepackage{parskip}
  }{% else
    \setlength{\parindent}{0pt}
    \setlength{\parskip}{6pt plus 2pt minus 1pt}}
}{% if KOMA class
  \KOMAoptions{parskip=half}}
\makeatother
\usepackage{xcolor}
\usepackage[margin=25mm]{geometry}
\usepackage{color}
\usepackage{fancyvrb}
\newcommand{\VerbBar}{|}
\newcommand{\VERB}{\Verb[commandchars=\\\{\}]}
\DefineVerbatimEnvironment{Highlighting}{Verbatim}{commandchars=\\\{\}}
% Add ',fontsize=\small' for more characters per line
\usepackage{framed}
\definecolor{shadecolor}{RGB}{48,48,48}
\newenvironment{Shaded}{\begin{snugshade}}{\end{snugshade}}
\newcommand{\AlertTok}[1]{\textcolor[rgb]{1.00,0.81,0.69}{#1}}
\newcommand{\AnnotationTok}[1]{\textcolor[rgb]{0.50,0.62,0.50}{\textbf{#1}}}
\newcommand{\AttributeTok}[1]{\textcolor[rgb]{0.80,0.80,0.80}{#1}}
\newcommand{\BaseNTok}[1]{\textcolor[rgb]{0.86,0.64,0.64}{#1}}
\newcommand{\BuiltInTok}[1]{\textcolor[rgb]{0.80,0.80,0.80}{#1}}
\newcommand{\CharTok}[1]{\textcolor[rgb]{0.86,0.64,0.64}{#1}}
\newcommand{\CommentTok}[1]{\textcolor[rgb]{0.50,0.62,0.50}{#1}}
\newcommand{\CommentVarTok}[1]{\textcolor[rgb]{0.50,0.62,0.50}{\textbf{#1}}}
\newcommand{\ConstantTok}[1]{\textcolor[rgb]{0.86,0.64,0.64}{\textbf{#1}}}
\newcommand{\ControlFlowTok}[1]{\textcolor[rgb]{0.94,0.87,0.69}{#1}}
\newcommand{\DataTypeTok}[1]{\textcolor[rgb]{0.87,0.87,0.75}{#1}}
\newcommand{\DecValTok}[1]{\textcolor[rgb]{0.86,0.86,0.80}{#1}}
\newcommand{\DocumentationTok}[1]{\textcolor[rgb]{0.50,0.62,0.50}{#1}}
\newcommand{\ErrorTok}[1]{\textcolor[rgb]{0.76,0.75,0.62}{#1}}
\newcommand{\ExtensionTok}[1]{\textcolor[rgb]{0.80,0.80,0.80}{#1}}
\newcommand{\FloatTok}[1]{\textcolor[rgb]{0.75,0.75,0.82}{#1}}
\newcommand{\FunctionTok}[1]{\textcolor[rgb]{0.94,0.94,0.56}{#1}}
\newcommand{\ImportTok}[1]{\textcolor[rgb]{0.80,0.80,0.80}{#1}}
\newcommand{\InformationTok}[1]{\textcolor[rgb]{0.50,0.62,0.50}{\textbf{#1}}}
\newcommand{\KeywordTok}[1]{\textcolor[rgb]{0.94,0.87,0.69}{#1}}
\newcommand{\NormalTok}[1]{\textcolor[rgb]{0.80,0.80,0.80}{#1}}
\newcommand{\OperatorTok}[1]{\textcolor[rgb]{0.94,0.94,0.82}{#1}}
\newcommand{\OtherTok}[1]{\textcolor[rgb]{0.94,0.94,0.56}{#1}}
\newcommand{\PreprocessorTok}[1]{\textcolor[rgb]{1.00,0.81,0.69}{\textbf{#1}}}
\newcommand{\RegionMarkerTok}[1]{\textcolor[rgb]{0.80,0.80,0.80}{#1}}
\newcommand{\SpecialCharTok}[1]{\textcolor[rgb]{0.86,0.64,0.64}{#1}}
\newcommand{\SpecialStringTok}[1]{\textcolor[rgb]{0.80,0.58,0.58}{#1}}
\newcommand{\StringTok}[1]{\textcolor[rgb]{0.80,0.58,0.58}{#1}}
\newcommand{\VariableTok}[1]{\textcolor[rgb]{0.80,0.80,0.80}{#1}}
\newcommand{\VerbatimStringTok}[1]{\textcolor[rgb]{0.80,0.58,0.58}{#1}}
\newcommand{\WarningTok}[1]{\textcolor[rgb]{0.50,0.62,0.50}{\textbf{#1}}}
\usepackage{longtable,booktabs,array}
\usepackage{calc} % for calculating minipage widths
% Correct order of tables after \paragraph or \subparagraph
\usepackage{etoolbox}
\makeatletter
\patchcmd\longtable{\par}{\if@noskipsec\mbox{}\fi\par}{}{}
\makeatother
% Allow footnotes in longtable head/foot
\IfFileExists{footnotehyper.sty}{\usepackage{footnotehyper}}{\usepackage{footnote}}
\makesavenoteenv{longtable}
\setlength{\emergencystretch}{3em} % prevent overfull lines
\providecommand{\tightlist}{%
  \setlength{\itemsep}{0pt}\setlength{\parskip}{0pt}}
\setcounter{secnumdepth}{-\maxdimen} % remove section numbering
% definitions for citeproc citations
\NewDocumentCommand\citeproctext{}{}
\NewDocumentCommand\citeproc{mm}{%
  \begingroup\def\citeproctext{#2}\cite{#1}\endgroup}
\makeatletter
 % allow citations to break across lines
 \let\@cite@ofmt\@firstofone
 % avoid brackets around text for \cite:
 \def\@biblabel#1{}
 \def\@cite#1#2{{#1\if@tempswa , #2\fi}}
\makeatother
\newlength{\cslhangindent}
\setlength{\cslhangindent}{1.5em}
\newlength{\csllabelwidth}
\setlength{\csllabelwidth}{3em}
\newenvironment{CSLReferences}[2] % #1 hanging-indent, #2 entry-spacing
 {\begin{list}{}{%
  \setlength{\itemindent}{0pt}
  \setlength{\leftmargin}{0pt}
  \setlength{\parsep}{0pt}
  % turn on hanging indent if param 1 is 1
  \ifodd #1
   \setlength{\leftmargin}{\cslhangindent}
   \setlength{\itemindent}{-1\cslhangindent}
  \fi
  % set entry spacing
  \setlength{\itemsep}{#2\baselineskip}}}
 {\end{list}}
\usepackage{calc}
\newcommand{\CSLBlock}[1]{\hfill\break\parbox[t]{\linewidth}{\strut\ignorespaces#1\strut}}
\newcommand{\CSLLeftMargin}[1]{\parbox[t]{\csllabelwidth}{\strut#1\strut}}
\newcommand{\CSLRightInline}[1]{\parbox[t]{\linewidth - \csllabelwidth}{\strut#1\strut}}
\newcommand{\CSLIndent}[1]{\hspace{\cslhangindent}#1}
\ifLuaTeX
\usepackage[bidi=basic]{babel}
\else
\usepackage[bidi=default]{babel}
\fi
\babelprovide[main,import]{british}
% get rid of language-specific shorthands (see #6817):
\let\LanguageShortHands\languageshorthands
\def\languageshorthands#1{}
% $HOME/.pandoc/defaults/latex-header-includes.tex
% Common header includes for both lualatex and xelatex engines.
%
% Preliminaries
%
% \PassOptionsToPackage{rgb,dvipsnames,svgnames}{xcolor}
% \PassOptionsToPackage{main=british}{babel}
\PassOptionsToPackage{english}{selnolig}
\AtBeginEnvironment{quote}{\small}
\AtBeginEnvironment{quotation}{\small}
\AtBeginEnvironment{longtable}{\centering}
%
% Packages that are useful to include
%
\usepackage{graphicx}
\usepackage{subcaption}
\usepackage[inkscapeversion=1]{svg}
\usepackage[defaultlines=4,all]{nowidow}
\usepackage{etoolbox}
\usepackage{fontsize}
\usepackage{newunicodechar}
\usepackage{pdflscape}
\usepackage{fnpct}
\usepackage{parskip}
  \setlength{\parindent}{0pt}
\usepackage[style=american]{csquotes}
% \usepackage{setspace} Use the <fontname-plus.tex> files for setspace
%
\usepackage{hyperref} % cleveref must come AFTER hyperref
\usepackage[capitalize,noabbrev]{cleveref} % Must come after hyperref
\let\longdivision\relax
\usepackage{longdivision}
% noto-plus.tex
% Font-setting header file for use with Pandoc Markdown
% to generate PDF via LuaLaTeX.
% The main font is Noto Serif.
% Other main fonts are also available in appropriately named file.
\usepackage{fontspec}
\usepackage{setspace}
\setstretch{1.3}
%
\defaultfontfeatures{Ligatures=TeX,Scale=MatchLowercase,Renderer=Node} % at the start always
%
% For English
% See also https://tex.stackexchange.com/questions/574047/lualatex-amsthm-polyglossia-charissil-error
% We use Node as Renderer for the Latin Font and Greek Font and HarfBuzz as renderer ofr Indic fonts.
%
\babelfont{rm}[Script=Latin,Scale=1]{NotoSerif}% Config is at $HOME/texmf/tex/latex/NotoSerif.fontspec
\babelfont{sf}[Script=Latin]{SourceSansPro}% Config is at $HOME/texmf/tex/latex/SourceSansPro.fontspec
\babelfont{tt}[Script=Latin]{FiraMono}% Config is at $HOME/texmf/tex/latex/FiraMono.fontspec
%
% Sanskrit, Tamil, and Greek fonts
%
\babelprovide[import, onchar=ids fonts]{sanskrit}
\babelprovide[import, onchar=ids fonts]{tamil}
\babelprovide[import, onchar=ids fonts]{greek}
%
\babelfont[sanskrit]{rm}[Scale=1.1,Renderer=HarfBuzz,Script=Devanagari]{NotoSerifDevanagari}
\babelfont[sanskrit]{sf}[Scale=1.1,Renderer=HarfBuzz,Script=Devanagari]{NotoSansDevanagari}
\babelfont[tamil]{rm}[Renderer=HarfBuzz,Script=Tamil]{NotoSerifTamil}
\babelfont[tamil]{sf}[Renderer=HarfBuzz,Script=Tamil]{NotoSansTamil}
\babelfont[greek]{rm}[Script=Greek]{GentiumBookPlus}
%
% Math font
%
\usepackage{unicode-math} % seems not to hurt % fallabck
\setmathfont[bold-style=TeX]{STIX Two Math}
\usepackage{amsmath}
\usepackage{esdiff} % for derivative symbols
% \renewcommand{\mathbf}{\symbf}
%
%
% Other fonts
%
\newfontfamily{\emojifont}{Symbola}
%

\usepackage{titling}
\usepackage{fancyhdr}
    \pagestyle{fancy}
    \fancyhead{}
    \fancyfoot{}
    \renewcommand{\headrulewidth}{0.2pt}
    \renewcommand{\footrulewidth}{0.2pt}
    \fancyhead[LO,RE]{\scshape\thetitle}
    \fancyfoot[CO,CE]{\footnotesize Copyright © 2006\textendash\the\year, R (Chandra) Chandrasekhar}
    \fancyfoot[RE,RO]{\thepage}
%
\usepackage{newunicodechar}
\newunicodechar{√}{\textsf{√}}
\ifLuaTeX
  \usepackage{selnolig}  % disable illegal ligatures
\fi
\usepackage{bookmark}
\IfFileExists{xurl.sty}{\usepackage{xurl}}{} % add URL line breaks if available
\urlstyle{sf}
\hypersetup{
  pdftitle={How Are Numbers Made?},
  pdfauthor={R (Chandra) Chandrasekhar},
  pdflang={en-GB},
  colorlinks=true,
  linkcolor={DarkOliveGreen},
  filecolor={Purple},
  citecolor={DarkKhaki},
  urlcolor={Maroon},
  pdfcreator={LaTeX via pandoc}}

\title{How Are Numbers Made?}
\author{R (Chandra) Chandrasekhar}
\date{2023-03-21 | 2023-04-21}

\begin{document}
\maketitle

\thispagestyle{empty}


``Is mathematics part of nature?'' This was the question that my
polymath friend Solus ``Sol'' Smkin asked me when we met at the opening
of a Music School.

``Sol, have a sense of occasion, of time and place, please! This is a
social occasion at a Music School, not the Agora of Athens! Your
question is too profound to be discussed here and now. We are planning
to go on a tour of Santorini in one month's time. Let our thoughts
mingle with those of the profound, ancient Greeks. Until then, I will
take a raincheck on your question,'' I remonstrated.

And so it was that Sol next resumed this conversational thread while we
gazed upon the azure sea, from under the shade olive trees, atop a
hillock on Santorini.

\subsection{How would you build a
universe?}\label{how-would-you-build-a-universe}

``If were given the power to build a world, how would you do it?'' Sol
fired away, without any segue.

``Why so outlandish a question? Enjoy the sun and the breeze, and the
bleats of sheep,'' I replied.

``Have you heard of the
\href{https://worldbuilding.stackexchange.com/}{Worldbuilding
Stackexchange}? It `is a question and answer site for writers/artists
using science, geography and culture to construct imaginary worlds and
settings'.

``No,'' I said.

``It is a serious place on the Web where imaginary worlds with negative
gravity and entropy may be conceived, discussed, and constructed. My
question is not a flippant one.''

``I stand educated. But what has mathematics to do with those flights of
fancy?'' I queried.

Sol said, ``Everything''. One cannot build a universe without the laws
of physics or the laws of mind. Or the laws of cause and effect. As long
as structure, consistence, repeatability, and durability are concerned,
one cannot do without numbers. More than atoms, numbers are the building
blocks of the world.''

We had at last launched into the ocean of discussion proper. And what a
majestic premise: that the world is built upon numbers before it could
be built upon atoms. I asked Sol to fire away with his canons of
unassailable argument, waiting passively to be informed and entertained.

\subsection{Lessons from observing
life}\label{lessons-from-observing-life}

``You must have heard of my paternal cousin, once removed, Hieronymus
Septimus Simkin, whom I affectionately call Seven. He it was who opened
my eyes first to the unguarded secrets staring at us from Nature. He
introduced me to books like D'Arcy Wentworth Thompsons classic \emph{On
Growth and Form} {[}\citeproc{ref-thompson-1992}{1}{]} and the
interestingly titled \emph{The Parsimonious Universe}
{[}\citeproc{ref-parsimonious-1996}{2}{]}. These books postulate, with
incontrovertible evidence, that the \href{}{Book of Nature} derives its
intelligence from adaptation, powered by mathematics.

\subsection{The integers have their
place}\label{the-integers-have-their-place}

Sol told me that he started off with the integers that Kronecker had so
exalted. ``The integers are fundamental because all mathematics begins
with counting. The quantitative fields are all founded on the natural
numbers we count with. And
\href{https://swanlotus.netlify.app/blogs/the-two-most-important-numbers-zero-and-one}{zero
and one are the two most important integers}---that I grant you,'' he
had told Seven. ``But we cannot stop with integers and exclude
everything else.''

\subsection{The square and the circle}\label{the-square-and-the-circle}

Sol had told Seven that the square is \emph{the} four-sided regular
polygon. If we consider a square with a side length equal to one unit,
by the theorem of Pythagoras, we know that its diagonal has a length
equal to \(\sqrt{1^2 + 1^2} = \sqrt{2}\) units. And there are proofs
aplenty on the Web that this number is in no way an integer. Indeed, it
is not even the ratio of two integers. How could something as basic as
the diagonal of a square cause the first chink in Kronecker's armour?

Moving from the finite to the infinite, the circle may be viewed as the
limiting case of a regular polygon of \(n\) sides as \(n \to \infty\).
And if we tried to find out how many diameters would fit into the
circumference of a circle, we do not get an integer, or even an exact
fraction, but rather a number that sits between \(3\) and \(4\), having
decimal places without end, namely, \(\pi \approx 3.141592654\). And
that number is not an integer by a country mile.

``The natural numbers, the integers, and the rationals---all of these
come under Kronecker's integers, but where do we stash \(\sqrt{2}\) and
\(2\pi\) amongst them?'' Sol had asked Seven.

He was met with bemused silence.

\subsection{\texorpdfstring{How about the number
\(e\)?}{How about the number e?}}\label{how-about-the-number-e}

Encouraged that he had stupefied Seven right at the start, Sol had
mounted his next hobby horse, and expounded on \(e\).

``The number \(e\) is probably \emph{the} most important number after
\(0\) and \(1\). And do you know what it is? It is both
\href{https://mathworld.wolfram.com/IrrationalNumber.html}{irrational}
and
\href{https://en.wikipedia.org/wiki/Transcendental_number}{transcendental}.
If you differentiate or integrate, you will find that the exponential
function \(\exp(x) = e^x\) is an eigenfunction of each operation. If you
look into Nature, \(e\) holds the pride of place in the
\href{https://www.khanacademy.org/math/statistics-probability/modeling-distributions-of-data/normal-distributions-library/a/normal-distributions-review}{normal
distribution}. If you are into
\href{https://www.cns.nyu.edu/~david/handouts/linear-systems/linear-systems.html}{linear
system theory} you cannot escape \(e\).

But what exactly is the value of \(e\)? Again it cannot be confined like
an integer: \(e \approx 2.718281828\) again in a never ending decimal
sequence. This number pervades all of Nature and yet it cannot be
bottled into a finite number of digits! Where are the legions of
integers to duel with this puny expeditionary force of three numbers?
\emph{It appears that Nature prefers the non-integers!}''

``Very poetic and ably put,'' I nodded in appreciation.

\subsection{Open secrets}\label{open-secrets}

``Helen Keller is reputed to have exclaimed, when she felt the warm glow
of a wood-fire, that it was the release of sunbeams that had been
trapped long ago in the wood. Her statement is remarkably perceptive,
poetic, and precise,'' Sol continued.

``Unlike ancient sunlight trapped in wood, \(\sqrt{2}\), \(\pi\), and
\(e\), cannot be caged in a finite box. These three numbers---that
pervade Nature---have decimal forms that clearly announce that they are
\emph{not} integers. Their value defies finite expression; only with
symbols may we do them justice.

``Do you know why they are open secrets? They are public, staring at us
from every square, circle, and signal, and yet, their full form is never
revealed. They cannot be contained except in infinity. To know the next
decimal place of \(\sqrt{2}\), or \(\pi\), or \(e\), one needs to
\emph{compute it} using some formula. Or one could look up a table. But
there is no \emph{knowing} that sought after next decimal place, as we
know \(\frac{1}{2} = 0.5\), with as many zeros stacked at the end as we
wish. That sort of closed form is not baked into nature. She prefers the
indescribable exactitude of numbers like \(e\).'' Sol had told Seven.

The rest of Sol's dialogue with Seven was intricately mathematical. I
have recorded it here, not as a dialogue, but as logical
exposition---complete with references---for the benefit of the casual
reader.

\subsection{The square root of two}\label{the-square-root-of-two}

Of the triad---\(\sqrt{2}\), \(\pi\), and \(e\)---we first consider
\(\sqrt{2}\). It is the most within our everyday grasp. It evokes
geometry rather than number for its precise expression. It is the
diagonal of a unit square. And we know that its square root must lie
between \(1\) and \(1.5\), as the latter squared is \(2.25\). It may be
evaluated painstakingly using algorithms from the
age-before-calculators. So, let us look at one of those first.

\subsubsection{Manual extraction of √2}\label{manual-extraction-of-2}

The manual extraction of square roots is analogous to long division. The
process is both tedious and error-prone. The algorithm uses the fact
that the factor \(2\) figures in any square, witness:
\((x + a) = x^2 + 2x +a^2\). So, this particular method makes use of
this fact at each step in the ``long division'' that is done. To see the
end result and the working, please see
\href{https://www.cuemath.com/algebra/square-root-of-2/}{this}
{[}\citeproc{ref-cuemathsqrt}{3}{]}. For a deeper explanation,
\href{https://www.cantorsparadise.com/the-square-root-algorithm-f97ab5c29d6d}{read
this blog} {[}\citeproc{ref-ujjwalsingh2021}{4}{]}. ``I consider this
form of working, with pencil and paper, a sophisticated form of torture.
Euler of Gauss might have revelled in such pursuits, but count me out!''
Sol added as a snide aside.

\subsection{Two ways of expressing the same
number}\label{two-ways-of-expressing-the-same-number}

``I came perilously close to losing the bet to Seven,'' Sol continued.
``You see, I had forgotten that the decimal system was not the only way
to represent irrationals and transcendentals in never-ending glory. And
I don't mean a change of base. Can you guess what I had forgotten?'' Sol
asked me during our conversation.

``Nothing from me to egg you on,'' I said in a sleepy tone. The time,
place, and weather had lulled me into a restful somnolence that was
ill-suited to mathematical head-scratching.

``It is something that we learn at high school, more as a curiosity than
as useful mathematics,'' Sol continued by way of enticing me with a
clue. ``Can you guess what it is?''

When I shook my head with a dazed stare, Sol said, ``Come on. One last
clue. It has to do with fractions.''

When I refused to be drawn into guessing what it was, Sol exclaimed,
``\href{https://en.wikipedia.org/wiki/Continued_fraction}{Continued
Fractions}!'' {[}\citeproc{ref-loya2017}{9}{]} rousing me into full
wakefulness with his thunderous voice.

``Apart from a change of base, there are basically \emph{two} ways of
representing real numbers: decimals, and continued fractions. Patterns
not discernible in the decimal representation suddenly pop out with
pellucid clarity when the same number is expressed as a continued
fraction. The advent of computers and 64-bit computation has diverted
our attention away from experiencing the periodic beauty of a
\href{https://en.wikipedia.org/wiki/Quadratic_irrational_number}{quadratic
irrational}, expressed as a continued fraction,'' Sol went on,
lyrically.

``Practically, every irrational, when pressed to computational use, is
really a rational approximation to the irrational, to an accuracy that
serves the purpose. In that sense, Kronecker was not far from the truth.
But the full glory of \(\sqrt{2}\), or \(\pi\), or \(e\) can only be
encapsulated by the symbols we use for them. Every other, rational
expression is but a costumed appearance, not the true persona.'' Sol was
in his element as he expounded.

\subsection{The charm of continued
fractions}\label{the-charm-of-continued-fractions}

Sol then went on to demonstrate his preferred method of evaluating
\(\sqrt{2}\), using continued fractions. The method seemed like sleight
of hand, but it is well-founded, and is also an example of how integers
are used to tame the irrationals.

Continued fractions are curious mathematical entities that have
surprising properties. They are an alternative rational number
representation of real numbers. No finite continued fraction can equate
to an irrational number. But a never-ending continued fraction can
indeed represent an irrational number. ``this is why I say that the
rationals and the irrationals meet at infinity,'' Sol said with panache.

\subsubsection{Continued fraction expansion of a rational
number}\label{continued-fraction-expansion-of-a-rational-number}

``Let us start modestly and try to expand a \emph{rational} number using
continued fractions,'' said Sol. ``Give me a scary or hairy rational
number, preferably larger than one,'' he said.

``What about \(\frac{3257}{106}\)?'' I answered, choosing the two
numbers that randomly came to mind.

``Taken,'' replied Sol. ``Here is a little program that I have written
to do it for me. I will then explain the algorithm. We start off by
doing plain long division to get:'' \[
\frac{3257}{106} = 30 + \frac{77}{106} = 30 + \frac{1}{\frac{106}{77}}\\
\] ``Why do we write it like this? we want to get whole number quotients
and whole number remainders and the trick is to always divide the larger
number by the smaller, by inverting the remainder fraction,'' Sol said.
``If you keep in mind that our goal is an improper fraction, you are
good to go.''

\subsubsection{Continued fraction expansion of
√2}\label{continued-fraction-expansion-of-2}

The number \(\sqrt{2}\) is particularly amenable to a simply derived
continued fraction expansion. Consider:
\begin{equation}\phantomsection\label{eq:sqrt2}{
\begin{aligned}
\sqrt{2} &= \sqrt{2} &\text{ [add and subtract $1$ on the RHS]}\\
&= 1 + \sqrt{2} - 1 &\text{ [multiply second term on RHS by $\frac{\sqrt{2}+1}{\sqrt{2}+1} = 1$}]\\
&= 1 + \frac{(\sqrt{2} - 1)(\sqrt{2} + 1)}{\sqrt{2} + 1} &\text{ [difference of two squares]}\\
&= 1 + \frac{1}{1 + \textcolor{Maroon}{\sqrt{2}}}\\
\end{aligned}
}\end{equation} Since the LHS in \cref{eq:sqrt2} is \(\sqrt{2}\), we may
substitute the entire RHS in place of the term
\(\textcolor{Maroon}{\sqrt{2}}\) on the RHS. So doing, we get the
following descending staircase of continued fractions:
\begin{equation}\phantomsection\label{eq:sqrt2cfinfty}{
\begin{aligned}
\sqrt{2} &= 1 + \frac{1}{1 + \textcolor{Maroon}{\sqrt{2}}}\\
&= 1 + \cfrac{1}{1 + \textcolor{Maroon}{1 + \cfrac{1}{1+\sqrt{2}}}}\\
&= 1 + \cfrac{1}{2 + \cfrac{1}{1 + \sqrt{2}}} &\text{ [and recursively substituting for $\sqrt{2}$ again]}\\
&= 1 + \cfrac{1}{2 + \cfrac{1}{1 + 1 + \cfrac{1}{1 + \sqrt{2}}}}\\
&= 1 + \cfrac{1}{2 + \cfrac{1}{2 + \cfrac{1}{1 + \sqrt{2}}}}\\
&= 1 + \cfrac{1}{2 + \cfrac{1}{2 + \cfrac{1}{2 + \cfrac{1}{1+\sqrt{2}}}}}\\
&\hskip 100pt\ddots\\ % Care!
\end{aligned}
}\end{equation} The \emph{congruents} or \emph{approximants} from a
continued fraction are partial sums that we may accumulate as
approximations to the irrational number, in our case, that we seek to
represent. Unfurling the continued fractions into partial sums is a
tricky exercise. There are also recurrence relations for them. In our
particular case, we ignore the \(\frac{1}{1 + \sqrt{2}}\) terms that
occur in the \emph{denominator} of \cref{eq:sqrt2cfinfty} but count the
numerator terms to get a sequence of fractions.

In this way, we start off with \(1\), followed by
\(1 + \frac{1}{2} = \frac{3}{2}\). Working our way down, we encounter
\(1 + \frac{1}{2 + \frac{1}{2}} = 1+\frac{1}{\frac{5}{2}} = 1 + \frac{2}{5} = \frac{7}{5}\).
The next convergent after this, when simplified, is
\(1 + \frac{1}{2+\frac{2}{5}} = 1 + \frac{5}{12} = \frac{17}{12}\).

Sol said that working out these fractions could be a form of torture
unless you are particularly fond of or adept at computing them. He
himself did not relish such hand computations but preferred to program
to get a solution. As it turns out, he was able to get a sequence of the
first eight successive convergents from the \texttt{Julia} code below:

\begin{Shaded}
\begin{Highlighting}[]
\ImportTok{using} \BuiltInTok{Pkg}
\BuiltInTok{Pkg}\NormalTok{.}\FunctionTok{add}\NormalTok{(}\StringTok{"RealContinuedFractions"}\NormalTok{)}
\CommentTok{\#}
\CommentTok{\# Use the above only for the first time.}
\CommentTok{\#}
\ImportTok{using} \BuiltInTok{RealContinuedFractions}
\FunctionTok{convergents}\NormalTok{(}\FunctionTok{contfrac}\NormalTok{(}\FunctionTok{sqrt}\NormalTok{(}\FloatTok{2}\NormalTok{), }\FloatTok{15}\NormalTok{))}
\end{Highlighting}
\end{Shaded}

which gave the following results:

\begin{Shaded}
\begin{Highlighting}[]
\FloatTok{15}\OperatorTok{{-}}\NormalTok{element }\DataTypeTok{Vector}\NormalTok{\{}\DataTypeTok{Rational}\NormalTok{\{}\DataTypeTok{Int64}\NormalTok{\}\}}\OperatorTok{:}
      \FloatTok{1}\OperatorTok{//}\FloatTok{1}
      \FloatTok{3}\OperatorTok{//}\FloatTok{2}
      \FloatTok{7}\OperatorTok{//}\FloatTok{5}
     \FloatTok{17}\OperatorTok{//}\FloatTok{12}
     \FloatTok{41}\OperatorTok{//}\FloatTok{29}
     \FloatTok{99}\OperatorTok{//}\FloatTok{70}
    \FloatTok{239}\OperatorTok{//}\FloatTok{169}
    \FloatTok{577}\OperatorTok{//}\FloatTok{408}
   \FloatTok{1393}\OperatorTok{//}\FloatTok{985}
   \FloatTok{3363}\OperatorTok{//}\FloatTok{2378}
   \FloatTok{8119}\OperatorTok{//}\FloatTok{5741}
  \FloatTok{19601}\OperatorTok{//}\FloatTok{13860}
  \FloatTok{47321}\OperatorTok{//}\FloatTok{33461}
 \FloatTok{114243}\OperatorTok{//}\FloatTok{80782}
 \FloatTok{275807}\OperatorTok{//}\FloatTok{195025}
\end{Highlighting}
\end{Shaded}

The rational fractions above are tabulated with their decimal versions
to provide an idea of how the convergents do indeed converge to the
``benchmark'' decimal value of \(\sqrt{2}\) as available on a
\texttt{Julia}
\href{https://en.wikipedia.org/wiki/Read\%E2\%80\%93eval\%E2\%80\%93print_loop}{REPL},
which is shown below:

\begin{Shaded}
\begin{Highlighting}[]
\FunctionTok{sqrt}\NormalTok{(}\FunctionTok{big}\NormalTok{(}\FloatTok{2}\NormalTok{))}
\FloatTok{1.414213562373095048801688724209698078569671875376948073176679737990732478462102}
\end{Highlighting}
\end{Shaded}

SEE Wolfram Alpha for repeating digits

\begin{longtable}[]{@{}
  >{\centering\arraybackslash}p{(\columnwidth - 4\tabcolsep) * \real{0.2340}}
  >{\raggedright\arraybackslash}p{(\columnwidth - 4\tabcolsep) * \real{0.2979}}
  >{\raggedleft\arraybackslash}p{(\columnwidth - 4\tabcolsep) * \real{0.4681}}@{}}
\caption{\label{tbl:sqrt2convergents}The first fifteen convergents for
\(\sqrt{2}\).}\tabularnewline
\toprule\noalign{}
\begin{minipage}[b]{\linewidth}\centering
Convergent
\end{minipage} & \begin{minipage}[b]{\linewidth}\raggedright
Decimal Value
\end{minipage} & \begin{minipage}[b]{\linewidth}\raggedleft
Period
\end{minipage} \\
\midrule\noalign{}
\endfirsthead
\toprule\noalign{}
\begin{minipage}[b]{\linewidth}\centering
Convergent
\end{minipage} & \begin{minipage}[b]{\linewidth}\raggedright
Decimal Value
\end{minipage} & \begin{minipage}[b]{\linewidth}\raggedleft
Period
\end{minipage} \\
\midrule\noalign{}
\endhead
\bottomrule\noalign{}
\endlastfoot
\(\frac{1}{1}\) & \(1.0\) & \(0\) \\
\(\frac{3}{2}\) & \(1.5\) & \(0\) \\
\(\frac{7}{5}\) & \(1.4\) & \(0\) \\
\(\frac{17}{12}\) & \(1.41\overline{6}\) & \(1\) \\
\(\frac{41}{29}\) & \(1.\overline{4137931034482758620689655172}\) &
\(28\) \\
\(\frac{99}{70}\) & \(1.4\overline{142857}\) & \(6\) \\
\(\frac{239}{169}\) &
\(1.\overline{414201183431952662721893491124260355029585798816568047337278106508875739644970}\)
& \(78\) \\
\(\frac{577}{408}\) & \(1.414\overline{2156862745098039}\) & \(16\) \\
\(\frac{1393}{985}\) & \(1.41421319796954314...\) & \(98\) \\
\(\frac{3363}{2378}\) & \(1.4142136248948696\) & \(140\) \\
\(\frac{8119}{5741}\) &
\(1.414213551646054778387906480929814279079437255859375...\) &
\(5740\) \\
\(\frac{19601}{13860}\) & \(1.41\overline{421356}\) & \(6\) \\
\(\frac{47321}{33461}\) &
\(1.414213562057320405784821559791453182697296142578125...\) &
\(4780\) \\
\(\frac{114243}{80782}\) &
\(1.4142135624272734024906538585328414745859226065212547349657101829...\)
& \(546\) \\
\(\frac{275807}{195025}\) &
\(1.4142135623637994701340403480571694672107696533203125...\) &
\(1876\) \\
\end{longtable}

The ``benchmark'' decimal avlue of \(\sqrt{2}\) as available on a
\texttt{Julia}
\href{https://en.wikipedia.org/wiki/Read\%E2\%80\%93eval\%E2\%80\%93print_loop}{REPL}
is shown below. There is agreement at best to four decimal places.

\begin{Shaded}
\begin{Highlighting}[]
\FunctionTok{sqrt}\NormalTok{(}\FunctionTok{big}\NormalTok{(}\FloatTok{2}\NormalTok{))}
\FloatTok{1.414213562373095048801688724209698078569671875376948073176679737990732478462102}
\end{Highlighting}
\end{Shaded}

The results are tabulated for comparison with the decimal value of
\(\sqrt{2}\) which, to 15 decimal places, is \(1.414213562373095\).
\(1.4142135623730951454746218587388284504413604736328125\) Julia
\(1.41421356237309504880168872420969807856967187537694807317667973799073247846210\)
Julia Big Int
\$1.414213562373095048801688724209698078569671875376948073176679737990732478462102
Julia Big Int
\$1.4142135623730950488016887242096980785696718753769480731766797379907324784621070388503875343276415727350138462309122970249248360\ldots{}
Wolfram Alpha \$1.4142135623730951454746218587388284504413604736328125
\#\# Where the rationals and the irrationals meet

``Infinity is where the rationals and irrationals meet,'' Sol had
continues in his discussion with Seven. ``And as far as I know, infinity
s not an integer. What it is, I do not precisely know.'' Take \[
\] https://r-knott.surrey.ac.uk/Fibonacci/cfINTRO.html\#section3.1
https://xlinux.nist.gov/dads/HTML/squareRoot.html
https://math.stackexchange.com/questions/2916718/calculating-the-square-root-of-2
https://medium.com/not-zero/how-to-calculate-square-roots-by-hand-21a78b6da9ae
https://www.cantorsparadise.com/the-square-root-algorithm-f97ab5c29d6d
https://nebusresearch.wordpress.com/2014/10/17/how-richard-feynman-got-from-the-square-root-of-2-to-e/
https://medium.com/i-math/how-to-find-square-roots-by-hand-f3f7cadf94bb
https://en.wikipedia.org/wiki/List\_of\_formulae\_involving\_\%CF\%80

https://mathworld.wolfram.com/PiFormulas.html

https://math.stackexchange.com/questions/2153619/where-do-mathematicians-get-inspiration-for-pi-formulas

Synesthtetic people can see grayness blurring a beautiful landscape if
there is wrong digit in the digit in the decimal expression for pi.

\subsection{Where the rationals and the irrationals
meet}\label{where-the-rationals-and-the-irrationals-meet}

\subsection{What is the next digit?}\label{what-is-the-next-digit}

\subsection{Fibonacci has closed form but
\ldots{}}\label{fibonacci-has-closed-form-but}

\subsection{Continues fractions pi}\label{continues-fractions-pi}

\subsection{How to construct a rational from two
irrationals}\label{how-to-construct-a-rational-from-two-irrationals}

\subsection{Acknowledgements}\label{acknowledgements}

\subsection{Feedback}\label{feedback}

Please \href{mailto:feedback.swanlotus@gmail.com}{email me} your
comments and corrections.

\noindent A PDF version of this article is
\href{./open-secrets.pdf}{available for download here}:

\begin{small}

\begin{sffamily}

\url{https://swanlotus.netlify.app/blogs/open-secrets.pdf}

\end{sffamily}

\end{small}

https://math.stackexchange.com/questions/586008/is-a-decimal-with-a-predictable-pattern-a-rational-number

https://math.stackexchange.com/questions/61937/how-can-i-prove-that-all-rational-numbers-are-either-terminating-decimal-or-repe

https://math.stackexchange.com/questions/1759007/which-real-numbers-have-2-possible-decimal-representations

https://en.wikipedia.org/wiki/Champernowne\_constant

https://math.stackexchange.com/questions/1259073/rational-irrational-numbers

https://en.wikipedia.org/wiki/Liouville\_number

https://en.wikipedia.org/wiki/Transcendental\_number

https://www.khanacademy.org/math/algebra/x2f8bb11595b61c86:irrational-numbers/x2f8bb11595b61c86:sums-and-products-of-rational-and-irrational-numbers/v/sums-and-products-of-irrational-numbers

https://www.google.com/search?q=example+of+a+\%2B+b+\%3D+r+where+a+is+irratioonal\%2C+b+is+irrational+and+r+is+rational

https://www.khanacademy.org/math/algebra/x2f8bb11595b61c86:irrational-numbers/x2f8bb11595b61c86:sums-and-products-of-rational-and-irrational-numbers/v/sums-and-products-of-irrational-numbers

https://www.youtube.com/watch?v=RpDWHlFKHy4

https://www.numberempire.com/3363

https://math.stackexchange.com/questions/730349/convergents-of-square-root-of-2

https://www.youtube.com/watch?v=E4b-k\_Dug\_E

https://www.youtube.com/watch?v=CaasbfdJdJg

https://www.youtube.com/watch?v=zCFF1l7NzVQ

https://math.stackexchange.com/questions/716944/how-to-find-continued-fraction-of-pi

https://perl.plover.com/classes/cftalk/INFO/gosper.html

https://tex.stackexchange.com/questions/334917/box-around-continued-fraction

https://www.quora.com/Why-can-some-irrational-numbers-be-expressed-as-continued-infinite-fractions

https://www.quora.com/What-is-the-difference-between-using-continued-fractions-to-represent-rationals-and-irrationals-Is-continued-fraction-unique

Euler and Lagrange proved that periodic continued fractions represent
quadratic irrational numbers. https://qr.ae/pKUeyc

https://www.quora.com/What-is-the-difference-between-using-continued-fractions-to-represent-rationals-and-irrationals-Is-continued-fraction-unique

https://math.stackexchange.com/questions/1349073/how-to-find-out-the-number-of-repeating-digits-of-a-rational-number-in-decimal-f

https://math.stackexchange.com/questions/140583/compute-the-period-of-a-decimal-number-a-priori

https://proofwiki.org/wiki/Continued\_Fraction\_Expansion\_of\_Irrational\_Square\_Root/Examples/2
Compute the period of a decimal number a priori

Are the numerator and the denominator of a convergent of a continued
fraction always coprime?
https://math.stackexchange.com/questions/1493902/are-the-numerator-and-the-denominator-of-a-convergent-of-a-continued-fraction-al

https://math.stackexchange.com/questions/1493902/are-the-numerator-and-the-denominator-of-a-convergent-of-a-continued-fraction-al

https://math.stackexchange.com/questions/4112417/continued-fraction-representation-of-quadratic-irrationals

https://r-knott.surrey.ac.uk/Fibonacci/cfINTRO.html

https://math.stackexchange.com/questions/265690/continued-fraction-of-a-square-root

Lists of pi and e expansions

https://mathworld.wolfram.com/PiFormulas.html

https://en.wikipedia.org/wiki/List\_of\_representations\_of\_e

https://math.stackexchange.com/questions/2205168/how-to-show-every-rational-number-can-be-expressed-as-two-different-continued-fr

\section*{References}\label{bibliography}
\addcontentsline{toc}{section}{References}

\phantomsection\label{refs}
\begin{CSLReferences}{0}{0}
\bibitem[\citeproctext]{ref-thompson-1992}
\CSLLeftMargin{{[}1{]} }%
\CSLRightInline{D'Arcy Wentworth Thompson. 1992. \emph{On growth and
form} (The Complete Revised Edition ed.). Dover Publications.}

\bibitem[\citeproctext]{ref-parsimonious-1996}
\CSLLeftMargin{{[}2{]} }%
\CSLRightInline{Stefan Hildebrandt and Anthony Tromba. 1996. \emph{{The
Parsimonious Universe: Shape and Form in the Natural World}} (1st ed.).
Copernicus Books.}

\bibitem[\citeproctext]{ref-cuemathsqrt}
\CSLLeftMargin{{[}3{]} }%
\CSLRightInline{---. 2023. {Square Root of 2}. Retrieved 9 December 2023
from \url{https://www.cuemath.com/algebra/square-root-of-2/}}

\bibitem[\citeproctext]{ref-ujjwalsingh2021}
\CSLLeftMargin{{[}4{]} }%
\CSLRightInline{Ujjwal Singh. 2021. {The Square Root Algorithm}.
Retrieved 8 December 2023 from
\url{https://www.cantorsparadise.com/the-square-root-algorithm-f97ab5c29d6d}}

\bibitem[\citeproctext]{ref-olds1963}
\CSLLeftMargin{{[}5{]} }%
\CSLRightInline{C D Olds. 1963. \emph{{Continued Fractions}}. Random
House.}

\bibitem[\citeproctext]{ref-niven1991}
\CSLLeftMargin{{[}6{]} }%
\CSLRightInline{Ian Niven, Herbert S Zuckerman, and Hugh L Montgomery.
1991. \emph{{An Introduction to the Theory of Numbers}} (5th ed.). John
Wiley \& Sons.}

\bibitem[\citeproctext]{ref-davenport2008}
\CSLLeftMargin{{[}7{]} }%
\CSLRightInline{Harold Davenport and James H Davenport. 2008. \emph{{The
Higher Arithmetic}: {An Introduction to the Theory of Numbers}} (8th
ed.). Cambridge University Press.}

\bibitem[\citeproctext]{ref-simoson2019}
\CSLLeftMargin{{[}8{]} }%
\CSLRightInline{Andrew J Simoson. 2019. \emph{{Exploring Continued
Fractions}: {From the Integers to Solar Eclipses}}. MAA Press/American
Mathematical Society.}

\bibitem[\citeproctext]{ref-loya2017}
\CSLLeftMargin{{[}9{]} }%
\CSLRightInline{Paul Loya. 2017. \emph{{Amazing and Aesthetic Aspects of
Analysis}}. Springer.}

\end{CSLReferences}



\end{document}
