% Options for packages loaded elsewhere
\PassOptionsToPackage{unicode,linktoc=all}{hyperref}
\PassOptionsToPackage{hyphens}{url}
\PassOptionsToPackage{dvipsnames,svgnames,x11names}{xcolor}
%
\documentclass[
  a4paper,
]{article}
\usepackage{amsmath,amssymb}
\usepackage{iftex}
\ifPDFTeX
  \usepackage[T1]{fontenc}
  \usepackage[utf8]{inputenc}
  \usepackage{textcomp} % provide euro and other symbols
\else % if luatex or xetex
  \usepackage{unicode-math} % this also loads fontspec
  \defaultfontfeatures{Scale=MatchLowercase}
  \defaultfontfeatures[\rmfamily]{Ligatures=TeX,Scale=1}
\fi
\usepackage{lmodern}
\ifPDFTeX\else
  % xetex/luatex font selection
\fi
% Use upquote if available, for straight quotes in verbatim environments
\IfFileExists{upquote.sty}{\usepackage{upquote}}{}
\IfFileExists{microtype.sty}{% use microtype if available
  \usepackage[]{microtype}
  \UseMicrotypeSet[protrusion]{basicmath} % disable protrusion for tt fonts
}{}
\makeatletter
\@ifundefined{KOMAClassName}{% if non-KOMA class
  \IfFileExists{parskip.sty}{%
    \usepackage{parskip}
  }{% else
    \setlength{\parindent}{0pt}
    \setlength{\parskip}{6pt plus 2pt minus 1pt}}
}{% if KOMA class
  \KOMAoptions{parskip=half}}
\makeatother
\usepackage{xcolor}
\usepackage[margin=25mm]{geometry}
\usepackage{longtable,booktabs,array}
\usepackage{calc} % for calculating minipage widths
% Correct order of tables after \paragraph or \subparagraph
\usepackage{etoolbox}
\makeatletter
\patchcmd\longtable{\par}{\if@noskipsec\mbox{}\fi\par}{}{}
\makeatother
% Allow footnotes in longtable head/foot
\IfFileExists{footnotehyper.sty}{\usepackage{footnotehyper}}{\usepackage{footnote}}
\makesavenoteenv{longtable}
\usepackage{graphicx}
\makeatletter
\def\maxwidth{\ifdim\Gin@nat@width>\linewidth\linewidth\else\Gin@nat@width\fi}
\def\maxheight{\ifdim\Gin@nat@height>\textheight\textheight\else\Gin@nat@height\fi}
\makeatother
% Scale images if necessary, so that they will not overflow the page
% margins by default, and it is still possible to overwrite the defaults
% using explicit options in \includegraphics[width, height, ...]{}
\setkeys{Gin}{width=\maxwidth,height=\maxheight,keepaspectratio}
% Set default figure placement to htbp
\makeatletter
\def\fps@figure{htbp}
\makeatother
\usepackage{svg}
\setlength{\emergencystretch}{3em} % prevent overfull lines
\providecommand{\tightlist}{%
  \setlength{\itemsep}{0pt}\setlength{\parskip}{0pt}}
\setcounter{secnumdepth}{-\maxdimen} % remove section numbering
% definitions for citeproc citations
\NewDocumentCommand\citeproctext{}{}
\NewDocumentCommand\citeproc{mm}{%
  \begingroup\def\citeproctext{#2}\cite{#1}\endgroup}
\makeatletter
 % allow citations to break across lines
 \let\@cite@ofmt\@firstofone
 % avoid brackets around text for \cite:
 \def\@biblabel#1{}
 \def\@cite#1#2{{#1\if@tempswa , #2\fi}}
\makeatother
\newlength{\cslhangindent}
\setlength{\cslhangindent}{1.5em}
\newlength{\csllabelwidth}
\setlength{\csllabelwidth}{3em}
\newenvironment{CSLReferences}[2] % #1 hanging-indent, #2 entry-spacing
 {\begin{list}{}{%
  \setlength{\itemindent}{0pt}
  \setlength{\leftmargin}{0pt}
  \setlength{\parsep}{0pt}
  % turn on hanging indent if param 1 is 1
  \ifodd #1
   \setlength{\leftmargin}{\cslhangindent}
   \setlength{\itemindent}{-1\cslhangindent}
  \fi
  % set entry spacing
  \setlength{\itemsep}{#2\baselineskip}}}
 {\end{list}}
\usepackage{calc}
\newcommand{\CSLBlock}[1]{\hfill\break#1\hfill\break}
\newcommand{\CSLLeftMargin}[1]{\parbox[t]{\csllabelwidth}{\strut#1\strut}}
\newcommand{\CSLRightInline}[1]{\parbox[t]{\linewidth - \csllabelwidth}{\strut#1\strut}}
\newcommand{\CSLIndent}[1]{\hspace{\cslhangindent}#1}
\ifLuaTeX
\usepackage[bidi=basic]{babel}
\else
\usepackage[bidi=default]{babel}
\fi
\babelprovide[main,import]{british}
% get rid of language-specific shorthands (see #6817):
\let\LanguageShortHands\languageshorthands
\def\languageshorthands#1{}
% $HOME/.pandoc/defaults/latex-header-includes.tex
% Common header includes for both lualatex and xelatex engines.
%
% Preliminaries
%
% \PassOptionsToPackage{rgb,dvipsnames,svgnames}{xcolor}
% \PassOptionsToPackage{main=british}{babel}
\PassOptionsToPackage{english}{selnolig}
\AtBeginEnvironment{quote}{\small}
\AtBeginEnvironment{quotation}{\small}
\AtBeginEnvironment{longtable}{\centering}
%
% Packages that are useful to include
%
\usepackage{graphicx}
\usepackage{subcaption}
\usepackage[inkscapeversion=1]{svg}
\usepackage[defaultlines=4,all]{nowidow}
\usepackage{etoolbox}
\usepackage{fontsize}
\usepackage{newunicodechar}
\usepackage{pdflscape}
\usepackage{fnpct}
\usepackage{parskip}
  \setlength{\parindent}{0pt}
\usepackage[style=american]{csquotes}
% \usepackage{setspace} Use the <fontname-plus.tex> files for setspace
%
\usepackage{hyperref} % cleveref must come AFTER hyperref
\usepackage[capitalize,noabbrev]{cleveref} % Must come after hyperref
\let\longdivision\relax
\usepackage{longdivision}
% noto-plus.tex
% Font-setting header file for use with Pandoc Markdown
% to generate PDF via LuaLaTeX.
% The main font is Noto Serif.
% Other main fonts are also available in appropriately named file.
\usepackage{fontspec}
\usepackage{setspace}
\setstretch{1.3}
%
\defaultfontfeatures{Ligatures=TeX,Scale=MatchLowercase,Renderer=Node} % at the start always
%
% For English
% See also https://tex.stackexchange.com/questions/574047/lualatex-amsthm-polyglossia-charissil-error
% We use Node as Renderer for the Latin Font and Greek Font and HarfBuzz as renderer ofr Indic fonts.
%
\babelfont{rm}[Script=Latin,Scale=1]{NotoSerif}% Config is at $HOME/texmf/tex/latex/NotoSerif.fontspec
\babelfont{sf}[Script=Latin]{SourceSansPro}% Config is at $HOME/texmf/tex/latex/SourceSansPro.fontspec
\babelfont{tt}[Script=Latin]{FiraMono}% Config is at $HOME/texmf/tex/latex/FiraMono.fontspec
%
% Sanskrit, Tamil, and Greek fonts
%
\babelprovide[import, onchar=ids fonts]{sanskrit}
\babelprovide[import, onchar=ids fonts]{tamil}
\babelprovide[import, onchar=ids fonts]{greek}
%
\babelfont[sanskrit]{rm}[Scale=1.1,Renderer=HarfBuzz,Script=Devanagari]{NotoSerifDevanagari}
\babelfont[sanskrit]{sf}[Scale=1.1,Renderer=HarfBuzz,Script=Devanagari]{NotoSansDevanagari}
\babelfont[tamil]{rm}[Renderer=HarfBuzz,Script=Tamil]{NotoSerifTamil}
\babelfont[tamil]{sf}[Renderer=HarfBuzz,Script=Tamil]{NotoSansTamil}
\babelfont[greek]{rm}[Script=Greek]{GentiumBookPlus}
%
% Math font
%
\usepackage{unicode-math} % seems not to hurt % fallabck
\setmathfont[bold-style=TeX]{STIX Two Math}
\usepackage{amsmath}
\usepackage{esdiff} % for derivative symbols
% \renewcommand{\mathbf}{\symbf}
%
%
% Other fonts
%
\newfontfamily{\emojifont}{Symbola}
%

\usepackage{titling}
\usepackage{fancyhdr}
    \pagestyle{fancy}
    \fancyhead{}
    \fancyfoot{}
    \renewcommand{\headrulewidth}{0.2pt}
    \renewcommand{\footrulewidth}{0.2pt}
    \fancyhead[LO,RE]{\scshape\thetitle}
    \fancyfoot[CO,CE]{\footnotesize Copyright © 2006\textendash\the\year, R (Chandra) Chandrasekhar}
    \fancyfoot[RE,RO]{\thepage}
%
\usepackage{newunicodechar}
\newunicodechar{√}{\textsf{√}}
\usepackage {caption}
    \captionsetup{font={sf,stretch=1.4}}
\ifLuaTeX
  \usepackage{selnolig}  % disable illegal ligatures
\fi
\IfFileExists{bookmark.sty}{\usepackage{bookmark}}{\usepackage{hyperref}}
\IfFileExists{xurl.sty}{\usepackage{xurl}}{} % add URL line breaks if available
\urlstyle{sf}
\hypersetup{
  pdftitle={The Wonder That Is Pi},
  pdfauthor={R (Chandra) Chandrasekhar},
  pdflang={en-GB},
  colorlinks=true,
  linkcolor={DarkGreen},
  filecolor={Purple},
  citecolor={Teal},
  urlcolor={Maroon},
  pdfcreator={LaTeX via pandoc}}

\title{The Wonder That Is Pi}
\author{R (Chandra) Chandrasekhar}
\date{2004-01-14 | 2024-07-25}

\begin{document}
\maketitle

\thispagestyle{empty}


This is a sequel to the blog
\href{https://swanlotus.netlify.app/blogs/the-pi-of-archimedes}{``The Pi
of Archimedes''}. Here, we look at π as a number---without explicit
reference to its geometric tethering---and explore its remarkable
ubiquity in mathematics. As an appetizer, see \cref{fig:pi-equations},
where the symbol for Pi is surmounted by two very disparate equations
defining it. How in all the world could these two different-looking
equations be true? But they are indeed!

\begin{figure}
\centering
\includesvg[width=0.6\textwidth,height=\textheight]{images/pi-equations.svg}
\caption{Pi expressed by two very different
equations.}\label{fig:pi-equations}
\end{figure}

\subsection{\texorpdfstring{The Number
\href{https://www.thefreedictionary.com/menagerie}{Menagerie}}{The Number Menagerie}}\label{the-number-menagerie}

Numbers may be compared to animals in a zoo. Each is different, and yet
they all share some things in common. The variety and diversity of zoo
animals can be challenging. That is why the big cats are grouped
together, the herbivores live in another part of the zoo, etc.

Numbers, like animals, have evolved over many centuries into what I call
the \emph{number menagerie}. A very elementary picture of this zoo is
outlined in my blog
\href{https://swanlotus.netlify.app/blogs/the-two-most-important-numbers-zero-and-one}{``The
Two Most Important Numbers: Zero and One''} in case you need to review
some definitions.

To appreciate \(\pi\) as a number, we need to be aware of the
\hyperref[taxonomy]{taxonomy} in the zoo of numbers. It turns out that
\(\pi\) is a real number that is transcendental and therefore also
irrational. Let us pause for a while to better understand what this
means.

\subsubsection{Real and Complex Numbers}\label{real-and-complex-numbers}

Broadly speaking, there are two classes of numbers: real numbers,
denoted by the set \(\mathbb{R}\), and complex numbers, denoted by the
set \(\mathbb{C}\). The difference between the two is that while a real
number is a single number, a complex number is a pair, composed of two
real numbers, conjoined by the
\href{https://en.wikipedia.org/wiki/Imaginary_unit}{imaginary unit}
\(i\), where \(i^2 = -1\). In set-theoretic notation, we write
\[\mathbb{C} = \{a + ib: a, b \in \mathbb{R}\}.
\]

What then are the reals? We will leave that question aside for the
moment, and look instead into what undivided or whole numbers are.

\subsubsection{The Integers and Friends}\label{the-integers-and-friends}

The set \(\mathbb{N}\) of \emph{natural or counting numbers} is defined
as \[
\mathbb{N} = \{1, 2, 3, \dots, n, n+1, \dots\}.
\] It is a \emph{countably infinite} set whose members begin with \(1\)
and progress by the addition of \(1\) to the predecessor. It goes on
without end.

Zero is not a natural number and is assigned its own, unnamed set,
\(\{0\}\).\footnote{Some folks include zero in \(\mathbb{N}\).}

The set of \emph{integers} \(\mathbb{Z}\) includes the negative numbers,
zero, and the positive numbers: \[
\mathbb{Z} = \{\dots -3, -2, -1, 0, 1, 2, 3, \dots\}
\] Like \(\mathbb{N}\), \(\mathbb{Z}\) is also a countably infinite set.

\subsubsection{Rational and Irrational
Numbers}\label{rational-and-irrational-numbers}

Every real number is \emph{either} rational or irrational. If the
universe of discourse is the real number set, the rational and
irrational numbers are complements of each other. In other words, the
\emph{union} of the set of rational numbers and the set of irrational
numbers \emph{is} the set of real numbers.

\subsubsection{Rational Numbers}\label{rational-numbers}

The \emph{rational numbers} are denoted by the set \(\mathbb{Q}\)
defined to be: \[
\mathbb{Q} = \{\tfrac{a}{b} \mbox{ where } a, b \in \mathbb{Z} \mbox{ and } b \neq 0\}.
\] The condition imposed on \(b\) arises from the stricture that
division by zero is not permitted among the integers and
reals.\footnote{See
  \href{https://swanlotus.netlify.app/blogs/the-two-most-important-numbers-zero-and-one}{``The
  Two Most Important Numbers: Zero and One''} for the reason why.}

Let us amplify the consequences of these definitions. Is the number
\(25\) rational? Yes, indeed. But where is the denominator? It is
\emph{implicit} and equals \(1\). The fact that \[
25 = \frac{25}{1}
\] makes it clear that \(25\) is a rational number. Every integer is a
rational number.

And it is obvious from the definition that \(\frac{2}{3}\) is a rational
number. But is \(-\frac{11}{16}\) a rational number? Yes, indeed,
because the definition depends upon the \emph{integer} \(a\) and the
\emph{non-zero integer} \(b\), where both integers, being drawn from
\(\mathbb{Z}\), can be signed.

When a rational number is expressed as a decimal, that decimal can
either terminate or recur without end.

For example, the fraction \(\frac{1}{3} = 0.\overline{3}\) has a
recurring decimal representation as revealed by division. The line on
top indicates the portion of the decimal which recurs---in this case, it
is the single digit \(3\).

When we look at the fraction \(\frac{1}{2} = 0.5\), we have an example
of a terminating decimal. We could, however, pad zeros after the first
decimal place, and claim that even a terminating decimal is recurring;
witness that \(\frac{1}{2} = 0.5 = 0.5000 \dots = 0.5\overline{0}\). But
that is not the whole story.

We can further show that: \[
\frac{1}{2} = 0.5 = 0.5\overline{0} = 0.4\overline{9}.
\] It does seem strange to claim that two different decimals can express
the \emph{same} rational number \(\frac{1}{2}\).

To see why, let us rewrite \(0.4\overline{9}\) as
\begin{equation}\phantomsection\label{eq:recur}{
\begin{aligned}
0.4\overline{9} = 0.4999\dots &= \frac{4}{10} + \frac{9}{100} + \frac{9}{1000} + \frac{9}{10000}\dots\\
&= \frac{4}{10} + 9\left[ \frac{1}{100} + \frac{1}{1000} + \frac{1}{10000} \dots\right]\\
\end{aligned}
}\end{equation} Consider now the expression in square brackets on the
right hand side (RHS) of \cref{eq:recur}. We can recognize it as a
\href{https://mathworld.wolfram.com/GeometricSeries.html}{geometric
series} with first term \(a = \frac{1}{100}\) and common ratio
\(r = \frac{1}{10}\). Since \(r < 1\), the series is \emph{convergent}
and its
\href{https://senecalearning.com/en-GB/revision-notes/a-level/maths/edexcel/pure-maths/4-2-9-sum-to-infinity-of-a-geometric-series}{sum
to infinity} {[}\citeproc{ref-seneca}{1}{]} is given by:
\begin{equation}\phantomsection\label{eq:sum}{
\begin{aligned}
\frac{a}{1 - r} &= \frac{\frac{1}{100}}{[1 - \frac{1}{10}]}\\
&= \frac{\left[\frac{1}{100}\right]}{\left[\frac{9}{10}\right]}\\
&= \left[\tfrac{1}{100}\right] \left[\tfrac{10}{9}\right]\\
& = \tfrac{1}{90}.
\end{aligned}
}\end{equation} Substituting for the terms in square brackets in
\cref{eq:recur}, we get \[
0.4\overline{9} = \frac{4}{10} + 9\left[\frac{1}{90}\right] = \frac{4}{10} + \frac{1}{10} = \frac{5}{10} = \frac{1}{2}.
\] Even if it seems counter-intuitive that
\(0.4\overline{9} = 0.5 = 0.5\overline{0} = \frac{1}{2}\), it is
mathematically consistent and correct. One may therefore hazard a guess,
and correctly so, that \emph{every rational number may be expressed as a
recurring decimal}.\footnote{In this case either the digit \(9\) or the
  digit \(0\) recurs.}

Infinite sums have this property of upending our ``intuition'' about
what is correct. So, we have to be extra careful when dealing with the
value of a limit as some variable goes to infinity. Moreover, infinity,
represented by \(\infty\) is \emph{not} a number and cannot be treated
as one. It is simply a convenient shorthand symbol. This caveat should
be kept in mind when we encounter infinite sums involving \(\pi\), as
shown for example, in \cref{fig:pi-equations}.

\subsubsection{Irrational Numbers}\label{irrational-numbers}

Irrational numbers are numbers which are \emph{not rational}. The
discovery that \(\sqrt{2}\)---which is the length of the diagonal of a
unit square---was not rational, caused the first ripples of disquiet in
the ancient mathematical world, because it upset the prevailing
philosophy that whole numbers alone ruled the world.

An irrational number like \(\sqrt{2}\) does not have any recurring
sequence of digits when expressed as a decimal. But the absence of
recurring sequences in the decimal representation of a number should not
solely be used to identify a number as irrational, because some
rationals with large denominators can and do have very long recurring
sequences, which may be difficult to detect by visual inspection. For
example, \(\frac{8119}{5741}\), which incidentally is a rational
approximation to \(\sqrt{2}\), has a recurring sequence of length
\(5740\).\footnote{Also called the \emph{period} of a repeating decimal.
  See \url{https://www.wolframalpha.com/input?i=8119\%2F5741}.}

\subsubsection{The irrationals exceed in number the
rationals}\label{the-irrationals-exceed-in-number-the-rationals}

The irrationals exceed in number the rational numbers
{[}\citeproc{ref-socratic}{2}{]}. This fact is stated baldly here
because going into the whys and wherefores of this claim will lead us
too far astray from our focus in this blog.

\subsubsection{Algebraic and Transcendental
Numbers}\label{algebraic-and-transcendental-numbers}

Another dichotomy that may be applied to the real numbers is to
segregate them into two classes of numbers: the algebraic numbers and
the transcendental numbers. There is one complication: both algebraic
and transcendental numbers may be complex. But if we restrict our
universe to the real numbers, then these two sets are disjoint, i.e.,
they do not overlap.

An algebraic number is the root of a non-zero polynomial with integer or
rational coefficients. Things have gotten abstract enough thus far for
eyes to be glazed. So, let us invoke some examples to revive attention.

We will steer clear of transcendental numbers that are complex because
we don't want to get dizzy right at the beginning \emojifont {😉}
\normalfont.

Real Rational Integer Irrational Algebraic Transcendental Complex etc.

Hark back to

Venn diagram showing the number taxonomy can be challenging
\href{https://mathmonks.com/transcendental-numbers}{Venn diagram is
shown here}. This Venn diagram is flawed.

\subsubsection{Taxonomy}\label{taxonomy}

\begin{enumerate}
\item
  The real numbers are a union of the algebraic and transcendental
  numbers.
\item
  The algebraic numbers can be either rational or irrational.
\item
  All rational numbers are algebraic.
\item
  No rational number is transcendental.
\item
  All real transcendental numbers are irrational.
\item
  The irrational numbers contain \emph{all} transcendental numbers ad a
  subset of the algebraic numbers.
\end{enumerate}

\subsection{Madhava-Gregory series}\label{madhava-gregory-series}

What does the equality sign mean for an infinite sum? How can a sum of
rationals equal and irrational number? A transcendental number? The
meaning of \(=\) is therefore not in the sense of \(2 + 2 = 4\) or even
of \(x - 5 = 0\). It is something similar and yet something different,
because, despite Cantor, we have not tamed the idea of infinity yet.

\subsection{Acknowledgements}\label{acknowledgements}

\subsection{Feedback}\label{feedback}

Please \href{mailto:feedback.swanlotus@gmail.com}{email me} your
comments and corrections.

https://math.stackexchange.com/questions/4675933/is-the-equal-symbol-in-an-infinite-series-misleading-notation

https://mathmonks.com/transcendental-numbers

https://gfredericks.com/blog/

https://www.quora.com/How-do-you-draw-a-Venn-diagram-showing-the-relationship-of-the-set-of-real-rational-irrational-integers-and-non-integer-numbers

https://www.reddit.com/r/math/comments/725nxu/how\_would\_you\_improve\_this\_types\_of\_numbers\_venn/

https://i.pinimg.com/736x/57/db/7f/57db7fb6dd9a4f2649b0d8ae5689ff98--math-teacher-math-class.jpg

https://study.com/skill/learn/how-to-construct-a-venn-diagram-to-classify-real-numbers-explanation.html

\section*{References}\label{bibliography}
\addcontentsline{toc}{section}{References}

\phantomsection\label{refs}
\begin{CSLReferences}{0}{0}
\bibitem[\citeproctext]{ref-seneca}
\CSLLeftMargin{{[}1{]} }%
\CSLRightInline{{Sum to Infinity of a Geometric Series---Maths: Edexcel
A Level Pure Maths}. Retrieved 28 July 2024 from
\url{https://senecalearning.com/en-GB/revision-notes/a-level/maths/edexcel/pure-maths/4-2-9-sum-to-infinity-of-a-geometric-series}}

\bibitem[\citeproctext]{ref-socratic}
\CSLLeftMargin{{[}2{]} }%
\CSLRightInline{George C. 2017. {Are there more rational numbers than
irrational numbers? \textbar{} Socratic}. Retrieved 28 July 2024 from
\url{https://socratic.org/questions/58c80a37b72cff29df40c794}}

\end{CSLReferences}



\end{document}
