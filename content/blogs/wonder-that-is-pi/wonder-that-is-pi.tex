% Options for packages loaded elsewhere
\PassOptionsToPackage{unicode,linktoc=all}{hyperref}
\PassOptionsToPackage{hyphens}{url}
\PassOptionsToPackage{dvipsnames,svgnames,x11names}{xcolor}
%
\documentclass[
  a4paper,
]{article}
\usepackage{amsmath,amssymb}
\usepackage{iftex}
\ifPDFTeX
  \usepackage[T1]{fontenc}
  \usepackage[utf8]{inputenc}
  \usepackage{textcomp} % provide euro and other symbols
\else % if luatex or xetex
  \usepackage{unicode-math} % this also loads fontspec
  \defaultfontfeatures{Scale=MatchLowercase}
  \defaultfontfeatures[\rmfamily]{Ligatures=TeX,Scale=1}
\fi
\usepackage{lmodern}
\ifPDFTeX\else
  % xetex/luatex font selection
\fi
% Use upquote if available, for straight quotes in verbatim environments
\IfFileExists{upquote.sty}{\usepackage{upquote}}{}
\IfFileExists{microtype.sty}{% use microtype if available
  \usepackage[]{microtype}
  \UseMicrotypeSet[protrusion]{basicmath} % disable protrusion for tt fonts
}{}
\makeatletter
\@ifundefined{KOMAClassName}{% if non-KOMA class
  \IfFileExists{parskip.sty}{%
    \usepackage{parskip}
  }{% else
    \setlength{\parindent}{0pt}
    \setlength{\parskip}{6pt plus 2pt minus 1pt}}
}{% if KOMA class
  \KOMAoptions{parskip=half}}
\makeatother
\usepackage{xcolor}
\usepackage[margin=25mm]{geometry}
\usepackage{longtable,booktabs,array}
\usepackage{calc} % for calculating minipage widths
% Correct order of tables after \paragraph or \subparagraph
\usepackage{etoolbox}
\makeatletter
\patchcmd\longtable{\par}{\if@noskipsec\mbox{}\fi\par}{}{}
\makeatother
% Allow footnotes in longtable head/foot
\IfFileExists{footnotehyper.sty}{\usepackage{footnotehyper}}{\usepackage{footnote}}
\makesavenoteenv{longtable}
\usepackage{graphicx}
\makeatletter
\def\maxwidth{\ifdim\Gin@nat@width>\linewidth\linewidth\else\Gin@nat@width\fi}
\def\maxheight{\ifdim\Gin@nat@height>\textheight\textheight\else\Gin@nat@height\fi}
\makeatother
% Scale images if necessary, so that they will not overflow the page
% margins by default, and it is still possible to overwrite the defaults
% using explicit options in \includegraphics[width, height, ...]{}
\setkeys{Gin}{width=\maxwidth,height=\maxheight,keepaspectratio}
% Set default figure placement to htbp
\makeatletter
\def\fps@figure{htbp}
\makeatother
\usepackage{svg}
\setlength{\emergencystretch}{3em} % prevent overfull lines
\providecommand{\tightlist}{%
  \setlength{\itemsep}{0pt}\setlength{\parskip}{0pt}}
\setcounter{secnumdepth}{-\maxdimen} % remove section numbering
% definitions for citeproc citations
\NewDocumentCommand\citeproctext{}{}
\NewDocumentCommand\citeproc{mm}{%
  \begingroup\def\citeproctext{#2}\cite{#1}\endgroup}
\makeatletter
 % allow citations to break across lines
 \let\@cite@ofmt\@firstofone
 % avoid brackets around text for \cite:
 \def\@biblabel#1{}
 \def\@cite#1#2{{#1\if@tempswa , #2\fi}}
\makeatother
\newlength{\cslhangindent}
\setlength{\cslhangindent}{1.5em}
\newlength{\csllabelwidth}
\setlength{\csllabelwidth}{3em}
\newenvironment{CSLReferences}[2] % #1 hanging-indent, #2 entry-spacing
 {\begin{list}{}{%
  \setlength{\itemindent}{0pt}
  \setlength{\leftmargin}{0pt}
  \setlength{\parsep}{0pt}
  % turn on hanging indent if param 1 is 1
  \ifodd #1
   \setlength{\leftmargin}{\cslhangindent}
   \setlength{\itemindent}{-1\cslhangindent}
  \fi
  % set entry spacing
  \setlength{\itemsep}{#2\baselineskip}}}
 {\end{list}}
\usepackage{calc}
\newcommand{\CSLBlock}[1]{\hfill\break#1\hfill\break}
\newcommand{\CSLLeftMargin}[1]{\parbox[t]{\csllabelwidth}{\strut#1\strut}}
\newcommand{\CSLRightInline}[1]{\parbox[t]{\linewidth - \csllabelwidth}{\strut#1\strut}}
\newcommand{\CSLIndent}[1]{\hspace{\cslhangindent}#1}
\ifLuaTeX
\usepackage[bidi=basic]{babel}
\else
\usepackage[bidi=default]{babel}
\fi
\babelprovide[main,import]{british}
% get rid of language-specific shorthands (see #6817):
\let\LanguageShortHands\languageshorthands
\def\languageshorthands#1{}
% $HOME/.pandoc/defaults/latex-header-includes.tex
% Common header includes for both lualatex and xelatex engines.
%
% Preliminaries
%
% \PassOptionsToPackage{rgb,dvipsnames,svgnames}{xcolor}
% \PassOptionsToPackage{main=british}{babel}
\PassOptionsToPackage{english}{selnolig}
\AtBeginEnvironment{quote}{\small}
\AtBeginEnvironment{quotation}{\small}
\AtBeginEnvironment{longtable}{\centering}
%
% Packages that are useful to include
%
\usepackage{graphicx}
\usepackage{subcaption}
\usepackage[inkscapeversion=1]{svg}
\usepackage[defaultlines=4,all]{nowidow}
\usepackage{etoolbox}
\usepackage{fontsize}
\usepackage{newunicodechar}
\usepackage{pdflscape}
\usepackage{fnpct}
\usepackage{parskip}
  \setlength{\parindent}{0pt}
\usepackage[style=american]{csquotes}
% \usepackage{setspace} Use the <fontname-plus.tex> files for setspace
%
\usepackage{hyperref} % cleveref must come AFTER hyperref
\usepackage[capitalize,noabbrev]{cleveref} % Must come after hyperref
\let\longdivision\relax
\usepackage{longdivision}
% noto-plus.tex
% Font-setting header file for use with Pandoc Markdown
% to generate PDF via LuaLaTeX.
% The main font is Noto Serif.
% Other main fonts are also available in appropriately named file.
\usepackage{fontspec}
\usepackage{setspace}
\setstretch{1.3}
%
\defaultfontfeatures{Ligatures=TeX,Scale=MatchLowercase,Renderer=Node} % at the start always
%
% For English
% See also https://tex.stackexchange.com/questions/574047/lualatex-amsthm-polyglossia-charissil-error
% We use Node as Renderer for the Latin Font and Greek Font and HarfBuzz as renderer ofr Indic fonts.
%
\babelfont{rm}[Script=Latin,Scale=1]{NotoSerif}% Config is at $HOME/texmf/tex/latex/NotoSerif.fontspec
\babelfont{sf}[Script=Latin]{SourceSansPro}% Config is at $HOME/texmf/tex/latex/SourceSansPro.fontspec
\babelfont{tt}[Script=Latin]{FiraMono}% Config is at $HOME/texmf/tex/latex/FiraMono.fontspec
%
% Sanskrit, Tamil, and Greek fonts
%
\babelprovide[import, onchar=ids fonts]{sanskrit}
\babelprovide[import, onchar=ids fonts]{tamil}
\babelprovide[import, onchar=ids fonts]{greek}
%
\babelfont[sanskrit]{rm}[Scale=1.1,Renderer=HarfBuzz,Script=Devanagari]{NotoSerifDevanagari}
\babelfont[sanskrit]{sf}[Scale=1.1,Renderer=HarfBuzz,Script=Devanagari]{NotoSansDevanagari}
\babelfont[tamil]{rm}[Renderer=HarfBuzz,Script=Tamil]{NotoSerifTamil}
\babelfont[tamil]{sf}[Renderer=HarfBuzz,Script=Tamil]{NotoSansTamil}
\babelfont[greek]{rm}[Script=Greek]{GentiumBookPlus}
%
% Math font
%
\usepackage{unicode-math} % seems not to hurt % fallabck
\setmathfont[bold-style=TeX]{STIX Two Math}
\usepackage{amsmath}
\usepackage{esdiff} % for derivative symbols
% \renewcommand{\mathbf}{\symbf}
%
%
% Other fonts
%
\newfontfamily{\emojifont}{Symbola}
%

\usepackage{titling}
\usepackage{fancyhdr}
    \pagestyle{fancy}
    \fancyhead{}
    \fancyfoot{}
    \renewcommand{\headrulewidth}{0.2pt}
    \renewcommand{\footrulewidth}{0.2pt}
    \fancyhead[LO,RE]{\scshape\thetitle}
    \fancyfoot[CO,CE]{\footnotesize Copyright © 2006\textendash\the\year, R (Chandra) Chandrasekhar}
    \fancyfoot[RE,RO]{\thepage}
%
\usepackage{newunicodechar}
\newunicodechar{√}{\textsf{√}}
\usepackage {caption}
    \captionsetup{font={sf,stretch=1.4}}
\ifLuaTeX
  \usepackage{selnolig}  % disable illegal ligatures
\fi
\IfFileExists{bookmark.sty}{\usepackage{bookmark}}{\usepackage{hyperref}}
\IfFileExists{xurl.sty}{\usepackage{xurl}}{} % add URL line breaks if available
\urlstyle{sf}
\hypersetup{
  pdftitle={The Wonder That Is Pi},
  pdfauthor={R (Chandra) Chandrasekhar},
  pdflang={en-GB},
  colorlinks=true,
  linkcolor={DarkGreen},
  filecolor={Purple},
  citecolor={Teal},
  urlcolor={Maroon},
  pdfcreator={LaTeX via pandoc}}

\title{The Wonder That Is Pi}
\author{R (Chandra) Chandrasekhar}
\date{2004-01-14 | 2024-07-25}

\begin{document}
\maketitle

\thispagestyle{empty}


This is a sequel to the blog
\href{https://swanlotus.netlify.app/blogs/the-pi-of-archimedes}{``The Pi
of Archimedes''}. Here, we look at π as a number---without explicit
reference to its geometric tethering---and explore its remarkable
ubiquity in mathematics. As an appetizer, see \cref{fig:pi-equations},
where the symbol for Pi is surmounted by two very disparate equations
defining it. How in all the world could these two different-looking
equations be true? But they are indeed!

\begin{figure}
\centering
\includesvg[width=0.6\textwidth,height=\textheight]{images/pi-equations.svg}
\caption{Pi expressed by two very different equations. Note that both
are sums to infinity of expressions involving
integers.}\label{fig:pi-equations}
\end{figure}

\subsection{\texorpdfstring{The Number
\href{https://www.thefreedictionary.com/menagerie}{Menagerie}}{The Number Menagerie}}\label{the-number-menagerie}

Numbers may be compared to animals in a zoo. Each is different, and yet
they all share some attributes in common. The variety and diversity of
zoo animals can be challenging. That is why the big cats are grouped
together, the herbivores live in another part of the zoo, etc.

Numbers, like animals, have evolved over many centuries into what I call
the \emph{number menagerie}. A very elementary picture of this zoo is
outlined in my blog
\href{https://swanlotus.netlify.app/blogs/the-two-most-important-numbers-zero-and-one}{``The
Two Most Important Numbers: Zero and One''} in case you need to review
some definitions.

To appreciate \(\pi\) as a number, we need to be aware of the
\href{https://www.britannica.com/science/taxonomy}{taxonomy} in the zoo
of numbers. It turns out that \emph{\(\pi\) is a real number that is
transcendental and therefore also irrational}. Let us make a short
detour to better understand what this means.

\subsubsection{Real and Complex Numbers}\label{real-and-complex-numbers}

There are two major sets of numbers: real numbers, denoted by the set
\(\mathbb{R}\), and complex numbers, denoted by the set \(\mathbb{C}\).
The difference between the two is that while a real number is a single
number, a complex number is a pair, composed of two real numbers,
conjoined by the
\href{https://en.wikipedia.org/wiki/Imaginary_unit}{imaginary unit}
\(i\), where \(i^2 = -1\). In set-theoretic notation, we write \[
\mathbb{C} = \{a + bi: a, b \in \mathbb{R}\}.
\] Sometimes, the complex number \(a + bi\) is written as the ordered
pair \((a, b)\), provided the context is clear.

What then are the reals? The real numbers are the union of the set of
rational numbers and the irrational numbers. Alternatively, the reals
are the union of the algebraic numbers and the transcendental
numbers.\footnote{Since both algebraic and transcendental numbers can be
  complex, we need the added condition that these do not involve the
  imaginary unit, \(i\). For example,
  \((1 + \frac{\sqrt{(-7)}}{2}) = (1 + \frac{\sqrt{7}}{2}i)\), and
  \(\pi i\) are examples of algebraic and transcendental numbers
  respectively that involve \(i\).}

We will define each of these terms below and how they relate to one
another. As always, we start with the known and proceed to the unknown.

\subsubsection{The Integers and Friends}\label{the-integers-and-friends}

The set \(\mathbb{N}\) of \emph{natural or counting numbers} is defined
as \[
\mathbb{N} = \{1, 2, 3, \dots, n, n+1, \dots\}.
\] It is a \href{https://en.wikipedia.org/wiki/Countable_set}{countably
infinite} set whose members begin with \(1\) and progress by the
addition of \(1\) to the predecessor. It is an infinite set, which means
it never ends, as denoted by the
\href{https://www.grammarly.com/blog/ellipsis/}{ellipsis} or dots at the
end of the definition.

Zero is not a natural number and is assigned its own, unnamed set,
\(\{0\}\).\footnote{Some folks include zero in \(\mathbb{N}\).}

The set of \emph{integers} \(\mathbb{Z}\) includes the negative numbers,
zero, and the positive numbers: \[
\mathbb{Z} = \{\ldots -3, -2, -1, 0, 1, 2, 3, \dots\}
\] Like \(\mathbb{N}\), \(\mathbb{Z}\) is also a countably infinite set.

\subsection{A first dichotomy}\label{a-first-dichotomy}

The real numbers may be partitioned into subsets in different ways: one
way is into the \emph{rational} and \emph{irrational} numbers.

Every real number is \emph{either} rational or irrational. If the
universe of discourse is the real number set, the rational and
irrational numbers are complements of each other. In other words, the
\emph{union} of the set of rational numbers and the set of irrational
numbers \emph{is} the set of real numbers.

\subsection{Rational Numbers}\label{rational-numbers}

The \emph{rational numbers} are denoted by the set \(\mathbb{Q}\)
defined to be: \[
\mathbb{Q} = \{\tfrac{a}{b} \mbox{ where } a, b \in \mathbb{Z} \mbox{ and } b \neq 0\}.
\] The condition imposed on \(b\) arises from the stricture that
division by zero is not permitted among the integers and
reals.\footnote{See
  \href{https://swanlotus.netlify.app/blogs/the-two-most-important-numbers-zero-and-one}{``The
  Two Most Important Numbers: Zero and One''} for the reason why.}

Let us amplify the consequences of these definitions. Is the number
\(25\) rational? Yes, indeed. But where is the denominator? It is
\emph{implicit} and equals \(1\). The fact that \[
25 = \frac{25}{1}
\] makes it clear that \(25\) is a rational number. Every integer is a
rational number.

And it is obvious from the definition that \(\frac{2}{3}\) is a rational
number. But is \(-\frac{11}{16}\) a rational number? Yes, indeed,
because the definition depends upon the \emph{integer} \(a\) and the
\emph{non-zero integer} \(b\), where both integers---being drawn from
\(\mathbb{Z}\)---can be signed.

When a rational number is expressed as a decimal, that decimal can
either terminate or recur without end.

For example, the fraction \(\frac{1}{3} = 0.\overline{3}\) has a
recurring decimal representation as revealed by division. The line on
top indicates the portion of the decimal which recurs---in this case, it
is the single digit \(3\).

When we look at the fraction \(\frac{1}{2} = 0.5\), we have an example
of a terminating decimal. We could, however, pad zeros after the first
decimal place, and claim that even a terminating decimal is recurring;
witness that \(\frac{1}{2} = 0.5 = 0.5000 \dots = 0.5\overline{0}\). But
that is not the whole story.

We can further show that: \[
\frac{1}{2} = 0.5 = 0.5\overline{0} = 0.4\overline{9}.
\] It does seem strange to claim that two different decimals can express
the \emph{same} rational number \(\frac{1}{2}\).

To see why, let us rewrite \(0.4\overline{9}\) as
\begin{equation}\phantomsection\label{eq:recur}{
\begin{aligned}
0.4\overline{9} = 0.4999\dots &= \frac{4}{10} + \frac{9}{100} + \frac{9}{1000} + \frac{9}{10000}\dots\\
&= \frac{4}{10} + 9\left[ \frac{1}{100} + \frac{1}{1000} + \frac{1}{10000} \dots\right]\\
\end{aligned}
}\end{equation} Consider now the expression in square brackets on the
right hand side (RHS) of \cref{eq:recur}. We can recognize it as a
\href{https://mathworld.wolfram.com/GeometricSeries.html}{geometric
series} with first term \(a = \frac{1}{100}\) and common ratio
\(r = \frac{1}{10}\). Since \(r < 1\), the series is \emph{convergent}
and its
\href{https://senecalearning.com/en-GB/revision-notes/a-level/maths/edexcel/pure-maths/4-2-9-sum-to-infinity-of-a-geometric-series}{sum
to infinity} {[}\citeproc{ref-seneca}{1}{]} is given by:
\begin{equation}\phantomsection\label{eq:sum}{
\begin{aligned}
\frac{a}{1 - r} &= \frac{\frac{1}{100}}{[1 - \frac{1}{10}]}\\
&= \frac{\left[\frac{1}{100}\right]}{\left[\frac{9}{10}\right]}\\
&= \left[\tfrac{1}{100}\right] \left[\tfrac{10}{9}\right]\\
& = \tfrac{1}{90}.
\end{aligned}
}\end{equation} Substituting for the terms in square brackets in
\cref{eq:recur}, we get \[
0.4\overline{9} = \frac{4}{10} + 9\left[\frac{1}{90}\right] = \frac{4}{10} + \frac{1}{10} = \frac{5}{10} = \frac{1}{2}.
\] Even if it seems counter-intuitive that
\(0.4\overline{9} = 0.5 = 0.5\overline{0} = \frac{1}{2}\), it is
mathematically consistent and correct. One may therefore hazard a guess,
and correctly so, that \emph{every rational number may be expressed as a
recurring decimal}.\footnote{In this case either the digit \(9\) or the
  digit \(0\) recurs.}

Infinite sums have this property of upending our ``intuition'' about
what is correct. So, we have to be extra careful when dealing with the
value of a limit as some variable goes to infinity. Moreover, infinity,
represented by \(\infty\) is \emph{not} a number and cannot be treated
as one. It is simply a convenient shorthand symbol. This caveat should
be kept in mind when we encounter infinite sums involving \(\pi\), as
shown for example, in \cref{fig:pi-equations}.

\subsection{Irrational Numbers}\label{irrational-numbers}

Irrational numbers are numbers which are \emph{not rational}. The
discovery that \(\sqrt{2}\)---which is the length of the diagonal of a
unit square---was not rational
{[}\citeproc{ref-HSM-SE}{2},\citeproc{ref-clegg2004}{3}{]}, caused the
first ripples of disquiet in the ancient mathematical world, because it
upset the prevailing philosophy that ratios of whole numbers alone ruled
the world.

There are many celebrated proofs that \(\sqrt{2}\) is not the ratio of
two integers and is therefore irrational
{[}\citeproc{ref-bogomolny2018}{4}{]}. Nevertheless, it took almost two
millennia for \(\sqrt{2}\) to be accepted into the fold of properly
defined numbers {[}\citeproc{ref-cepelewicz2024}{5}{]}.

An irrational number like \(\sqrt{2}\) does not have any recurring
sequence of digits when expressed as a decimal. But the absence of
recurring sequences in the decimal representation of a number should not
solely be used to identify a number as irrational, because some
rationals with large denominators can and do have very long recurring
sequences, which may be difficult to detect by visual inspection . For
example, \(\frac{8119}{5741}\)---which incidentally is a rational
approximation to \(\sqrt{2}\)---has a recurring sequence of length
\(5740\).\footnote{Also called the \emph{period} of a repeating decimal.
  See \url{https://www.wolframalpha.com/input?i=8119\%2F5741}.}

\subsubsection{The irrationals exceed in number the
rationals}\label{the-irrationals-exceed-in-number-the-rationals}

If you are curious, you might wonder which are the more numerous: the
rationals or the irrationals. You might guess that the familiar
rationals are more numerous than the obscure irrationals. But you would
be mistaken.

In fact,
\href{https://socratic.org/questions/58c80a37b72cff29df40c794}{the
irrationals far exceed in number the rational numbers}
{[}\citeproc{ref-socratic}{6}{]}. This fact is stated baldly here,
because going into the whys and wherefores of this claim will lead us
too far astray from our focus on \(\pi\). It is an interesting fact,
though, that you should stash away for future use.

\subsection{A second dichotomy}\label{a-second-dichotomy}

The real numbers may also be split another way into two mutually
exclusive sets: the \emph{algebraic numbers} and the
\emph{transcendental numbers}. Every real number is \emph{either} an
algebraic number or a transcendental number; it cannot be both.

It bears noting though, that both the algebraic and the transcendental
numbers may be complex, i.e, have an imaginary part. But in this blog,
we have restricted our universe to the real numbers. In this blog, we
will not consider algebraic or transcendental numbers that embody the
imaginary unit.

\subsection{The Algebraic Numbers}\label{the-algebraic-numbers}

An algebraic number is the root of a non-zero polynomial with integer or
rational coefficients. Things have gotten abstract enough thus far for
eyes to be glazed. So, let us invoke some examples to revive attention.

The simplest algebraic number is an integer. Let us take \(5\) as an
example. If the polynomial \(p(x) = x - 5\), its root is when
\(p(x) = 0\), i.e., when \(x - 5 = 0\). This implies \(x = 5\) and we
have shown that \(5\) is algebraic \emph{by definition}.

Note that we could have used any other polynomial with the same root,
such as \(q(x) = 2x - 10\). All we need do is find \emph{one} polynomial
whose root equals the number and we have shown that the number is
algebraic.

Likewise, the rational number \(-(\frac{2}{3})\) is the root of the
polynomial \(3x + 2\) and is therefore algebraic.

We may assert that \emph{every rational number is algebraic and
therefore not transcendental}.

But what about an irrational number like \(\sqrt{2}\)? Is it algebraic?
The polynomial \((x^2 - 2)\) has a zeros at \(\pm\sqrt{2}\), thereby
demonstrating that both \(\pm\sqrt{2}\) are algebraic.

Can an algebraic number be a complex root of a real polynomial? Let us
find the roots for the real polynomial \(x^2 - 10x +34\): \[
\begin{aligned}
x^2 - 10x + 34 &= 0\\
(x^2 -10x + 25) + 9 &= 0\\
(x - 5)^2 + 9 = 0\\
(x - 5)^2 &= -9\\
(x - 5) &= \pm3i\\
x &= 5 \pm 3i\\
\end{aligned}
\] We have just shown that an algebraic number can be a complex root of
a real polynomial. While we will not consider complex algebraic numbers
in this blog, it is useful to know that they do exist.

\subsection{The Transcendental
Numbers}\label{the-transcendental-numbers}

Numbers which are \emph{not algebraic} are assigned the rather exalted
title of transcendental numbers. Numbers like
\href{https://www.wolframalpha.com/input?i=is+pi+transcendental}{\(\pi\)},
\href{https://www.wolframalpha.com/input?i=is+e+transcendental}{\(e\)},
and
\href{https://www.wolframalpha.com/input?i=is+ln\%282\%29+transcendental}{\(\ln 2\)}
are transcendental. But proving that a particular number is
transcendental is no mean task. We will accept \(\pi\) as transcendental
if it has been
\href{https://fermatslibrary.com/s/the-transcendence-of-pi}{proved to be
so by professional mathematicians}
{[}\citeproc{ref-niven1939}{7}--\citeproc{ref-morris-jones-pearson-2022}{9}{]}.

\emph{All transcendental numbers are perforce irrational}.

Transcendental numbers can also be complex, e.g., \(e^{i}\), but we will
steer clear of that category here, because we don't want to get more
dizzy (mathematically) than we already are right now! \emojifont {😉}
\normalfont.

\subsection{Taxonomy via Tetrachotomy}\label{taxonomy-via-tetrachotomy}

We have established a
\href{https://www.collinsdictionary.com/dictionary/english/tetrachotomy}{tetrachotomy}
among the real numbers. But the four parts are not mutually exclusive.
They overlap. There are two non-overlapping dichotomies: the rationals
and irrationals as one pair, and the algebraic and transcendental
numbers as the other.

It is noteworthy that irrational numbers like \(\sqrt{2}\) and
transcendental numbers like \(\pi\) and \(e\) are denoted, not by
values, but by \emph{symbols}.

This classification of the real numbers seems to be crying out for a
Venn diagram to depict it visually. But before we do that, let us
marshal the facts we have gathered so far:

\begin{enumerate}
\item
  The real numbers are represented by the standard set \(\mathbb{R}\).
\item
  The rationals are represented by the standard set \(\mathbb{Q}\).
\item
  There is no assigned symbol for the set of irrationals. Because it is
  the set difference between the reals and the rationals, it is often
  denoted as \(\mathbb{R}\setminus\mathbb{Q}\). But this notation is
  cumbersome. So, let us define a non-standard set \(\mathbb{I}\) and
  let it stand for the irrationals:
  \(\mathbb{I} = \mathbb{R}\setminus\mathbb{Q}\).
\item
  Let us introduce the non-standard symbol \(\mathbb{A_{\mathbb{R}}}\)
  for the set of real algebraic numbers.
\item
  Let us introduce the non-standard symbol \(\mathbb{T_{\mathbb{R}}}\)
  for the set of real transcendental numbers.
\item
  The real numbers are the union of the rational and the irrational
  numbers: \(\mathbb{R} = \mathbb{Q} \cup \mathbb{I}\).
\item
  The real numbers are also the union of the algebraic and
  transcendental numbers that do not embody the imaginary unit \(i\):
  \(\mathbb{R} = \mathbb{A_{\mathbb{R}}} \cup \mathbb{T_{\mathbb{R}}}\)
\item
  Algebraic numbers can be either rational or irrational:
  \(\mathbb{A_{\mathbb{R}}} \subseteq (\mathbb{Q} \cup \mathbb{I})\).
\item
  All rational numbers are algebraic:
  \(\mathbb{Q} \subset \mathbb{A_{\mathbb{R}}}\)
\item
  No rational number is transcendental:
  \(\mathbb{Q} \cap \mathbb{T_{\mathbb{R}}} = \emptyset\)
\item
  All real transcendental numbers are irrational:
  \(\mathbb{T_{\mathbb{R}}} \subset \mathbb{I}\).
\item
  The irrational numbers contain \emph{all} transcendental numbers and a
  subset of the algebraic numbers, again excluding those that embody
  \(i\):
  \((\mathbb{T_{\mathbb{R}}} \subset \mathbb{I}) \wedge (\mathbb{A}_{\mathbb{R}} \cap \mathbb{I} \neq \emptyset)\).
\end{enumerate}

That was quite mouthful even with mathematical symbols. We are now ready
to draw the Venn diagram for the tetrachotomy of the real numbers.

And surprise! surprise! There are only \emph{three} regions in the Venn
diagram that are populated. So, taking mathematical liberties, we may
say that our tetrachotomy was not ``linearly independent''.

\begin{figure}
\centering
\includesvg[width=0.9\textwidth,height=\textheight]{images/venn-tetra.svg}
\caption{Venn diagram showing the rationals, \(\mathbb{Q}\), the
irrationals, \(\mathbb{I}\), the real algebraics,
\(\mathbb{A_{\mathbb{R}}}\), and the real transcendentals
\(\mathbb{T_{\mathbb{R}}}\). From this diagram, we may assert that
\(\mathbb{R} = \mathbb{A_{\mathbb{R}}} \cup \mathbb{I}\). Note where
\(\pi\) resides, and also that there are only \emph{three} populated
regions in the Venn diagram: \(\mathbb{Q}\),
\((\mathbb{A_{\mathbb{R}}} \cap \mathbb{I})\), and
\(\mathbb{T_{\mathbb{R}}}\).}\label{fig:venn-tetra}
\end{figure}

\subsection{Enter π}\label{enter-ux3c0}

We have gone through all this huffing and puffing to place \(\pi\)
contextually among the real numbers. Let us list its characteristics:

\begin{enumerate}
\item
  It is not a rational number, which means that it cannot be expressed
  as the ratio of two whole numbers, the denominator being non-zero.
\item
  Its decimal representation is neither finite nor does it contain a
  recurring segment, regardless of how long the decimal is.
\item
  It is also not the root to any non-zero polynomial equation whose
  coefficients are integers or rational numbers.
\item
  When Pi is used in equations, the placeholder symbol \(\pi\) is used.
\end{enumerate}

These properties have earned for \(\pi\) the rather exalted title of
transcendental number, which it shares with other pivotal numbers like
\(e\). Pi is not only important, it is also tantalizing. Pi is like a
beautiful butterfly that cannot be caught in the net of finitude. It is
like a rainbow that is beautiful to behold from afar, but can never be
reached.

One could almost say that \(\pi\) is not numerically friendly. And you
would not be too wrong. Rational approximations for \(\pi\), like
\(\frac{22}{7}\), are used in practice. And the matter would have rested
there were it not for the human quest for beauty.

The unpredictability of successive decimal places of \(\pi\) has
enchanted mathematicians and still continues to engross them. Pi has
been calculated to an unprecedented number of decimal places, and such a
quest is certainly driven, not by practical necessity, but possibly by
the need for aesthetic fulfilment.

The search for increasingly more accurate values for \(\pi\) has
resulted in many approaches to solve the problem. Geometric\footnote{See
  the
  \href{https://swanlotus.netlify.app/blogs/the-pi-of-archimedes}{``Pi
  of Archimedes''}.} and analytical approaches to estimate \(\pi\) have
both borne fruit. Interestingly, \(\pi\) may also be estimated by
repeatedly performing a random---or probabilistic---experiment, whose
precise outcome cannot be predicted, but whose average behaviour may be
estimated. Such an experiment is called a
\href{https://www.ibm.com/topics/monte-carlo-simulation}{Monte Carlo
simulation}. Thus the quest for \(\pi\) brings together the mathematical
sub-fields of geometry, analysis, and probabilistic simulation.

This quest for the unattainable---but supremely beautiful---has engaged
human minds to seek \(\pi\) in countless infinite sums, such are shown
in \cref{fig:pi-equations}. These equations are sometimes starkly simple
and at other times thoroughly mystifying, and embody the paradox that is
\(\pi\) more succinctly than all the words in the world.

\subsection{The Madhava-Gregory-Leibnitz
series}\label{the-madhava-gregory-leibnitz-series}

It must be obvious by now that trigonometry, circles, and the number
\(\pi\) are inextricably entwined.\footnote{If this sounds unfamiliar, I
  invite you to read my blogs
  \href{https://swanlotus.netlify.app/blogs/a-tale-of-two-measures-degrees-and-radians}{``A
  tale of two measures: degrees and radians''} and
  \href{https://swanlotus.netlify.app/blogs/the-pi-of-archimedes}{``The
  Pi of Archimedes''}.} The quest for more accurate values of \(\pi\)
continued to fascinate mathematicians in the centuries after Archimedes.
This time though, rather than geometric iteration, \emph{sums of
successive terms} were used to approximate \(\pi\).

For our purposes, a \emph{sequence} is an \emph{ordered} procession of
numbers, and a \emph{series} is a sum of successive terms that obey some
specific rule. If the summation stops at some particular term, we have a
\emph{partial sum}; if the summation goes on indefinitely, we have an
\emph{infinite series}. If this infinite sum approaches ever closer to a
finite value, the series is said to \emph{converge}. To see what all
this means in practice, let us look at the Madhava-Gregory-Leibniz
series.

\subsubsection{Why triple-barrelled
names}\label{why-triple-barrelled-names}

The triple-barrelled names for the series we will look at arose because
the discoveries of medieval Indian mathematicians remained unknown to
the larger world and were only recently uncovered and promulgated
{[}\citeproc{ref-roy1990}{10}--\citeproc{ref-madhava-wiki}{12}{]}.

This series was originally called the Gregory series or the
Gregory-Libnitz series, but recent scholarship has given
\href{https://en.wikipedia.org/wiki/Scientific_priority}{priority} to
Madhava of Sangamagrama for this and other infinite trigonometric series
{[}\citeproc{ref-roy1990}{10}--\citeproc{ref-madhava-wiki}{12}{]}.\footnote{It
  is interesting that Madhava used the ratio of the circle arc to the
  radius when denoting angles, thus establishing the naturalness of
  radians as angular measure. See also {[}``A tale of two measures:
  degrees and
  radians''{]}((https://swanlotus.netlify.app/blogs/a-tale-of-two-measures-degrees-and-radians).}

\href{https://en.wikipedia.org/wiki/James_Gregory_(mathematician)}{James
Gregory} was the first Professor of Mathematics at the University of
Edinburgh and in 1671, he published the series that was called the
\emph{Gregory series} or the
\href{https://en.wikipedia.org/wiki/Arctangent_series}{arctangent
series}.

\href{https://www.google.com/search?q=Leibnitz}{Gottfried Wilhelm
Leibniz} evaluated the arctangent series at \(\frac{\pi}{4}\) to get an
estimate of \(\pi\); the result was known as the Gregory-Leibnitz series
or the Leibnitz Formula.

Recent scholarship has accorded
\href{https://en.wikipedia.org/wiki/Scientific_priority}{priority} to
\href{https://en.wikipedia.org/wiki/Madhava_of_Sangamagrama}{Madhava of
Sangamagrama} who anticipated both these developments by more than 250
years {[}\citeproc{ref-roy1990}{10}--\citeproc{ref-madhava-wiki}{12}{]}.
He also developed other trigonometric series and used arc length over
radius to denote an angle, thereby using radian measure.

This series was originally called the Gregory series or the
Gregory-Libnitz series, but recent scholarship has given
\href{https://en.wikipedia.org/wiki/Scientific_priority}{priority} to
Madhava of Sangamagrama for this and other infinite trigonometric series
{[}\citeproc{ref-roy1990}{10}--\citeproc{ref-madhava-wiki}{12}{]}.\footnote{It
  is interesting that Madhava used the ratio of the circle arc to the
  radius when denoting angles, thus establishing the naturalness of
  radians as angular measure. See also {[}``A tale of two measures:
  degrees and
  radians''{]}((https://swanlotus.netlify.app/blogs/a-tale-of-two-measures-degrees-and-radians).}

James Gregory was the first Professor of Mathematics at the University
of Edinburgh and in 1671, he published the series now known by his name.
Rather than draw Gregory's formula out of a hat, we will sketch its
derivation, and show its origins in integral calculus. Gregory found
that:

\subsection{Acknowledgements}\label{acknowledgements}

Wolfram Alpha for several results.

\subsection{Feedback}\label{feedback}

Please \href{mailto:feedback.swanlotus@gmail.com}{email me} your
comments and corrections.

https://math.stackexchange.com/questions/4675933/is-the-equal-symbol-in-an-infinite-series-misleading-notation

https://mathmonks.com/transcendental-numbers

https://gfredericks.com/blog/

https://www.quora.com/How-do-you-draw-a-Venn-diagram-showing-the-relationship-of-the-set-of-real-rational-irrational-integers-and-non-integer-numbers

https://www.reddit.com/r/math/comments/725nxu/how\_would\_you\_improve\_this\_types\_of\_numbers\_venn/

https://i.pinimg.com/736x/57/db/7f/57db7fb6dd9a4f2649b0d8ae5689ff98--math-teacher-math-class.jpg

https://study.com/skill/learn/how-to-construct-a-venn-diagram-to-classify-real-numbers-explanation.html

Real Rational Integer Irrational Algebraic Transcendental Complex etc.

Hark back to

Venn diagram showing the number taxonomy can be challenging
\href{https://mathmonks.com/transcendental-numbers}{Venn diagram is
shown here}. This Venn diagram is flawed.

https://www.quora.com/Why-cant-we-draw-a-Venn-diagram-for-4-sets-with-circles-and-how-can-we-solve-it

https://www.gauthmath.com

https://tex.stackexchange.com/questions/668565/how-can-i-draw-a-shape-with-overlapping-regions-with-opacity

https://www.cut-the-knot.org/proofs/sq\_root.shtml

https://news.ycombinator.com/item?id=40750230

\section*{References}\label{bibliography}
\addcontentsline{toc}{section}{References}

\phantomsection\label{refs}
\begin{CSLReferences}{0}{0}
\bibitem[\citeproctext]{ref-seneca}
\CSLLeftMargin{{[}1{]} }%
\CSLRightInline{---. {Sum to Infinity of a Geometric Series---Maths}.
{Edexcel A Level Pure Maths}. Retrieved 28 July 2024 from
\url{https://senecalearning.com/en-GB/revision-notes/a-level/maths/edexcel/pure-maths/4-2-9-sum-to-infinity-of-a-geometric-series}}

\bibitem[\citeproctext]{ref-HSM-SE}
\CSLLeftMargin{{[}2{]} }%
\CSLRightInline{Various. 2014. {Irrationality of the square root of 2}.
{History of Science and Mathematics Stack Exchange}. Retrieved 29 July
2024 from
\url{https://hsm.stackexchange.com/questions/2/irrationality-of-the-square-root-of-2}}

\bibitem[\citeproctext]{ref-clegg2004}
\CSLLeftMargin{{[}3{]} }%
\CSLRightInline{Brian Clegg. 2004. {The Dangerous Ratio}. Retrieved 29
July 2024 from \url{https://nrich.maths.org/2671}}

\bibitem[\citeproctext]{ref-bogomolny2018}
\CSLLeftMargin{{[}4{]} }%
\CSLRightInline{Alexander Bogomolny. 2018. {Square root of 2 is
irrational}. Retrieved 29 July 2024 from
\url{https://www.cut-the-knot.org/proofs/sq_root.shtml}}

\bibitem[\citeproctext]{ref-cepelewicz2024}
\CSLLeftMargin{{[}5{]} }%
\CSLRightInline{Jordana Cepelewicz. 2024. {How the Square Root of 2
Became a Number}. {Quanta Magazine}. Retrieved 29 July 2024 from
\url{https://www.quantamagazine.org/how-the-square-root-of-2-became-a-number-20240621/}}

\bibitem[\citeproctext]{ref-socratic}
\CSLLeftMargin{{[}6{]} }%
\CSLRightInline{George C. 2017. {Are there more rational numbers than
irrational numbers?}. {Socratic}. Retrieved 28 July 2024 from
\url{https://socratic.org/questions/58c80a37b72cff29df40c794}}

\bibitem[\citeproctext]{ref-niven1939}
\CSLLeftMargin{{[}7{]} }%
\CSLRightInline{Ivan Niven. 1939. The transcendence of π. \emph{{The
American Mathematical Monthly}} 46, 8 (1939), 469--471.
DOI:https://doi.org/\href{https://doi.org/10.1080/00029890.1939.11998903}{10.1080/00029890.1939.11998903}}

\bibitem[\citeproctext]{ref-chow2024}
\CSLLeftMargin{{[}8{]} }%
\CSLRightInline{Timothy Y Chow. 2024. {A Well-Motivated Proof That Pi Is
Irrational}. Retrieved 31 July 2024 from
\url{https://arxiv.org/html/2403.20140v1}}

\bibitem[\citeproctext]{ref-morris-jones-pearson-2022}
\CSLLeftMargin{{[}9{]} }%
\CSLRightInline{Kenneth R Pearson Sidney A Morris Arthur Jones. 2022.
\emph{{Abstract Algebra and Famous Impossibilities}. {Squaring the
Circle, Doubling the Cube, Trisecting an Angle, and Solving Quintic
Equations}} (2nd ed.). Springer.}

\bibitem[\citeproctext]{ref-roy1990}
\CSLLeftMargin{{[}10{]} }%
\CSLRightInline{Ranjan Roy. 1990. {The Discovery of the Series Formula
for π by Leibniz, Gregory and Nilakantha}. \emph{{Mathematics Magazine}}
63, 5 (1990), 291--306.
DOI:https://doi.org/\href{https://doi.org/10.1080/0025570X.1990.11977541}{10.1080/0025570X.1990.11977541}}

\bibitem[\citeproctext]{ref-roy2011}
\CSLLeftMargin{{[}11{]} }%
\CSLRightInline{Ranjan Roy. 2011. {Power Series in Fifteenth-Century
Kerala}. In \emph{{Sources in the Development of Mathematics}: {Series
and Products from the Fifteenth to the Twenty-first Century}}.
{Cambridge University Press }.
DOI:https://doi.org/\href{https://doi.org/10.1017/CBO9780511844195}{10.1017/CBO9780511844195}}

\bibitem[\citeproctext]{ref-madhava-wiki}
\CSLLeftMargin{{[}12{]} }%
\CSLRightInline{Wikipedia contributors. 2024. {Madhava of Sangamagrama}.
{Wikipedia, The Free Encyclopedia}. Retrieved 31 July 2024 from
\url{https://en.wikipedia.org/wiki/Madhava_of_Sangamagrama}}

\end{CSLReferences}



\end{document}
