% Options for packages loaded elsewhere
\PassOptionsToPackage{unicode,linktoc=all}{hyperref}
\PassOptionsToPackage{hyphens}{url}
\PassOptionsToPackage{dvipsnames,svgnames,x11names}{xcolor}
%
\documentclass[
  a4paper,
]{article}
\usepackage{amsmath,amssymb}
\usepackage{iftex}
\ifPDFTeX
  \usepackage[T1]{fontenc}
  \usepackage[utf8]{inputenc}
  \usepackage{textcomp} % provide euro and other symbols
\else % if luatex or xetex
  \usepackage{unicode-math} % this also loads fontspec
  \defaultfontfeatures{Scale=MatchLowercase}
  \defaultfontfeatures[\rmfamily]{Ligatures=TeX,Scale=1}
\fi
\usepackage{lmodern}
\ifPDFTeX\else
  % xetex/luatex font selection
\fi
% Use upquote if available, for straight quotes in verbatim environments
\IfFileExists{upquote.sty}{\usepackage{upquote}}{}
\IfFileExists{microtype.sty}{% use microtype if available
  \usepackage[]{microtype}
  \UseMicrotypeSet[protrusion]{basicmath} % disable protrusion for tt fonts
}{}
\makeatletter
\@ifundefined{KOMAClassName}{% if non-KOMA class
  \IfFileExists{parskip.sty}{%
    \usepackage{parskip}
  }{% else
    \setlength{\parindent}{0pt}
    \setlength{\parskip}{6pt plus 2pt minus 1pt}}
}{% if KOMA class
  \KOMAoptions{parskip=half}}
\makeatother
\usepackage{xcolor}
\usepackage[margin=25mm]{geometry}
\usepackage{longtable,booktabs,array}
\usepackage{calc} % for calculating minipage widths
% Correct order of tables after \paragraph or \subparagraph
\usepackage{etoolbox}
\makeatletter
\patchcmd\longtable{\par}{\if@noskipsec\mbox{}\fi\par}{}{}
\makeatother
% Allow footnotes in longtable head/foot
\IfFileExists{footnotehyper.sty}{\usepackage{footnotehyper}}{\usepackage{footnote}}
\makesavenoteenv{longtable}
\usepackage{graphicx}
\makeatletter
\def\maxwidth{\ifdim\Gin@nat@width>\linewidth\linewidth\else\Gin@nat@width\fi}
\def\maxheight{\ifdim\Gin@nat@height>\textheight\textheight\else\Gin@nat@height\fi}
\makeatother
% Scale images if necessary, so that they will not overflow the page
% margins by default, and it is still possible to overwrite the defaults
% using explicit options in \includegraphics[width, height, ...]{}
\setkeys{Gin}{width=\maxwidth,height=\maxheight,keepaspectratio}
% Set default figure placement to htbp
\makeatletter
\def\fps@figure{htbp}
\makeatother
\setlength{\emergencystretch}{3em} % prevent overfull lines
\providecommand{\tightlist}{%
  \setlength{\itemsep}{0pt}\setlength{\parskip}{0pt}}
\setcounter{secnumdepth}{-\maxdimen} % remove section numbering
% definitions for citeproc citations
\NewDocumentCommand\citeproctext{}{}
\NewDocumentCommand\citeproc{mm}{%
  \begingroup\def\citeproctext{#2}\cite{#1}\endgroup}
\makeatletter
 % allow citations to break across lines
 \let\@cite@ofmt\@firstofone
 % avoid brackets around text for \cite:
 \def\@biblabel#1{}
 \def\@cite#1#2{{#1\if@tempswa , #2\fi}}
\makeatother
\newlength{\cslhangindent}
\setlength{\cslhangindent}{1.5em}
\newlength{\csllabelwidth}
\setlength{\csllabelwidth}{3em}
\newenvironment{CSLReferences}[2] % #1 hanging-indent, #2 entry-spacing
 {\begin{list}{}{%
  \setlength{\itemindent}{0pt}
  \setlength{\leftmargin}{0pt}
  \setlength{\parsep}{0pt}
  % turn on hanging indent if param 1 is 1
  \ifodd #1
   \setlength{\leftmargin}{\cslhangindent}
   \setlength{\itemindent}{-1\cslhangindent}
  \fi
  % set entry spacing
  \setlength{\itemsep}{#2\baselineskip}}}
 {\end{list}}
\usepackage{calc}
\newcommand{\CSLBlock}[1]{\hfill\break#1\hfill\break}
\newcommand{\CSLLeftMargin}[1]{\parbox[t]{\csllabelwidth}{\strut#1\strut}}
\newcommand{\CSLRightInline}[1]{\parbox[t]{\linewidth - \csllabelwidth}{\strut#1\strut}}
\newcommand{\CSLIndent}[1]{\hspace{\cslhangindent}#1}
\ifLuaTeX
\usepackage[bidi=basic]{babel}
\else
\usepackage[bidi=default]{babel}
\fi
\babelprovide[main,import]{british}
% get rid of language-specific shorthands (see #6817):
\let\LanguageShortHands\languageshorthands
\def\languageshorthands#1{}
% $HOME/.pandoc/defaults/latex-header-includes.tex
% Common header includes for both lualatex and xelatex engines.
%
% Preliminaries
%
% \PassOptionsToPackage{rgb,dvipsnames,svgnames}{xcolor}
% \PassOptionsToPackage{main=british}{babel}
\PassOptionsToPackage{english}{selnolig}
\AtBeginEnvironment{quote}{\small}
\AtBeginEnvironment{quotation}{\small}
\AtBeginEnvironment{longtable}{\centering}
%
% Packages that are useful to include
%
\usepackage{graphicx}
\usepackage{subcaption}
\usepackage[inkscapeversion=auto]{svg}
\usepackage{nowidow}
\usepackage{etoolbox}
\usepackage{fontsize}
\usepackage{newunicodechar}
\usepackage{pdflscape}
\usepackage{fnpct}
\usepackage{parskip}
  \setlength{\parindent}{0pt}
\usepackage[style=american]{csquotes}
% \usepackage{setspace} Use the <fontname-plus.tex> files for setspace
%
\usepackage{hyperref} % cleveref must come AFTER hyperref
\usepackage[capitalize,noabbrev]{cleveref} % Must come after hyperref
\let\longdivision\relax
\usepackage{longdivision}
\newcommand{\dd}{\ensuremath{mathrm d}}
%
% Assume that amsmath is already loaded via \usepackage{amsmath}
% in the standard LaTeX template foe Pandoc
% 
\DeclareMathOperator{\sech}{sech}
\DeclareMathOperator{\csch}{csch}
\DeclareMathOperator{\arcsec}{arcsec}
\DeclareMathOperator{\arccot}{arccot}
\DeclareMathOperator{\arccsc}{arccsc}
\DeclareMathOperator{\arccosh}{arccosh}
\DeclareMathOperator{\arcsinh}{arcsinh}
\DeclareMathOperator{\arctanh}{arctanh}
\DeclareMathOperator{\arcsech}{arcsech}
\DeclareMathOperator{\arccsch}{arccsch}
\DeclareMathOperator{\arccoth}{arccoth} 
% noto-plus.tex
% Font-setting header file for use with Pandoc Markdown
% to generate PDF via LuaLaTeX.
% The main font is Noto Serif.
% Other main fonts are also available in appropriately named file.
\usepackage{fontspec}
\usepackage{setspace}
\setstretch{1.3}
%
\defaultfontfeatures{Ligatures=TeX,Scale=MatchLowercase,Renderer=Node} % at the start always
%
% For English
% See also https://tex.stackexchange.com/questions/574047/lualatex-amsthm-polyglossia-charissil-error
% We use Node as Renderer for the Latin Font and Greek Font and HarfBuzz as renderer ofr Indic fonts.
%
\babelfont{rm}[Script=Latin,Scale=1]{NotoSerif}% Config is at $HOME/texmf/tex/latex/NotoSerif.fontspec
\babelfont{sf}[Script=Latin]{SourceSansPro}% Config is at $HOME/texmf/tex/latex/SourceSansPro.fontspec
\babelfont{tt}[Script=Latin]{FiraMono}% Config is at $HOME/texmf/tex/latex/FiraMono.fontspec
%
% Sanskrit, Tamil, and Greek fonts
%
\babelprovide[import, onchar=ids fonts]{sanskrit}
\babelprovide[import, onchar=ids fonts]{tamil}
\babelprovide[import, onchar=ids fonts]{greek}
%
\babelfont[sanskrit]{rm}[Scale=1.1,Renderer=HarfBuzz,Script=Devanagari]{NotoSerifDevanagari}
\babelfont[sanskrit]{sf}[Scale=1.1,Renderer=HarfBuzz,Script=Devanagari]{NotoSansDevanagari}
\babelfont[tamil]{rm}[Renderer=HarfBuzz,Script=Tamil]{NotoSerifTamil}
\babelfont[tamil]{sf}[Renderer=HarfBuzz,Script=Tamil]{NotoSansTamil}
\babelfont[greek]{rm}[Script=Greek]{GentiumBookPlus}
%
% Math font
%
\usepackage{unicode-math} % seems not to hurt % fallabck
\setmathfont[bold-style=TeX]{STIX Two Math}
\usepackage{amsmath}
\usepackage{esdiff} % for derivative symbols
% \renewcommand{\mathbf}{\symbf}
%
%
% Other fonts
%
\newfontfamily{\emojifont}{Symbola}
%

\usepackage{titling}
\usepackage{fancyhdr}
    \pagestyle{fancy}
    \fancyhead{}
    \fancyfoot{}
    \renewcommand{\headrulewidth}{0.2pt}
    \renewcommand{\footrulewidth}{0.2pt}
    \fancyhead[LO,RE]{\scshape\thetitle}
    \fancyfoot[CO,CE]{\footnotesize Copyright © 2006\textendash\the\year, R (Chandra) Chandrasekhar}
    \fancyfoot[RE,RO]{\thepage}
%
\usepackage{newunicodechar}
\newunicodechar{√}{\textsf{√}}
\usepackage {caption}
    \captionsetup{font={sf,stretch=1.4}}
\ifLuaTeX
  \usepackage{selnolig}  % disable illegal ligatures
\fi
\IfFileExists{bookmark.sty}{\usepackage{bookmark}}{\usepackage{hyperref}}
\IfFileExists{xurl.sty}{\usepackage{xurl}}{} % add URL line breaks if available
\urlstyle{sf}
\hypersetup{
  pdftitle={Some Mathematical Fundamentals},
  pdfauthor={R (Chandra) Chandrasekhar},
  pdflang={en-GB},
  colorlinks=true,
  linkcolor={DarkGreen},
  filecolor={Purple},
  citecolor={Teal},
  urlcolor={Maroon},
  pdfcreator={LaTeX via pandoc}}

\title{Some Mathematical Fundamentals}
\author{R (Chandra) Chandrasekhar}
\date{2024-02-14 | 2024-02-14}

\begin{document}
\maketitle

\thispagestyle{empty}


\subsection{An unforeseen challenge}\label{an-unforeseen-challenge}

My dear friend, Solus ``Sol'' Simkin, casually asked me one summer day
if I would write a blog demystifying the meanings and uses of four
mathematical terms: expression, equation, formula, and differential
equation. I thought he spoke in jest, and let his request lie in a dusty
corner of my mind, as a memento to his humour.

Imagine my surprise when he accosted me again after two months and asked
if I had put pen to paper to explain the four mathematical terms.

``Surely, you cannot be serious, Sol'', I said. ``Who would want to know
something as fundamental as this? With the exception of differential
equations, it should have been mostly taught by the fourth year of
elementary school mathematics.''

``You would be astounded to know how many so-called
\href{https://en.wikipedia.org/wiki/Science,_technology,_engineering,_and_mathematics}{STEM}
\emph{graduates} and \emph{postgraduates}---who have passed through the
degree mill---are ignorant of these \emph{definitions}, let alone their
\emph{purpose},'' replied Sol. ``As an added bonus, write your blog so
that it is also perfectly clear to elementary school students, going on
to middle school. It will serve as a valuable review for them.''

Incredulously, I took up his challenge, complete with its stipulations.
This blog was born after much cogitation, and is really my first attempt
at presenting and exemplifying fundamental definitions, usually taught
in elementary school\footnote{I usually find it easier to explain
  concepts to students in middle school and beyond. rather than to
  elementary school students.}. Any reader who still finds it
conceptually muddy or murky is cordially invited to
\href{mailto:feedback.swanlotus@gmail.com}{write to me}.

I have borrowed liberally from material contained in my book-manuscript
\href{https://swanlotus.netlify.app/sas-manuscript/SAS-partial.pdf}{\emph{Secrets
of Academic Success}}, henceforth referred to as \emph{SAS}. Perhaps the
earnest student will be inspired to look for clarification there as
well. \emojifont {😉} \normalfont

\subsection{Starting at the beginning}\label{starting-at-the-beginning}

My decades of muddling in matters scholastic have convinced me that
there are \emph{four} stages in all learning, as shown in
\cref{fig:four-stages}. These have been explained \emph{in extenso} in
my \emph{SAS} book, and the interested reader is directed to the first
chapter of that book {[}\citeproc{ref-sas}{1}{]} for a more substantial
discussion.

\begin{figure}
\centering
\includegraphics[width=0.9\linewidth,height=\textheight,keepaspectratio]{images/four-stages-of-learning.png}
\caption{Learning any subject involves four stages as shown above
{[}\citeproc{ref-sas}{1}{]}.}\label{fig:four-stages}
\end{figure}

All knowledge begins with \emph{naming}. You cannot analyze or
understand what you cannot name. In specialized subject areas, names are
called \emph{definitions}. In this blog, we have the following
\emph{four} mathematical names to define, understand, analyze, and
apply:

\begin{enumerate}
\tightlist
\item
  expression;
\item
  equation;
\item
  formula; and
\item
  differential equation.
\end{enumerate}

After naming, we move to \emph{knowing}. At this stage, we
systematically study the subject that has been defined to the extent
that we are familiar with it \emph{ourselves}, without recourse to a
teacher, a textbook, or other reference material.

The third stage, \emph{doing}, involves \emph{application} of the
newfound concept that has already been defined and studied. If you were
learning to fly an aircraft, you could not claim to be a pilot, based on
mere theoretical knowledge. You must practise flying---first under
supervision, and later solo---so that you accumulate enough experience
to claim competence in that art.

Once the doing stage has been mastered, it becomes effortless: this is
the \emph{being} stage of knowledge. You are now a master at what you
started out to learn. You can start teaching others.

Every subject of study---whether academic like mathematics, or practical
like surgery---involves these four steps and their mastery. By steadily
moving from one stage to another, and finally by graduating to the being
stage, you achieve mastery of your subject.

This blog is mainly concerned with the naming stage, but our discussion
will not be complete without a modicum of knowing and doing as well. Let
us set to.

\subsection{Expressions}\label{expressions}

The word
\href{https://www.etymonline.com/search?q=expression}{expression}
literally means ``(something) that is pressed out''. In the context of
mathematics, an expression is a collection of numbers or symbols that
are written out or expressed. Sometimes, the expression might seem
complicated, but it might also be amenable to simplification.

Let us start with something basic:
\begin{equation}\phantomsection\label{eq:two-plus-four}{
2 + 4
}\end{equation} It is a mathematical expression for adding four to two.
But is that not \(6\)? So, is the expression \(2 + 4\) or is it \(6\)?
The \emph{expression} itself is \emph{two plus four}. Its \emph{value}
is six.

But if we know that \(2 + 4 = 6\), why can't we say that the expression
is \(6\)? We \emph{may} if we were asked to \emph{simplify} the
expression. But the expression itself remains as it was originally
written.

Let us move up a notch. Look at:
\begin{equation}\phantomsection\label{eq:sqrt5}{
\sqrt{25}
}\end{equation} What does it mean? Now you need to know the language of
mathematics. What does \(\surd\) stand for? It is a stylized letter
``r'' for the word
\href{https://en.wikipedia.org/wiki/Square_root}{radix} which stands for
the positive square root of the number inside the symbol. What number
multiplied by itself will give us \(25\)? Well, \(5 \times 5\) equals
\(25\).

But is that all? What about \((-5) \times (-5) = 25\)? That too is
correct. So, what does \(\sqrt{25}\) really stand for? It is
\emph{defined} to be the \emph{positive} square root of \(25\) which is
\(5\).

The case of \(-5\) is catered for by the expression \(-\sqrt{25}\). We
may write \(-\sqrt{25} = -5\); thus, we do not have notational
ambiguity.

\subsubsection{Simplifying an
expression}\label{simplifying-an-expression}

In school, you might have been asked to \emph{simplify an expression}.
In that case, you are being asked to produce a result that is the same
as the original expression but is simpler in form and appearance. For
example, we could write: \begin{equation}\phantomsection\label{eq:six}{
2 + 4 = 6
}\end{equation} Look! What have we done? We have produced an
\emph{equation}. The sum of the two numbers on the left hand side (LHS)
equals the single number on the right hand side (RHS).

We will consider \emph{equations} a little later, but for now, bear in
mind, that to simplify an expression, we need to find a
\emph{mathematical alias} for it that \emph{equals} the original
expression, but is simpler in form.

\subsubsection{Enter algebra}\label{enter-algebra}

After we mature a little more mathematically, we start dealing with
numbers whose values are not known. We use \emph{letters} to denote
these unknown quantities, much like we use \emph{pronouns} instead of
\emph{proper nouns} for the names of people we do not know. Let us take
a look at a potentially confusing expression:
\begin{equation}\phantomsection\label{eq:abc}{
\frac{\frac{a}{b}}{c}
}\end{equation} What does it mean? Can it be simplified? If so, what is
its simplified form? Does it convey any meaning?

Mathematics is a language in which ambiguity is prohibited by strictly
enforced conventions. We already saw that with the \(\surd\) sign.

Does \cref{eq:abc}\footnote{It is not an equation but an expression; my
  software did not allow that degree of customization. Please excuse
  this inaccuracy.} mean more than one thing? Not if we know our
conventions. The expression consists of a value on top divided by a
value at the bottom. But the value at the top is itself a fraction:
\begin{equation}\phantomsection\label{eq:a-over-b}{
\frac{a}{b}
}\end{equation} This is now divided by the value \(c\).
\href{https://swanlotus.netlify.app/blogs/the-two-most-important-numbers-zero-and-one\#the-multiplicative-inverse-in-mathbbz-mathbbq-and-mathbbr}{We
know} that \emph{dividing} by \(c\) amounts to \emph{multiplying} by
\(\frac{1}{c}\). The expression may therefore be simplified
so\footnote{Refer to the chapter ``Arithmetic Revisited'' in the
  \emph{SAS} book {[}\citeproc{ref-sas}{1}{]} if you are still unclear
  about what follows.}:
\begin{equation}\phantomsection\label{eq:simplified}{
\begin{aligned}
\frac{\frac{a}{b}}{c} &= \frac{a}{b} \times \frac{1}{c}\\
&= \frac{a \times 1}{b \times c}\\
&= \frac{a}{bc}
\end{aligned}
}\end{equation} Note that the horizontal line separating the numerator
and the denominator is called the \emph{vinculum} and it is long enough
to cover \emph{both} \(b\) and \(c\) in the denominator.

If we did not have access to mathematical typesetting, this fraction
would be written unambiguously as \(a/(bc)\) where the two terms in the
denominator must be grouped together by parentheses. If instead, this
was written as \(a/bc\) the expression could also be correctly read as
\((a/b) \times c = (ac)/b\) which is different from \(a/(bc)\). This is
reason enough to justify the use of brackets in mathematical
expressions, which we take a look at next.

\subsubsection{BIDMAS}\label{bidmas}

When a mathematical expression is evaluated, we work from right to left,
respecting
\href{https://en.wikipedia.org/wiki/Order_of_operations}{operator
precedence}. This is a convention that lays down a hierarchy or protocol
about which operation is performed before which. It is often reduced to
the \href{https://www.dictionary.com/browse/mnemonic}{mnemonic}
\href{https://en.wikipedia.org/wiki/Order_of_operations\#Mnemonics}{BIDMAS}.

The initial \emph{B} stands for brackets, or parentheses. Bracketed
expressions are evaluated first. Then we evaluate \emph{I} or indices:
powers and square roots. The \emph{DMAS} stands for division,
multiplication, addition, and subtraction in that order.

This \emph{convention} ensures that everyone is
\href{https://www.gingersoftware.com/content/phrases/on-the-same-page}{on
the same page} when evaluating mathematical expressions. All will get
the same result. Ambiguity is hence exiled from the mathematical
landscape.

If you love mathematical symbols, you might wish to remember this
unpronounceable visual mnemonic instead: \[
()x^y\div\times+-
\] Choose whichever mnemonic appeals more to you.

\subsubsection{A visual metaphor for mathematical
expressions}\label{a-visual-metaphor-for-mathematical-expressions}

My preferred visual image for a mathematical expression is a tied-up
bundle of clothes:

\begin{figure}
\centering
\includegraphics[width=0.8\linewidth,height=\textheight,keepaspectratio]{images/bundle-of-clothes-in-disarray.jpg}
\caption{Bundle of clothes as a vsiual metaphor for a mathematical
expression.}\label{fig:clothes-bundle}
\end{figure}

\subsection{Equations}\label{equations}

We now look at \emph{equations}. All equations embody the \(=\) symbol,
which is called an \emph{equals sign}. It is a mathematical shorthand to
denote that what is on the LHS of this symbol is equal to what is on the
RHS, however different they may appear to be. We have previously
encountered this symbol in the very simple equation \[
2 + 4 = 6.
\]

\subsubsection{Operations and relations}\label{operations-and-relations}

Before venturing further, we need to distinguish between
\emph{operations} and \emph{relations}\footnote{I have avoided a set
  theoretic framework and notation to keep this blog within the grasp of
  young students.}.

Addition, multiplication, exponentiation, etc., are familiar
\href{https://en.wikipedia.org/wiki/Binary_operation}{binary
operations}, which take two inputs or \emph{operands} and produce a
single output or result, as in \[
2 + 4 = 6.
\]

Equality is an
\href{https://en.wikipedia.org/wiki/Equivalence_relation}{equivalence
relation} that is
\href{https://en.wikipedia.org/wiki/Reflexive_relatio}{reflexive},
\href{https://en.wikipedia.org/wiki/Symmetric_relation}{symmetric}, and
\href{https://en.wikipedia.org/wiki/Transitive_relation}{transitive}.

\begin{enumerate}
\tightlist
\item
  Reflexivity means that every number is equal to itself:
  \(2 + 4 = 2 + 4\) and \(6 = 6\).
\item
  Symmetry means that if \(2 + 4 = 6\), then \(6 = 2 + 4\). Note that
  this is \emph{not}
  \href{https://en.wikipedia.org/wiki/Commutative_property}{commutativity}
  which applies to operands, not to relations.
\item
  Transitivity means that if \(2 + 4 = 6\) and if \(6 = 3 + 3\), then
  \(2 + 4 = 3 + 3\).
\end{enumerate}

You might think that these definitions and explanations are absurd and
were probably invented by
\href{https://www.wordhippo.com/what-is/the-plural-of/crazy.html}{crazies}
because they state the obvious---and you would not be far wrong. But the
power of these ideas lies in their ability to be generalized beyond the
immediate context in which they arose: something you would appreciate as
you plumb the deeper depths and higher heights of mathematics.

In sum, a binary operation works on two inputs to produce a third
output. A relation like equality, on the other hand, establishes the
\emph{sameness} of two mathematical entities.

\subsubsection{A visual metaphor for
equality}\label{a-visual-metaphor-for-equality}

A two-pan balance is an excellent visual metaphor for equality. Even
though the material in each pan might be different, when the pans
balance, we have equality. This means each pan contains the same weight
or mass. It is the principle behind how we buy foodstuffs. And it is
identical to the principle of equality as a mathematical relation.

\begin{figure}
\centering
\includegraphics[width=0.9\linewidth,height=\textheight,keepaspectratio]{images/two-pan-balance-in-equilibrium.jpg}
\caption{A two-pan balance in equilibrium, indicating that the mass on
the left hand side equals that on the right hand side, even though the
contents differ.}\label{fig:two-pan-balance}
\end{figure}

\subsection{Acknowledgements}\label{acknowledgements}

\subsection{Feedback}\label{feedback}

Please \href{mailto:feedback.swanlotus@gmail.com}{email me} your
comments and corrections.

\noindent A PDF version of this article is
\href{./math-fundamentals.pdf}{available for download here}:

\begin{small}

\begin{sffamily}

\url{https://swanlotus.netlify.app/blogs/math-fundamentals.pdf}

\end{sffamily}

\end{small}

\section*{References}\label{bibliography}
\addcontentsline{toc}{section}{References}

\phantomsection\label{refs}
\begin{CSLReferences}{0}{0}
\bibitem[\citeproctext]{ref-sas}
\CSLLeftMargin{{[}1{]} }%
\CSLRightInline{R (Chandra) Chandrasekhar. 2025. {Secrets of Academic
Success}: {Timeless Principles for Lifelong Learning}. Retrieved 11
February 2025 from
\url{https://swanlotus.netlify.app/sas-manuscript/SAS-partial.pdf}}

\end{CSLReferences}



\end{document}
